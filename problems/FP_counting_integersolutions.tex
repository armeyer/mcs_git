\documentclass[problem]{mcs}

\begin{pcomments}
  \pcomment{FP_counting_integersolutions}
  \pcomment{from: Sharon/Paul? S13}
  \pcomment{format 4/27/18 ARM}
\end{pcomments}

\pkeywords{
  donut
  binomial_coefficient
  integer_solution
}

%%%%%%%%%%%%%%%%%%%%%%%%%%%%%%%%%%%%%%%%%%%%%%%%%%%%%%%%%%%%%%%%%%%%%
% Problem starts here
%%%%%%%%%%%%%%%%%%%%%%%%%%%%%%%%%%%%%%%%%%%%%%%%%%%%%%%%%%%%%%%%%%%%%

\begin{problem}
  Please give \textbf{numerical answers} together with brief \textbf{explanations} for each of the following questions.

  \bparts
  
  \ppart How many \emph{positive} integer solutions are there to
  equation~(sumx)?
  
  \begin{equation}\tag{sumx}
    x_1+x_2+x_3+x_4+x_5 = 20
  \end{equation}
  
  \examspace[2in]
  
  \begin{solution}
    \[
      \binom{19}{4}.
    \]
    
    We are basically doing the donut problem with 15 donuts and 5 people.
    We have 15 donuts because each person has already been given one
    because each person eats a positive number of donuts.
  \end{solution}
  
  \ppart How many nonnegative \emph{even} integer solutions are there
  to~(sumx)?
  \examspace[2in]
  
  \begin{solution}
    \[
      \binom{14}{4}.
    \]
    
    This is the same problem as having five nonnegative integers that add
    up to 10.  There is a bijection because we can simply double each
    number to a solution in even numbers.
  \end{solution}
  
  \ppart How many nonnegative \emph{odd} integer solutions are there
  to~(sumx)?
  
  \examspace[2in]
  
  \begin{solution}
    0. If all five numbers are odd, their sum is odd.
  \end{solution}
  
  \eparts
  
\end{problem}

%%%%%%%%%%%%%%%%%%%%%%%%%%%%%%%%%%%%%%%%%%%%%%%%%%%%%%%%%%%%%%%%%%%%% 
% Problem ends here
%%%%%%%%%%%%%%%%%%%%%%%%%%%%%%%%%%%%%%%%%%%%%%%%%%%%%%%%%%%%%%%%%%%%%

\endinput

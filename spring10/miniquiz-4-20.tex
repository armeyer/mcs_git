\documentclass[11pt,twoside]{article}    
\usepackage{latex-macros/course}% !! use the macros 
\usepackage{subfigure}

%\hidesolutions
%\showsolutions

\instatements{
\renewcommand{\exampreamble}{	% !! renew \exampreamble 
  \textbf{Circle the name of your TA/LA}:
  \begin{center}
  \renewcommand{\arraystretch}{1.5}
  \begin{tabular}{c}
        \courseassistants
  \end{tabular}
  \end{center}
  \begin{itemize}

  \item 
  This quiz is \textbf{closed book}.  Total time is 25 minutes.

  \item 
  There are four (4) problems totalling 15 points.

  \item 
  Write your solutions in the space provided.  If you need more
  space, write on the back of the sheet containing the problem.
  Please keep your entire answer to a problem on that problem's page.

\iffalse
  \item  
  \textbf{Be neat and write legibly}.  You will be graded not only on
  the correctness of your answer, but also on the clarity with which
  you express it.
\fi

  \item 
  GOOD LUCK!
  \end{itemize}}
}

\begin{document} 
\miniquiz{Apr. 20}			% !! start the paper


\begin{problem}[5] \ 
%[Asymptotics]
%[CP9.W.3]

Circle each of the true statements below:
\begin{itemize}
\item $3n = o(n^2)$
%True
\item $\ln n = O(n^k)$, $k > 1$ a constant
%True
\item $3^{n/2} = O(3^n)$ 
%True
\item $(3n-7)/(n+4) = \Theta(1)$
%True
\item $(3n-7)/(n+4) \sim 1$
%False
\item $n^k = O(k^n)$, $k > 1$ a constant
%True
\item $k^n = O(n^k)$, $k > 1$ a constant
%False
\item $3^n = O(2^n)$ 
%False
\item $2^n = o(3^n)$
%True
\item $\sum_{i = 1}^n i = O(n)$
%False
\end{itemize}

\solution{Items 1, 2, 3, 4, 6, and 9 are true.
}

\end{problem}



\instatements{\newpage}

\begin{problem}[2]
%[Stirling/asymptotics]
%[PS8, 1d]

Show that 
\[
\ln (n^2!) = \Theta(n^2 \ln n)
\]


\solution{By Stirling's formula:
\[
n^2! \sim \sqrt{2 \pi n^2} \left(\frac{n^2}{e}\right)^{n^2}
\]
Taking logarithms gives:
%
\begin{align*}
\ln(n^2!)
    & \sim \ln(\sqrt{2 \pi n^2} \left(\frac{n^2}{e}\right)^{n^2}) \\
    & = \ln(\sqrt{2 \pi n^2}) + \ln\left(\frac{n^2}{e}\right)^{n^2} \\
    & = \frac{1}{2}\ln 2\pi + \ln n + n^2 \ln (\frac{n^2}{e}) \\
    & = \frac{1}{2}\ln 2\pi + \ln n + n^2(2 \ln n - 1)
\end{align*}
%
It is then easy to see that this expression and $n^2 \ln n$ are big-O of each other, so we conclude that $\ln (n^2!) = \Theta(n^2 \ln n)$.
}

\end{problem}



\instatements{\newpage}

\begin{problem}[4]
%[Sum/product rules]
%[CP9F.1, TP9.5]
A license plate consists of either:

\begin{itemize}

\item 3 letters followed by 3 digits (standard plate)

\item 5 letters (vanity plate)

\item 2 characters -- letters or numbers (big shot plate)


\end{itemize}

Let $L$ be the set of all possible license plates.

\begin{problemparts}

\problempart[2] Express $L$ in terms of
%
\begin{align*}
{\cal A} & = \set{ A, B, C, \dots, Z} \\
{\cal D} & = \set{ 0, 1, 2, \dots, 9} \\
\end{align*}
%
using unions ($\cup$) and set products ($\times$ or the compact exponent notation).

\solution[\vspace{1.5in}]{
\[
L = (A^3 \times D^3) \union A^5 \union (A\union D)^2
\]
}

\problempart[2] Compute $\card{L}$, the number of different license plates,
using the sum and product rules.

\solution{
\begin{align*}
\card{L}
  & = \card{(A^3 \times D^3) \union A^5 \union (A\union D)^2} \\
  & = \card{(A^3 \times D^3)} + \card{A^5} + \card{(A\union D)^2}  &  \text{Sum Rule} \\
    & = \card{A}^3 \cdot \card{D}^3 + \card{A}^5 + \card{A\union D}^2 & \text{Product Rule} \\ 
    & = \card{A}^3 \cdot \card{D}^3 + \card{A}^5 + (\card{A}+\card{D})^2 & \text{Sum Rule} \\ 
    & = 26^3 \cdot 10^3 + 26^5 + 36^2 = 29458672
\end{align*}
}

\end{problemparts}
\end{problem}



\instatements{\newpage}

\begin{problem}[4]
%[Bijections]
%[PS8.4c,e,CP9F.2b]
There are 20 different books arranged in a row on a shelf.
\begin{problemparts}
\problempart[1] 
Describe a bijection between ways of choosing $x$ distinct books, $0 \leq x \leq 20$, and 20-bit sequences with exactly $x$ ones.

\solution[\vspace{2in}]{There is a bijection from 20-bit sequences with $x$ ones to book selections: map each 1 to a chosen book and each 0 to a non-chosen book.}

\problempart[1]
How many ways are there of choosing at least 2 books from the 20? (Hint: How many ways are there of not choosing at least 2 books?)
\solution[\newpage]{If you do not choose at least 2 books, you have chosen 0 or 1 book. 
There is only 1 way of choosing no books, and 20 ways of choosing 1 book.
Since there are $2^{20}$ 20-bit sequences, there are $2^{20}-20-1$ 20-bit sequences with at least one 1.}

\problempart[1]
Describe a
bijection between ways of choosing 6 of these books so that no two
adjacent books are selected and $15$-bit sequences with exactly 6 ones.

\solution[\vspace{4in}]{There is a bijection from 15-bit sequences with exactly six
1's to valid book selections: given such a sequence, map each zero to
a non-chosen book, each of the first five 1's to a chosen book
followed by a non-chosen book, and the last 1 to a chosen book.  
}

\problempart[1]
Using your answer from part (c), give the bit representation corresponding to the selection of the 1st, 3rd, 5th, 10th, 13th, and 19th books.

\solution{111000101000010}

\end{problemparts}
\end{problem}




\instatements{\newpage}
\section*{Appendix}

\begin{lemma*}[Stirling's Formula]
\[
n! \sim \left(\frac{n}{e}\right)^n \sqrt{2 \pi n},
\]
\end{lemma*}

For functions $f,g: \reals \to \reals$, we say $f$ is \emph{asymptotically
equal} to $g$, in symbols,
\[
f(x) \sim g(x)
\]
iff
\[
\lim_{x \rightarrow \infty} f(x)/g(x) = 1.
\]

For functions $f,g: \reals \to \reals$, we say $f$ is \emph{asymptotically
smaller} than $g$, in symbols,
\[
f(x) = o(g(x)),
\]
iff
\[
\lim_{x \rightarrow \infty} f(x)/g(x) = 0.
\]

Given functions $f, g : \reals \mapsto \reals$, with $g$ nonnegative, we
say that\footnote{
\[
\limsup_{x \rightarrow \infty} h(x) \eqdef \lim_{x \rightarrow \infty}
\text{lub}_{y \geq x} h(y).
\]}
\[
f = O(g)
\]
iff
\[
\limsup_{x \rightarrow \infty} \abs{f(x)}/g(x) < \infty.
\]

An alternative, equivalent, definition is
\[
f = O(g)
\]
iff there exists a constant $c \geq 0$ and an $x_0$ such that for all $x \geq
x_0$, $\abs{f(x)} \leq c g(x)$.

Finally, we say
\[
f = \Theta(g) \quad\text{iff}\quad f=O(g) \land g=O(f).
\]



\end{document}

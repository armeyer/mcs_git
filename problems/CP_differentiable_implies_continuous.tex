\documentclass[problem]{mcs}

\begin{pcomments}
  \pcomment{from: S09.cp2m}
%  \pcomment{}
%  \pcomment{}
\end{pcomments}

\pkeywords{
  translating_english_statements
  logic
  implies
}

%%%%%%%%%%%%%%%%%%%%%%%%%%%%%%%%%%%%%%%%%%%%%%%%%%%%%%%%%%%%%%%%%%%%%
% Problem starts here
%%%%%%%%%%%%%%%%%%%%%%%%%%%%%%%%%%%%%%%%%%%%%%%%%%%%%%%%%%%%%%%%%%%%%

\begin{problem}
When the Mathematician says to his student ``If a function is not
continuous, then it is not differentiable,'' then letting $D$ stand for
``differentiable'' and $C$ for continuous, the only proper translation of the
Mathematician's statement would be
\[
\QNOT(C)\ \QIMP\ \QNOT(D),
\]
or equivalently,
\[
D\ \QIMP\ C.
\]

But when a Mother says to her son, ``If you don't do your homework, then
you can't watch TV,'' then letting $T$ stand for ``watch TV'' and $H$ for
``do your homework'', a sensible transation of the Mother's statement would
be
\[
\QNOT(H)\ \QIFF\ \QNOT(T),
\]
or equivalently,
\[
H\ \QIFF\ T.
\]

Explain why it is reasonable to translate these two IF-THEN statements in
different ways into propositional formulas.

\solution{We know that a differentiable function must be continuous, so
  when a function is not continuous, it is also not differentiable.  Now
  Mathematicians use $\QIMP$ in the technical way given by its truth table.
  In particular, if a function \emph{is} continuous then to a
  Mathematician, the implication
\[
\QNOT(C)\ \QIMP\ \QNOT(D),
\]
is automatically true since the hypothesis (left hand side of the IMPLIES)
is false.  So whether or not continuity holds, the Mathematician could
comfortably assert the $\QIMP$ statement knowing it is correct.

And of course a Mathematician does \emph{not} mean $\QIFF$, since she
knows a function that is not differentiable may well be continuous.

On the other hand, while the Mother certainly means that her son cannot
watch TV if he does not do his homework, both she and her son \emph{most
  likely} understand that if he \emph{does} do his homework, then he
\emph{will} be allowed watch TV.  In this case, even though the Mother
uses an IF-THEN phrasing, she really means $\QIFF$.

On the other hand, circumstances in the household might be that the boy may
watch TV when he has not only done his homework, but \emph{also} cleaned up
his room.  In this case, just doing homework would not imply being allowed
to watch TV --the boy won't be allowed to watch TV if he hasn't cleaned his
room, even if he has done his homework.

The general point here is that semantics (meaning) trumps syntax (sentence
structure): even though the Mathematician's and Mother's statements have
the same structure, their meaning may warrant different translations into
precise logical language.
}

\end{problem}

%%%%%%%%%%%%%%%%%%%%%%%%%%%%%%%%%%%%%%%%%%%%%%%%%%%%%%%%%%%%%%%%%%%%%
% Problem ends here
%%%%%%%%%%%%%%%%%%%%%%%%%%%%%%%%%%%%%%%%%%%%%%%%%%%%%%%%%%%%%%%%%%%%%

\endinput

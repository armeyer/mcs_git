\documentclass[handout]{mcs}

\begin{document}

\inclassproblems{3, Wed.}

%%%%%%%%%%%%%%%%%%%%%%%%%%%%%%%%%%%%%%%%%%%%%%%%%%%%%%%%%%%%%%%%%%%%%
% Problems start here
%%%%%%%%%%%%%%%%%%%%%%%%%%%%%%%%%%%%%%%%%%%%%%%%%%%%%%%%%%%%%%%%%%%%%

\pinput{CP_undescribable_language}
\pinput{CP_foundation_axiom}

%%%%%%%%%%%%%%%%%%%%%%%%%%%%%%%%%%%%%%%%%%%%%%%%%%%%%%%%%%%%%%%%%%%%%
% Problems end here
%%%%%%%%%%%%%%%%%%%%%%%%%%%%%%%%%%%%%%%%%%%%%%%%%%%%%%%%%%%%%%%%%%%%%

\inhandout{
\newpage
\textbox{
\begin{center}
\large Cantor's Theorem

\textbf{There is no bijection between any set $A$ and its powerset
$\power(A)$.}
\end{center}

\begin{proof}
  We show that if $g$ is a total function from $A$ to $\power(A)$,
  then $g$ does not have the $[\ge 1\ \text{in}]$, surjection
  property, and so is certainly not a bijection.

  Define
  \[
  A_g \eqdef \set{a \in A \suchthat a \notin g(a)}.
  \]
  Since $g$ is total, $A_g$ is a well-defined subset of $A$, which
  means it is a member of $\power(A)$.  We claim $A_g$ is not in the
  range of $g$, and so $g$ is not a surjection.

  To prove that $A_g \notin \range{g}$, assume to the contrary that it
  was in $\range{g}$.  That is,
\[
A_g = g(a_0)
\]
for some $a_0 \in A$.  Then by definition of $A_g$,
\[
a \in g(a_0) \qiff a \in A_g \qiff a \notin g(a)
\]
for all $a \in A$.  Now letting $a = a_0$ yields the contradiction
\[
a_0 \in g(a_0) \qiff a_0 \notin g(a_0).
\]

\end{proof}
}
}

\end{document}

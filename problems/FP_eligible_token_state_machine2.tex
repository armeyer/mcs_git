\documentclass[problem]{mcs}

\begin{pcomments}
  \pcomment{FP_eligible_token_state_machine2}
  \pcomment{parts of PS_eligible_token_state_machine}
  \pcomment{S18.final zabel}
\end{pcomments}

\pkeywords{
  state_machine
  invariant
  preserved_invariant
  reachable
  induction
  remainder
  derived_variable
  constant
}

%%%%%%%%%%%%%%%%%%%%%%%%%%%%%%%%%%%%%%%%%%%%%%%%%%%%%%%%%%%%%%%%%%%%%
% Problem starts here
%%%%%%%%%%%%%%%%%%%%%%%%%%%%%%%%%%%%%%%%%%%%%%%%%%%%%%%%%%%%%%%%%%%%%

\begin{problem}

\emph{Token replacing-1-3} is a single player game using a set of tokens,
each colored black or white.  In each move, a player can replace a
black token with three white tokens, or replace a white token with
three black tokens.  We can model this game as a state machine whose
states are pairs $(b,w)$ of nonnegative integers, where $b$ is the
number of black tokens and $w$ the number of white ones.

The game has two possible start states: $(5,4)$ or $(4,3)$.  We call a
state \emph{reachable} if it is reachable from \emph{at least one} of
the two start states.

We call a state $(b,w)$ \emph{eligible} when
\begin{align}
\rem{b-w}{4} & = 1, \QAND\\
\min\set{b,w} & \geq 3.
\end{align}
(Recall that $\rem{n}{4}$ denotes the number $0\le r \le 3$ such that
$n = 4q+r$ for some $q\in\integers$.)

This problem examines the connection between eligible states and reachable states.

\bparts

\ppart Give an example of a reachable state that is not eligible.  Explain.

\exambox{0.7in}{0.4in}{0in}

\begin{solution}
The state $(b,w) = (7,2)$ is directly reachable from the start state
$(4,3)$, but it is not eligible because $\min\set{7,2} < 3$.
\end{solution}

\examspace[1in]

\ppart Show that the derived variable $b+w$ is strictly increasing.
Conclude that state $(3,2)$ is not reachable.

\examspace[2in]
    
\begin{solution}
Each transition increases $b+w$ by $2$, so $b+w$ is strictly
increasing.  Also, $b+w$ for the start states is $\geq 7$, so all
reachable states have $b+w \geq 7$. But $3+2 < 7$, so $(3,2)$ is not
reachable.
\end{solution}


\ppart Verify that $\rem{3b+w}{8}$ is a derived variable that is
constant.  Conclude that no state is reachable from both start states.

\examspace[1.0in]

\begin{solution}
The state $(b,w)$ transitions to either $(b+1, w-3)$ or $(b-3,
w+1)$. In the first case, $3b+w$ transitions to $3(b+1)+(w-3) = 3b+w$,
and in the second case, it becomes $3(b-3)+(w+1) = 3b+w-8$. In both
cases, $3b+w$ changes by a multiple of $8$ and so $\rem{3b+w}{8}$
stays constant.

This derived variable takes the value 3 at start state $(5,4)$ and
therefore has value $3$ at every state reachable from $(5,4)$. By
contrast, it takes the value $7$ at start state $(4,3)$ and every
state reachable from it, so these two sets do not overlap.
\end{solution}

\examspace
\ppart\label{bwebg6} Suppose $(b,w)$ is eligible and $b \geq 6$.
Verify that $(b-3,w+1)$ is eligible.

\begin{solution}
We have $w \geq 3$ and $\rem{b-w}{4} = 1$ since $(b,w)$ is eligible.
Therefore,
\begin{align*}
\min\set{{b-3,w+1}}
    & \geq 3   & \text{since}\ b \geq 6,\\
\rem{(b-3) - (w+1)}{4}
    & = \rem{(b-w) -4}{4}\\
    &  = \rem{b-w}{4} = 1,
\end{align*}
which proves that $(b-3,w+1)$ is eligible.
\end{solution}

\eparts

\examspace[1.5in]

We now prove that every eligible state is reachable.  For the rest of
the problem, you may---and should---\textbf{assume} the following
Fact:
\begin{fact*}
If $\max\set{b,w} \leq 5$ and $(b,w)$ is eligible, then $(b,w)$ is
reachable.
\end{fact*}
(This is easy to verify since there are only nine states with $b,w \in
\set{3,4,5}$, but don't waste time doing this.)

\bparts

\ppart\label{indhypP} Define the predicate $P(n)$ to be:
\[
\forall (b,w). [b+w = n \QAND\ (b,w)\ \text{is eligible}] \QIMPLIES
(b,w)\ \text{is reachable}.
\]
Prove \textbf{carefully by strong induction} that $P(n)$ is true for
every integer $n\ge 0$.  \hint Prove and use the fact that $P(n-1)
\QIMPLIES P(n+1)$.

\examspace[3.5in]

\begin{solution}
\begin{proof}
As suggested, we'll use strong induction, with $P(n)$ as our inductive
hypothesis.  The given Fact proves that $P(0)$ through $P(5)$ are true,
because $b+w \le 5$ implies that $\max(b,w) \le 5$, so we may take
these as base cases.

For the inductive step, assume $n\ge 5$ and assume that $P(0)$ through
$P(n)$ are all true---we must prove $P(n+1)$.

To do this, we must show that for any $(b,w)$ with $b+w = n+1$, if
$(b,w)$ is eligible, then $(b,w)$ is reachable.

The Fact means we may assume $\max\set{b,w} \geq 6$.  There are two
cases, $b \geq 6$ and $w \geq 6$.

In the case that $b \geq 6$, part~\eqref{bwebg6} implies that
\[
(b-3,w+1) \text{ is eligible}.
\]
Moveover $(b-3)+(w+1) = (b+w-2) = n-1$.  Now we know that $P(n-1)$ is
true because $n\ge 1$, and $P(n-1)$ implies that $(b-3,w+1)$ is
reachable.  But $(b-3,w+1)$ transitions to $(b,w)$ in one step,
proving that $(b,w)$ is reachable.

In the case that $w \geq 6$, the same reasoning above shows that
$(b+1, w-3)$ is reachable, and therefore $(b,w)$ is reachable.

So in any case, $(b,w)$ is reachable, as required.
\end{proof}
\end{solution}

\eparts

\end{problem}

\endinput

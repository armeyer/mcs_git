\documentclass[problem]{mcs}

\begin{pcomments}
  \pcomment{CP_n5_last_digit}
  \pcomment{excerpted from obsolete PS_RSA_correctness}
  \pcomment{ARM 3/28/13}
\end{pcomments}

\pkeywords{
  number_theory
  modular_arithmetic
  Euler_theorem
  phi
}

%%%%%%%%%%%%%%%%%%%%%%%%%%%%%%%%%%%%%%%%%%%%%%%%%%%%%%%%%%%%%%%%%%%%%
% Problem starts here
%%%%%%%%%%%%%%%%%%%%%%%%%%%%%%%%%%%%%%%%%%%%%%%%%%%%%%%%%%%%%%%%%%%%%

\begin{problem}
Prove that $n$ and $n^5$ have the same last digit.  For example:
%
\[
\underline{2}^5 = 3\underline{2}
\hspace{1in}
7\underline{9}^5 = 307705639\underline{9}
\]

\begin{solution}
Two numbers have the same last digit iff their remainders are equal in
$\Zmod{10}$, so we need to show that
\begin{equation}\label{n5nz10}
n^5 = n \inzmod{10}
\end{equation}
for all $n\in [0,10)$.  This isn't too hard to check exhaustively,
  especially with the help of Euler's Theorem.

Namely, $\phi(10) = 4$, so $n^4 = 1 \inzmod{10}$, and
therefore~\eqref{n5nz10} holds, for $n$ relatively prime to 10,
namely, 1,3,7, and 9.  We saw in the problem statement that 2
satisfies~\eqref{n5nz10}.  Then,
  \[
    4^5 = \paren{2^5}^2 = 2^2 = 4 \inzmod{10}.
  \]
We know of course that for $n = 0$ or 5, the low order digit of any
positive power $n$ is $n$.  Next, using the results for 2 and 3, we
have
\[
6^5 = 2^53^5=2\cdot 3 =6 \inzmod{10},
\]
and
\[
8^5 = (2^5)^3 = 2^3 = 8 \inzmod{10}.
\]
This covers all the numbers in $[0,10)$.

The more general way to do this is to let $p$ and $q$ be 2 and 5 in
the equation used in verifying RSA correctness:
\[
m^{1+ \phi{pq}} = m \inzmod{pq}
\]
for all primes $p \neq q$, see
Problem~\bref{CP_RSA_proving_correctness}.

\end{solution}
\end{problem}

\endinput

\documentclass[problem]{mcs}

\begin{pcomments}
  \pcomment{from: S09.cp4r}
%  \pcomment{}
%  \pcomment{}
\end{pcomments}

\pkeywords{
  induction
  ordinary_induction
}

%%%%%%%%%%%%%%%%%%%%%%%%%%%%%%%%%%%%%%%%%%%%%%%%%%%%%%%%%%%%%%%%%%%%%
% Problem starts here
%%%%%%%%%%%%%%%%%%%%%%%%%%%%%%%%%%%%%%%%%%%%%%%%%%%%%%%%%%%%%%%%%%%%%

\begin{problem}
%
Use induction to prove that
\begin{equation}\label{CP_cubic_series:P}
1^3 + 2^3 + \cdots + n^3  =  \paren{\frac{n(n+1)}{2}}^2.
\end{equation}
for all $n \geq 1$.

\iffalse
(1 + 2 + \cdots + n)^2.
\hint
\href{http://theory.csail.mit.edu/classes/6.042/fall05/ln2.pdf}
{Week 2 Notes} includes a proof that $(1 + 2 + \cdots + n)^2 =
\paren{\frac{n(n+1)}{2}}^2$.  You may assume this in your proof here.
\fi

Remember to formally 
\begin{enumerate}
\item Declare proof by induction.
\item Identify the induction hypothesis $P(n)$.
\item Establish the base case.
\item Prove that $P(n)\Rightarrow P(n+1)$.
\item Conclude that $P(n)$ holds for all $n\geq 1$.  
\end{enumerate}
as in the five part template.  

\begin{solution}
We proceed by induction.  The induction hypothesis, $P(n)$, will
be the equation~\eqref{CP_cubic_series:P}.

\textbf{Base case:} First, we must show that $P(1)$ is true.  This is 
immediate, since:

\begin{eqnarray*}
1^3 & = & \left(\frac{1 (1+1)}{2}\right)^2
\end{eqnarray*}

\textbf{Inductive step:} Next, we must show that $P(n)$ implies $P(n + 1)$
for all $n \geq 1$.  Assuming that $P(n)$ is true, we can reason as
follows:

\begin{eqnarray*}
1^3 + 2^3 + \cdots + n^3 + (n + 1)^3
    & = & \paren{\frac{n(n+1)}{2}}^2 + (n + 1)^3 \\
    & = & \paren{\frac{(n + 1)(n+2)}{2}}^2
\end{eqnarray*}

The first step uses the assumption $P(n)$, and the second
uses only algebra.  This shows that $P(n + 1)$ is true.  
Therefore, $P(n)$ is true for all $n \geq 1$ by induction.
\end{solution}

\end{problem}

%%%%%%%%%%%%%%%%%%%%%%%%%%%%%%%%%%%%%%%%%%%%%%%%%%%%%%%%%%%%%%%%%%%%%
% Problem ends here
%%%%%%%%%%%%%%%%%%%%%%%%%%%%%%%%%%%%%%%%%%%%%%%%%%%%%%%%%%%%%%%%%%%%%

\endinput

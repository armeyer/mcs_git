\documentclass[problem]{mcs}

\begin{pcomments}
  \pcomment{CP_DAG_to_WPO}
  \pcomment{from: S09.cp7r, F06.cp5m}
  \pcomment{commented out in S09 - proofread before using}
  \pcomment{This problem could potentially be revised to use the
            positive walk relation and strict partial orders instead
            or in addition to the walk relation and weak POs.}
\end{pcomments}

\pkeywords{
  DAGs
  partial_orders
  transitive_closure
  digraphs
  relations
}

%%%%%%%%%%%%%%%%%%%%%%%%%%%%%%%%%%%%%%%%%%%%%%%%%%%%%%%%%%%%%%%%%%%%%
% Problem starts here
%%%%%%%%%%%%%%%%%%%%%%%%%%%%%%%%%%%%%%%%%%%%%%%%%%%%%%%%%%%%%%%%%%%%%

\begin{problem}
Prove that if $R$ is the graph of a DAG, then the walk relation $\mrel{R^*}$ is
a weak partial order on the vertices.

\begin{solution}
\emph{reflexive}: Every vertex $a$ is connected to itself by
a 0-length path, so $a\mrel{R^*}a$ holds for all $a$.

\emph{antisymmetric}: Suppose $a\neq b$ and $a\mrel{R^*}b$, that is,
there is a (necessarily positive length) walk from $a$ to $b$.  If
$b\mrel{R^*}a$ also held, then there is also a walk from $b$ to $a$,
and the two paths could be merged into a positive length closed walk.
Then Lemma~\bref{shortestclosedwalk_lem} implies the graph has a
cycle, contradicting the fact that the graph is acyclic.  We conclude
that $\QNOT(b\mrel{R^*}a)$, which proves antisymmetry.

\emph{transitive}: $a\mrel{R^*}b$ means there is a walk from $a$ to $b$, and
likewise, $b\mrel{R^*}c$ means there is a walk from $b$ to $c$.  Then the
first walk followed by the second would form a walk from $a$ to $c$.
This implies that $a\mrel{R^*}c$, proving transitivity.
\end{solution}

\end{problem}

%%%%%%%%%%%%%%%%%%%%%%%%%%%%%%%%%%%%%%%%%%%%%%%%%%%%%%%%%%%%%%%%%%%%%
% Problem ends here
%%%%%%%%%%%%%%%%%%%%%%%%%%%%%%%%%%%%%%%%%%%%%%%%%%%%%%%%%%%%%%%%%%%%%
 
\endinput

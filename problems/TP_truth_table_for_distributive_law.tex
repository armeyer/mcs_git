\documentclass[problem]{mcs}

\begin{pcomments}
  \pcomment{TP_truth_table_for_distributive_law}
  \pcomment{from: S17.cp2f, S09.cp2m}
\end{pcomments}

\pkeywords{
  truth_table
  logic
  proposition
  distributive
}

%%%%%%%%%%%%%%%%%%%%%%%%%%%%%%%%%%%%%%%%%%%%%%%%%%%%%%%%%%%%%%%%%%%%%
% Problem starts here
%%%%%%%%%%%%%%%%%%%%%%%%%%%%%%%%%%%%%%%%%%%%%%%%%%%%%%%%%%%%%%%%%%%%%

\begin{problem}
Prove by truth table that $\QOR$ distributes over $\QAND$, namely,
\begin{equation}\label{dist}
%\brac{P \; \QOR\ (Q\ \QAND\ R)} \quad\text{is equivalent to}\quad
P \QOR (Q \QAND R) \quad\text{is equivalent to}\quad
%\brac{(P\ \QOR\ Q)\ \QAND\ (P\ \QOR\ R)}
(P \QOR Q) \QAND (P \QOR R)
\end{equation}

\begin{staffnotes}
Students often copy subformulas into their own truth tables instead of
just writing the truth values under the principal operator of the
subformula.  Have them do it w/o copying subformulas as described in
Section~\bref{truthtablecalc} of the text and illustrated below.  Note
that this method is much more economical for formulas with many
subformulas.
\end{staffnotes}

\begin{solution}
\[
\begin{array}{|c|c|c|ccccc|ccccccc|}
\hline
P 			& Q 			& R 			& P & \QOR 			& \text{(}Q & \QAND 	& R\text{)} & \text{(}P & \QOR    & Q\text{)} & \QAND    	& \text{(}P & \QOR    & R\text{)} \\  \hline
\true  	&	\true 	&	\true  	&   & \lgtrue  	& 					&	\true   & 					&						& \true		&						& \lgtrue		&						& \true		&	\\  \hline
\true  	& \true		& \false	&   & \lgtrue  	&						& \false  & 					&						&	\true		&						& \lgtrue		&						& \true		&	\\  \hline
\true  	& \false	& \true		&   & \lgtrue  	&						& \false  &						&						&	\true 	&						& \lgtrue		&						& \true		&	\\  \hline
\true  	& \false 	& \false	&   & \lgtrue  	&						& \false  &						&						& \true		&						& \lgtrue		&						& \true		&	\\  \hline
\false 	& \true		& \true		&   & \lgtrue  	&						& \true   &						&						& \true		&						& \lgtrue		&						& \true		&	\\  \hline
\false 	& \true		& \false	&   & \lgfalse 	&						& \false  &						&						& \true		&						& \lgfalse	&						& \false	&	\\  \hline
\false 	& \false	& \true		&   & \lgfalse 	&						& \false  &						&						& \false	&						& \lgfalse	&						& \true		&	\\  \hline
\false 	& \false	& \false	&   & \lgfalse 	&						& \false  &						&						& \false	&						& \lgfalse	&						& \false	&	\\  \hline
\end{array}
\]
The highlighted column giving the truth values of the first formula is
the same as the corresponding column of the second formula, so the two
propositional formulas are equivalent.
\end{solution}


\end{problem}

%%%%%%%%%%%%%%%%%%%%%%%%%%%%%%%%%%%%%%%%%%%%%%%%%%%%%%%%%%%%%%%%%%%%%
% Problem ends here
%%%%%%%%%%%%%%%%%%%%%%%%%%%%%%%%%%%%%%%%%%%%%%%%%%%%%%%%%%%%%%%%%%%%%

\endinput

%\[
%\begin{array}{|ccccc|}
%\hline
%P      & \underline{\QOR} & \text{(}Q    & \QAND & R\text{)}   \\ \hline
%\true  &\lgtrue       		& \true  & \true      & \true \\ \hline
%\true  &\lgtrue       	& \true  & \false     & \false\\ \hline
%\true  &\lgtrue       & \false & \false     & \true \\ \hline
%\true  &\lgtrue       & \false & \false     & \false\\ \hline
%\false &\lgtrue       & \true  & \true      & \true \\ \hline
%\false &\lgfalse      & \true  & \false     & \false\\ \hline
%\false &\lgfalse      & \false & \false     & \true \\ \hline
%\false &\lgfalse      & \false & \false     & \false\\ \hline
%\end{array}
%\]
%
%\[
%\begin{array}{|ccccccc|}
%\hline
%\text{(}P  & \QOR      & Q\text{)} & \QAND       & \text{(}P & \QOR      & R\text{)} \\  \hline
%     \true  & \true     & \true     & \lgtrue      &   \true   & \true     & \true \\  \hline
%     \true  & \true     & \true     & \lgtrue      &   \true   & \true     & \false\\  \hline
%     \true  & \true     & \false    & \lgtrue      &   \true   & \true     & \true \\  \hline
%     \true  & \true     & \false    & \lgtrue      &   \true   & \true     & \false\\  \hline
%     \false & \true     & \true     & \lgtrue      &   \false  & \true     & \true \\  \hline
%     \false & \true     & \true     & \lgfalse     &   \false  & \false    & \false\\  \hline
%     \false & \false    & \false    & \lgfalse     &   \false  & \true     & \true \\  \hline
%     \false & \false    & \false    & \lgfalse     &   \false  & \false    & \false\\  \hline
%\end{array}
%\]

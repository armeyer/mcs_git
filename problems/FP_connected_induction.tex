\documentclass[problem]{mcs}

\begin{pcomments}
  \pcomment{FP_connected_induction}
  \pcomment{first part of FP_inbook_connected_induction}
  \pcomment{S17.final, S01.final}
  \pcomment{added, edited ARM 5/20/17}

\end{pcomments}

\pkeywords{
  connect
  connected
}

%%%%%%%%%%%%%%%%%%%%%%%%%%%%%%%%%%%%%%%%%%%%%%%%%%%%%%%%%%%%%%%%%%%%%
% Problem starts here
%%%%%%%%%%%%%%%%%%%%%%%%%%%%%%%%%%%%%%%%%%%%%%%%%%%%%%%%%%%%%%%%%%%%%

\begin{problem}
MIT Information Services \& Technology (IS\&T) wants to assemble a
cluster of $n$ computers using wires and hubs.  Each computer must
have exactly one wire attached to it, while each hub can have up to
five wires attached.  There must be a path of wires between every pair
of computers in the cluster.


Prove by induction that $\ceil{(n-2)/3}$ hubs are sufficient for IS\&T
to assemble the cluster of $n$ computers.

\examspace[4.0in]

\begin{solution}

\begin{proof}
The proof will be by strong induction on $n$ with hypothesis
\[
P(n) \eqdef n \text{ computers can be clustered using } \ceil{\frac{n-2}{3}} \text{ degree-5 hubs}.
\]

\inductioncase{Base Cases}.
($n = 0,1,2$): Zero hubs are needed, and $0 = \ceil{(0-2)/3} = \ceil{(1-2)/3} = \ceil{(2-2)/3}$.

($n = 3$): The three computers can be connected to one hub, and $1 = \ceil{(3-2)/3}$.

\inductioncase{Induction step}.  We must prove $P(n+1)$ assuming $P(k)$ for $0 \leq
k \leq n$, where $n\geq 3$.  In particular, we may assume $P(n-2)$.  That is, there is a
cluster $C$ of $n-2\geq 1$ computers using at most
\[
\ceil{\frac{(n-2)-2}{3}} = \ceil{\frac{(n+1)-2}{3}}-1
\]
hubs.  If we can rearrange the cluster $C$ to include one more hub and
three more computers, we will have a cluster with $n+1$ computers and
$\ceil{((n+1)-2)/3}$ hubs, thereby proving $P(n+1)$.

To rearrange $C$ in this way, we replace a computer $R$ with a new hub
$H$.  This leaves room to attach four more wires to $H$, allowing us
to attach $R$ and three new computers to the ends of these wires.
\end{proof}

\end{solution}

\end{problem}

\endinput

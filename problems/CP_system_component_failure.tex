\documentclass[problem]{mcs}

\begin{pcomments}
  \pcomment{CP_system_component_failure}
  \pcomment{from: S09.cp12r, F09}
\end{pcomments}

\pkeywords{
  probability
  union_bound
}

%%%%%%%%%%%%%%%%%%%%%%%%%%%%%%%%%%%%%%%%%%%%%%%%%%%%%%%%%%%%%%%%%%%%%
% Problem starts here
%%%%%%%%%%%%%%%%%%%%%%%%%%%%%%%%%%%%%%%%%%%%%%%%%%%%%%%%%%%%%%%%%%%%%

\begin{problem}
Suppose there is a system with $n$~components, and we know from past
experience that any particular component will fail in a given year
with probability~$p$.  That is, letting $F_i$ be the event that the
$i$th component fails within one year, we have
\[
\pr{F_i} = p
\]
for $1 \leq i \leq n$.  The \emph{system} will fail if \emph{any one} of
its components fails.  What can we say about the probability that the
system will fail within one year?

Let $F$ be the event that the system fails within one year.  Without any
additional assumptions, we can't get an exact answer for $\pr{F}$.
However, we can give useful upper and lower bounds, namely,
\begin{equation}\label{cp12r_pFnp}
p \le \pr{F} \le np.
\end{equation}
We may as well assume $p < 1/n$, since the upper bound is trivial
otherwise.  For example, if $n = 100$ and $p = 10^{-5}$, we conclude that
there is at most one chance in 1000 of system failure within a year and at
least one chance in 100,000.

Let's model this situation with the sample space $\sspace \eqdef
\power(\Zintv{1}{n})$ whose outcomes are subsets of positive integers
$\le n$, where $s \in \sspace$ corresponds to the indices of exactly those
components that fail within one year.  For example, $\set{2, 5}$ is the
outcome that the second and fifth components failed within a year and none
of the other components failed.  So the outcome that the system did not
fail corresponds to the empty set $\emptyset$.

\begin{staffnotes}
Encourage students to begin by stating explicitly what outcomes
(subsets of $\Zintv{1}{n}$) are in the event $F_i$ and $F$.
\end{staffnotes}

\bparts

\ppart Show that the probability that the system fails could be as small
as $p$ by describing appropriate probabilities for the outcomes.  Make
sure to verify that the sum of your outcome probabilities is 1.

\begin{solution}
According to the description,
\begin{align*}
F_i & = \set{s \in \sspace \suchthat i \in s},\\
F & = \union_{i=1}^n F_i = \set{s \in \sspace \suchthat s \neq \emptyset}.
\end{align*}

Now there could be a probability $p$ of system failure, that is
$\pr{F} = p$, if the individual failures always occur together.
Formally, we model this by assigning outcome probabilities as follows:
\begin{align*}
\pr{\,\Zintv{1}{n}\,} & \eqdef p,\\
\pr{\emptyset}        &\eqdef 1-p,\\
\pr{s}                & \eqdef 0 & \text{for } \emptyset \neq s \subset \Zintv{1}{n}.
\end{align*}
Now the sum of the probabilities of all the outcomes is one, so this
is a well-defined probability space.  Then the only outcome in $F_i$
with nonzero probability is $\Zintv{1}{n}$, so
\[
\pr{F_i} = \pr{\,\Zintv{1}{n}\,} = p,
\]
as specified.  Likewise, the only outcome in $F$ with nonzero
probability is also $\Zintv{1}{n}$, so
\[
\pr{F} = \pr{\,\Zintv{1}{n}\,} = p,
\]
as required.
\end{solution}

\ppart Show that the probability that the system fails could actually be
as large as $np$ by describing appropriate probabilities for the outcomes.
Make sure to verify that the sum of your outcome probabilities is 1.

\begin{solution}
Suppose at most one component ever fails at a time.  Formally, we
model this by assigning outcome probabilities as follows:
\begin{align*}
\pr{\,\set{i}\,} & \eqdef p & \text{for } i \in \Zintv{1}{n},\\
\pr{\emptyset} & \eqdef 1-np,\\
\pr{s}  & \eqdef 0 & \text{for } s \in \sspace \text{ such that } \card{s} > 1.
\end{align*}
Now the sum of the probabilities of all the outcomes is one, so this is a
well-defined probability space.  Also, the only outcome in $F_i$ with
positive probability is $\set{i}$, so
\[
\pr{F_i} = \pr{\,\set{i}\,} = p,
\]
as specified.  Finally, the only outcomes in $F$ with positive
probability are the $n$ outcomes $\set{i}$ for $i \in \Zintv{1}{n}$,
so
\[
\pr{F} = \pr{\,\set{1}\,} + \pr{\,\set{2}\,}+ \cdots + \pr{\,\set{n}\,} = p+p+\cdots + p = np,
\]
as required.
\end{solution}

\ppart Prove inequality~\eqref{cp12r_pFnp}.

\begin{solution}
$F = \lgunion_{i=1}^n F_i$ so
\begin{align*}
p & = \pr{F_1}          & \text{(given)}\\
  & \le \pr{F}          & \text{(since $F_1 \subseteq F$)}\\
  & = \pr{\lgunion F_i} & \text{(def.\ of $F$)} \\
  & \le \sum_{i=1}^n \pr{F_i} & \text{(Union Bound)}\\
  & = np.
\end{align*}
\end{solution}

\eparts
\end{problem}


%%%%%%%%%%%%%%%%%%%%%%%%%%%%%%%%%%%%%%%%%%%%%%%%%%%%%%%%%%%%%%%%%%%%%
% Problem ends here
%%%%%%%%%%%%%%%%%%%%%%%%%%%%%%%%%%%%%%%%%%%%%%%%%%%%%%%%%%%%%%%%%%%%%

\endinput

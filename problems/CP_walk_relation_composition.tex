\documentclass[problem]{mcs}

\begin{pcomments}
  \pcomment{CP_walk_relation_composition}
  \pcomment{better version of deleted PS_walk_relation_composition}
  \pcomment{from: digraph notes}
\end{pcomments}

\pkeywords{
  binary_relation
  composition
  walk
  walk_relation }

%%%%%%%%%%%%%%%%%%%%%%%%%%%%%%%%%%%%%%%%%%%%%%%%%%%%%%%%%%%%%%%%%%%%%
% Problem starts here
% %%%%%%%%%%%%%%%%%%%%%%%%%%%%%%%%%%%%%%%%%%%%%%%%%%%%%%%%%%%%%%%%%%%%

\begin{problem}
Let $R$ be a binary relation on a set $A$.  Regarding $R$ as a
digraph, let $W^{(n)}$ denote the length-$n$ walk relation in the digraph
$R$, that is,
\[
a \mrel{W^{(n)}} b \eqdef \mbox{there is a length $n$ walk from $a$ to
    $b$ in $R$}.
\]

\bparts

\ppart
Prove that
\begin{equation}\label{RnRpnp}
W^{(n)} \compose W^{(m)} = W^{(m+n)}
\end{equation}
for all $m,n \in \nngint$, where $\compose$ denotes relational
composition.\inhandout{
\footnote{ The composition of binary relations $R: B\to C$ with
  $S: A \to B$ is the binary relation $(R \compose S): A\to C$ defined
  by the rule
\[
a \mrel{(R \compose S)} c \eqdef\ \exists b \in B.\, (a \mrel{S} b)
\QAND (b \mrel{R} c).
\]
}}

\begin{solution}

\begin{proof}
  Any length-$(m+n)$ walk between vertices $u$ and $v$ begins with a
  length $m$ walk starting at $u$ and ending at some vertex $w$
  followed by a length $n$ walk starting at $w$ and ending at $v$.  So
 \[
u \mrel{W^{(m+n)}} v 
   \qiff \exists w.\, u \mrel{W^{(m)}} w \QAND w \mrel{W^{(n)}} v
   \qiff  u \mrel{W^{(n)} \compose W^{(m)}} v
\]
\end{proof}

\end{solution}

\ppart  Let $R^n$ be the composition
of $R$ with itself $n$ times for $n \ge 0$.  So $R^0 \eqdef
\ident{A}$, and $R^{n+1} \eqdef R \compose R^n $.  

Conclude that
\begin{equation}\label{Rn=Wn}
R^n = W^{(n)}
\end{equation}
for all $n \in \nngint$.

\begin{solution}

\begin{proof}
By induction on $n$ with equation~\eqref{Rn=Wn} as induction
hypothesis.

\inductioncase{Base case} ($n=0$): $R^0 = \ident{A}$ by definition,
and $W^{(0)}$ is the length-0 walk relation which also equals
$\ident{A}$ by definition.

\inductioncase{Inductive step:} Suppose~\eqref{Rn=Wn} holds for some $n\ge 0$.
We want to prove that
\[
R^{n+1} = W^{(n+1)}.
\]

We first observe that
\begin{equation}\label{R=W1}
R = W^{(1)}
\end{equation} by definition.

Now we have
\begin{align*}
R^{n+1} & \eqdef R \compose R^n\\
       & = W^{(1)} \compose W^{(n)} 
           & \text{(by~\eqref{R=W1} and ind. hyp.~\eqref{Rn=Wn})}\\
       & = W^{(n+1)}
           & \text{(by~\eqref{RnRpnp})}
\end{align*}

This completes the proof by induction, and we conclude that $\forall n
\in \nngint.\, R^n = W^{(n)}$.

\end{proof}
\end{solution}

\ppart Conclude that
\[
R^{+} = \lgunion_{i=1}^{\card{A}} R^i
\]
where $R^{+}$ is the positive length walk relation determined by $R$
on the set $A$.

\begin{solution}
By Theorem~\bref{shortestwalk_thm}, there is a positive length walk
from vertex $u$ to a different vertex $v$ iff there is a positive
length path from $u$ to $v$.  Likewise, by
Lemma~\bref{shortestclosedwalk_lem}, there is a positive length walk
from $u$ to itself iff $u$ is on a cycle.  Since no path or can be
longer than $n-1$ and no cycle can be longer than $n$, it follows that
\begin{align*}
u \mrel{R^{+}} v
  & \qiff \exists i \in \Zintv{1}{n}.\, \text{[there is a length
       $i$ walk from $u$ to $v$]}\\
  & \qiff \exists i \in \Zintv{1}{n}.\ u \mrel{ W^{(i)}} v\\
  & \qiff u \mrel{\paren{\lgunion_{i=1}^n W^{(i)}}} v\\
  & \qiff u \mrel{\paren{\lgunion_{i=1}^n R^i}} v & \text{by~\eqref{Rn=Wn}}.
\end{align*}

\end{solution}

\eparts
\end{problem}

%%%%%%%%%%%%%%%%%%%%%%%%%%%%%%%%%%%%%%%%%%%%%%%%%%%%%%%%%%%%%%%%%%%%%
% Problem ends here
%%%%%%%%%%%%%%%%%%%%%%%%%%%%%%%%%%%%%%%%%%%%%%%%%%%%%%%%%%%%%%%%%%%%%

\endinput

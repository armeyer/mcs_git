\documentclass[problem]{mcs}

\begin{pcomments}
  \pcomment{from: S09.ps3}
  \pcomment{revise ref to notes}
\end{pcomments}

\pkeywords{
  partial_orders
}

%%%%%%%%%%%%%%%%%%%%%%%%%%%%%%%%%%%%%%%%%%%%%%%%%%%%%%%%%%%%%%%%%%%%%
% Problem starts here
%%%%%%%%%%%%%%%%%%%%%%%%%%%%%%%%%%%%%%%%%%%%%%%%%%%%%%%%%%%%%%%%%%%%%

\begin{problem} The class notes explained how partial orders can be
  represented by a collection of sets under the subset relation.  In
  particular, if $R$ is a \emph{weak} partial order on a set, $A$, then
\begin{equation}\label{ps3_rgb}
a \mrel{R} b  \qiff  R\set{a} \subseteq R\set{b}.
\end{equation}
holds for all $a,b \in A$.  Carefully prove statement~\eqref{ps3_rgb},
starting from the definitions of weak partial order and the subset
relation.

\begin{solution}
\iffalse
  \begin{theorem}
    $a = b \qiff R\set{a} = R\set{b}$.
  \end{theorem}
  \begin{proof}
    ($\Rightarrow$) Suppose $a = b$.  Then $\set{a} = \set{b}$, and
    hence $R\set{a} = R\set{b}$. 

    ($\Leftarrow$) Suppose $R\set{a} = R\set{b}$, and let $X =
    R\set{a} = R\set{b}$.  Since $R$ is a weak partial order, it is
    reflexive (i.e., $a \mrel{R} a$ and $b \mrel{R} b$), and thus $a,b \in X$.
    Since $X = \set{ x | x \mrel{R} a}$, and $b \in X$, it follows that $b R
    a$.  Similarly, $a \mrel{R} b$ by swapping $a$ and $b$ in the previous
    sentence.  Since $R$ is a weak partial order, it must be
    antisymmetric (i.e., $a \mrel{R} b \implies \neg(b \mrel{R} a) \vee (a = b)$).
    Since $a \mrel{R} b$ and $b \mrel{R} a$, we conclude that $a = b$.
  \end{proof}
\fi

  \begin{theorem}
    $a \mrel{R} b \qiff R\set{a} \subseteq R\set{b}$.
  \end{theorem}
  \begin{proof}
    ($\Rightarrow$) Suppose $a \mrel{R} b$.  Then by transitivity of
    partial order, $x \mrel{R} a \implies x \mrel{R} b$.  Thus, $x \in R
    \set{a} \implies x \in R\set{b}$, and we conclude $R\set{a} \subseteq
    R\set{b}$.  Notice that this direction holds for all partial orders.

    ($\Leftarrow$) Suppose $R\set{a} \subseteq R\set{b}$.  Since $R$ is a
    weak partial order, $a \mrel{R} a$ is true, so we have $a \in
    R\set{a}$.  Since $R\set{a} \subseteq R\set{b}$, it follows that $a \in
    R\set{b}$.  This means $a \mrel{R} b$, By definition of $R\set{b}$.
  \end{proof}

\end{solution}

\end{problem}

%%%%%%%%%%%%%%%%%%%%%%%%%%%%%%%%%%%%%%%%%%%%%%%%%%%%%%%%%%%%%%%%%%%%%
% Problem ends here
%%%%%%%%%%%%%%%%%%%%%%%%%%%%%%%%%%%%%%%%%%%%%%%%%%%%%%%%%%%%%%%%%%%%%

\documentclass[problem]{mcs}

\begin{pcomments}
    \pcomment{TP_Eulers_Theorem}
    \pcomment{Converted from euler-theorem.scm by scmtotex and dmj
              on Sat 12 Jun 2010 09:09:34 PM EDT}
    \pcomment{subsumed by MQ_modular_arithmetic}
\end{pcomments}

\begin{problem}

\bparts

\ppart
What is the value of $\phi(1250)$, where $\phi$ is Euler's function?

\begin{solution}
$1250 = 2 \cdot 5^{4}$, so $\phi(1250)=(2 - 1)(5^{4} - 5^{3}) = 1 \cdot (625-125) = 500$.
\end{solution}

\examspace[1.0in]

\ppart
What is the remainder of $39^{10001}$ divided by 1250?

\begin{solution}
\begin{align*}
39^{10001} &= 39^{(500 \cdot 20) +1} \\
           & = (39^{500})^{20}  \cdot  39 \\
           & \equiv  1^{20}  \cdot  39 \bmod 1250\\
           & \equiv  39 \bmod 1250.
\end{align*}
\end{solution}

\examspace[2.0in]

\ppart Find the inverse of 39 modulo 1250 in the interval
$\Zintvcc{1}{1249}$?

\begin{solution}
We first use the Pulverizer to find $s,t$ such that $\gcd(39, 1250) = s\cdot 39 + t\cdot 1250$:
\[
\begin{array}{ccccrcl}
x & \quad & y & \quad & \rem{x}{y} & = & x - q \cdot y \\ \hline
1250 && 39 && 2  & = &   1250 - 32\cdot 39 \\
39 && 2 && 1 & = &   39 - 19\cdot 2 \\
&&&&            & = &   39 - 19\cdot (1250 - 32\cdot 39)\\
&&&&            & = &  609\cdot 39 - 19\cdot 1250\\
\end{array}
\]

\[
1 = 609 \cdot 39 - 19 \cdot 1250.
\]
This implies that $s = 609$ is an inverse of 39 modulo 1250.
\end{solution}

\examspace[4.0in]

\iffalse
\ppart Find the inverse of 22 modulo 175 in the interval
$\Zintvcc{1}{174}$?

\begin{solution}
Note that $8 \cdot 22 - 175 = 1$. Therefore, $8$ is an inverse of $22$ modulo $175$.

\begin{staffnotes}
Change to need Pulverizer.
\end{staffnotes}
\end{solution}
\fi

\iffalse
%\examspace[2.0in]

\ppart
What is the remainder of $22^{11999}$ divided by 175?

\begin{solution}

\begin{align*}
22^{11999} & \equiv 22^{11999} \cdot 22 \cdot 8 \bmod 175 \\
           & \equiv 22^{12000} \cdot 8 \bmod 175 \\
           &= 22^{(120 \cdot 100)} \cdot 8 \bmod 175 \\
           & = (22^{120})^{100} \cdot 8 \bmod 175 \\
           & \equiv  1^{100} \cdot 8 \bmod 175\\
           & \equiv  8 \bmod 175.
\end{align*}
\end{solution}

\examspace[2.0in]
\fi

\eparts

\end{problem}

\endinput

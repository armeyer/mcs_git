
\documentclass[problem]{mcs}

\begin{pcomments}
  \pcomment{CP_proving_probability_rules}
  \pcomment{from: S09.cp12r, F09}
\end{pcomments}

\pkeywords{
  probability
  disjoint_sum
  inclusion-exclusion  
}

%%%%%%%%%%%%%%%%%%%%%%%%%%%%%%%%%%%%%%%%%%%%%%%%%%%%%%%%%%%%%%%%%%%%%
% Problem starts here
%%%%%%%%%%%%%%%%%%%%%%%%%%%%%%%%%%%%%%%%%%%%%%%%%%%%%%%%%%%%%%%%%%%%%

\begin{problem}
Here are some handy rules for reasoning about probabilities that all
follow directly from the Disjoint Sum Rule.  Prove them.

\begin{align*}
\pr{A-B}
   & = \pr{A}-\pr{A \intersect B}  & \text{(Difference Rule)}\\
\pr{\setcomp{A}}
   &  = 1 - \pr{A}                 & \text{(Complement Rule)}\\
\pr{A \union B}
   &  = \pr{A}+\pr{B} - \pr{A \intersect B} & \text{(Inclusion-Exclusion)}\\
\pr{A \union B}
   & \leq \pr{A}+\pr{B}            & \text{(2-event Union Bound)}\\
A \subseteq B & \QIMPLIES \pr{A} \leq \pr{B} & \text{(Monotonicity)}
\end{align*}

\begin{solution}
(Difference Rule): Any set $A$ is the disjoint union of $A-B$ and $A \intersect B$, so
\[
\pr{A}= \pr{A-B}+\pr{A \intersect B}
\]
by the Disjoint Sum Rule.

(Complement Rule):
$\setcomp{A} \eqdef \sspace - A$, so by the Difference Rule
\[
\pr{\setcomp{A}} = \pr{\sspace} - \pr{A} =  1 - \pr{A}.
\]

(Inclusion-Exclusion): $A \union B$ is the disjoint union of $A$ and
$B-A$ so
\begin{align*}
\pr{A \union B} & = \pr{A} + \pr{B-A}  & \text{(Disjoint Sum Rule)}\\
     & = \pr{A} + (\pr{B} - \pr{A \intersect B})    & \text{(Difference Rule)}
\end{align*}

(2-event Union Bound):
This follows immediately from Inclusion-Exclusion and the fact that
$\pr{A \intersect B} \geq 0$.

(Monotonicity):
\begin{align*}
\pr{A} & = \pr{B} - (\pr{B} - \pr{A})\\
       & = \pr{B} - (\pr{B} - \pr{A \intersect B})
                & (\text{since } A = A \intersect B)\\
       & = \pr{B} - \pr{B - A}   & \text{(difference rule)}\\
       & \leq \pr{B} & (\text{since } \pr{B-A} \geq 0).
\end{align*}
\end{solution}

\end{problem}

%%%%%%%%%%%%%%%%%%%%%%%%%%%%%%%%%%%%%%%%%%%%%%%%%%%%%%%%%%%%%%%%%%%%%
% Problem ends here
%%%%%%%%%%%%%%%%%%%%%%%%%%%%%%%%%%%%%%%%%%%%%%%%%%%%%%%%%%%%%%%%%%%%%

\endinput

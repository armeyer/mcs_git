\documentclass[problem]{mcs}

\begin{pcomments}
\pcomment{MQ_sets_to_membership}
\pcomment{long-winded version of MQ_sets_to_membership_no_intro}
\pcomment{Created by Joliat and Kazerani; variant of CP_proving_basic_set_id, 2/27/11}
\pcomment{revised by ARM, 3/1/11 and 8/31/11}
\end{pcomments}

\pkeywords{
  logic
  set_theory
  identity
  propositional
  chain_of_iff
  difference
}

%%%%%%%%%%%%%%%%%%%%%%%%%%%%%%%%%%%%%%%%%%%%%%%%%%%%%%%%%%%%%%%%%%%%%
% Problem starts here
%%%%%%%%%%%%%%%%%%%%%%%%%%%%%%%%%%%%%%%%%%%%%%%%%%%%%%%%%%%%%%%%%%%%%

\begin{problem}
Below is a familiar ``chain of \QIFF's'' proof of the set equality
\begin{equation}\label{AUBIAA}
A \union (B \intersect A) = A.
\end{equation}

\begin{proof}
\begin{align*}
x \in A \union (B \intersect A)
& \QIFF x \in A \QOR x \in (B \intersect A) & \text{(def of $\union$)}\\
& \QIFF x \in A \QOR (x \in B \QAND x \in A) & \text{(def of $\intersect$)}\\
& \QIFF x \in A,
\end{align*}
where the last \QIFF\ follows from the fact that
\begin{quote}
the propositional formulas $P \QOR (Q \QAND P)$ and $P$ are
equivalent.
\end{quote}
\end{proof}

State a similar propositional equivalence that would justify the key
step in a chain of \QIFF's proof for the following set equality. 
\begin{equation}\label{A-BcompA-compC}
\setcomp{A-B} =
\paren{\setcomp{A} - \setcomp{C}} \union
\paren{B \intersect C}    \union
\paren{\paren{\setcomp{A} \union B} \intersect \setcomp{C}}
\end{equation}

(You are \emph{not} being asked to write out a \QIFF\ proof of the
equality or a proof of the propositional equivalence.  Just state the
equivalence.)

\iffalse
Your formula must contain only propositional variables and their
negations, parentheses, and operators such as $\QNOT$, $\QOR$
$\QAND$, and $\QIFF$.

\hint You may find it useful to define three propositions describing
the membership of $x$ in each of the sets $A$, $B$ and $C$:
\begin{eqnarray*}
P & \eqdef & x\in A\\
Q & \eqdef & x\in B\\
R & \eqdef & x\in C
\end{eqnarray*}
\fi

\begin{solution}

The stated set equality holds iff membership in $\setcomp{A-B}$ implies and is
implied by membership in
$\paren{\setcomp{A}-\setcomp{C}}\union\paren{B\intersect C}\union\paren{\paren{\setcomp{A}\union B}\intersect\setcomp{C}}$.  That is, the set equality
holds iff, for all $x$,
\[
x \in \setcomp{A-B} \qiff
x \in \paren{\setcomp{A}-\setcomp{C}} \union \paren{B\intersect C} \union
      \paren{\paren{\setcomp{A}\union B}\intersect\setcomp{C}}.
\]
Define three propositions describing the membership of $x$ in each of
the sets $A$, $B$ and $C$:
\begin{eqnarray*}
P & \eqdef & x\in A\\
Q & \eqdef & x\in B\\
R & \eqdef & x\in C\\
\end{eqnarray*}
Now, express membership in $\setcomp{A-B}$ in terms of $P$, $Q$ and $R$:
\begin{align*}
\lefteqn{x \in \setcomp{A-B}}\\
 & \qiff \QNOT\paren{x \in \paren{A\intersect\setcomp{B}}}\\
 & \qiff \QNOT\paren{x \in A \QAND x \in \setcomp{B}}\\
 & \qiff \QNOT\paren{x \in A \QAND \QNOT\paren{x \in B}}\\
 & \qiff \QNOT\paren{P \QAND \QNOT\paren{Q}}
\end{align*}
Then express membership in
\[
\paren{\setcomp{A} - \setcomp{C}} \union \paren{B \intersect C} \union
\paren{\paren{\setcomp{A}\union B} \intersect \setcomp{C}}
\]
in terms of $P$, $Q$ and $R$:
\begin{align*}
\lefteqn{x \in \paren{\setcomp{A}-\setcomp{C}}\union\paren{B\intersect C}\union\paren{\paren{\setcomp{A}\union B}\intersect\setcomp{C}}}\\
 & \qiff x \in \paren{\setcomp{A}-\setcomp{C}} \QOR x \in \paren{B\intersect C} \QOR x \in \paren{\paren{\setcomp{A}\union B}\intersect\setcomp{C}}\\
 & \qiff x \in \paren{\setcomp{A}\intersect \setcomp{\setcomp{C}}} \QOR x \in \paren{B\intersect C} \QOR \paren{x \in \paren{\setcomp{A}\union B}\QAND x\in \setcomp{C}}\\
 & \qiff x \in \paren{\setcomp{A}\intersect C} \QOR x \in \paren{B\intersect C} \QOR \paren{x \in \paren{\setcomp{A}\union B}\QAND x\in \setcomp{C}}\\
 & \qiff \paren{x \in\setcomp{A}\QAND x\in C} \QOR \paren{x \in B\QAND
  x\in C}\\
&\qquad\qquad \QOR \paren{\paren{x \in \setcomp{A} \QOR x \in B}\QAND x\in \setcomp{C}}\\
 & \qiff \paren{\QNOT\paren{x \in A} \QAND x\in C} \QOR \paren{x \in B
  \QAND x \in C}\\
 &\qquad\qquad \QOR \paren{\paren{\QNOT\paren{x \in A} \QOR x \in B}\QAND \QNOT\paren{x\in C}}\\
 & \qiff \paren{\bar{P}\QAND R} \QOR \paren{Q\QAND R} \QOR \paren{\paren{\bar{P} \QOR Q}\QAND \bar{R}}
\end{align*}
So the stated set equality holds if and only if the following two
propositional formulas are equivalent
\[
\QNOT\paren{P \QAND \bar{Q}}
\]
and
\[
\paren{\paren{\bar{P} \QAND R} \QOR \paren{Q \QAND R} \QOR
  \paren{\paren{\bar{P} \QOR Q}\QAND \bar{R}}}.
\]

Notice that you were \textbf{not} expected to write out a proof like
this.  We've written this out to remind you how the propositional
equivalence would be used in such a proof.

The point is that there is a clear correspondence between the set
equality and the needed propositional equivalence in such proofs, and
once you've recognized this, you can read off the propositional
equivalence from the set equality without having to go through any
long derivation.

\end{solution}

\end{problem}

%%%%%%%%%%%%%%%%%%%%%%%%%%%%%%%%%%%%%%%%%%%%%%%%%%%%%%%%%%%%%%%%%%%%%
% Problem ends here
%%%%%%%%%%%%%%%%%%%%%%%%%%%%%%%%%%%%%%%%%%%%%%%%%%%%%%%%%%%%%%%%%%%%%

\endinput

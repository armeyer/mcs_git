\documentclass[problem]{mcs}

\begin{pcomments}
 \pcomment{Question obtained from:}
  \pcomment{PS_counting_graphs}
  \pcomment{from: F08 PS9 -> S09 PS9; S09 CP7R}
  \pcomment{part(c) rewritten 4/8/11 by ARM}
  \pcomment{edited 11/3/11 by ARM }
\end{pcomments}

\pkeywords{
  counting
  graph
  asymmetric
  linear
  partial_order
  permutation
  simple_graph
  digraph
}

%%%%%%%%%%%%%%%%%%%%%%%%%%%%%%%%%%%%%%%%%%%%%%%%%%%%%%%%%%%%%%%%%%%%%
% Problem starts here
%%%%%%%%%%%%%%%%%%%%%%%%%%%%%%%%%%%%%%%%%%%%%%%%%%%%%%%%%%%%%%%%%%%%%

\begin{problem}

\bparts

\ppart How many simple graphs are there with vertex set $\set{1,2,\dots,n}$?\hfill \examrule[0.7in]

Briefly explain.

\examspace[2in]

\begin{solution}
There are $\binom{n}{2}$ potential edges, each of which may or
may not appear in a given graph.  Therefore, the number of graphs is:
\[
2^{\binom{n}{2}}
\]
\end{solution}


\ppart How many strict, linear partial orders with vertices
$\set{1,2,\dots,n}$ are there?\hfill \examrule[0.7in]

Briefly explain.


\begin
{solution}
\[
n!
\]

Since the partial order is linear, there is a unique listing of
the elements in decreasing partial order.  This listing defines a
bijection between the linear strict partial orders and the $n!$
permutations of $\Zintv{1}{n}$.
\end{solution}

\eparts
\end{problem}

%%%%%%%%%%%%%%%%%%%%%%%%%%%%%%%%%%%%%%%%%%%%%%%%%%%%%%%%%%%%%%%%%%%%%
% Problem ends here
%%%%%%%%%%%%%%%%%%%%%%%%%%%%%%%%%%%%%%%%%%%%%%%%%%%%%%%%%%%%%%%%%%%%%

\endinput

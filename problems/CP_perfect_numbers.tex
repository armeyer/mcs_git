\documentclass[problem]{mcs}

\begin{pcomments}
  \pcomment{from: S09.cp8m, S06.cp6w}
%  \pcomment{}
\end{pcomments}

\pkeywords{
  series
  divides
  number_theory
}

%%%%%%%%%%%%%%%%%%%%%%%%%%%%%%%%%%%%%%%%%%%%%%%%%%%%%%%%%%%%%%%%%%%%%
% Problem starts here
%%%%%%%%%%%%%%%%%%%%%%%%%%%%%%%%%%%%%%%%%%%%%%%%%%%%%%%%%%%%%%%%%%%%%

% S09, S06

\begin{problem}
A number is \term{perfect} if it is equal to the sum of its
positive divisors, other than itself.  For example, 6 is perfect,
because $6 = 1 + 2 + 3$.  Similarly, 28 is perfect, because $28 = 1 +
2 + 4 + 7 + 14$.  Explain why $2^{k-1} (2^k - 1)$ is perfect if $2^k -
1$ is prime.

\solution{If $2^k - 1$ is prime, then the only divisors of
$2^{k-1} (2^k - 1)$ are:
\[
1,\quad 2,\quad 4,\quad \ldots,\quad 2^{k-1}
\]
which sum to $2^k-1$ (using the formula for a geometric series; in this
case there's a direct ``Computer Science'' proof: think about the binary
representations of $2^j$ and $2^k - 1$), and also
\[
1 \cdot (2^k - 1),\quad 2 \cdot (2^k - 1),\quad 4 \cdot (2^k - 1),\quad
   \ldots,\quad 2^{k-2} \cdot (2^k - 1)
\]
which sum to $(2^{k-1} - 1) \cdot (2^k - 1)$.  Adding these two sums
gives $2^{k-1} (2^k - 1)$, so the number is perfect.

Euclid knew this, and also was able to show that all \emph{even} perfect numbers
are of exactly this form.  To this day, no one knows if there are any
odd perfect numbers!}

\end{problem}

%%%%%%%%%%%%%%%%%%%%%%%%%%%%%%%%%%%%%%%%%%%%%%%%%%%%%%%%%%%%%%%%%%%%%
% Problem ends here
%%%%%%%%%%%%%%%%%%%%%%%%%%%%%%%%%%%%%%%%%%%%%%%%%%%%%%%%%%%%%%%%%%%%%

\endinput

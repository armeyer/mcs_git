\documentclass[problem]{mcs}

\begin{pcomments}
  \pcomment{from: S09.cp4t, S08.cp4m}
\end{pcomments}

\pkeywords{
  well-ordering
  WOP
  postage_stamps
}

%%%%%%%%%%%%%%%%%%%%%%%%%%%%%%%%%%%%%%%%%%%%%%%%%%%%%%%%%%%%%%%%%%%%%
% Problems start here
%%%%%%%%%%%%%%%%%%%%%%%%%%%%%%%%%%%%%%%%%%%%%%%%%%%%%%%%%%%%%%%%%%%%%

\begin{problem}
  The proof below uses the Well Ordering Principle to prove that every
  amount of postage that can be paid exactly using only 6 cent and 15
  cent stamps, is divisible by 3.  That is, letting $S(n)$ mean that
  exactly $n$ cents postage can be paid using only 6 and 15 cent
  stamps, the proof shows that
%
\begin{equation}\tag{*}
S(n) \QIMPLIES [\text{n is divisable by 3, for all nonnegative integers $n$}].
\end{equation}
Fill in the missing portions (indicated by ``\dots'') of the following
proof of~(*).

\begin{quote}
Let $C$ be the set of \emph{counterexamples} to~(*), namely
\[
C \eqdef \set{n \suchthat \dots}
\]

\begin{solution}
 $n$ is a counterexample to~(*) if $n$ cents postage can be
  made and $n$ is not divisible by 3, so the predicate
\[
S(n)\text{ and } \QNOT(3 \divides n)
\]
defines the set, $C$, of counterexamples.
\end{solution}

Assume for the purpose of obtaining a contradiction that $C$ is
nonempty.  Then by the WOP, there is a smallest number, $m \in C$.
This $m$ must be positive because\dots.

\begin{solution}
\dots $3 \divides 0$, so 0 is not a counterexample.
\end{solution}

But if $S(m)$ holds and $m$ is positive, then $S(m-6)$ or $S(m-15)$
must hold, because\dots.

\begin{solution}
\dots if $m>0$ cents postage is made from 6 and 15 cent
  stamps, at least one stamp must have been used, so removing this
  stamp will leave another amount of postage that can be made.
\end{solution}

So suppose $S(m-6)$ holds.  Then $3 \divides (m-6)$, because\dots
\begin{solution}
\dots if $\QNOT(3 \divides (m-6))$, then $m-6$ would be
  a counterexample smaller than $m$, contradicting the minimality of
  $m$.
\end{solution}

But if $3 \divides (m-6)$, then obviously $3 \divides m$,
contradicting the fact that $m$ is a counterexample.

Next suppose $S(m-15)$ holds.  Then the proof for $m-6$ carries over
directly for $m-15$ to yield a contradiction in this case as well.
Since we get a contradiction in both cases, we conclude that\dots

\begin{solution}
\dots $C$ must be empty.  That is, there are no
  counterexamples to~(*),
\end{solution}

which proves that (*) holds.

\end{quote}
\end{problem}

\begin{problem}
Use the Well Ordering Principle to prove that
\begin{equation}\label{sum-of-sq}
\sum_{k=0}^n k^2 = \frac{n(n+1)(2n+1)}{6}.
\end{equation}
for all nonnegative integers, $n$.


\begin{solution}
The proof is by contradiction.

Suppose to the contrary that equation~\eqref{sum-of-sq} failed for some $n
\geq 0$.  Then by the WOP, there is a \emph{smallest} nonnegative integer,
$m$, such that~\eqref{sum-of-sq} does not hold when $n = m$.

But~\eqref{sum-of-sq} clearly holds when $n = 0$, which means that $m \geq
1$.  So $m-1$ is nonegative, and since it is smaller than $m$,
equation~\eqref{sum-of-sq} must be true for $n = m-1$.  That is,
\begin{equation}\label{sum-to-m-1}
\sum_{k=0}^{m-1} k^2 = \frac{(m-1)((m-1) + 1)(2(m-1)+1)}{6}.
\end{equation}
Now add $m^2$ to both sides of equation~\eqref{sum-to-m-1}.
Then the left hand side equals
\[
\sum_{k=0}^{m} k^2
\]
and the right hand side equals
\[
\frac{(m-1)((m-1) + 1)(2(m-1)+1)}{6} + m^2 
\]
Now a little algebra (given below) shows that the right hand side equals
\[
\frac{m(m+1)(2m+1)}{6}.
\]
That is,
\[
\sum_{k=0}^{m} k^2 = \frac{m(m+1)(2m+1)}{6},
\]
contradicting the fact that equation~\eqref{sum-of-sq} does not hold for
$m$.

It follows that there is no smallest nonnegative integer for which
equation~\eqref{sum-of-sq} fails.  Hence~\eqref{sum-of-sq} must hold for
all nonnegative integers.

Here's the algebra:
\textbox{
\begin{align*}
\frac{(m-1)((m-1) + 1)(2(m-1)+1)}{6} + m^2 
&= \frac{(m-1)m(2m-1)}{6} + m^2\\
 &  = \frac{(m^2-m)(2m-1)}{6} + m^2\\
 & = \frac{(2m^3-3m^2 +m)}{6} + \frac{6m^2}{6}\\
 &  = \frac{(2m^3 +3m^2 +m)}{6}\\
 &  = \frac{m(m+1)(2m+1)}{6}
\end{align*}}

\end{solution}

\end{problem}

%last used S09 ps1

\begin{problem}
  Class Notes Chapter~\ref{a4} described Euler's Conjecture:
  there are no positive integer solutions to the equation
\[
a^4 + b^4 + c^4 =  d^4.
\]
The Notes also gave integer values for $a,b,c,d$ that do satisfy this
equation, discovered more than two hundred years after the Conjecture.  So
Euler guessed wrong.

Now let's consider Lehman's\footnote{Suggested by Eric Lehman, a former
  6.042 Lecturer.} equation, similar to Euler's but with some
coefficients:
\begin{eqnarray}\label{wc}
8 a^4 + 4 b^4 + 2 c^4 & = & d^4
\end{eqnarray}

Prove that Lehman's equation~\eqref{wc} really does not have any positive
integer solutions.

\hint Consider the minimum value of $a$ among all possible solutions
to~\eqref{wc}.

\begin{solution}
Suppose that there exists a solution.  Then there must be a
  solution in which $a$ has the smallest possible value.  We will show
  that, in this solution, $a$, $b$, $c$, and $d$ must all be even.
  However, we can then obtain another solution over the positive integers
  with a smaller $a$ by dividing $a$, $b$, $c$, and $d$ in half.  This is
  a contradiction, and so no solution exists.

All that remains is to show that $a$, $b$, $c$, and $d$ must all be even.
The left side of Lehman's equation is even, so $d^4$ is even, so $d$ must
be even.  Substituting $d = 2 d'$ into Lehman's equation gives:

\begin{eqnarray}
8 a^4 + 4 b^4 + 2 c^4 & = & 16 d'^4
\end{eqnarray}

Now $2 c^4$ must be a multiple of 4, since every other term is a
multiple of 4.  This implies that $c^4$ is even and so $c$ is also
even.  Substituting $c = 2 c'$ into the previous equation gives:

\begin{eqnarray}
8 a^4 + 4 b^4 + 32 c'^4 & = & 16 d'^4
\end{eqnarray}

Arguing in the same way, $4 b^4$ must be a mutliple of 8, since every
other term is.  Therefore, $b^4$ is even and so $b$ is even.
Substituting $b = 2 b'$ gives:

\begin{eqnarray}
8 a^4 + 64 b'^4 + 32 c'^4 & = & 16 d'^4
\end{eqnarray}

Finally, $8 a^4$ must be a multiple of 16, $a^4$ must be even, and so
$a$ must also be even.  Therefore, $a$, $b$, $c$, and $d$ must all be
even, as claimed.
\end{solution}
\end{problem}

%%%%%%%%%%%%%%%%%%%%%%%%%%%%%%%%%%%%%%%%%%%%%%%%%%%%%%%%%%%%%%%%%%%%%
% Problems end here
%%%%%%%%%%%%%%%%%%%%%%%%%%%%%%%%%%%%%%%%%%%%%%%%%%%%%%%%%%%%%%%%%%%%%
\end{document}


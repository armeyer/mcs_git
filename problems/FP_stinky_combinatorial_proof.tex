\documentclass[problem]{mcs}

\begin{pcomments}
  \pcomment{FP_stinky_combinatorial_proof}
  \pcomment{third part of FP_more_counting}
  \pcomment{part of FP_counting_fall11}
  \pcomment{from: F03.final.prob5; F01.final}
\end{pcomments}

\pkeywords{
  counting
  combinatorial_proof
  binomial_coefficient
  factorial
  bijection
}

%%%%%%%%%%%%%%%%%%%%%%%%%%%%%%%%%%%%%%%%%%%%%%%%%%%%%%%%%%%%%%%%%%%%%
% Problem starts here
%%%%%%%%%%%%%%%%%%%%%%%%%%%%%%%%%%%%%%%%%%%%%%%%%%%%%%%%%%%%%%%%%%%%%

\begin{problem}
Below is a combinatorial proof of an equation.  Fill in the
empty boxes in the Theorem statement with the proper expressions.

\begin{theorem*}
\begin{eqnarray*}
\fbox{\begin{minipage}{2.5in}
\vspace{1in}
\hspace{1in}
\end{minipage}}
& = &
\fbox{\begin{minipage}{2.5in}
\vspace{1in}
\hspace{1in}
\end{minipage}}
\end{eqnarray*}
\end{theorem*}

\begin{proof}
Stinky Peterson owns $n$ newts, $t$ toads, and $s$ slugs.
Conveniently, he lives in a dorm with $n + t + s$ other students.
(The students are distinguishable, but creatures of the same variety
are not distinguishable.)  Stinky wants to put one creature in each
neighbor's bed.  Let $W$ be the set of all ways in which this can be
done.

On one hand, he could first determine who gets the slugs.  Then, he
could decide who among his remaining neighbors has earned a toad.
Therefore, $\card{W}$ is equal to the expression on the left.

On the other hand, Stinky could first decide which people deserve
newts and slugs and then, from among those, determine who truly merits
a newt.  This shows that $\card{W}$ is equal to the expression on the
right.

Since both expressions are equal to $\card{W}$, they must be equal to each
other.
\end{proof}

\begin{solution}
\[
\binom{n + t + s}{s} \cdot \binom{n + t}{t}
 = 
\binom{n + t + s}{n + s} \cdot \binom{n + s}{n}
\]
\end{solution}

\end{problem}

%%%%%%%%%%%%%%%%%%%%%%%%%%%%%%%%%%%%%%%%%%%%%%%%%%%%%%%%%%%%%%%%%%%%%
% Problem ends here
%%%%%%%%%%%%%%%%%%%%%%%%%%%%%%%%%%%%%%%%%%%%%%%%%%%%%%%%%%%%%%%%%%%%%

\endinput

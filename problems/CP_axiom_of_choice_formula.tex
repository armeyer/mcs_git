\documentclass[problem]{mcs}

\begin{pcomments}
  \pcomment{CP_axiom_of_choice_formula}
  \pcomment{9/23/09 by ARM, from logic notesproblem}
  \pcomment{edited ARM 2/11/11}
\end{pcomments}

\pkeywords{
 Choice
 Set Theory
 ZFC
 member
}

%%%%%%%%%%%%%%%%%%%%%%%%%%%%%%%%%%%%%%%%%%%%%%%%%%%%%%%%%%%%%%%%%%%%%
% Problem starts here
%%%%%%%%%%%%%%%%%%%%%%%%%%%%%%%%%%%%%%%%%%%%%%%%%%%%%%%%%%%%%%%%%%%%%

\def\goesto{\quad\text{translates into}\quad}

\begin{problem} 
  The axiom of choice%
\index{axiom!ZFC axioms!axiom of choice} 
says that if $s$ is a set whose members are
  nonempty sets that are \emph{pairwise disjoint}---that is, no two
  sets in $s$ have an element in common---then there is a set $c$
  consisting of exactly one element from each set in $s$.

In formal logic, we could describe $s$ with the formula,
\begin{align*}
\text{pairwise-disjoint}(s)
 \eqdef & \forall x \in s.\, x \neq \emptyset\  \QAND\\
        & \forall x,y \in s.\, x \neq y \QIMPLIES x \intersect y = \emptyset.
\end{align*}
 Similarly we could describe $c$ with the formula
 \[
 \text{choice-set}(c,s) \eqdef \quad \forall x \in s.\, \exists! z.\ z \in c
 \intersect x.
 \]
 Here ``$\exists!\, z.$'' is fairly standard notation for ``there exists a
 \emph{unique} $z$.''

Now we can give the formal definition:
\begin{definition*}[Axiom of Choice]
\[
\forall s.\, \text{pairwise-disjoint}(s) \QIMPLIES \exists c.\,
\text{choice-set}(c,s).
\]
\end{definition*}

The only issue here is that set theory is technically supposed to be
expressed in terms of \emph{pure} formulas in the language of sets,
which means formula that uses only the membership relation $\in$
propositional connectives, the two quantifies $\forall$ and $\exists$,
and variables ranging over all sets.  Verify that the axiom of choice
can be expressed as a pure formula, by explaining how to replace all
impure subformulas above with equivalent pure formulas.

For example, the formula $x = y$ could be replaced with the pure formula
$\forall z.\, z \in x \QIFF z \in y$.

\begin{solution}
Here is how the impure subformulas used in the above definition of the
axiom of choice can be translated into pure formulas:

\[
x \neq \emptyset \goesto \exists y/\, y \in x.
\]

\[
[x \intersect y = \emptyset] \goesto \QNOT(\exists z.\, z \in x \QAND z \in y).
\]

\[
[z \in x \intersect y] \goesto z \in x \QAND z \in y.
\]

\[
\exists! z.\, P(z) \goesto  \exists z.\, P(z) \QAND \forall w.\, P(w)
\QIMPLIES w = z.
\]
This last formula is not pure because it uses $=$, but this is ok since we
know it can be replaced by a pure formula.

\end{solution}
\end{problem}

%%%%%%%%%%%%%%%%%%%%%%%%%%%%%%%%%%%%%%%%%%%%%%%%%%%%%%%%%%%%%%%%%%%%%
% Problem ends here
%%%%%%%%%%%%%%%%%%%%%%%%%%%%%%%%%%%%%%%%%%%%%%%%%%%%%%%%%%%%%%%%%%%%%

\endinput

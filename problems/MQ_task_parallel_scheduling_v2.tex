\documentclass[problem]{mcs}

\begin{pcomments}
  \pcomment{MQ_task_parallel_scheduling_v2}
  \pcomment{written for Spring 2013 Miniquiz 5 (afternoon)}
\end{pcomments}

\pkeywords{
  DAG
  scheduling
  chains_and_antichains
}

%%%%%%%%%%%%%%%%%%%%%%%%%%%%%%%%%%%%%%%%%%%%%%%%%%%%%%%%%%%%%%%%%%%%%
% Problem starts here
%%%%%%%%%%%%%%%%%%%%%%%%%%%%%%%%%%%%%%%%%%%%%%%%%%%%%%%%%%%%%%%%%%%%%

\begin{problem}
Let $\set{A, ..., H}$ be a set of tasks that we must complete.  The
following DAG describes which tasks must be done before others, where
there is an arrow from $a$ to $b$ iff $a$ must be done before $b$.

\begin{figure}[h]
\graphic[height=2.5in]{miniquiz5-p2-afternoon}
\end{figure}

\bparts
\ppart
Write the longest chain.
\begin{solution}
A, D, E, G, H
\end{solution}
\examspace[3cm]

\ppart
Write the longest antichain.
\begin{solution}
A, B, C
\end{solution}
\examspace[3cm]

\ppart If we allow parallel scheduling, and each task takes 1 minute
to complete, what is the minimum amount of time needed to complete all
tasks?

\begin{solution}
This should be the length of the longest chain, 5 minutes.
\end{solution}

\eparts

\end{problem}


%%%%%%%%%%%%%%%%%%%%%%%%%%%%%%%%%%%%%%%%%%%%%%%%%%%%%%%%%%%%%%%%%%%%%
% Problem ends here
%%%%%%%%%%%%%%%%%%%%%%%%%%%%%%%%%%%%%%%%%%%%%%%%%%%%%%%%%%%%%%%%%%%%%

\endinput

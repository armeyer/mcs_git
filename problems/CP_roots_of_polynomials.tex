\documentclass[problem]{mcs}

\begin{pcomments}
  \pcomment{from: S09.cp1r, S08.cp2m(?), S99 Tutorial 1 Notes}
  \pcomment{Has a reference to another problem that should be resolved.}
%  \pcomment{}
\end{pcomments}

\pkeywords{
  irrational
  primes
  polynomials
}

%%%%%%%%%%%%%%%%%%%%%%%%%%%%%%%%%%%%%%%%%%%%%%%%%%%%%%%%%%%%%%%%%%%%%
% Problem starts here
%%%%%%%%%%%%%%%%%%%%%%%%%%%%%%%%%%%%%%%%%%%%%%%%%%%%%%%%%%%%%%%%%%%%%

\begin{problem}
  Here is a generalization of Problem~\ref{generprob} that you may not have
  thought of:

\begin{lemma}\label{poly}
Let the coefficients of the polynomial
$a_0+a_1x+a_2x^2+\cdots + a_{n-1}x^{n-1} + x^n$ be
integers.  Then any real root of the polynomial is either integral or
irrational.
\end{lemma}

\bparts

\ppart Explain why Lemma~\eqref{poly} immediately implies that
$\sqrt[n]{k}$ is irrational whenever $k$ is not an $n$th power of some integer.

\begin{solution}
Saying that an integer, $k$, is not the $n$th power of an
  integer, is equivalent to saying that the equation $x^n = k$ has no
  integer solutions.  Another way to say this is that the polynomial $x^n
  - k$ has no integer root.  Lemma~\eqref{poly} therefore implies that any
  root of $x^n-k$ is irrational.  But $\sqrt[n]{k}$ is, by definition, a
  root of this polynomial, so it is irrational.
\end{solution}

\ppart Collaborate with your tablemates to write a clear, textbook quality
proof of Lemma~\ref{poly} on your whiteboard.  (Besides clarity and
correctness, textbook quality requires good English with proper
punctuation.  When a real textbook writer does this, it usually takes
multiple revisions; if you're satisfied with your first draft, you're
probably misjudging.)  You may find it helpful to appeal to the following:
\begin{lemma}\label{ppow}
  If a prime, $p$, is a factor some power of an integer, then it is a
  factor of that integer.
\end{lemma}
You may assume Lemma~\ref{ppow} without writing down its proof, but see if
you can explain why it is true.

\begin{solution}
%From S08 class problem 2M(?) and S99 Tutorial 1 Notes:

\begin{proof}

Let $r$ be a real root of the polynomial, so that
\[
a_0+a_1r+a_2r^2+\cdots+  a_{n-1}r^{n-1}+ r^n = 0.
\]
There are three cases: either $r$ is an integer, or $r$ is irrational, or
$r = s/t$ for integers $s$ and $t$ which have no common factors and such
that $t >1$.  We want to eliminate the last case, so assume for the sake
of contradiction that it held for some $r$.

Substituting $s/t$ for $r$ and multiplying both sides of the above
equation by $t^n$ yields:
\begin{eqnarray}
a_0t^n+a_1st^{n-1}+a_2s^2t^{n-2}+ \cdots + a_{n-1}s^{n-1}t+s^n & = & 0,\\
a_0t^n+a_1st^{n-1}+a_2s^2t^{n-2}+ \cdots + s^{n-1}t & = & -s^n.\label{e2}
\end{eqnarray}

Now since $t>1$, it must have a prime factor, $p$.  The prime, $p$,
therefore divides each term of the lefthand side of equation~\eqref{e2},
so $p$ also divides the righthand side, $-s^n$.  This means that $p$
divides $s^n$, so by Lemma~\ref{ppow}, $p$ is also a factor of
$s$.  So $p$ is a common factor of $s$ and $t$, contradicting the
fact that $s$ and $t$ have no common factors.
\end{proof}

Lemma~\ref{ppow} is a simple consequence of the \emph{Fundamental Theorem
  of Arithmetic} which says that every integer $> 1$ factors into a
product of primes that is \emph{unique} except for the order in which the
primes are multiplied.

For example, here are some ways to express 140 as a product of primes:
\[
140 = 2\cdot 2 \cdot 5 \cdot 7 = 2\cdot 5 \cdot 7 \cdot 2 = 7 \cdot 5 \cdot
2\cdot 2 = \cdots.
\]
By the Fundamental Theorem, every such product will have exactly two
occurrences of 2 and one each of 5 and 7.  Next, we can obviously get a
product of primes equal to, say, the third power of 140 by taking a
product that equals $140$ and repeating it three times.  For example,
\[
(140)^3 = 2\cdot 2 \cdot 5 \cdot 7 \ \ \cdot \ \  2\cdot 2 \cdot 7 \cdot 5
\ \ \cdot \ \ 2\cdot 2 \cdot 7 \cdot 5.
\]
The Fundamental Theorem now says that \emph{every} prime product equal to
the third power of 60 must have the same primes as this repeated product,
namely, six occurrences of 2 and three occurrences each of 5 and 7.  In
particular, the \emph{only} primes that are factors of $(140)^3$ are the
primes 2, 5 and 7 that are factors of 140.  This reasoning applies equally
well with any other integer greater than 1 in place of 140 and any power
greater than 0 in place of 3, proving that if $p$ is a prime factor of
$s^n$, then $p$ must have been a factor of $s$.

The Fundamental Theorem of Arithmetic is also know as the \emph{Unique
  Prime Factorization Theorem}.  It one of those familiar Mathematical
facts that is not exactly obvious.  We'll work out a proof of the
Fundamental Theorem later in the term.
\end{solution}

\eparts

\end{problem}

%%%%%%%%%%%%%%%%%%%%%%%%%%%%%%%%%%%%%%%%%%%%%%%%%%%%%%%%%%%%%%%%%%%%%
% Problem ends here
%%%%%%%%%%%%%%%%%%%%%%%%%%%%%%%%%%%%%%%%%%%%%%%%%%%%%%%%%%%%%%%%%%%%%

\endinput

\documentclass[handout]{mcs}

\begin{document}

\inclassproblems{5, Mon.}

%%%%%%%%%%%%%%%%%%%%%%%%%%%%%%%%%%%%%%%%%%%%%%%%%%%%%%%%%%%%%%%%%%%%%
% Problems start here
%%%%%%%%%%%%%%%%%%%%%%%%%%%%%%%%%%%%%%%%%%%%%%%%%%%%%%%%%%%%%%%%%%%%%
\begin{staffnotes}
Infinite Cardinality; The Halting Problem, Ch. 7
\end{staffnotes}

\pinput{CP_smallest_infinite_set}
%\pinput{CP_set_product_bijection}
\pinput{CP_countable_surjection}
\pinput{CP_rationals_are_countable}
\hint Use Problem 2.

%\pinput{CP_Schroeder_Bernstein_theorem}

\textbf{Supplemental problem:}

\pinput{CP_recognizable_sets}

%\inhandout{\instatements{\newpage}}
%\begin{center}
%\textbf{
%If you have time, you might enjoy tackling the following
%problem.
%}
%\end{center}
%\pinput{FP_infinite_binary_sequences}

%%I would also suggest the classic Hilbert's Infinite Hotel
%%problems. These basically call for bijection between the natural
%%numbers and the whole numbers, the natural numbers and the even
%%natural numbers, the natural numbers and the rationals, but they are
%%all pharased as lots of mathmaticions showing up to a hotel for a
%%confrence.

%%%%%%%%%%%%%%%%%%%%%%%%%%%%%%%%%%%%%%%%%%%%%%%%%%%%%%%%%%%%%%%%%%%%%
% Problems end here
%%%%%%%%%%%%%%%%%%%%%%%%%%%%%%%%%%%%%%%%%%%%%%%%%%%%%%%%%%%%%%%%%%%%%

\end{document}



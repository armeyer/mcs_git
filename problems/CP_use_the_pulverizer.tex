\documentclass[problem]{mcs}

\begin{pcomments}
  \pcomment{from: S09.cp8m, S06.cp6w}
  \pcomment{references pulverizer in appendix which should be in notes instead}
%  \pcomment{}
\end{pcomments}

\pkeywords{
  gcd
  linear_combinations
  Pulverizer
  number_theory
}

%%%%%%%%%%%%%%%%%%%%%%%%%%%%%%%%%%%%%%%%%%%%%%%%%%%%%%%%%%%%%%%%%%%%%
% Problem starts here
%%%%%%%%%%%%%%%%%%%%%%%%%%%%%%%%%%%%%%%%%%%%%%%%%%%%%%%%%%%%%%%%%%%%%

\begin{problem}
\bparts

\ppart Use the Pulverizer (see the Appendix) to find integers $x,y$ such
that
\[
x50 + y21 = \gcd(50,21).
\]

\solution{Here is the table produced by the Pulverizer:
\[
\begin{array}{ccccrcl}
x & \quad & y & \quad & \rem{x}{y} & = & x - q \cdot y \\ \hline
 50 &&  21 &&  8 & = &    50 - 2 \cdot  21 \\
 21 &&   8 &&  5 & = &    21 - 2 \cdot  8 \\
&&&&             & = &    21 - 2 \cdot (50 - 2 \cdot  21) \\
&&&&             & = &   -2 \cdot 50 + 5 \cdot 21 \\
  8 &&   5 &&  3 & = &    8 - 1 \cdot 5  \\
&&&&             & = &   (50 - 2 \cdot  21) 
                         -1 \cdot (-2 \cdot 50 + 5 \cdot 21) \\
&&&&             & = &   3\cdot 50 -7 \cdot 21 \\
  5 &&   3 &&  2 & = &    5 - 1\cdot 3 \\
&&&&             & = &   (-2 \cdot 50 + 5 \cdot 21) 
                         -1 \cdot (3\cdot 50 -7 \cdot 21) \\
&&&&             & = &   -5\cdot 50 + 12 \cdot 21 \\
  3 &&   2 &&  1 & = &    3 - 1\cdot 2 \\
&&&&             & = &   (3\cdot 50 -7 \cdot 21) 
                         -1\cdot (-5\cdot 50 + 12 \cdot 21) \\
&&&&             & = &   \fbox{$8\cdot 50 - 19 \cdot 21$} \\
  2 &&   1 &&  0 &   & 
\end{array}
\]
}

\ppart Now find integer $x',y'$ with $y'>0$ such that
\[
x'50 + y'21 = \gcd(50,21)
\]
\solution{
since $(x,y) = (8,-19)$ works, so does $(8 - 21n,-19+50n)$ for any $n \in
\integers$, so letting $n=1$, we have
\[
-13 \cdot 50 + 31 \cdot 21 = 1
\]
}

\eparts
\end{problem}


%%%%%%%%%%%%%%%%%%%%%%%%%%%%%%%%%%%%%%%%%%%%%%%%%%%%%%%%%%%%%%%%%%%%%
% Problem ends here
%%%%%%%%%%%%%%%%%%%%%%%%%%%%%%%%%%%%%%%%%%%%%%%%%%%%%%%%%%%%%%%%%%%%%

\endinput

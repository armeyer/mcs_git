\documentclass[problem]{mcs}

\begin{pcomments}
  \pcomment{FP_mersenne_primes}
  \pcomment{zabel s18}
\end{pcomments}

\pkeywords{
  number_theory
  modular_arithmetic
  primes
}

%%%%%%%%%%%%%%%%%%%%%%%%%%%%%%%%%%%%%%%%%%%%%%%%%%%%%%%%%%%%%%%%%%%%%
% Problem starts here
%%%%%%%%%%%%%%%%%%%%%%%%%%%%%%%%%%%%%%%%%%%%%%%%%%%%%%%%%%%%%%%%%%%%%

\begin{problem}
  
  \bparts
  
  \ppart\label{mersenne-number-div} If $a,b$ are positive integers and
  $a \divides b$, prove that $2^a-1 \mid 2^b-1$. \hint $2^a\equiv 1
  \pmod {2^a-1$}

  \begin{solution}
    TBD
  \end{solution}
  
  \ppart Use part~\eqref{mersenne-number-div} to prove that if $n$ is
  a positive integer and $2^n-1$ is prime, then $n$ must itself be
  prime.
  
  \begin{solution}
    TBD

    Note that the converse is not true: even if $p$ is prime, $2^p-1$
    might not be prime. For example, $2^{11}-1 = 2047 = 23\cdot
    89$.  Primes of the form $2^n-1$ are called \emph{Mersenne primes},
    and currently only 50 examples are known.  The largest example,
    $2^{77,232,917}-1$, was found in December 2017 and has more than
    23 million digits.
  \end{solution}

  \eparts
\end{problem}

%%%%%%%%%%%%%%%%%%%%%%%%%%%%%%%%%%%%%%%%%%%%%%%%%%%%%%%%%%%%%%%%%%%%%
% Problem ends here
%%%%%%%%%%%%%%%%%%%%%%%%%%%%%%%%%%%%%%%%%%%%%%%%%%%%%%%%%%%%%%%%%%%%%

\endinput

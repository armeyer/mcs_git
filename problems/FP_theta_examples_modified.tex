\documentclass[problem]{mcs}

\begin{pcomments}
  \pcomment{FP_theta_examples_modified}
 \pcomment{variant of CP_theta_examples}
 \pcomment{f17 midterm4}
\end{pcomments}

\pkeywords{
 Theta
 asymptotic
 Stirling
}

%%%%%%%%%%%%%%%%%%%%%%%%%%%%%%%%%%%%%%%%%%%%%%%%%%%%%%%%%%%%%%%%%%%%%
% Problem starts here
%%%%%%%%%%%%%%%%%%%%%%%%%%%%%%%%%%%%%%%%%%%%%%%%%%%%%%%%%%%%%%%%%%%%%

\begin{problem}
Determine which of these choices 
\[
\Theta(n), \quad
\Theta(n^2 \log n), \quad
\Theta(n^2), \quad
\Theta(1), \quad
\Theta(2^n), \quad
\Theta(2^{n \ln n}), \quad
\textbf{None}\ \text{of these}
\]
describes each function's asymptotic behavior. Full proofs
  are not required, but briefly explain your answers.


\bparts
\ppart  
\[
n + \ln n + (\ln n)^2
\]

\begin{solution}
$\Theta(n)$

Both $n > \ln n$ and $n > (\ln n)^2$ hold for all
sufficiently large $n$.  Thus, for all sufficiently large $n$:
\[
n < n + \ln n + (\ln n)^2 < n + n + n
\]
So $n + \ln n + (\ln n)^2 = \Theta(n)$.

\end{solution}

\examspace[.85in]
\ppart 
\[
\frac{n^2 + 2n - 3}{n^2 - 7}
\]

\begin{solution}
$\Theta(1)$

Observe that:
\[
\lim_{n \to \infty} \frac{n^2 + 2n - 3}{n^2 - 7} = 1
\]
This means that, for all sufficiently large $n$, the fraction lies, for
example, between $0.99$ and $1.01$ and is therefore $\Theta(1)$.
\end{solution}

\examspace[.85in]
\ppart 
\[
\sum_{i = 0}^n 2^{i+1}
\]

\begin{solution}
$\Theta(2^n)$

From the geometric series formula, this sum simplifies to $2(2^{n+1}-1) = 4\cdot 2^n - 2$, which is $\Theta(2^n)$.
\end{solution}


\examspace[.85in]
\ppart
\[
2^{n\log n} - 2^n
\]

\begin{solution}
  $\Theta(2^{n\log n})$

  We have $2^n = o(2^{n\log n})$ because $\lim_{n\to\infty} 2^{n\log n}/2^n = \lim_{n\to\infty}2^{n(\log(n)-1)} = \infty$, so the $2^{n\log n}$ term dominates.
\end{solution}


\examspace[.85in]
\ppart
\[
(n^2+3)(n^2+4) - (n^2+1)(n^2+2)
\]

\begin{solution}
  $\Theta(n^2)$

  Though the expression looks like a $4$th degree polynomial, expanding reveals that it equals $(n^4 + 7n^2 + 12) - (n^4 + 3n^2 + 2) = 4n^2 + 10 = \Theta(n^2)$.
\end{solution}

\eparts

\end{problem}

%%%%%%%%%%%%%%%%%%%%%%%%%%%%%%%%%%%%%%%%%%%%%%%%%%%%%%%%%%%%%%%%%%%%%
% Problem ends here
%%%%%%%%%%%%%%%%%%%%%%%%%%%%%%%%%%%%%%%%%%%%%%%%%%%%%%%%%%%%%%%%%%%%%

\endinput

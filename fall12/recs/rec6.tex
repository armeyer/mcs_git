\documentclass[12pt]{article}
\usepackage{light}
\usepackage{enumerate}

\setlength{\parindent}{0pt}

%\hidesolutions
\showsolutions

\newcommand{\edge}[2]{#1\text{---}#2}
\newcommand{\mfigure}[3]{\bigskip\centerline{\resizebox{#1}{#2}{\includegraphics{#3}}}\bigskip}
\newcommand{\eqdef}{\mathbin{::=}}

\begin{document}

\recitation{6}{September 28, 2012}

%%%%%%%%%%%%%%%%%%%%%%%%%%%%%%%%%%%%%%%%%%%%%%%%%%%%%%%%%%%%%%%%%%%%%%%%%%%%%%%
%\insolutions{

%\stamp

\section*{Hall's theorem}

Let $G=(V,E)$ be a bipartite graph, with left vertex set $L$ and right vertex set $R$.
Recall that for a subset $S$ of the vertices, $N(S)$ is the set of vertices which are adjacent to some vertex in $S$:
\[ N(S) = \{ r \in V \mid \{r, s\} \in E \text{ for some } s \in S \}. \]

\textbf{Halls' theorem} says that if for every subset $S$ of $L$ we have $|N(S)| \geq |S|$, then there is a matching in $G$ that covers $L$.

\section*{Problem 1}

Recall that a graph is called \emph{$d$-regular} if every vertex in the graph has degree exactly $d$.
Let $G=(V,E)$ be a $d$-regular bipartite graph, with the same number of vertices in the left part $L$ as in the right part $R$.

Prove, using Hall's theorem and induction, that $G$ can be partitioned into $d$ perfect matchings. In other words, we can find $E_1, E_2, \ldots, E_d \subseteq E$, all disjoint ($E_i \cap E_j = \emptyset$) and which together form $E$, so that $E_i$ is a perfect matching of $G$ for each $1 \leq i \leq d$.

\solution{
The proof is by induction on $d$. So let
\begin{multline*} P(d) = \text{``Any $d$-regular bipartite graph with $n/2$ left nodes and $n/2$ right nodes}\\
    \text{can be partitioned into $d$ perfect matchings''.} \end{multline*}


Let us take $d=0$ for the base case (the statement still makes sense). In a $0$-regular graph, there are no edges. We can indeed partition the empty set into $0$ perfect matchings! So $P(0)$ is true.

So assume $P(d)$ holds; we want to prove $P(d+1)$, so let $G$ be a $(d+1)$-regular bipartite graph, with left subset $L$ and right subset $R$, where $|L|=|R|=n/2$.

Let $S$ be any subset of $L$; we want to show that $|N(S)| \geq |S|$. Consider the subgraph $G'=(S \cup N(S),E')$ induced by $S \cup N(S)$. Every node of $S$ has degree $d$ in $G'$ (since we kept all the neighbours of $S$). Every node in $N(S)$ has degree at most $d+1$ in $G'$ (since it had degree $d+1$ in $G$, and $G'$ is a subgraph). Now count the number of edges of $G'$ in two ways, just like in class: we get
\[ (d+1)|S| = |E'| = \sum_{v \in N(S)} \deg_{G'}(v) \leq (d+1)|N(S)|. \]
Simplifying, $|N(S)| \geq |S|$. %(The above is the key part of the argument, and needs the most emphasis.)

So Hall's condition is satisfied, and there exists some perfect matching $M$ on $G$. Now look at the subgraph $H=(V, E_H)$ obtained by removing $M$ from $G$. This is $d$-regular, since for each vertex we remove a single edge adjacent to it. So by induction, $H$ can be partitioned into $d$ perfect matchings $E_1, E_2, \ldots, E_d$. Writing $E_{d+1} = M$, we obtain that $E = E_H \cup M = E_1 \cup \cdots \cup E_{d+1}$ (and all the $E_i$'s are disjoint).

This proves $P(d+1)$, and so by induction, $P(d)$ holds for all $d \geq 0$.
}

\section*{Problem 2}

Given the preference lists of each boy and girl, there can be in general many different stable matchings.

Consider a particular boy $i$, and let $P_i$ be the set of girls for which there is \emph{some} stable matching where this girl is matched to $i$.
We say that boy $i$'s favorite girl in $P_i$ is his \emph{optimal mate}; this represents the best outcome for boy $i$, given that only stable matchings are allowed.

Prove that The Mating Algorithm returns a matching where every boy is matched with his optimal mate.

\solution{See the book, pp~140--142.}

\section*{Problem 3}
Similarly to the previous problem, we say that the \emph{pessimal mate} of girl $j$ is her least favorite boy from the set $P_j$ of boys she can be matched to in some stable matching. 

Prove that The Mating Algorithm returns a matching where every girl is matched with her pessimal mate.

\solution{See the book, pp~140--142.}


\end{document}

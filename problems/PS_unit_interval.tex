\documentclass[problem]{mcs}

\begin{pcomments}
  \pcomment{from: S09.ps2}
\end{pcomments}

\pkeywords{
  mapping_lemma
  functions
  surjections
}

%%%%%%%%%%%%%%%%%%%%%%%%%%%%%%%%%%%%%%%%%%%%%%%%%%%%%%%%%%%%%%%%%%%%%
% Problem starts here
%%%%%%%%%%%%%%%%%%%%%%%%%%%%%%%%%%%%%%%%%%%%%%%%%%%%%%%%%%%%%%%%%%%%%

\begin{problem} In this problem you will prove a fact that may surprise
  you ---or make you even more convinced that set theory is nonsense: the
  half-open unit interval,
\[
(0,1] \eqdef \set{r \in \reals \suchthat 0 < r \leq 1}
\]
is actually the \emph{same size} as the half-open solid unit square, 
$(0,1]^2$.

\bparts

\ppart\label{surjection1} Describe a surjection from $(0,1]^2$ onto $(0,1]$.

\solution{A surjection, $f$, can be defined by:
\[
f(\ang{x,y}) \eqdef x
\]}

\ppart\label{surjection2a} An infinite sequence of the decimal digits
$\set{\texttt{0},\texttt{1},\dots,\texttt{9}}$ will be called \emph{long}
if it has infinitely many occurrences of some digit other than 0.  Let $L$
be the set of all such long sequences.  Describe a bijection from $L$ to
the half-open real interval $(0,1]$.

\hint Consider using a decimal expansion.

\solution{Putting a decimal point in front of a long sequence defines a
  bijection from $L$ to $(0,1]$.  This follows because every real number
in $(0,1]$ has a unique long decimal expansion.  Note that if we didn't
  exclude the non-long sequences, namely, those sequences ending with all
  zeroes, this wouldn't be a bijection.  For example, the sequences
  \texttt{1000}\dots and \texttt{099999}\dots would both map to the same
  real number, namely, $1/10$.}

\ppart\label{surjection2b} Describe a surjection from $L$ onto $L^2$ that
involves alternating digits from two long sequences. 

\hint The surjection need not be total.

\solution{Given any long sequence $s = x_1, x_2, x_3, x_4, \ldots$, let
$h_1(s) = x_1, x_3, x_5, \ldots$ be the sequence of digits in odd 
positions and $h_2(s ) = x_2, x_4, x_6, \ldots$ the sequence of digits
in even positions.  The following mapping is then surjective:
\begin{eqnarray}
h(s) \eqdef \begin{cases}
  (h_1(s),h_2(s)), &  \text{if $h_1(s) \in L$ and $h_2(s) \in L$,}\\
  \text{undefined}, & \text{otherwise.}
 \end{cases}
\end{eqnarray}
}

\ppart\label{surjection2c} Prove the following lemma and use it to
conclude that $L^2$ is the same size as $(0,1]^2$.

\begin{lemma}\label{product-map}
Let $A$ and $B$ be nonempty sets.  If $A$ and $B$ are the same size then
so are $A \times A$ and $B \times B$.
\end{lemma}

\hint Show how a bijection $f : A \to B$ can be used to construct a 
bijection $g : A^2 \to B^2$.

\solution{\mbox{}

\begin{proof}
By the definition of \emph{same size} there must exist a
bijection $f : A \to B$.  Let $g : A^2 \to B^2$ be defined by the rule
$g(x,y) = (f(x),f(y))$.  In order to prove $g$ is a bijection we must
show that it is (1) total, (2) surjective and (3) injective.  

\begin{enumerate}
\item Since $f$ is total, $f(a_1)$ and $f(a_2)$ are defined 
$\forall a_1, a_2 \in A$ and $g$ is therefore a total function. 

\item Since $f$ is surjective, for any $(b_1,b_2) \in B^2$ there 
exists an $a_1 \in A$ such that $b_1 = f(a_1)$ and an $a_2 \in A$ 
such that $b_2 = f(a_2)$. Thus $(b_1,b_2) = g(a_1,a_2)$ and $g$ is a 
surjection as well.

\item Since $f$ is injective $f(a_1)=f(a_1^{\prime})$ iff 
$a_1 = a_1^{\prime}$ and $f(a_2)=f(a_2^{\prime})$ iff 
$a_2 = a_2^{\prime}$.  Therefore, 
$g(a_1,a_2) = g(a_1^{\prime},a_2^{\prime})$ iff 
$(a_1,a_2) = (a_1^{\prime},a_2^{\prime})$ and so $g$ is injective.
\end{enumerate}
\end{proof}

Since it was shown in part \eqref{surjection2a} that $(0,1]$ and $L$ are
the same size, an immediate corollary of lemma \ref{product-map} is that
$L^2$ is the same size as $(0,1]^2$.} 

\ppart Complete the proof that $(0,1]$ and $(0,1]^2$ are the same 
size by appealing to the Schr\"oder-Bernstein Theorem.

\solution{

Part \eqref{surjection1} showed the existence of a surjection from
$(0,1]^2$ onto $(0,1]$ and so $(0,1]^2$ is as big as $(0,1]$.  Because
      ``as big as'' is a transitive relation, parts \eqref{surjection2a},
      \eqref{surjection2b} and \eqref{surjection2c} imply conversely that
      $(0,1]$ is as big as $(0,1]^2$.  The Schr\"oder-Bernstein Theorem
          then implies that $(0,1]$ and $(0,1]^2$ are the same size.

We leave it as an easy optional exercise to show that the unit interval is
as big as the whole real line, $\reals$.  Combined with the previous
parts, it follows that the unit interval is the same size as the whole
plane, $\reals^2$!}

\eparts
\end{problem}

%%%%%%%%%%%%%%%%%%%%%%%%%%%%%%%%%%%%%%%%%%%%%%%%%%%%%%%%%%%%%%%%%%%%%
% Problem ends here
%%%%%%%%%%%%%%%%%%%%%%%%%%%%%%%%%%%%%%%%%%%%%%%%%%%%%%%%%%%%%%%%%%%%%

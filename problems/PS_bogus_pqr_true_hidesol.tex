\documentclass[problem]{mcs}

\begin{pcomments}
  \pcomment{PS_bogus_pqr_true_hidesol}
  \pcomment{S95/S98/F14/F16/S17.ps1}
  \pcomment{revised ARM 2/3/17 for S17.ps1}
\end{pcomments}

\pkeywords{
  logic
  formula
  proposition
  implies
}

%%%%%%%%%%%%%%%%%%%%%%%%%%%%%%%%%%%%%%%%%%%%%%%%%%%%%%%%%%%%%%%%%%%%%
% Problems start here
%%%%%%%%%%%%%%%%%%%%%%%%%%%%%%%%%%%%%%%%%%%%%%%%%%%%%%%%%%%%%%%%%%%%%

\begin{problem}
Sloppy Sam is trying to prove a certain proposition $P$.  He defines
two related propositions $Q$ and $R$, and then proceeds to prove three
implications:
\[
P\ \QIMPLIES\ Q, \qquad Q\ \QIMPLIES\ R, \qquad R\ \QIMPLIES\ P.
\]
He then reasons as follows:
\begin{quote}
If $Q$ is true, then since I proved $(Q \QIMPLIES R)$, I can conclude
that $R$ is true.  Now, since I proved $(R \QIMPLIES P)$, I can
conclude that $P$ is true.  Similarly, if $R$ is true, then $P$ is
true and so $Q$ is true.  Likewise, if $P$ is true, then so are $Q$
and $R$.  So any way you look at it, all three of $P,Q$ and $R$ are
true.
\end{quote}

\bparts
\ppart Exhibit truth tables for
\begin{equation}\tag{*}
(P\ \QIMPLIES\ Q) \QAND (Q\ \QIMPLIES\ R) \QAND (R\ \QIMPLIES\ P)
\end{equation}
and for 
\begin{equation}\tag{**}
P \QAND Q \QAND R.
\end{equation}
Use these tables to find a truth assignment for $P,Q,R$ so that~(*)
is~\true\ and~(**) is~\false.

\begin{solution}
\TBA{Tues, Feb 21, 1PM}
\end{solution}

\ppart You show these truth tables to Sloppy Sam and he says ``OK, I'm
wrong that $P,Q$ and $R$ all have to be true, but I still don't see
the mistake in my reasoning.  Can you help me understand my mistake?''
How would you explain to Sammy where the flaw lies in his reasoning?

\begin{solution}
\TBA{Tues, Feb 21, 1PM}
\end{solution}
\eparts
\end{problem}

%%%%%%%%%%%%%%%%%%%%%%%%%%%%%%%%%%%%%%%%%%%%%%%%%%%%%%%%%%%%%%%%%%%%%
% Problems end here
%%%%%%%%%%%%%%%%%%%%%%%%%%%%%%%%%%%%%%%%%%%%%%%%%%%%%%%%%%%%%%%%%%%%%

\endinput


\documentclass[problem]{mcs}

\begin{pcomments}
  \pcomment{TP_satisfiable_vs_valid}
  \pcomment{trivial variant of one part of CP_valid_vs_satisfiable}
  \pcomment{ARM 9/20/15}
\end{pcomments}

\pkeywords{
  valid
  satisfiable
  propositional
}

%%%%%%%%%%%%%%%%%%%%%%%%%%%%%%%%%%%%%%%%%%%%%%%%%%%%%%%%%%%%%%%%%%%%%
% Problem starts here
%%%%%%%%%%%%%%%%%%%%%%%%%%%%%%%%%%%%%%%%%%%%%%%%%%%%%%%%%%%%%%%%%%%%%

\begin{problem}
Explain why a logical formula $P$ is satisfiable iff its
negation $\QNOT(P)$ is \emph{not} valid.

\begin{solution}
Essentially the same as the solution to~\bref{CP_valid_vs_satisfiable}.

To prove the iff, we prove that the left hand statement implies the
right hand one and vice versa.

\textbf{(left-to-right case)}: If $P$ is satisfiable, then $\QNOT(P)$ is \emph{not}
valid.

\begin{proof}
$P$ is true in an environment iff $\QNOT(P)$ is false in that
environment.  Since $P$ is satisfiable, it is true in some environment, which
means that $\QNOT(P)$ is false in some environment.  So not all environments make
$\QNOT(P)$ true, which means that $\QNOT(P)$ is \emph{not} valid.
\end{proof}

\textbf{(right-to-left case)}: If $\QNOT(P)$ is \emph{not}
valid, then $P$ is satisfiable.

\begin{proof}
If $\QNOT(P)$ is not valid, some environment makes it false, and
therefore this environment make $P$ true.  This means that $P$ that
$P$ is satisfiable by defintion.
\end{proof}

\end{solution}

\end{problem}

\endinput

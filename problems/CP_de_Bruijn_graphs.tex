\documentclass[problem]{mcs}

\begin{pcomments}
  \pcomment{from: S09.cp7r}
  \pcomment{commented out in S09 but should be in pretty good shape}
\end{pcomments}

\pkeywords{
  digraphs
  Euler_circuit
  binary
  strings
}

%%%%%%%%%%%%%%%%%%%%%%%%%%%%%%%%%%%%%%%%%%%%%%%%%%%%%%%%%%%%%%%%%%%%%
% Problem starts here
%%%%%%%%%%%%%%%%%%%%%%%%%%%%%%%%%%%%%%%%%%%%%%%%%%%%%%%%%%%%%%%%%%%%%

\begin{problem}

  A $3$-bit string is a string made up of $3$ characters, each a $0$
  or a $1$.  Suppose you'd like to write out, in one string, all eight
  of the 3-bit strings in any convenient order.  For example, if you
  wrote out the $3$-bit strings in the usual order starting with
  000 001 010\dots, you could concatenate them together to get a
  length $3\cdot 8 = 24$ string that started 000001010\dots.

  But you can get a shorter string containing all eight $3$-bit
  strings by starting with 00010\dots.  Now $000$ is present as bits
  $1$ through $3$, $001$ is present as bits $2$ through $4$, $010$ is
  present as bits $3$ through $5$, \dots.

\bparts

\ppart Take a few moments to see how short a string you can make that
contains every $3$-bit string as $3$ consecutive bits somewhere in it.
Can you see why $10$ bits is the absolute minimum length for such a
string?

\solution{$0001110100$ does it with $10$ bits and you can't do better:
  there must be two bits to start and each additional bit can yield at
  most one new $3$-bit string.}

\mfigure{!}{3in}{figures/debruijn}

\ppart Imagine that the labels on the vertices of the graph above
represent the last two digits in a string you build by adding one bit
at a time.  Convince yourself that the graph completely describes how
the last two digits of your string can change throughout this process.


\solution{No matter what the last two bits of your current string $x$
  are, say $ab$, there is a vertex representing that.  And, no matter
  what bit you want to use to extend $x$, say $c$, there is an edge
  directed from your current state, $ab$, to the vertex $bc$ that
  represents your new state, the last two bits of $xc$.  Furthermore,
  these are the only types of vertices and edges in the graph.  As a
  bonus, every edge $(s,s')$ is also labeled with the bit that was
  added to the string when going from the state $s$ to $s'$.}


\ppart Find a directed path in this graph starting at some vertex, $v$,
that traverses every edge exactly once.  Note that vertices will have to
be used more than once and the path wil have to end in $v$.

\solution{Many: $00,00,01,11,11,10,01,10,00$ is one.}

\ppart Explain how such a path provides a shortest possible solution
to the original problem.

\solution{Build a solution $x$ using the path.  Start with $x$ equal
  to the two digits in the label of the first vertex on the path.
  Then for every edge used after that, add on the bit labelling that
  edge to the string.  For example, the path given in last part's
  solution, yields the string $x=0001110100$.
  
  Sincce there are $8$ edges, the string will be of length $10$, the
  minimum possible.

  Furthermore, the label of the first vertex of the graph, followed by
  the label of the first edge, give the first $3$-bit substring in
  $x$. The next vertex and edge, give the next $3$-bit substring, and
  so on.  Since all possible $2$-bit vertex labels appear exactly
  once, and each vertex has a $1$ edge and a $0$ edge departing from
  it, every $3$-bit string corresponds to a unique vertex/out-edge
  combination.  The path uses every possible vertex/out-edge
  combination exactly once, so the string contains every $3$-bit
  sequence exactly once.}

\ppart What about $k$-bit substrings, $k = 4, 5, \ldots$?  Can you define
the appropriate generalization of the useful graph above?  (They're called
de Bruijn graphs.) If you do it sucessfully, you should be able to see
that the in-degree (as well as the out-degree) of every vertex is $2$.

It is a theorem that if the in-degree is equal to the out-degree at
every vertex of a digraph (and if the graph is connected when all the
edges are considered undirected edges) then a directed path can be
drawn in that digraph that uses every edge exactly once.  You might
want to think about why this should be true or how you might find such
a path.

But if you do believe it, you should be able to see why all $2^k$
$k$-bit strings can be written as substrings of a string of length
$2^k + k-1$.  (These strings are essentially de Bruijn strings.)

\eparts

\end{problem}

%%%%%%%%%%%%%%%%%%%%%%%%%%%%%%%%%%%%%%%%%%%%%%%%%%%%%%%%%%%%%%%%%%%%%
% Problem ends here
%%%%%%%%%%%%%%%%%%%%%%%%%%%%%%%%%%%%%%%%%%%%%%%%%%%%%%%%%%%%%%%%%%%%%

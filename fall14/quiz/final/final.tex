\documentclass[12pt]{article}
\usepackage{amsmath}
\usepackage{light}
\usepackage{xcolor, colortbl}
\usepackage{graphicx}
\usepackage{verbatim}

\hidesolutions
\showsolutions

\newcommand{\prcond}[2]{\Pr[#1 \mid #2]}
\newcommand{\prob}[1]{\Pr[#1]}
\renewcommand{\pr}[1]{\Pr \left[ #1 \right]}
\newcommand{\var}[1]{\mathrm{Var}(#1)}
\renewcommand{\ex}[1]{\mathop{\textup{Ex}}[#1]}


\newcommand{\quizz}[1]{
    \vspace*{-2cm}
  \noindent \coursename \hfill #1 \newline
  \coursestaff \vspace{-1.5ex} \newline
  \mbox{} \hrulefill \mbox{}
%  \vspace{-0.15in}
  \begin{center}
    \ifthenelse{\boolean{showsolutions}}
               {\Large \textbf{Final Exam Solutions}}
               {\Large \textbf{Final Exam}}
  \end{center}
  \vspace{-.1in}
  \thispagestyle{plain}
  \pagestyle{myheadings}
  \thispagestyle{empty}
  %\markboth{Quiz #1}{Quiz #1}
  \markboth{Final Exam}{Final Exam}
  %\coursecopyright
  }

\begin{document}




\quizz{}



\begin{itemize}\itemsep1.5pt
\item  The exam is \textbf{closed book}, but you may have two $8.5''  \times 11''$ sheet with notes (either printed or in your own handwriting) on both sides.

\item Calculators and electronic devices (including cell phones) are not allowed.

\item You may assume all of the results presented in class. This does \textbf{not} include results demonstrated in practice quiz material.

 \item Please show your work. Partial credit cannot be given for a wrong answer if your work isn't shown.

 \item Write your solutions in the space provided. If you need more space, write on the back of the sheet containing the problem. Please keep your entire answer to a problem on that problem's page.

 \item Be neat and write legibly. You will be graded not only on the correctness of your answers, but also on the clarity with which you express them.

 \item  If you get stuck on a problem, move on to others. The problems are not arranged in order of difficulty.\\


 \textbf{NAME:} \rule{5in}{0.5pt}\\
 
 \textbf{TA:} \rule{5.34in}{0.5pt}\\
 
\centering
\scalebox{1.3}{
\begin{tabular}{|c|c|c|c|}
\hline
\textbf{Problem} & \textbf{Value} & \textbf{Score} & \textbf{Grader} \\\hline
 1 & 10   & &\\ \hline
 2 & 13   & &\\ \hline
 3 & 18   & &\\ \hline
 4 & 10   & &\\ \hline
 5 & 24   & &\\ \hline
 6 & 15   & &\\ \hline
 7 & 10   & &\\ \hline
 8 & 15   & &\\ \hline
 9 & 15   & &\\ \hline
10 & 15   & &\\ \hline
11 & 15   & &\\ \hline
12 & 10   & &\\ \hline
13 & 10   & &\\ \hline
\textbf{Total} & 180 & & \\\hline
\end{tabular}
}
\end{itemize}

\newpage

%%%%%%%%%%%%%%%%%%%%%%%%%%%%%%%%%%
% Warmup / conditional probability
%%%%%%%%%%%%%%%%%%%%%%%%%%%%%%%%%%
\begin{problem}{10}
    {\it Finalphobia} is a rare disease in which the victim has the delusion that he or she is being subjected to an intense mathematical examination.
\begin{itemize}\itemsep-.3em\vspace{-1em}
    \item A person selected uniformly at random has finalphobia with probability $\dfrac{1}{50}$.
    \item A person with finalphobia has shaky hands with probability $\dfrac{9}{10}$.
    \item A person without finalphobia has shaky hands with probability $\dfrac{1}{20}$.
\end{itemize}
What is the probability that a person selected uniformly at random has finalphobia, given that he or she has shaky hands?

Your answer should be expressed as a ratio of integers.

\solution[\newpage]{
    We first compute:
    \begin{align*}
        \pr{\text{shaky hands}} &= \pr{\text{shaky hands} \mid \text{finalphobia}} \cdot \pr{\text{finalphobia}} \\
                                &+ \pr{\text{shaky hands} \mid \text{no finalphobia}} \cdot \pr{\text{no finalphobia}} \\
                              &= \frac{9}{10} \cdot \frac{1}{50} + \frac{1}{20} \cdot \frac{49}{50} \\
                              &= \frac{67}{1000}.
    \end{align*}
    By Bayes's theorem,
    \begin{align*}
        \pr{\text{finalphobia} \mid \text{shaky hands}} &= \frac{\pr{\text{shaky hands} \mid \text{finalphobia}} \pr{\text{finalphobia}}}{\pr{\text{shaky hands}}} \\
                                                      &= \frac{\frac{9}{10} \cdot \frac{1}{50}}{\frac{67}{1000}} \\
                                                      &= \frac{18}{67}.
    \end{align*}
}

\end{problem}
\newpage




%*************
% Summations
%*************
%
% Find a closed form for:
%  sum_1 ^ n       1 + i
%  sum sum         i*j
%  prod prod       2^ij   OR 2^i 3^j
%
% Give a theta bound (Stirlings approximation)
%  prod prod       (i + j)

\begin{problem}{13}
    \bparts
        %\ppart{4} Give a closed form for
        %\begin{equation*}
        %    \sum_{i=1}^n (1 + i)
        %\end{equation*}
%
%        \solution[\vspace{3in}]{
%            \begin{align*}
%                \sum_{i=1}^n (1 + i) &= \sum_{i=1}^n 1 + \sum_{i=1}^n i \\
%                                     &= n + \frac{n (n+1)}{2} \\
%                                     &= \frac{n (n+3)}{2}
%            \end{align*}
%        }

        
        \ppart{3} Give a closed form for
        \begin{equation*}
            \sum_{i=1}^n \sum_{j=1}^n i \cdot j
        \end{equation*}

        \solution[\vspace{3in}]{
%XXX        \solution[\newpage]{
            \begin{align*}
                \sum_{i=1}^n \sum_{j=1}^n i \cdot j &= \sum_{i=1}^n (i \cdot \sum_{j=1}^n j) \\
                                                    &= \sum_{i=1}^n i \cdot \frac{n(n+1)}{2} \\
                                                    &= \frac{n(n+1)}{2} \cdot \sum_{i=1}^n i \\
                                                    &= \left( \frac{n (n+1)}{2} \right)^2.
            \end{align*}
        }

        \ppart{5} 
        Give a closed form, as a ratio of polynomials in $n$, for
        $$ \sum_{j=1}^n \sum_{i=j}^n j. $$
        For convenience, we provide the identity:
        $\displaystyle \sum_{i=1}^n i^2 = \frac{n(n+1)(2n+1)}{6}. $

        \solution[\newpage]{
            \begin{align*}
                \sum_{j=1}^n \sum_{i=j}^n j &= \sum_{j=1}^n j \sum_{i=j}^n 1 \\
                                            &= \sum_{j=1}^n j (n-j+1) \\
                                            &= n \sum_{j=1}^n j - \sum_{j=1}^n j^2 + \sum_{j=1}^n j \\
                                            &= (n+1) \sum_{j=1}^n j - \sum_{j=1}^n j^2 \\
                                            &= (n+1) \cdot \frac{n(n+1)}{2} - \frac{n(n+1)(2n+1)}{6} \\
                                            &= \frac{n(n+1)(3n+3) - n(n+1)(2n+1)}{6} \\
                                            &= \frac{n(n+1)(n+2)}{6}.
            \end{align*}
        }

        %\ppart{3} Give a closed form for
        %\begin{equation*}
        %    \prod_{i=1}^n \prod_{j=1}^n 2^i 3^{2j}
        %\end{equation*}
        %\solution[\vspace{3in}]{
        %    \begin{align*}
        %        \prod_{i=1}^n \prod_{j=1}^n 2^i 3^{2j} &=  \left( \prod_{i=1}^n 2^i \right)^n \;\cdot\; \left( \prod_{j=1}^n 3^{2j} \right)^n \\
        %                                               &=  \left( 2^{\sum_{i = 1}^n i} \right)^n \;\cdot\; \left( 3^{\sum_{j=1}^n 2j} \right)^n \\
        %                                               &=  \left( 2^{\frac{n(n+1)}{2}} \right)^n \;\cdot\; \left( 3^{n(n+1)} \right)^n \\
        %                                               &= 2^{\frac{n^2(n+1)}{2}} \;\cdot\; 3^{n^2(n+1)}.
        %    \end{align*}
        %}

        \ppart{5} Let $f(n) = \prod_{i=1}^n 2i$. Give a function $g$ in closed form such that
        $f(n) = \Theta(g(n))$. You may not include factorials in your answer.
        \solution{
            We can write
            $$ f(n) = 2^n \prod_{i=1}^n i = 2^n \cdot n! $$
            Using Stirling's approximation, we have that
            \begin{equation*}
                f(n) \sim 2^n \sqrt{2 \pi n} \left( \frac{n}{e} \right) ^n
            \end{equation*}
            This asymptotic on the right is an adequate answer. Another expression (which equals the same thing up to a constant) is
            $$\sqrt{n} \left( \frac{2n}{e} \right) ^n.$$
        }
    \eparts
\end{problem}
\newpage


%************
% Counting
%************
% Battleship
% How many ways can you place two pieces
%
% 5x5 with a -- and ---
%
% Then opponent gets to make 15 random shots (independent)
% what is the probability that both of ships survive

\begin{problem}{18}
We play a simplified game of battleship. We are given a 4x4 board
\begin{center}
    \begin{tabular}{| c | c | c | c |}
        \hline
        & & & \\ \hline
        & & & \\ \hline
        & & & \\ \hline
        & & & \\ \hline
    \end{tabular}
\end{center}
on which you have placed
% must place
two pieces. Your destroyer is $1 \text{ square} \times  2 \text{ squares}$
    \begin{tabular}{| c | c |}
        \rowcolor[gray]{0.8}
        \hline
        & \\ \hline
    \end{tabular}
    and your submarine is $1 \text{ square} \times 3 \text{ squares}$
    \begin{tabular}{| c | c | c |}
        \rowcolor[gray]{0.8}
        \hline
        & & \\ \hline
    \end{tabular}.
The pieces
%must
lie entirely on the board, cannot overlap, and are
%must be
arranged either vertically or horizontally.

Your opponent picks 8 of the 16 squares uniformly at random and then shoots at those
8 squares. A ship is sunk if all the squares it occupies are shot at. 

For this problem, you may leave your answer as the sum of expressions that are products or ratios of integers, exponentials, factorials and/or choose expressions.

\bparts
%\ppart{10} How many ways are there to place your pieces on the board?
%\solution[\newpage]{
%  First we count the number of ways to place the destroyer. If it is horizontal, there are
%  12 possible locations and by symmetry, there are 12 vertical locations, for a total of 24 locations.
%  Similarly, there are 16 locations for the submarine.
%  The total number of ways to place both pieces is the product of all locations for the destroyer and
%  the submarine, subtracting away the number of illegal overlapping positions.
%
%  Suppose the pieces are both horizontal or both vertical. There are 24 horizontal and 24 vertical
%  arrangements that give the overlap. Next suppose the submarine is horizontal and the destroyer is
%  vertical. If the submarine is in the top or bottom row, there are 4 positions for it and 3
%  overlapping positions for the destroyer for each. If the submarine is in a middle row, there are
%  4 positions for it and 6 overlapping positions for the destroyer for each. This adds to 12 + 24 = 36
%  overlapping positions. By symmetry, if the submarine is vertical and the destroyer is horizontal, there
%  are 36 overlapping positions.
%
%  The total number of ways to place the pieces on the board is $24 \cdot 16 - 8 - 8 - 36 - 36 = \fbox{264}$
%}

\ppart{8}
What is the probability that both of your ships are sunk?
\solution[\newpage]{
    There are $16 \choose 8$ total choices for your opponent. There are ${16 - 5 \choose 8 - 5}$
    choices that pick all of your squares. So the probability that both of your ships 
    are sunk is
    $\frac{{11 \choose 3}}{{16 \choose 8}}$.
}

\ppart{10} What is the probability of sinking the submarine but not the destroyer? 
\solution{
    To sink the submarine (and perhaps also the destroyer), your opponent must choose the $3$ submarine squares and any $5$ of the $13$ others; so the probability of this event is
    $$ \frac{{13 \choose 5}}{{16 \choose 8}}. $$
    To exclude the possibility of also sinking the destroyer, we subtract off our answer to part (a), and conclude with
    $$ \frac{{13 \choose 5} - {11 \choose 3}}{{16 \choose 8}}. $$
}

\eparts

\end{problem}
\newpage


%************
% Counting
%************
% Card Shuffle
%
% Probability of getting a pair with a random 5 card hand
% Probability of getting a pair with a random 10 card hand
% Probability of getting a pair with a random 15 card hand

\begin{problem}{10}
    %\bparts
    %\ppart{5}
    You get 5 cards at random from a standard 52 card deck. What is the probability that
    you have exactly one pair? (This means that exactly 2 cards share the same rank, so two pairs or three of
    a kind would not count.)

    You may express your answer as the product or ratio of integers, factorials, exponentials, and/or choose expressions.
    \solution[\vspace{3in}]{
        There are 13 different ranks that could have a pair. For each rank, you need
        2 out of the 4 cards. There are 12 remaining ranks for the last three cards
        and each has four possible suits, for a final answer of:
        $$ \frac{{13 \choose 1} {4 \choose 2} {12 \choose 3}{ 4 \choose 1}^3 }{{52 \choose 5}} $$
    }

    %\ppart{2} You get 10 cards at random from a standard 52 card deck. What is the probability that
    %you have exactly one pair?
    %\solution[\vspace{1in}]{
    %    Similarly to above, except there are now 8 other cards.
    %    $$ \frac{{13 \choose 1} {4 \choose 2} {12 \choose 8}{ 4 \choose 1 }^8 }{{52 \choose 5}} $$
    %}

    %\ppart{5} You get 15 cards at random from a standard 52 card deck. What is the probability that
    %you have at exactly one pair?
    %\solution{
    %    There are 13 possible ranks and 15 cards in your hand. The pigeonhole principle states
    %    that there must be at least two pairs or a three of a kind. Given 15 cards, the probability
    %    that you have exactly one pair is $\fbox{0}$
    %}
    %\eparts
\end{problem}
\newpage

%************
% Variance
%************
% Stocks have the following possible outcomes:
% What is the variance
% What is the EV
% What is the probability that we end up with at least X dollars

\begin{problem}{24}
    Alice decides to play the lottery.
    She bought 10,000 tickets, each with a probability of $\dfrac{1}{1,000,000}$ of winning a payout of
    $\$ 1,000,000$, probability $\dfrac{1}{10}$ of paying out $\$ 10$, and probability $1 - \dfrac{1}{10} - \dfrac{1}{1,000,000}$ of being worth nothing.
    If multiple tickets win, the payouts remain as above for each ticket. The tickets are mutually independent.

    \bparts
    \ppart{5}
        What is Alice's expected return? Express your answer as an integer.
        \solution[\vspace{2in}]{
            Each ticket has an expected return of $\frac{1}{1,000,000} \cdot 1,000,000  + \frac{1}{10} \cdot 10 = 2$ dollars. By the
            linearity of expectation, Alice's expected return is the sum of the expected returns of $10,000$ such tickets, which gives a total of $\$ 20,000$.
        }

    \ppart{2}
        Does your answer to part (a) depend on the tickets being mutually independent?
        \solution[\vspace{1in}]{
            Part (a) uses only the linearity of expectation, so independence is not required.
        }

    \ppart{5}
        What is the variance of Alice's return? Express your answer as an integer.
        \solution[\newpage]{
            Because the payoffs $X_i$ of individual tickets are mutually independent, the
            variance of the sum is the sum of the variances. So we compute the
            variance of an individual ticket:
            \begin{align*}
                \var{X_i} &= \expect{(X_i - \expect{X_i})^2} \\
                          &= (-2)^2 \cdot \left( 1 - \frac{1}{10} - \frac{1}{1,000,000} \right)  + 999,998^2 \cdot \frac{1}{1,000,000} + 8^2 \cdot \frac{1}{10} \\
                          &= 1,000,006.
            \end{align*}
            An arithmetically easier way to compute this is as follows:
            $$ \expect{X_i^2} = 1,000,000^2 \cdot \frac{1}{1,000,000} + 10^2 \cdot \frac{1}{10} = 1,000,010, $$
            $$ \var{X_i} = \expect{X_i^2} - (\expect{X_i})^2 = 1,000,010 - 4 = 1,000,006. $$

            The total variance of $R = \sum_{i=1}^{100} X_i$ is then $10,000$ times this, for a final answer of $10,000,060,000$ (in units of dollars squared).
        }

    \ppart{2}
        Does your answer to part (c) depend on the tickets being mutually independent?
        \solution[\vspace{1in}]{
            Part (c) uses the additivity of variance, so independence is required.
        }

    \ppart{4}
        Give a Markov upper bound for the probability that Alice wins at least $\$ 100,000,000$. Your answer should be expressed as a ratio of integers.
        \solution[\vspace{3in}]{
            Markov's Theorem states that, since $R$ is non-negative,
            \begin{equation*}
                \pr{R \ge x} \le \frac{\expect{R}}{x}.
            \end{equation*}
            In this case,
            $$\pr{R \ge 100,000,000} \le \frac{20,000}{100,000,000} = \frac{1}{5,000}.$$
        }

%    \ppart{4}
%        Does the bound in part (e) lie within $0.1$ of the actual value of the probability that Alice wins at least $\$1000$? Explain.
%        \solution{
%            No. For Alice to win at least $1000$, she must either have one of the very rare tickets with payoff $1000000$, or else all $100$ of her tickets must have payoff $10$. Both of these events are much less likely than the bound in (e). To be very explicit, we could use the Union Bound to see that the probability is at most
%            $$ 100 \cdot 0.000001 + (0.1)^100 = \frac{1}{10000} + 10^{-100}, $$
%            which is clearly much smaller than $0.2 - 0.1$.
%        }
        
    \ppart{6}
        Give a Chebyshev upper bound for the probability that Alice wins at least $\$ 100,000,000$.
        Your answer should be expressed in terms of basic arithmetic operations on integers, but need not be simplified further. 
        %Your answer should be expressed as a ratio of integers.

        \solution{
            \begin{align*}
                \pr{R \geq 100,000,000} &= \pr{|R-20,000| \geq 99,980,000} \\
                                        &= \pr{|R - \expect{R}| \geq 99,980,000} \\
                                        &\leq \frac{\Var{R}}{99,980,000^2} \\
                                        &= \frac{10,000,060,000}{99,980,000^2} \\
                                        &= \frac{1,000,006}{999,800^2} \\
                                        &= \frac{1,000,006}{999,600,040,000} \\
                                        &= \frac{500,003}{499,800,020,000}.
            \end{align*}
        }

    \eparts
\end{problem}
\newpage


%************
% Combinatorial Proof
%************
% Vandermondes Identity
%\begin{problem}{10}
%    Give a combinatorial proof of the following equality by counting something two ways.
%    \begin{equation*}
%        {m + n \choose r} = \sum_{k=0}^r {m \choose k} {n \choose r-k}
%    \end{equation*}
%
%    \solution{
%        We count the number of ways to make a committee of $r$ people out of
%        $m$ men and $n$ women. The left side is just choosing $r$ out of the
%        union of the set of men and the set of women.
%
%        On the right side, we iterate over the number of men on the committee.
%        If there are $k$ men on the committee, then there are $r-k$ women on
%        the committee. There are $m \choose k$ ways to pick the men and
%        $n \choose n-k$ ways to pick the women. The number of men on the committee
%        can range from $0$ to $r$, so we sum over all choices of $k$.
%
%        The left and right sides are both counting the number of ways to
%        choose $r$ objects from the union of a set with $m$ objects and a set
%        with $n$ objects.
%    }
%
%\end{problem}
%\newpage

%\begin{problem}{15}
%
%Give a combinatorial proof of the following identity by letting
%$S$ be the set of all length-$n$ sequences of letters $a$, $b$ and
%exactly one $c$, and counting $|S|$ two different ways.
%
%\begin{equation}\label{n2n-1comproof}
%n \cdot 2^{n-1} = \sum_{k=1}^n k \binom{n}{k}
%\end{equation}
%
%\solution{
%    Let $P = \{ 0, \ldots, n-1 \} \times \{ a, b \}^{n-1}$.  On
%the one hand, there is a bijection from $P$ to $S$ by mapping $(k,x)$ to
%the word obtained by inserting a $c$ just after the $k$th letter in the
%length-$(n-1)$ word, $x$, of $a$'s and $b$'s.  So
%\begin{equation}\label{SPn2n-1}
%|S| = |P| = n \cdot 2^{n-1},
%\end{equation}
%by the Product Rule.
%
%On the other hand, every sequence in $S$ contains between $1$ and $n$
%entries not equal to $a$ (since the $c$, at least, is not $a$).  
%We map each sequence in $S$ with exactly $k$ non-$a$ entries to the
%pair consisting of the choice of $k$ positions of the non-$a$ entries and the
%position of the $c$ among these entries. This map is a bijection, and the number
%of such pairs is $\binom{n}{k}k$ by the Generalized Product Rule.
%Thus, by the Sum Rule:
%\[
%|S| = \sum_{k=1}^n k \binom{n}{k}.
%\]
%Equating this expression and the expression~\eqref{SPn2n-1} for $|S|$
%proves the theorem.
%}
%
%\end{problem}
%\newpage


%\begin{problem}{15}
%    Give a combinatorial proof of the following identity:
%\[
%\sum_{i=0}^{n} \binom{n}{i}(-1)^i = \begin{cases}
%0 & \text{if }n>0,\\
%1 & \text{if }n=0.
%\end{cases}
%\]
%In particular, use of the Binomial Theorem will {\bf not} be considered a combinatorial proof.
%
%{\it Hint: Consider the zero-one strings with an even number of ones versus with an odd
%number of ones.}
%
%\solution{
%Consider the $n$-bit zero-one strings, and divide
%them into two sets, those with an even number of ones (even $i$ terms) and
%those with an odd number of ones (odd $i$ terms).  The sum is then equal
%to the number of strings with an even number of ones, minus the number of
%strings with an odd number of ones.
%
%Next, we note that for $n >0$, the number of $n$-bit strings with an
%even number of ones is equal to the number with an odd number of ones.
%This can be seen by establishing a bijection between the two sets: any
%string in one set can be made into a string in the other set by
%complementing the first bit in the string.  Since the number of strings
%with an even number of ones is equal to the number with an odd number,
%the entire expression must be equal to 0.
%
%The step of complementing the first bit breaks down for $n=0$, where
%there is no first bit. We can see that the alternating sum of binomial
%coefficients is ${0 \choose 0} = 1$ in this case.
%}
%
%\end{problem}
%\newpage



%********************
% Inclusion-exclusion
%********************
\begin{problem}{15}

    How many of the numbers $1, 2, \ldots, 3000$ are divisible by one or more of $4$, $5$, or $6$?

    \solution{
        We will apply the inclusion--exclusion principle.

        Let $S_i$ be the set of numbers $1,2,\ldots,3000$ that are divisible by $i$; and let $T$ denote the set of numbers $1,2,\ldots,3000$ divisible by one or more of $4$, $5$, or $6$; so we are trying to count $T$. Notice that $T = S_4 \cup S_5 \cup S_6$. Then
%
%        \begin{align*}
%            \#(\text{divisible by $4$, $5$, or $6$}) &= \#(\text{divisible by $4$}) + \#(\text{divisible by $5$}) + \#(\text{divisible by $6$}) \\
%            &\quad - \#(\text{divisible by $4$ and $5$}) - \#(\text{divisible by $4$ and $6$}) \\
%            &\quad - \#(\text{divisible by $5$ and $6$}) + \#(\text{divisible by $4$, $5$, and $6$}) \\
%            &=  \#(\text{divisible by $4$}) + \#(\text{divisible by $5$}) + \#(\text{divisible by $6$}) \\
%            &\quad - \#(\text{divisible by $20$}) - \#(\text{divisible by $12$}) - \#(\text{divisible by $30$}) \\
%            &\quad + \#(\text{divisible by $60$}) \\
%            &= \lfloor \frac{3000}{4} \rfloor + \lfloor \frac{3000}{5} \rfloor + \lfloor \frac{3000}{6} \rfloor - \lfloor \frac{3000}{20} \rfloor - \lfloor \frac{3000}{12} \rfloor - \lfloor \frac{3000}{30} \rfloor + \lfloor \frac{3000}{60} \rfloor \\
%            &= 750 + 600 + 500 - 150 - 250 - 100 + 50 \\
%            &= 1400.
%        \end{align*}
%        (Note that the numbers $20$, $12$, $30$, and $60$ here are LCMs, not just products.)
%
        \begin{align*}
            |T| &= |(S_4 \cup S_5) \cup S_6| \\
                &= |S_4 \cup S_5| + |S_6| - |(S_4 \cup S_5) \cap S_6| & \text{(by inclusion--exclusion)} \\
                &= |S_4| + |S_5| + |S_6| - |S_4 \cap S_5| - |(S_4 \cup S_5) \cap S_6| & \text{(by inclusion--exclusion)} \\
                &= |S_4| + |S_5| + |S_6| - |S_4 \cap S_5| - |(S_4 \cap S_6) \cup (S_5 \cap S_6)| \\
                &= |S_4| + |S_5| + |S_6| - |S_4 \cap S_5| - |S_4 \cap S_6| \\
                &\qquad - |S_5 \cap S_6| + |(S_4 \cap S_6) \cap (S_5 \cap S_6)| & \text{(by inclusion--exclusion)} \\
                &= |S_4| + |S_5| + |S_6| - |S_4 \cap S_5| - |S_4 \cap S_6| \\
                &\qquad - |S_5 \cap S_6| + |S_4 \cap S_5 \cap S_6| \\
                &= |S_4| + |S_5| + |S_6| - |S_{20}| - |S_{12}| - |S_{30}| + |S_{60}| & \text{(by taking LCMs)} \\
                &= \lfloor \frac{3000}{4} \rfloor + \lfloor \frac{3000}{5} \rfloor + \lfloor \frac{3000}{6} \rfloor \\
                &\qquad - \lfloor \frac{3000}{20} \rfloor - \lfloor \frac{3000}{12} \rfloor - \lfloor \frac{3000}{30} \rfloor + \lfloor \frac{3000}{60} \rfloor \\
                &= 750 + 600 + 500 - 150 - 250 - 100 + 50 \\
                &= 1400.
        \end{align*}

        A more concise solution might directly apply the general inclusion--exclusion formula for three sets, whereas here we deal with unions of two sets at a time.
    }

\end{problem}
\newpage




%************
%Probability (counter intuitive results) or draw out game tree
%************
%Nontransitive dice (but Leighton made a mistake)
%argue that there is a die that the student should pick
% 2,5,9    1,8,6     3,7,4
%\begin{problem}{15}
%    Professor Leighton has tried to setup his non-transitive dice game to play against the
%    students. The dice each have equal probability of landing on their 6 sides and are
%    mutually independent. Die A has values $\{2,2,5,5,9,9\}$. Die B has values $\{1,1,6,6,8,8\}$.
%    Die C has values $\{3,3,4,4,7,7\}$. You believe that he has made a mistake when making
%    these dice and that they are in fact transitive. Use a tree diagram to show that if you pick A,
%    the probability that your roll is higher than his is greater than $\frac{1}{2}$ no matter
%    which die he chooses.
%
%    \solution{
%        We use a tree diagram for each choice of die that Professor Leighton could have made.
%        If he picks B, our diagram shows that we win with probability $\frac{5}{9}$.
%        If he picks C, our diagram shows that we win with probability $\frac{5}{9}$.
%
%        [ draw out the tree ]
%    }
%
%\end{problem}
%\newpage


%************
%Induction
%************
%Fibonacci identity
%\begin{problem}{15}
%    Recall that the Fibonacci numbers are defined by the equation $F_n = F_{n-1} + F_{n-2}$, 
%    with $F_1 = F_2 = 1$. Prove using induction that for $n \ge 2$,
%    \begin{equation*}
%        F_{n+1}F_{n-1} - F_{n}^2 = (-1)^{n}.
%    \end{equation*}
%
%    \solution{
%        First we check the base case: $F_3 F_1 - F_2^2 = 2 - 1 = 1 = (-1)^2.$
%
%        Next we assume the inductive hypothesis holds for $n$. $F_{n+1}F_{n-1} - F_{n}^2 = (-1)^{n}$
%        We need to show that the identity holds for $n+1$.
%
%        \begin{align*}
%            F_{n+2}F_{n} - F_{n+1}^2 &= (F_{n} + F_{n+1})F_{n} - F_{n+1}^2 \\
%                                     &= F_n^2 + F_{n+1}F_n - F_{n+1}^2 \\
%                                     &= F_n^2 + F_{n+1}(F_{n+1} - F_{n-1}) - F_{n+1}^2 \\
%                                     &= F_n^2 - F_{n+1}F_{n-1} \\
%                                     &= -1 (-1)^{n} \\
%                                     &= (-1)^{n+1}.
%        \end{align*}
%    }
%\end{problem}
%\newpage

% induction, factorial identity
\begin{problem}{10}

    Prove the following identity for all natural numbers $n$:
    $$ \sum_{i=1}^n (i \cdot i!) = (n+1)! - 1. $$

    \solution{
        We proceed by induction. The base case $n=1$ is clear:
        $$ 1 \cdot 1! = 2! - 1. $$
        For the inductive step, suppose that the identity holds for $n$; we prove it for $n+1$:
        \begin{align*}
            \sum_{i=1}^{n+1} (i \cdot i!) &= (n+1)! - 1 + (n+1)(n+1)! & \text{(by inductive hypothesis)} \\
                                          &= (n+2)(n+1)! - 1 & \text{(factoring)} \\
                                          &= (n+2)! - 1.
        \end{align*}
        This completes the inductive step. By induction, the identity holds for all natural numbers $n$.
    }

\end{problem}
\newpage


% ***********
% Recurrences
% ***********
\begin{problem}{15}

    \bparts

    \ppart{8} Give an exact solution to the following recurrence:
    $$ T(n) = 5 T(n-1) - 6 T(n-2),\quad T(0) = 0,\; T(1) = 1. $$
    \solution[\vspace{3in}]{
        This is a linear recurrence with characteristic equation $ x^2 = 5x - 6$, which factors as
        $$ (x-2)(x-3) = 0. $$
        We therefore expect a solution of the form $T(n) = a \cdot 2^n + b \cdot 3^n$. From the initial conditions, we have $a+b=0$ and $2a+3b=1$, from which $b = 1$ and $a = -1$. So the exact solution is:
        $$ T(n) = 3^n - 2^n. $$
    }

    \ppart{7} Give an asymptotic expression for the following recurrence, in $\Theta$ notation:
    $$ T(n) = 4 T\left( \frac{n}{2} \right) + n^2, \quad T(1) = 0. $$
    \solution[\newpage]{
        We apply the Akra--Bazzi method with $g(n) = n^2$, $a = 4$, $b = \frac{1}{2}$. Solving $a b^p = 1$ for $p$, we find $p = 2$. Then
        \begin{align*}
            T(n) &= \Theta \left( n^2 \left( 1 + \int_1^n \frac{u^2}{u^3} \,du \right) \right) \\
                 &= \Theta \left( n^2 \left( 1 + \log n \right) \right) \\
                 &= \Theta ( n^2 \log n ).
        \end{align*}

        An alternative is to apply the Master Theorem.
    }

    %\ppart{6} Give an asymptotic expression for the following recurrence, in $\Theta$ notation:
    %$$ T(n) = T\left( \frac{1}{9} n \right) + T\left( \frac{4}{9} n \right) + \sqrt{n}, \quad T(1) = 0. $$
    %\solution{
    %    Again we apply the Akra--Bazzi method, now with $g(n) = \sqrt{n}$, $a_1 = 1 = a_2$, $b_1 = \frac{1}{9}$, and $b_2 = \frac{4}{9}$. Solving $\sum_i a_i b_i^p = 1$, we see that $p = \frac{1}{2}$ satisfies $(\frac{1}{9})^p + (\frac{4}{9})^p = 1$. So we have
    %    \begin{align*}
    %        T(n) &= \Theta \left( n^{1/2} \left( 1 + \int_1^n \frac{\sqrt{u}}{u^{3/2}}\, du \right) \right) \\
    %             &= \Theta \left( n^{1/2} \left( 1 + \log n \right) \right) \\
    %             &= \Theta \left( \sqrt{n} \log n \right).
    %    \end{align*}
    %}

    \eparts

\end{problem}
\newpage



\begin{problem}{15}
    Suppose that we build a graph on $N$ vertices as follows: for each (unordered) pair of distinct vertices, we independently toss a fair coin, and draw an edge between that pair of vertices if the coin lands `heads'.

    Justify your responses for all parts below.

    \bparts

    \ppart{6} Given a vertex, what is the probability that the vertex has degree exactly $3$? 
    \solution[\vspace{4in}]{
        This vertex is a member of $N-1$ vertex pairs, and this vertex has degree $3$ precisely when exactly $3$ of these vertex pairs are connected by edges. Considering the $2^{N-1}$ equally probable outcomes for the edge status of these $N-1$ vertex pairs, there are $\binom{N-1}{3}$ outcomes for which the vertex has degree $3$. Thus the probability is
        $$ 2^{-N+1} \binom{N-1}{3}. $$
    }

    \ppart{3} Let $D_i$ denote the degree of vertex $i$ as a random variable. Which standard family of probability distributions does $D_i$ belong to?
    \solution[\newpage]{
        The degree follows a binomial distribution with parameters $n = N-1$ and $p = \frac12$. The argument of the part above essentially shows this; we could replace `$3$' with any other value in part (a) and recover the pdf for a binomial distribution.
        
        For a more formal argument, let $E_{ij}$ denote the indicator random variable taking on value $1$ if $(i,j)$ is an edge, which is true with probability $\frac12$; this is a Bernoulli random variable with parameter $\frac12$, and the random variables $E_{ij}$ are independent. Fixing a vertex $i$, the degree of $i$ is $\sum_j E_{ij}$, and a sum of $N-1$ independent Bernoulli variables with parameter $\frac12$ is a binomial distribution as described.
    }

    \ppart{6} What is the expected number of vertices in the graph with degree exactly $3$? 

    Partial credit will be given for answers written in terms of $p$, where $p$ represents the correct answer to part (a).

    \solution{
        Let $X_i$ be the indicator that vertex $i$ has degree $3$; then the expectation of $X_i$ is the same as the probability that vertex $i$ has degree $3$, which we computed in part (a). The number of degree $3$ vertices in the graph is $X = \sum_{i=1}^N X_i$. Its expectation is
        $$ \expect{X} = \expect{\sum_{i=1}^N X_i} = \sum_{i=1}^N \expect{X_i} = N \cdot p = N \cdot 2^{-N+1} \binom{N-1}{3}, $$
        by the linearity of expectation. 
    }

    \eparts
\end{problem}
\newpage



\begin{problem}{15}

    \bparts

    \ppart{8} Let $d \geq 1$ be a given constant, and consider the sequence $(a_k)_{k \geq 0}$ defined by
    $$ a_k = \begin{cases} 1 & \text{if }1 \leq k \leq d, \\ 0 & \text{otherwise}. \end{cases} $$
    Give a closed form expression for the generating function $f_d(x)$ of this sequence, expressed as a ratio of polynomials in the variable $x$. The answer will depend on $d$.

    {\it Hint: Be careful of the first term, $a_0 = 0$.}

    \solution[\newpage]{
        If we were to shift the sequence left by one, so that $a_0 = 1$ but $a_d = 0$, the generating function would be a finite geometric series with initial term $1$ and closed form
        $$ \frac{1 - x^d}{1-x}. $$
        Shifting the sequence right corresponds to multiplying by $x$, so the actual generating function is
        $$ \frac{x(1-x^d)}{1-x}. $$
    }

    \ppart{7} Suppose we have two dice, one with values $1, 2, \ldots, d_1$ on its sides, and the other with values $1, 2, \ldots, d_2$. Define the sequence $(b_k)_{k \geq 0}$ as the number of outcomes for the dice that yield a total sum of $k$. What is the generating function of this sequence?

    {\it Hint: Your answer to part (a) might be helpful. }

    Partial credit will be given for answers written in terms of $f_d(x)$, the generating function that correctly answers part (a).

    \solution{
        The sequence $(b_k)$ is a convolution of two copies of $(a_k)$, where the two copies have $d_1$ and $d_2$ respectively substituted for $d$. Convolution of sequences corresponds to multiplication of generating functions, so the generating function is:
        \begin{align*}
            f_{d_1}(x) \cdot f_{d_2}(x) &= \frac{x(1-x^{d_1})}{1-x} \cdot \frac{x(1-x^{d_2})}{1-x} \\
                                        &= \left( \frac{x}{1-x} \right)^2 (1-x^{d_1})(1-x^{d_2}). 
        \end{align*}
    }

    \eparts

\end{problem}
\newpage


\begin{problem}{15}
    Suppose we choose a random $9$-digit MIT ID number, taken uniformly from the numbers 000000000 to 999999999.
    
    Each answer should be expressed as a ratio of integers or in scientific notation.
    \bparts
    \ppart{8} What is the expected number of total occurrences of the digit sequence `6042' in the ID number?
    \solution[\vspace{4in}]{
        For $1 \leq i \leq 6$, let $X_i$ be the indicator random variable with value $1$ precisely when the $i$th digit of the ID is a $6$, the $(i+1)$th digit is a $0$, the $(i+2)$th is a $4$, {\bf and} the $(i+3)$th is a $2$.

        Each $X_i$ is $1$ with probability $\frac{1}{10000}$. By the linearity of expectation, the expected number of occurrences of `6042' is
        $$ \expect{X_1 + \ldots + X_6} = \expect{X_1} + \ldots + \expect{X_6} = \frac{6}{10000}. $$ 
    }

    \ppart{7} What is the expected total number of occurrences of `6042' or `6041'? That is,
    $$ \expect{\text{\# occurrences of `6042'} + \text{\# occurrences of `6041'}}. $$
    \solution{
        Again we use the linearity of expectation to rewrite this as 
        $$ \expect{\text{\# occurrences of `6042'}} + \expect{\text{\# occurrences of `6041'}}. $$
        The first term is the answer to part (a); the second term is the same, since we would not have answered part (a) differently for the sequence `6041'. Thus the answer is $\dfrac{12}{10000}$.
    }
    
    \eparts
\end{problem}
\newpage



\begin{problem}{10}

    Suppose that we randomly scramble (uniformly) the letters of the word `ABRACADABRA'; what is the probability that the word remains `ABRACADABRA' afterwards? Express your answer as a ratio of integers, exponentials, factorials, and/or choose expressions.

    \solution{
        We first count the letters in the given word: five As, two Bs, two Rs, a C and a D; 11 letters in total. In order to shuffle letters around and still obtain the word `ABRACADABRA', we can only permute the five As among themselves, permute the two Bs, and permute the two Rs, for which there are $5! \cdot 2! \cdot 2! = 480$ possible outcomes. On the other hand, there are $11!$ total ways to shuffle the letters. So the final probability is
        $$ \frac{5! \cdot 2! \cdot 2!}{11!}. $$
    }

\end{problem}
\newpage



\begin{problem}{10}

    Identify each of the following asymptotic statements as true or false, with brief justification.

    \bparts

    \ppart{2} $4^x = O(3^x)$.
    \solution[\vspace{1in}]{ False. The ratio
    $$ \frac{4^x}{3^x} = \left(\frac43\right)^x $$
tends to $\infty$ as $x \to \infty$. }

    \ppart{2} $x^2 = O(x^3)$.
    \solution[\vspace{1in}]{ True. We can multiply $x^3$ by a constant (say, $1$) such that it eventually forms an upper bound for $x^2$ (say, for all $x \geq 1$). }

    \ppart{2} $x^3 - x^2 = o(5x^3)$.
    \solution[\vspace{1in}]{ False. The ratio $\dfrac{5x^3}{x^3 - x^2}$ tends to $5$, not to $\infty$, as $x \to \infty$. }

    \ppart{2} $\frac{1}{x} = \Theta(1)$.
    \solution[\vspace{1in}]{ False. $\frac{1}{x}$ is not bounded below by any \emph{positive} constant times $1$. }

    \ppart{2} $e^{\ln^2 x} = \Omega(x^{100})$.
    \solution[\vspace{1in}]{ True. We can write $e^{\ln^2 x} = x^{\ln x}$, which exceeds $x^{100}$ for all sufficiently (very!) large $x$. }

    \eparts

\end{problem}



\end{document}


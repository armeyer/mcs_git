\documentclass[problem]{mcs}

\begin{pcomments}
  \pcomment{CP_multinomial_fermat}
  \pcomment{old name: CP_multinomial_euler}
  \pcomment{old name: CP_multinomial_theorem_2}
  \pcomment{from: S09.cp12m}
  \pcomment{revised slightly by ARM 4/24/16}
\end{pcomments}

\pkeywords{
  multinomial
  Fermat
  mod
  counting
}

%%%%%%%%%%%%%%%%%%%%%%%%%%%%%%%%%%%%%%%%%%%%%%%%%%%%%%%%%%%%%%%%%%%%%
% Problem starts here
%%%%%%%%%%%%%%%%%%%%%%%%%%%%%%%%%%%%%%%%%%%%%%%%%%%%%%%%%%%%%%%%%%%%%

\begin{problem}  Let $p$ be a \textbf{prime number}.

\bparts

\ppart\label{pk1k2} Explain why the multinomial coefficient
\[
%\begin{equation}\label{pk1k2}
\binom{p}{k_1, k_2,  \dots, k_n}
\]
%\end{equation}
is divisible by $p$ if all the $k_i$'s are nonnegative integers less
than $p$.

\examspace[2.0in]

\begin{solution}
The multinomial coefficient is an integer equal to the quotient of
$p!$ divided by the product $k_1 !, k_2 !, \dots, k_n !$.  If all the
$k_i$'s are less than $p$, then none of the denominator terms divides
the numerator $p$ and so the multinomial coefficient is divisible by
$p$.
\end{solution}

\ppart Conclude from part~\eqref{pk1k2} that
\begin{equation}\label{xp}
(x_1 + x_2 + \cdots +x_n)^p \equiv x_1^p + x_2^p + \cdots +x_n^p \pmod p.
\end{equation}
(Do not prove this using Fermat's ``little'' Theorem.  The point of
this problem is to offer an independent proof of Fermat's theorem.)

\examspace[2.0in]

\begin{solution}
  By the Multinomial Theorem~\bref{multinom-thm}, $(x_1 + x_2 + \cdots
  +x_n)^p$ is a sum of monomials in $x_1,\dots,x_n$ whose coefficients
  are of the form given in part\eqref{pk1k2}.

  Since the sum of the $k_i$'s is $p$, the only coefficients not
  divisible by $p$ are the coefficients where some $k_i =p$ and all
  the other $k_j$'s are zero.  That is, the only coefficients not
  $\equiv 0 \pmod p$ are the coefficients 
  \[
  \binom{p}{0,0,\dots,0,p,0,\dots,0} = 1
  \]
  of the monomials $x_i^p$.  The congruence~\eqref{xp} follows
  immediately.
\end{solution}

\ppart Explain how~\eqref{xp} immediately proves Fermat's Little
Theorem\inbook{~\bref{fermat_little}}:
\[
n^{p-1} \equiv 1 \pmod p
\]
when $n$ is not a multiple of $p$.

\begin{solution}
  Let $x_1=x_2=\cdots x_n = 1$.  Then~\eqref{xp} implies $n^p \equiv
  n\cdot 1^p =n \pmod p$.  If $n$ is not a multiple of $p$, then we
  can then cancel $n$ to get $n^{p-1} \equiv 1 \pmod p$.
\end{solution}

\eparts
\end{problem}


%%%%%%%%%%%%%%%%%%%%%%%%%%%%%%%%%%%%%%%%%%%%%%%%%%%%%%%%%%%%%%%%%%%%%
% Problem ends here
%%%%%%%%%%%%%%%%%%%%%%%%%%%%%%%%%%%%%%%%%%%%%%%%%%%%%%%%%%%%%%%%%%%%%
\endinput

\documentclass[problem]{mcs}

\begin{pcomments}
\pcomment{TP_simpinj}
\pcomment{From class 2/27/17}
\pcomment{part(b) is TP_simpinj}
\end{pcomments}

\pkeywords{
 Functions
 Injections
 Surjections
}

%%%%%%%%%%%%%%%%%%%%%%%%%%%%%%%%%%%%%%%%%%%%%%%%%%%%%%%%%%%%%%%%%%%%%
% Problem starts here
%%%%%%%%%%%%%%%%%%%%%%%%%%%%%%%%%%%%%%%%%%%%%%%%%%%%%%%%%%%%%%%%%%%%%

\newcommand{\simpinj}{\mrelt{simpinj}}
\newcommand{\simpsurj}{\mrelt{simpsurj}}

\begin{problem}

\bparts

\ppart The definition of $A \inj B$ is that there is a total injective
relation from $A$ to $B$. Suppose the total condition was dropped, so
that we have a simpler definition $A \simpinj B$ iff there is an
injective $[\leq 1\ \text{in}]$ relation from $A$ to $B$. Describe
which sets $A$, $B$ are related by $\simpinj$ and explain why.

\begin{solution}
$A \simpinj B$ is true for all sets $A$, $B$, because the empty relation 
(with no arrows) from $A$ to $B$ has $[\leq 1\ \text{in}]$, and is therefore
injective.
\end{soution}
\examspace[2in]

\ppart Similarly, $A \surj B$ requires that there be a surjective function 
(not necessarily total) from $A$ to $B$. Suppose the function condition was
dropped, so we have a simpler definition $A \simpsurj B$ iff there is a 
surjective $[\geq 1\ \text{in}]$ relation from $A$ to $B$. Describe which sets
$A$, $B$ are related by $\simpsurj$. Explain.

\begin{solution}
If $A$ is nonempty, then $A \simpsurj B$ is true for all sets $B$,
because there is an element $e$ in $A$, and the relation that has
arrows from $e$ to every element of $B$ has $[\geq 1\ \text{in}]$ and
is therefore surjective.

Also $\emptyset \simpsurj B$ is true iff $B = \emptyset$.
\end{solution}

\eparts

\end{problem}

%%%%%%%%%%%%%%%%%%%%%%%%%%%%%%%%%%%%%%%%%%%%%%%%%%%%%%%%%%%%%%%%%%%%%
% Problem ends here
%%%%%%%%%%%%%%%%%%%%%%%%%%%%%%%%%%%%%%%%%%%%%%%%%%%%%%%%%%%%%%%%%%%%%
\endinput

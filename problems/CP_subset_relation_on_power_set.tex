\documentclass[problem]{mcs}

\begin{pcomments}
  \pcomment{CP_subset_relation_on_power_set}
  \pcomment{subsumed by CP_partial_order_on_power_set}
  \pcomment{from: S09.ps3}
\end{pcomments}

\pkeywords{
  partial_orders
  chains_and_antichains
  minimal_vs_minimum
}

%%%%%%%%%%%%%%%%%%%%%%%%%%%%%%%%%%%%%%%%%%%%%%%%%%%%%%%%%%%%%%%%%%%%%
% Problem starts here
%%%%%%%%%%%%%%%%%%%%%%%%%%%%%%%%%%%%%%%%%%%%%%%%%%%%%%%%%%%%%%%%%%%%%

\begin{problem}
The proper subset relation $\subset$ defines a partial order on the
power set $\power{\set{1,2,\dots,6}}$.

\bparts
\ppart What is the size of a maximal chain in this partial order?
Describe one.

\begin{solution}
Size 7, for example,
\[
\set{\emptyset, \set{1}, \set{1,2},
  \set{1,2,3},\set{1,2,3,4},\set{1,2,3,4,5}, \set{1,2,3,4,5,6}}.
\]
\end{solution}

\ppart Describe the largest antichain you can find in this partial order.

\begin{solution}
All the size 3 subsets of $\set{1,2,\dots, 6}$ form an antichain
  of size 20.  These are actually the largest, though proving this can be
  a challenge, especially trying to generalize to the power set of an $n$
  element set.
\end{solution}

\ppart  What are the maximal and minimal elements?  Are they maximum and
minimum?

\begin{solution}
$\emptyset$ is minimum and $\set{1,2,\dots,6}$ is maximum.
\end{solution}

\ppart Answer the previous part for the $\subset$ partial order on the set
$\power{\set{1,2,\dots, 6}}$ with $\emptyset$ removed.

\begin{solution}
Now the six size 1 subsets are minimal and there is no minimum.
$\set{1,2,\dots,6}$ is still maximum.
\end{solution}

\eparts
\end{problem}

%%%%%%%%%%%%%%%%%%%%%%%%%%%%%%%%%%%%%%%%%%%%%%%%%%%%%%%%%%%%%%%%%%%%%
% Problem ends here
%%%%%%%%%%%%%%%%%%%%%%%%%%%%%%%%%%%%%%%%%%%%%%%%%%%%%%%%%%%%%%%%%%%%%

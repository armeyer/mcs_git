\documentclass[problem]{mcs}

\begin{pcomments}
  \pcomment{FP_RSA_TF_f15}
  \pcomment{parts of FP_RSA_TF}
  \pcomment{proposed for Midterm 3, F15; ZDz 10/24/15}
\end{pcomments}

\pkeywords{
  RSA
  public key
  private key
  Euler's theorem
}

%%%%%%%%%%%%%%%%%%%%%%%%%%%%%%%%%%%%%%%%%%%%%%%%%%%%%%%%%%%%%%%%%%%%%
% Problem starts here
%%%%%%%%%%%%%%%%%%%%%%%%%%%%%%%%%%%%%%%%%%%%%%%%%%%%%%%%%%%%%%%%%%%%%

\begin{problem}
Ben Bitdiddle decided to encrypt all his data using RSA.
Unfortunately, he lost his private key.  He has been looking for it
all night, and suddenly a genie emerges from his lamp.  He offers Ben
a quantum computer that can perform exactly one procedure on large
numbers $e,d,n$.

Which of the following procedures should Ben choose to recover his data? Explain.

\begin{itemize}
%\begin{enumerate}[(a)]
\item  Find $\gcd(e, d)$.
%\item  Find the prime factorization of $n$.
\item  Determine whether $n$ is prime.
\item  Find $\rem{e^d}{n}$.
\item  Find the inverse of $e$ modulo $n$ (the inverse of $e$ in $\Zmod{n})$.
\item  Find the inverse of $e$ modulo $\phi(n)$.
%\end{enumerate}
\end{itemize}

\begin{solution}
\begin{itemize}
%\item Find the prime factorization of $n$.
\item Find the inverse of $e$ modulo $\phi(n)$.
\end{itemize}
When the RSA public key is $(e,n)$, we could find the secret key the
same way the receiver does: by calculating $\phi(n)$.  Then
the secret key is the inverse of $e$ modulo $\phi(n)$.

We already know ``fast'' procedures---with running time proportional
to a at most a little more than the square of the length of the
decimal representations of $e,d,n$---for all the other items.
\end{solution}

\examspace[2in]

\end{problem}

\endinput

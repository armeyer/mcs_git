\documentclass[handout]{mcs}

\begin{document}

\inclassproblems{3, Mon.}

%%%%%%%%%%%%%%%%%%%%%%%%%%%%%%%%%%%%%%%%%%%%%%%%%%%%%%%%%%%%%%%%%%%%%
% Problems start here
%%%%%%%%%%%%%%%%%%%%%%%%%%%%%%%%%%%%%%%%%%%%%%%%%%%%%%%%%%%%%%%%%%%%%

\pinput{CP_smallest_infinite_set}

\pinput{CP_mapping_rule}

\pinput{CP_set_product_bijection}

\pinput{CP_rationals_are_countable}


% Problems end here
%%%%%%%%%%%%%%%%%%%%%%%%%%%%%%%%%%%%%%%%%%%%%%%%%%%%%%%%%%%%%%%%%%%%%

%\inhandout{\newpage}
%\appendix
\begin{center}
{\Large Appendix}
\end{center}

Let $R: A \to B$ be a binary relation.

For any subset $X \subseteq A$
\[
XR \eqdef \set{b \in B \suchthat \exists x \in X.\, x\mrel{R}b}
\]
In other words, $XR$ is the set of endpoints of arrows that start in $X$.

\begin{lemma}[Mapping Rule] \mbox{}

\begin{enumerate}

\item If $R$ is a surjective function, then
\[
\card{A} \geq \card{B}.
\]

\begin{proof}
  Since $R$ is a function, every arrow in the diagram for $R$ comes from
  exactly one element of $A$, so the number of arrows is at most the
  number of elements in $A$.  That is, since $R$ is a function,
\[
\card{A} \geq \#\text{arrows}.
\]
Similarly, since $R$ is surjective, every element of $B$ has at least one
arrow into it, so there must be at least as many arrows as the number of
elements of $B$.  That is, since $R$ is surjective,
\[
\#\text{arrows} \geq \card{B}.
\]
Combining these inequaulties immediately implies that $\card{A} \geq \card{B}.$

\end{proof}

\item If $R$ is total and injective, then $\card{A} \leq \card{B}$.

\item If $R$ is a bijection, then $\card{A} = \card{B}$.

\end{enumerate}

\end{lemma}

\end{document}

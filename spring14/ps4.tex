\documentclass[handout]{mcs}

\begin{document}

\renewcommand{\reading}{ 
\begin{itemize}
\item Chapter~\bref{infinite_chap}.\ \emph{Infinite Sets: The
    Halting Problem}.
\item Chapter~\bref{number_theory_chap}.\ \emph{Number Theory}
  through~\bref{Turing_sec}.\ \emph{Alan Turing}.
\end{itemize}
}

\problemset{4}

\begin{staffnotes}
Lectures covered: Infinite Sets, Ch. 7; Number Theory: GCD's, Ch. 8-8.5.
\end{staffnotes}

%%%%%%%%%%%%%%%%%%%%%%%%%%%%%%%%%%%%%%%%%%%%%%%%%%%%%%%%%%%%%%%%%%%%%
% Problems start here
%%%%%%%%%%%%%%%%%%%%%%%%%%%%%%%%%%%%%%%%%%%%%%%%%%%%%%%%%%%%%%%%%%%%%

%Infinite sets
%\pinput{CP_finite_strings_of_nonneg}
%\pinput{CP_recognizable_sets} % used as a class problem in cp5f
%\pinput{FP_countable_quadratics}
%\pinput{FP_countable_sets}

\pinput{FP_infinite_binary_sequences}
%\pinput{FP_uncountable_infinite_sequences}
%\pinput{FP_uncountable_ones}

%Number theory
%\pinput{FP_divide_using_3}
%\pinput{FP_gcd_linear_combination_induction}
%\pinput{PS_gcd_properties}
\pinput{PS_gcd_three_integers_hint}
\pinput{PS_pulverizer_machine}

\end{document}

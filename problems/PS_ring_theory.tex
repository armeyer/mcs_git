\documentclass[problem]{mcs}

\newcommand{\ringz}{\ensuremath{\mathbf{0}}}
\newcommand{\ringw}{\ensuremath{\mathbf{1}}}

\begin{pcomments}
  \pcomment{PS_ring_theory}
  \pcomment{by ARM 10/3/12}
\end{pcomments}

\pkeywords{
  ring
  axioms
  commutative
  inverse
  unit
}

%%%%%%%%%%%%%%%%%%%%%%%%%%%%%%%%%%%%%%%%%%%%%%%%%%%%%%%%%%%%%%%%%%%%%
% Problem starts here
%%%%%%%%%%%%%%%%%%%%%%%%%%%%%%%%%%%%%%%%%%%%%%%%%%%%%%%%%%%%%%%%%%%%%

\begin{problem}
A commutative ring is a set $R$ of elements along with two
binary operations $\oplus$ and $\otimes$ from $R \cross R$ to $R$.
There is an element in $R$ called the zero-element, \ringz, and
another element called the unit-element, \ringw.  The operations in a
commutative ring satisfy the following \emph{ring axioms} for $r,s,t
\in R$:
\begin{align*}
(r \otimes s) \otimes t & = r \otimes (s \otimes t)
  & \text{(associativity of $\otimes$)},\\
  (r \oplus s) \oplus t & = r \oplus (s \oplus t)
  & \text{(associativity of $\oplus$)},\\
  r \oplus s & = s \oplus r
  & \text{(commutativity of $\oplus$)}\\
  r \otimes s & = s \otimes r
  & \text{(commutativity of $\otimes$)},\\
  \ringz \oplus r  & = r
  & \text{(identity for $\oplus$)},\\
  \ringw \otimes r  & = r
  & \text{(identity for $\otimes$)},\\
  \exists r' \in R.\ r \oplus r' & = \ringz
  & \text{(inverse for $\oplus$)},\\
  r \otimes (s \oplus t) & = (r \otimes s) \oplus (r \otimes t)
  & \text{(distributivity)}.
\end{align*}

\bparts

\ppart  Show that the zero-element is unique, that is,
show that if $z \in R$ has the property that
\begin{equation}\label{zoplusrr}
z \oplus r  = r,
\end{equation}
then $z = \ringz$.

\begin{solution}
Let $r'$ be an $\oplus$ inverse for $r$, that is,
\[
 r \oplus r' = \ringz.
\]

Then
\begin{align*}
z & = \ringz \oplus z
       &  \text{(identity for $\oplus$)},\\
  & = (r \oplus r') \oplus z
       & \text{(inverse for $\oplus$)},\\
  & = (r' \oplus r) \oplus z 
       & \text{(commutativity of $\oplus$)}\\
  & = r' \oplus (r \oplus z)
        & \text{(associativity of $\oplus$)},\\
  & = r' \oplus (z \oplus r)
       & \text{(commutativity of $\oplus$)}\\
  & = r' \oplus r
       & \text{(by~\eqref{zoplusrr})}\\
  & = r \oplus r'
       & \text{(commutativity of $\oplus$)}\\
  & = \ringz
         & \text{(inverse for $\oplus$)}.
\end{align*}

\end{solution}

\ppart\label{addinvuniq} Show that additive inverses are unique, that
is, show that
\begin{align}
r \oplus r_1  & = \ringz \quad \text{and}\label{ror1}\\
r \oplus r_2  & = \ringz\label{ror2}
\end{align}
implies $r_1 = r_2$.

\begin{solution}
\begin{align*}
r_1
  & = \ringz \oplus r_1
       &  \text{(identity for $\oplus$)},\\
  & = (r \oplus r_2) \oplus r_1
       & \text{(by~\eqref{ror2})},\\
  & = (r_2 \oplus r) \oplus r_1
       & \text{(commutativity of $\oplus$)}\\
  & = r_2 \oplus (r \oplus r_1)
        & \text{(associativity of $\oplus$)},\\
  & = r_2 \oplus \ringz
       & \text{(by~\eqref{ror1})},\\
  & = \ringz \oplus r_2
       & \text{(commutativity of $\oplus$)}\\
  & = r_2
       &  \text{(identity for $\oplus$)}.
\end{align*}

\end{solution}

\ppart Show that multiplicative inverses are unique, that is, show that
\begin{align*}
r \otimes r_1  & = \ringw\\
r \otimes r_2  & = \ringw
\end{align*}
implies $r_1 = r_2$.

\begin{solution}
The proof of uniqueness of additive inverses uses only properties of
addition that have corresponding properties of multiplication, and so
the additive proof in part~\eqref{addinvuniq} carries over directly to
multiplicative inverses.

\end{solution}

\eparts
\end{problem}


%%%%%%%%%%%%%%%%%%%%%%%%%%%%%%%%%%%%%%%%%%%%%%%%%%%%%%%%%%%%%%%%%%%%%
% Problem ends here
%%%%%%%%%%%%%%%%%%%%%%%%%%%%%%%%%%%%%%%%%%%%%%%%%%%%%%%%%%%%%%%%%%%%%
\endinput

\documentclass[problem]{mcs}

\begin{pcomments}
  \pcomment{PS_recursive_set_data_type}
  \pcomment{ARM 3/8/16, revised 3/9/17, revised 10/14/17}
  \pcomment{basically a rephrasing of Fundamental Thm for win-lose
    games in the text}
  \pcomment{extends PS_recursive_set_data_type_alt}
\end{pcomments}

\pkeywords{
  set_theory
  recursive_data
  structural_induction
  game
  strategy
}

%\newcommand{\recset}{\text{Recset}}

%%%%%%%%%%%%%%%%%%%%%%%%%%%%%%%%%%%%%%%%%%%%%%%%%%%%%%%%%%%%%%%%%%%%%
% Problem starts here
%%%%%%%%%%%%%%%%%%%%%%%%%%%%%%%%%%%%%%%%%%%%%%%%%%%%%%%%%%%%%%%%%%%%%

\begin{problem}
In this problem structural induction and the Foundation Axiom of set
theory provide simple proofs about some utterly infinite objects.  

\begin{definition*}
The class of \emph{recursive sets} \recset\ is defined as follows:

\inductioncase{Base case}: The empty set $\emptyset$ is a \recset.

\inductioncase{Constructor step}: If $S$ is a nonempty set of
\recset's, then $S$ is a \recset.
\end{definition*}

The two problem parts below are independent.

\bparts

\ppart Use the Foundation
axiom\inbook{\footnote{Section~\bref{ZFC_sec}}} to prove that every
set is a \recset.

\begin{solution}
\begin{proof}
Suppose a set $S$ is not in \recset.  So $S$ must have an element $N_0
\in S - \recset$, since otherwise $S$ would by definition be in
\recset.  Likewise, $N_0$ must have an element $N_1 \in N_0 -
\recset$, and $N_1$ must have an element $N_2 \in N_1 - \recset$, and
so on.  So we have an infinite sequence
\[
S \ni N_0 \ni N_1 \ni N_2 \ni \dots.
\]
Therefore the set 
\[
T \eqdef \set{S,N_0, N_1, N_2, \dots}
\]
has no member-minimal element, in violation of Foundation.
\end{proof}
\end{solution}

\ppart Every \recset\ defines a two-person game in which a player's move
consists of choosing any element of the game.  The two players
alternate moves, and a player loses when it is their turn to move and
there is no move to make.  That is, whoever moves to the empty set is
a winner, because the next player has no move.

So we think of $R \in \recset$ as the initial ``board position'' of a
game.  The player who goes first in $R$ is called the \emph{Next}
player, and the player who moves second in $R$ is called the
\emph{Previous} player.  When the Next player moves to an $S \in R$,
the game continues with the new game $S$ in which the Previous player
moves first.

Prove that for every game in \recset, either the Previous player or
the Next player has a winning strategy.\footnote{\recset\ games are
  called ``uniform'' because the two players have the \emph{same}
  objective: to leave the other player stuck with no move to make.  In
  more general games, the two players have different objectives, for
  example, one wants to maximize the final payoff and the other wants
  to minimize it (Problem~\bref{PS_VG}).}

\hint Structural induction.

\begin{solution}
Call a \recset\ a \emph{P-game} if the Previous player has a winning
strategy and an \emph{N-game} if the Next player has a winning
strategy.  We prove that every \recset\ $R$ is a P-game or and
N-game, by structural induction on the definition of \recset.

\inductioncase{Base case} ($R = \emptyset$): $R$ is an P-game since
the Next player cannot move.

\inductioncase{Constructor step}: ($R$ is a nonempty set of
\recset's): By structural induction, we may assume that every $s \in
R$ is a P-Game or an N-game.  If there is an $s \in R$ that is a
P-Game, then the Next player wins by picking $s$ (since he will be the
Previous player in $s$), and following the P-winning strategy in $s$.
This makes $R$ an N-game.

Otherwise, every $s \in R$ is an N-game, and the Previous player will
have a winning strategy to follow regardless of which game the Next
player selects.  This makes $R$ a P-game.
\end{solution}

\eparts

\end{problem}

%%%%%%%%%%%%%%%%%%%%%%%%%%%%%%%%%%%%%%%%%%%%%%%%%%%%%%%%%%%%%%%%%%%%%
% Problem ends here
%%%%%%%%%%%%%%%%%%%%%%%%%%%%%%%%%%%%%%%%%%%%%%%%%%%%%%%%%%%%%%%%%%%%%

\endinput

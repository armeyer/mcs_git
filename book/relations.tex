\chapter{Binary Relations}

\emph{Relations} are another fundamental Mathematical data type.  Equality
and ``less-than'' are the most familiar examples of Mathematical relations.
These are called \emph{binary} relations because they apply to a pair
$(a,b)$ of objects; the equality relation holds for the pair when $a=b$,
and less-than holds when $a$ and $b$ are real numbers and $a < b$.

In this chapter we'll define some basic vocabulary and properties of binary
relations.

\hyperdef{func}{rel}{\section{Binary Relations and Functions}}
Binary relations are far more general than equality or less-than.
Here's the official definition:
\begin{definition}\label{reldef}
A \emph{binary relation}, $R$, consists of a set, $A$, called
the \emph{domain} of $R$, a set, $B$, called the \emph{codomain} of $R$, and
a subset of $A \cross B$ called the \emph{graph} of $R$.
\end{definition}

Notice that Definition~\ref{reldef} is exactly the same as the definition
in section~\ref{funcsubsec} of a {\emph{function}}, except that it doesn't
require the functional condition that, for each domain element, $a$, there
is \emph{at most} one pair in the graph whose first coordinate is $a$.  So
a function is a special case of a binary relation.

A relation whose domain is $A$ and codomain is $B$ is said to be
``between $A$ and $B$'', or ``from $A$ to $B$.''  When the domain and
codomain are the same set, $A$, we simply say the relation is ``on $A$.''
It's common to use infix notation ``$a \mrel{R} b$'' to mean that the pair
$(a,b)$ is in the graph of $R$.

For example, we can define an ``in-charge of'' relation, $T$, for MIT in
Spring '09 to have domain equal to the set, $F$, of names of the faculty
and codomain equal to all the set, $N$, of subject numbers in the current
catalogue.  The graph of $T$ contains precisely the pairs of the form
\[
(\ang{\text{instructor-name}}, \ang{\text{subject-num}})
\]
such that the faculty member named $\ang{\text{instructor-name}}$ is
in charge of the subject with number $\ang{\text{subject-num}}$ in Spring '09.
So $\graph{T}$ contains pairs like

\[\begin{array}{ll}
(\texttt{A. R. Meyer}, & \texttt{6.042}),\\
(\texttt{A. R. Meyer}, & \texttt{18.062}),\\
(\texttt{A. R. Meyer}, & \texttt{6.844}),\\
(\texttt{T. Leighton}, & \texttt{6.042}),\\
(\texttt{T. Leighton}, & \texttt{18.062}),\\
(\texttt{G, Verghese}, & \texttt{6.011}),\\
(\texttt{G, Verghese}, & \texttt{6.UAT}),\\
(\texttt{G. Verghese}, & \texttt{6.881})\\
(\texttt{G. Verghese}, & \texttt{6.882})\\
(\texttt{T. Eng},      & \texttt{6.UAT})\\
(\texttt{J. Guttag},  & \texttt{6.00})\\
\qquad \vdots
\end{array}\]

This is a surprisingly complicated relation: Meyer is in charge of
subjects with three numbers.  Leighton is also in charge of subjects with
two of these three numbers ---because the same subject, Mathematics for
Computer Science, has two numbers: 6.042 and 18.062, and Meyer and
Leighton are co-in-charge of the subject.  Verghese is in-charge of even
more subjects numbers (around 20), since as Department Education Officer,
he is in charge of whole blocks of special subject numbers.  Some
subjects, like 6.844 and 6.00 have only one person in-charge.  Some
faculty, like Guttag, are in charge of only one subject number, and no
one else is co-in-charge of his subject, 6.00.

Some subjects in the codomain, $N$, do not appear in the list ---that is,
they are not an element of any of the pairs in the graph of $T$; these are
the Fall term only subjects.  Similarly, there are faculty in the domain,
$F$, who do not appear in the list because all their in-charge subjects
are Fall term only.

\section{Images and Inverse Images}

The faculty in charge of 6.UAT in Spring '09 can be found by taking the
pairs of the form
\[
(\ang{\text{instructor-name}}, 6.UAT)
\]
in the graph of the teaching relation, $T$, and then just listing the left
hand sides of these pairs; these turn out to be just Eng and Verghese.
Similarly to find out who is in-charge of introductory course 6 subjects
this term, take all the pairs of the form $(\ang{\text{instructor-name}},
6.0\dots)$ and list the left hand sides of these pairs.  This works because
the set $D$, of introductory course 6 subject numbers are the ones that
start with ``6.0''.  For example, from the part of the graph of $T$ shown
above, we can see that Meyer, Leighton, Verghese, and Guttag are in-charge
of introductory subjects this term.

These are all examples of taking an \emph{inverse image} of a set under a
relation.  If $R$ is a binary relation from $A$ to $B$, and $X$ is any
set, define the \emph{inverse image} of $X$ under $R$, written simply as
$RX$ to be the set elements of $A$ that are related to something in $X$.

For example, $TD$, the inverse image of the set $D$ under the relation,
$T$, is the set of all faculty members in-charge of introductory course 6
subjects in Spring '09.  Notice that in inverse image notation, $D$ gets
written to the right of $T$ because, to find the faculty members in $TD$,
we're looking at the right hand side of pairs in the graph of $T$ for
subject numbers in $D$.

Here's a concise definition of the inverse image of a set $X$ under
a relation, $R$:
\[
RX \eqdef \set{a \in A \suchthat aRx \text{ for some } x \in X}.
\]

Similarly, the \emph{image} of a set $Y$ under $R$, written $YR$, is the
set of elements of the codomain, $B$, that are related to some element in
$Y$, namely,
\[
YR \eqdef \set{b \in B \suchthat yRb \text{ for some } y \in Y}.
\]

So, $\set{\text{A. Meyer}}T$ gives the subject numbers that Meyer is in
charge of in Spring '09.  In fact, $\set{\text{A. Meyer}}T = \set{6.042,
  18.062, 6.844}$.  Since the domain, $F$, is the set of all in-charge
faculty, $FT$ is exactly the set of \emph{all} Spring '09 subjects being
taught.  Similarly, $TN$ is the set of people in-charge of a Spring '09
subject.

It gets interesting when we write composite expressions mixing images,
inverse images and set operations.  For example, $(TD)T$ is the set of
Spring '09 subjects that have people in-charge who also are in-charge of
introductory subjects.  So $(TD)T - D$ are the advanced subjects with
someone in-charge who is also in-charge of an introductory subject.
Similarly, $TD \intersect T(N-D)$ is the set of faculty teaching both an
introductory \emph{and} an advanced subject in Spring '09.

\textbf{Warning:} When $R$ happens to be a function, the pointwise
application, $R(Y)$, of $R$ to a set $Y$ described in
Section~\ref{funcsubsec} is exactly the same as the image of $Y$ under
$R$.  That means that when $R$ is a function, $R(Y) = YR$ ---\emph{not}
$RY$.  Both notations are common in Math texts, so you'll have to live
with the fact that they clash.  Sorry about that.

\section{Surjective and Injective Relations}

There are a few properties of relations that will be useful when we take
up the topic of counting because they imply certain relations between the
\emph{sizes} of domains and codomains.  We say a binary relation $R : A
\to B$ is:

\begin{itemize}

\item \term{total} when every element of $A$ is assigned to some element of
  $B$; more concisely, $R$ is total iff $A=RB$.

\item \term{surjective} when every element of $B$ is mapped to \textit{at
least once}\footnote{
The names ``surjective'' and ``injective'' are unmemorable and
nondescriptive.  Some authors use the term \term{onto} for surjective and
\emph{one-to-one} for injective, which are shorter but arguably no more
memorable.}; more concisely, $R$ is surjective iff $AR=B$.

\item \term{injective} if every element of $B$ is mapped to \textit{at
most once}, and

\item \term{bijective} if $R$ is total, surjective, and injective
  \emph{function}.

\end{itemize}

Note that the definition of total for functions is a special case of the
above definition for binary relations.

If $R$ is a binary relation from $A$ to $B$, we define $AR$ to to be the
\emph{range} of $R$.  So a relation is surjective iff its range equals its
codomain.

\subsection{Relation Diagrams}
We can explain all these properties of a relation $R:A \to B$ in terms of
a diagram where all the elements of the domain, $A$, appear in one column
(a very long one if $A$ is infinite) and all the elements of the codomain,
$B$, appear in another column, and we draw an arrow from a point $a$ in
the first column to a point $b$ in the second column when $a$ is related
to $b$ by $R$.  For example, here are diagrams for two functions:

\begin{center}
\begin{tabular}{ccc}

\unitlength = 2pt
\begin{picture}(50,60)(-10,-5)
\thinlines
\put(-5,50){\makebox(0,0){$A$}}
  \put(35,50){\makebox(0,0){$B$}}
\put(-5,40){\makebox(0,0){a}}
  \put(0,40){\vector(1,0){28}}
  \put(35,40){\makebox(0,0){1}}
\put(-5,30){\makebox(0,0){b}}
  \put(0,30){\vector(3,-1){28}}
  \put(35,30){\makebox(0,0){2}}
\put(-5,20){\makebox(0,0){c}}
  \put(0,20){\vector(3,-1){28}}
  \put(35,20){\makebox(0,0){3}}
\put(-5,10){\makebox(0,0){d}}
  \put(0,10){\vector(3,2){28}}
  \put(35,10){\makebox(0,0){4}}
\put(-5,0){\makebox(0,0){e}}
  \put(0,0){\vector(3,2){28}}
\end{picture}

& \hspace{0.5in} &

\unitlength = 2pt
\begin{picture}(50,60)(-10,-5)
\thinlines
\put(-5,50){\makebox(0,0){$A$}}
  \put(35,50){\makebox(0,0){$B$}}
\put(-5,40){\makebox(0,0){a}}
  \put(0,40){\vector(1,0){28}}
  \put(35,40){\makebox(0,0){1}}
\put(-5,30){\makebox(0,0){b}}
  \put(0,30){\vector(3,-1){28}}
  \put(35,30){\makebox(0,0){2}}
\put(-5,20){\makebox(0,0){c}}
  \put(0,20){\vector(3,-2){28}}
  \put(35,20){\makebox(0,0){3}}
\put(-5,10){\makebox(0,0){d}}
  \put(0,10){\vector(3,2){28}}
  \put(35,10){\makebox(0,0){4}}
\put(35,0){\makebox(0,0){5}}
\end{picture}

\end{tabular}
\end{center}

Here is what the definitions say about such pictures:
\begin{itemize}

\item ``$R$ is a function'' means that every point in the domain column,
  $A$, has \emph{at most one arrow out of it}.

\item ``$R$ is total'' means that \emph{every} point in the $A$ column has
  \emph{at least one arrow out of it}.  So if $R$ is a function, being
  total really means every point in the $A$ column has
  \emph{exactly one arrow out of it}.

\item ``$R$ is surjective'' means that \emph{every} point in the codomain
  column, $B$, has \emph{at least one arrow into it}.

\item ``$R$ is injective'' means that every point in the codomain column,
  $B$, has \emph{at most one arrow into it}.

\item ``$R$ is bijective'' means that \emph{every} point in the $A$ column
      has exactly one arrow out of it, and \emph{every} point in the $B$ column
      has exactly one arrow into it.

\end{itemize}

So in the diagrams above, the relation on the left is a total, surjective
function (every element in the $A$ column has exactly one arrow out, and
every element in the $B$ column has at least one arrow in), but not
injective (element 3 has two arrows going into it).  The relation on the
right is a total, injective function (every element in the $A$ column has
exactly one arrow out, and every element in the $B$ column has at most one
arrow in), but not surjective (element 4 has no arrow going into it).

Notice that the arrows go in a diagram for $R$ precisely correspond to the
pairs in the graph of $R$.  But $\graph{R}$ does not determine by itself
whether $R$ is total or surjective; we also need to know what the domain
is to determine if $R$ is total, and we need to know the codomain to tell
if it's surjective.
\begin{example}
  The function defined by the formula $1/x^2$ is total if its domain is
  $\reals^+$ but partial if its domain is some set of real numbers
  including 0.  It is bijective if its domain and codomain are both
  $\reals^+$, but neither injective nor surjective if its domain and
  codomain are both $\reals$.
\end{example}


\section{The Mapping Rule}

The relational properties above are useful in figuring out the relative
sizes of domains and codomains.  If $A$ is a finite set, we let $\card{A}$
be its size, that is, the number of elements in $A$.

For example, if $R$ is a function, then every arrow in the diagram comes
from exactly one element of $A$, so the number of arrows is at most the
number of elements in $A$.  That is, if $R$ is a function, then
\[
\card{A} \geq \#\text{arrows} = \card{\graph{R}}.
\]
Similarly, if $R$ is surjective, then every element of $B$ has an arrow
into it, so there must at least as many arrows in the diagram as the size
of $B$.  That is,
\[
\#\text{arrows} \geq card{B}.
\]
Combining these inequalities implies that if $R$ is a surjective function,
then $\card{A} \geq \card{B}$.  This is the first rule about relational
properties and domain sizes.  The following lemma lists this and two
simlar rules (with similar proofs left to you).

\begin{lemma}\label{mapruldef}
\hyperdef{mapping-rule}{lemma}{[Mapping Rule]} \mbox{}
Let $R: A \to B$ be a binary relation.

\begin{itemize}

\item If $R$ is a surjective function, then $\card{A} \geq \card{B}$.

\item If $R$ is total and injective, then $\card{A} \leq \card{B}$.

\item If $R$ is a bijection, then $\card{A} = \card{B}$.

\end{itemize}
\end{lemma}

\section{The sizes of infinite sets}

A finite set, $A$, may have no elements (the empty set), or one element,
or two elements,\dots or any nonnegative integer number of elements.  The
\emph{size}, $\card{A}$, of $A$ is defined to be the number of elements in
$A$.  The Mapping Rule implies that the size of a finite set, $A$, is
greater than or equal to the size of another finite set, $B$, \emph{iff}
there is a surjective function, $f:A \to B$.  Likewise, they have the same size iff
there is a bijection from $A$ to $B$.  This idea generalizes in an
interesting way to infinite sets:
\begin{definition}\label{bigger}
Let $A,B$ be sets.  Then
\begin{enumerate}
\item $A$ is \emph{as big as} $B$ iff there is a surjective
  function from $A$ to $B$,

\item $A$ is \emph{the same size} as $B$ iff there is a bijection from $A$
  to $B$.

\item $A$ is \emph{strictly} bigger than $B$ iff $A$ is as big as $B$,
  but $B$ is not as big as $A$.
\end{enumerate}
\end{definition}

\textcolor{red}{\textbf{Warning}}: We haven't, and won't, define what the
``size'' of an infinite is.  The definition of infinite ``sizes'' is
cumbersome and technical, and we can get by just fine without it.  All we
need are the ``as big as'' and ``same size'' relations between sets.

But there's something else to \textcolor{red}{watch out for}, because
we've taken an ordinary phrase and given it a purely technical meaning
---something Mathematicians do all the time.  There are a lot of
properties you'd expect an ``as big as'' relation to have, but you can't
assume \emph{any} of them until they're proved.  Of course most of the
``as big as'' and ``same size'' properties of finite sets actually do
carry over to infinite sets, but some important ones don't, as we're about
to show.

The first thing to confirm is that Definition~\ref{bigger} does match the
usual one for finite sets.  Since the Mapping Rule motivated
Definition~\ref{bigger} in the first place, this should be no surprise.
Namely,
\begin{lemma}\label{finbig}
For finite sets, $A,B$,
\[
A \text{ is as big as } B \qiff \card{A} \geq \card{B},
\]
and
\[
A \text{ is the same size as } B \qiff \card{A} = \card{B}. 
\]
\end{lemma}
This Lemma is just a rephrasing of the surjection and bijection parts of
the Mapping Rule, so it follows immediately.

Several further familiar properties of the ``as big as'' and ``same size''
relations on finite sets carry over exactly to infinite sets:
\begin{lemma}\label{translem}
 For any sets, $A,B,C$,
\begin{enumerate}

\item \label{bigtrans}
$A \text{ is as big as } B \text{ and }B\text{ is as big as } C, \text{
implies} A \text{ is as big as } C.$


\item \label{sametrans}
$A \text{ is the same size as } B \text{ and }B\text{ is the same size as }
C,\text{ implies } A \text{ is the same size as } C.$

\item\label{sameABA}
$A \text{ is the same size as } B \text{ implies } B\text{ is the same size
  as } A.$
\end{enumerate}
\end{lemma}

Lemma~\ref{translem}.\ref{bigtrans} and~\ref{translem}.\ref{sametrans}
follow immediately from the fact that compositions of surjections are
surjections, and likewise for bijections, and part~\ref{sameABA}.\ follows
from the fact that the inverse of a bijection is a bijection.  These facts
make a good exercise:

\begin{notesproblem}
  Let $f:A \to B$ and $g: B \to C$ be functions and $h:A \to C$ be their
  composition, namely, $h(a) \eqdef g(f(a))$ for all $a \in A$.
\bparts
  \ppart Prove that if $f$ and $g$ are surjections, then so is $h$.

  \ppart Prove that if $f$ and $g$ are bijections, then so is $h$.

  \ppart If $f$ is a bijection, then define $f':B \to A$ so that
  \[
  f'(b) \eqdef\text{ the unique } a \in A \text{ such that } f(a)=b.
  \]
  Prove that $f'$ is a bijection.  (The function $f'$ is called the
  \emph{inverse} of $f$.  The notation $f^{-1}$ is often used for the
  inverse of $f$.)
\eparts
\end{notesproblem}

Another familiar property of finite sets carries over to infinite sets,
but this time it's not so obvious:
\begin{theorem} [Schr\"oder-Bernstein] For any sets $A,B$, if $A$ is as
  big as $B$, and $B$ is as big as $A$, then $A$ is the same size as
  $B$.
\end{theorem}
The Schr\"oder-Bernstein Theorem is actually pretty technical: it says
that if there are surjections (which need not be bijections), $f:A\to B$
and $g:B \to A$, then there is a bijection $e:A \to B$.  The idea is to
construct $e$ from parts of both $f$ and $g$, but we won't go into the
details here.

\subsubsection{Infinity is different}

A basic property of finite sets that does \emph{not} carry over to
infinite sets is that adding something new makes a set bigger.  That is,
if $A$ is a finite set and $b \notin A$, then $\card{A \union \set{b}} =
\card{A}+1$, and so $A$ and $A \union \set{b}$ are not the same size.  But
if $A$ is infinite, then these two sets \emph{are} the same size!

\begin{lemma}\label{AUb}
  Let $A$ be a set and $b \notin A$.  Then $A$ is infinite iff $A$ is the
  same size as $A \union \set{b}$.
\end{lemma}
\begin{proof}
  Since $A$ is \emph{not} the same size as $A \union \set{b}$ when $A$ is
  finite, we only have to show that $A \union \set{b}$ \emph{is} the same
  size as $A$ when $A$ is infinite.

That is, we have to find a bijection between $A \union \set{b}$ and $A$
when $A$ is infinite.  Here's how: since $A$ is infinite, it certainly has
at least one element; call it $a_0$.  But since $A$ is infinite, it has at
least two elements, and one of them must not be equal to $a_0$; call this
new element $a_1$.  But since $A$ is infinite, it has at least three
elements, one of which must not equal $a_0$ or $a_1$; call this new
element $a_2$.  Continuing in the way, we conclude that there is an
infinite sequence $a_0,a_1,a_2,\dots,a_n,\dots$ of different elements of
$A$.  Now it's easy to define a bijection $e: A \union \set{b} \to A$:
\begin{align*}
e(b) & \eqdef a_0,\\
e(a_n) & \eqdef a_{n+1}  &\text{ for } n \in \naturals,\\
e(a) & \eqdef a & \text{ for } a \in A - \set{b,a_0,a_1,\dots}.
\end{align*}
\end{proof}

\begin{notesproblem}
Prove that every infinite set is as big as the set, $\naturals$, of
nonnegative integers.
\hint The proof of Lemma~\ref{AUb}.
\end{notesproblem}


\subsection{Infinities in Computer Science}

We've run into a lot of Computer Science students who wonder why they need
to learn all this abstract theory about infinite sets, and this is a good
question.

Sometimes these students overstate their doubts by claiming that only
finite sets come up in Computer Science, but it ain't so: the standard
programming data types of integers, floating point numbers, strings, for
example, each have a potentially infinite number of data items of that
type, although each individual datum of these types of course is finite.
But higher order types, for example, the type of string-to-integer
procedures, not only has an infinite number of data items, but each
procedure datum generally behaves differently on different inputs, so that
a single datum may embody an infinite number of behaviors.

\endinput

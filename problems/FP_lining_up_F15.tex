\documentclass[problem]{mcs}

\begin{pcomments}
  \pcomment{FP_lining_up_F15}
  \pcomment{by Rich F09}
\end{pcomments}

\pkeywords{
 predicate_calculus
 counting
}

%%%%%%%%%%%%%%%%%%%%%%%%%%%%%%%%%%%%%%%%%%%%%%%%%%%%%%%%%%%%%%%%%%%%%
% Problem starts here
%%%%%%%%%%%%%%%%%%%%%%%%%%%%%%%%%%%%%%%%%%%%%%%%%%%%%%%%%%%%%%%%%%%%%

\begin{problem}

%\textbf{First-order Logic and Counting}
  There are 10 students $A, B, \dots, J$ who will be lined up left to
  right according to the some rules below.

  Translate each of the following rules into predicate formulas with the
  set of 10 students as the domain of discourse.  The only predicates
  you may use are
\begin{itemize}
\item equality and,
\item $F(x,y)$, meaning that ``$x$ is to the left of $y$.''
For example, in the lineup ``CDA'', both $F(C,A)$ and $F(C,D)$ are true.
\end{itemize}

\bparts

\ppart
Rule I: Student A must not be rightmost.


\exambox{3in}{0.5in}{0.3in}

\examspace[0.5in]

\begin{solution}
  \[ \exists x. F(A,x) \]
\end{solution}

\ppart
Rule II: Student B must be adjacent to C (directly to the left or right of C).

\exambox{3in}{0.5in}{0.3in}

\examspace[0.5in]

\begin{solution}
  \[ \forall x.
       ((x \neq B) \QAND (x \neq C)) \QIMPLIES
       (F(x,B) \QIFF F(x,C))
  \]
\end{solution}

\ppart
Rule III: Student D is always second.

\exambox{3in}{0.5in}{0.3in}


\begin{solution}
  \[ \exists x. 
       F(x, D) \QAND
       (\forall y. 
         ((y \neq D) \QAND (y \neq x)) \QIMPLIES 
         F(D,y))
  \]
\end{solution}
\examspace[0.5in]

\ppart
How many possible lineups are there that satisfy all three of these rules?

\exambox{2.0in}{0.5in}{0.5in}


\begin{solution}
   We may consider the number of ways to place $A, B, C, D$ by cases:
   
   Suppose that $B$ or $C$ is last. There are still $2$ ways to place $B$ and $C$.
   A may then take any of the remaining 7 spots. So the total is $2 \cdot 7$.
   
   Suppose that neither $B$ or $C$ is last.  There are $6$ (they may be anywhere
   between 3rd to 9th) places to place them together and $2$ ways to order them.
   $A$ may be placed in any of the remaining $6$ places (discounting the 
   last place and the places B and C occupy).  The total is therefore 
   $6 \cdot 2 \cdot 6$.
   
   We know that, in both cases, there are $6!$ ways to place the remaining 6 
   people into the remaining spots, so the total number of possible lineups is:
   \[
     6! \cdot (6 \cdot 2 \cdot 6 + 2 \cdot 7) =
      6! \cdot 86 =
      61920
   \]
\end{solution}


\ppart
How many possible lineups are there that satisfy at least one of these rules?

% seems like too much work for an inclusion-exclusion problem?

\examspace[3in]
\begin{solution}
   
   \# lineups satisfying rule I: 
   There are 9 places to place A, and $9!$ ways to place the remaining people.
   
   \[ 9 \cdot 9! \]
   
   \# lineups satisfying rule II:
   We can consider $B$ and $C$ to be one person, so there are $9!$ ways to place them, and
   2 ways to order $B$ and $C$.
   
   \[ 2 \cdot 9! \]
   
   \# lineups satisfying rule III:
   There are $9!$ ways to order the remaining 9 people:
   
   \[ 9! \]
   
   \# lineups satisfying rule I and II:
   If B or C is last, there are $8!$ ways to order the remaining people.
   Otherwise, there are 8 places to place B and C; 7 places to place A; and
   $7!$ ways to order the remaining numbers.
   
   \[ 2 \cdot 8! + 2 \cdot 8 \cdot 7 \cdot 7! \]
   
   \# lineups satisfying rule I and III:
   There are 8 places to put $A$ and $8!$ ways to order the remaining people.
   
   \[ 8 \cdot 8! \]
   
   \# lineups satisfying rule II and III:
   There are 7 places to place $B$ and $C$ together, and $7!$ ways to order
   the remaining people.
   
   \[ 2 \cdot 7 \cdot 7! \]
   
   \# lineups satisfying rule I, II and III:
   From the previous part.
   
   \[ 6! \cdot 86 \]
   
   So the total is, using inclusion-exclusion:
   \[ 
   9 \cdot 9! + 2 \cdot 9! + 9! - (2 \cdot 8! + 2 \cdot 8 \cdot 7 \cdot 7! + 8 \cdot 8! + 2 \cdot 7 \cdot 7!) + 6! \cdot 86 =
   4692 \cdot 6! =
   3378240
   \]
\end{solution}

\eparts
\end{problem}

\endinput

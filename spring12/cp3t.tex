\documentclass[handout]{mcs}

\begin{document}

\inclassproblems{3, Tue.}

%%%%%%%%%%%%%%%%%%%%%%%%%%%%%%%%%%%%%%%%%%%%%%%%%%%%%%%%%%%%%%%%%%%%%
% Problems start here
%%%%%%%%%%%%%%%%%%%%%%%%%%%%%%%%%%%%%%%%%%%%%%%%%%%%%%%%%%%%%%%%%%%%%

\pinput{CP_proving_basic_set_identity}
\pinput{CP_subset_take_away}

\instatements{\newpage}
%\pinput{PS_binary_relations_on_a_set}  renamed TP; stupid problem

%\pinput{CP_mapping_rule}

\pinput{CP_web_search}

\pinput{CP_distributive-law-for-sets-by-WOP}

\iffalse
\inhandout{
\examspace[1.0in]
\textbox{
\textboxheader{Arrow Properties}

\begin{definition*}
A binary relation, $R$ is
\begin{itemize}

\item is a \term{function} when it has the $[\le 1\ \text{arrow
    \textbf{out}}]$ property.

\item is \term{surjective} when it has the $[\ge 1\ \text{arrows
    \textbf{in}}]$ property.  That is, every point in the righthand,
     codomain column has at least one arrow pointing to it.

\item is \term{total} when it has the $[\ge 1\ \text{arrows
       \textbf{out}}]$ property.

\item is \term{injective} when it has the $[\le 1\ \text{arrow
    \textbf{in}}]$ property.

\item is \term{bijective} when it has both the $[=1\ \text{arrow
    \textbf{out}}]$ and the $[=1\ \text{arrow \textbf{in}}]$ property.
\end{itemize}
\end{definition*}
}}\fi



%%%%%%%%%%%%%%%%%%%%%%%%%%%%%%%%%%%%%%%%%%%%%%%%%%%%%%%%%%%%%%%%%%%%%
% Problems end here
%%%%%%%%%%%%%%%%%%%%%%%%%%%%%%%%%%%%%%%%%%%%%%%%%%%%%%%%%%%%%%%%%%%%%
\end{document}


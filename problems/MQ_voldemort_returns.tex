\documentclass[problem]{mcs}

\begin{pcomments}
  \pcomment{MQ_voldemort_returns}
  \pcomment{similar to CP_conditional_prob_says_so_bug}
  \pcomment{Peter Huang and ARM 5/5/12}
\end{pcomments}

\pkeywords{
  conditional_probability
  tree_diagram
  four-step_method
}

%%%%%%%%%%%%%%%%%%%%%%%%%%%%%%%%%%%%%%%%%%%%%%%%%%%%%%%%%%%%%%%%%%%%%
% Problem starts here
%%%%%%%%%%%%%%%%%%%%%%%%%%%%%%%%%%%%%%%%%%%%%%%%%%%%%%%%%%%%%%%%%%%%%

\begin{problem}
A guard is going to release exactly two of the three prisoners,
Sauron, Voldemort, and Bunny Foo Foo, and he's equally likely to
release any set of two prisoners.
\bparts

\ppart\label{Vreleaseprob} What is the probability that Voldemort will
be released?

\exambox{0.5in}{0.5in}{0.2in}

\eparts

The guard will truthfully tell Voldemort the name of one of the
prisoners to be released.  We're interested in the following events:
\begin{description}

\item[\ \ $V$:] \emph{V}oldemort is released.

\item[\textbf{``$F$''}:] The guard tells Voldemort that \emph{F}oo Foo will be released.

\item[\textbf{``$S$''}:] The guard tells Voldemort that \emph{S}auron will be released.

\end{description}

The guard has two rules for choosing whom he names:
\begin{itemize}

\item never say that Voldemort will be released,

\item if both Foo Foo and Sauron are getting released, say ``Foo Foo.''

\end{itemize}

\bparts

\iffalse

\ppart Draw a tree to represent the  sample space, clearly indicating
which outcomes in the the events $V$, \textbf{``$F$''}, and
\textbf{``$S$''}.

\examspace[3.5in]

\begin{solution}

\begin{center}
\begin{picture}(360,175)(0,-40)
%\put(0,-40){\dashbox(360,175){}} % bounding box
\put(0,60){\line(1,1){60}}
\put(0,60){\line(1,0){60}}
\put(0,60){\line(1,-1){60}}
\put(30,-10){\makebox(0,0){released}}
\put(60,120){\line(1,0){60}}
\put(60,60){\line(1,0){60}}
\put(60,0){\line(1,0){60}}
\put(90,-25){\makebox(0,0){guard says}}
\put(11,90){\makebox(0,0){$F,V$}}
\put(40,68){\makebox(0,0){$F,S$}}
\put(11,30){\makebox(0,0){$V,S$}}
\put(52,96){\makebox(0,0){$1/3$}}
\put(40,50){\makebox(0,0){$1/3$}}
\put(52,24){\makebox(0,0){$1/3$}}
\put(90,128){\makebox(0,0){$F$}}
\put(90,68){\makebox(0,0){$F$}}
\put(90,-10){\makebox(0,0){$S$}}
\put(90,110){\makebox(0,0){$1$}}
\put(90,50){\makebox(0,0){$1$}}
\put(90,8){\makebox(0,0){$1$}}
\put(150,120){\makebox(0,0){$1/3$}}
\put(150,60){\makebox(0,0){$1/3$}}
\put(150,0){\makebox(0,0){$1/3$}}
\put(150,-20){\makebox(0,0){prob.}}
\put(210,120){\makebox(0,0){$\times$}}
\put(210,68){\makebox(0,0){$\times$}}
\put(210,0){\makebox(0,0){}}
\put(210,-25){\makebox(0,0){\shortstack{guard says\\"Foo-foo"}}}
\put(270,120){\makebox(0,0){}}
\put(270,68){\makebox(0,0){}}
\put(270,0){\makebox(0,0){$\times$}}
\put(270,-25){\makebox(0,0){\shortstack{guard says\\"Sauron"}}}
\put(330,120){\makebox(0,0){$\times$}}
\put(330,68){\makebox(0,0){}}
\put(330,0){\makebox(0,0){$\times$}}
\put(330,-25){\makebox(0,0){\shortstack{Voldemort\\released}}}
\end{picture}
\end{center}

\end{solution}
\end{editingnotes}

\bparts
\fi

\problempart\label{VgiveFF} What is $\prcond{V}{\text{``$F$''}}$?

\exambox{0.5in}{0.5in}{0.2in}

\begin{solution}
\[
\frac{1}{2}\ ,
\]
because
\[
\prcond{V}{\textbf{``$F$''}} = \frac{ \pr{V \intersect \textbf{``$F$''} }}{ \pr{\textbf{``$F$''}} }
       = \frac{ 1/3 }{ 2/3 } = \frac{1}{2}\ .
\]
\end{solution}

\problempart\label{VgiveSa}
What is $\prcond{V}{\textbf{``$S$''}}$?

\exambox{0.5in}{0.5in}{0in}
\examspace[0.2in]

\begin{solution}
\[
\prcond{V}{\text{``S''}}
   = \frac{ \pr{V \QAND \text{``S''}} }{\pr{\text{``S''}} }
  =\frac{ 1/3 }{ 1/3 } = 1.
\]
\end{solution}

\problempart Show how to use the Law of Total Probability to combine
your answers to parts~\eqref{VgiveFF} and~\eqref{VgiveSa} to verify
that the result matches the answer to part~\eqref{Vreleaseprob}.

\examspace[2in]

\begin{solution}
\begin{align*}
\frac{2}{3}
 & = \prob{V}
      & \text{(part~\eqref{Vreleaseprob})}\\
 & = \prcond{V}{\text{``F''}}
              \cdot \prob{\text{``F''}} + \prcond{V}{\text{``S''}} \cdot \prob{\text{``S''}}
      & \text{(Total Probability)}\\
 & = \frac{1}{2} \cdot \frac{2}{3} + 1 \cdot \frac{1}{3}
       & \text{(parts~\eqref{VgiveFF} and~\eqref{VgiveSa})}\\
 & = \frac{2}{3}
\end{align*}

\end{solution}
\end{problemparts}

\end{problem}


%%%%%%%%%%%%%%%%%%%%%%%%%%%%%%%%%%%%%%%%%%%%%%%%%%%%%%%%%%%%%%%%%%%%%
% Problem ends here
%%%%%%%%%%%%%%%%%%%%%%%%%%%%%%%%%%%%%%%%%%%%%%%%%%%%%%%%%%%%%%%%%%%%%

\endinput

\documentclass[problem]{mcs}

\begin{pcomments}
  \pcomment{CP_domain_of_discourse}
  \pcomment{from: S09.cp2t, S10.cp3w}
%  \pcomment{}
%  \pcomment{}
\end{pcomments}

\pkeywords{
  quantifiers
  predicate_calculus
  domain_of_discourse
  logical_formula
}

%%%%%%%%%%%%%%%%%%%%%%%%%%%%%%%%%%%%%%%%%%%%%%%%%%%%%%%%%%%%%%%%%%%%%
% Problem starts here
%%%%%%%%%%%%%%%%%%%%%%%%%%%%%%%%%%%%%%%%%%%%%%%%%%%%%%%%%%%%%%%%%%%%%

\begin{problem}
For each of the logical formulas, indicate whether or not it is true
when the domain of discourse is $\nngint$, (the nonnegative integers
0, 1, 2, \dots), $\integers$ (the integers), $\rationals$ (the
rationals), $\reals$ (the real numbers), and $\complexes$ (the complex
numbers).  Add a brief explanation to the few cases that merit one.

\begin{align*}
          & \exists x.\,  x^2 =  2\\
\forall x. & \exists y.\, x^2  =  y\\
\forall y. & \exists x.\, x^2  =  y\\
\forall x\neq 0. & \exists y.\,  xy  =  1\\
\exists x. & \exists y.\,  x + 2y  =  2\ \QAND\ 2 x + 4 y = 5
\end{align*}

\begin{staffnotes}
The few brief explanations for entries below are sufficient.

Intervene if teams start to go overboard with adding explanations
(unlikely).  After the problem has been team-approved (team check on
their board), you can challenge a team member to provide an omitted
explanation.  If they had sufficient explanations (common), I like to
challenge a team member with a ``meta''-question, ``Which was the
hardest entry to fill in, and why?''
\end{staffnotes}


\begin{solution}
\[
\begin{array}{llllll}
Statement & \nngint & \integers & \rationals & \reals & \complexes\\
\exists x.\, x^2=2 & \false & \false & \false & \true\ (x=\sqrt{2}) & \true\\
\forall x.\, \exists y.\, x^2 = y & \true & \true & \true & \true\ (y=x^2)& \true\\
\forall y.\, \exists x.\, x^2=y & \false & \false & \false & \false\ (\text{take }y<0) & \true\\
\forall x\neq 0.\, \exists y.\, xy=1 & \false & \false & \true & \true\ (y=1/x) & \true\\
\exists x.\, \exists y.\, x + 2y = 2\ \QAND\ 2 x + 4 y = 5 & \false & \false & \false & \false  & \false
\end{array}
\]

\end{solution}

\end{problem}

%%%%%%%%%%%%%%%%%%%%%%%%%%%%%%%%%%%%%%%%%%%%%%%%%%%%%%%%%%%%%%%%%%%%%
% Problem ends here
%%%%%%%%%%%%%%%%%%%%%%%%%%%%%%%%%%%%%%%%%%%%%%%%%%%%%%%%%%%%%%%%%%%%%

\endinput


\documentclass[problem]{mcs}

\begin{pcomments}
  \pcomment{FP_eligible_token_state_machine}
  \pcomment{first parts of PS_eligible_token_state_machine}
  \pcomment{S18, final, S16.mid2}
  \pcomment{ARM 3/10/16, revised 5/21/18}
  \pcomment{leaves unresolved the reachability of eligible states that
    are not big enough}
\end{pcomments}

\pkeywords{
  state_machine
  invariant
  preserved_invariant
  reachable
  induction
  remainder
  derived_variable
  constant
  increasing
}

%%%%%%%%%%%%%%%%%%%%%%%%%%%%%%%%%%%%%%%%%%%%%%%%%%%%%%%%%%%%%%%%%%%%%
% Problem starts here
%%%%%%%%%%%%%%%%%%%%%%%%%%%%%%%%%%%%%%%%%%%%%%%%%%%%%%%%%%%%%%%%%%%%%

\begin{problem}

\emph{Token replacing-1-3} is a single player game using a set of tokens,
each colored black or white.  In each move, a player can replace a
black token with three white tokens, or replace a white token with
three black tokens.  We can model this game as a state machine whose
states are pairs $(b,w)$ of nonnegative integers, where $b$ is the
number of black tokens and $w$ the number of white ones.

The game has two possible start states: $(5,4)$ or $(4,3)$.

We call a state $(b,w)$ \emph{eligible} when
\begin{align}
\rem{b-w}{4} & = 1
\end{align}
This problem examines the connection between eligible states and
states that are \emph{reachable} from either of the possible start
states.

\bparts

\ppart Prove that the predicate
\[
P(b,w)\eqdef \rem{b-w}{4} = 1,
\]
is a Preserved Invariant.

\examspace[1.0in]

\begin{solution}
$P$ is a preserved invariant, because if state $(b,w)$ transitions to
  $(b',w')$, then by definition $(b',w') = (b+3,w-1)\text{ or }(b-1,
  w+3)$.  But
\[
\rem{(b+3)-(w-1)}{4} = \rem{(b-1)-(w+3)}{4} = \rem{b-w}{4}.
\]
So if $\rem{b-w}{4} = 1$ then $\rem{b'-w'}{4} = 1$.
\end{solution} 

\ppart Conlude that every reachable state is eligible.

\examspace[0.6in]

\begin{solution}
$P$ is a preserved invariant that is obviously true for both start
  states, so by the Invariant Principle, $P$ is true for all reachable
  states.
\end{solution}

\ppart Prove that the eligible state $(3,2)$ is not reachable.  \hint
$b+w$ is strictly increasing.

\examspace[0.8in]
    
\begin{solution}
Each transition increases $b+w$ by $2$, so $b+w$ is a strictly
increasing derived variable.  

Since $b+w$ for both start states is $\geq 7$, no state with $b+w <7$
is reachable.  In particular, $3+2 < 7$, so $(3,2)$ is not reachable.
\end{solution}

\eparts

\medskip

Say that a state $(b,w)$ is \emph{big enough} when $\min(b,w) > 2$.
We now prove the
\begin{claim*}
Every big enough eligible state is reachable.
\end{claim*}

To begin, observe the
\begin{fact*}
If $\max(b,w) \leq 5$ and $(b,w)$ is eligible and big enough, then
$(b,w)$ is reachable.
\end{fact*}

Among the nine states with $3 \leq \min(b,w)$ and
    $\max(b,w) \leq 5$, it is easy to check that only the start state
    $(5,4)$ satisfies the preserved invariant.  Now the Fact holds
    because by definition the start state is reachable.

\bparts

\ppart Prove the Claim by strong induction using the induction
hypothesis:
\begin{equation}\label{indhypP}
Q(n) \eqdef \forall (b,w). [b+w = n \QAND\ (b,w)\ \text{is big enough
    and eligible}] \QIMPLIES (b,w)\ \text{is reachable}.
\end{equation}
You may assume the Fact given above.

\hint $Q(n-2) \QIMPLIES Q(n)$.

\examspace[3.0in]

\begin{solution}
\begin{proof}
To prove $Q(n)$, we must show that if $b+w = n$, and $(b,w)$ is big
enough and eligible, then $(b,w)$ is reachable.

The Fact confirms that this is true for $\max(b,w) \leq 5$, so we may
assume that $\max(b,w) \geq 6$.

Suppose $b \geq w$.  Since $(b,w)$ is big enough, $\min(b-3,w+1) \geq
3$, so $(b-3,w+1)$ is big enough.  Also, $\rem{(b-3,w+1)}{4} =
\rem{(b,w)}{4}$ and $(b,w)$ is eligible, so $(b-3,w+1)$ is also
eligible.  But $(b-3)+(w+1) = n-2 > 0$, and the induction hypothesis
$Q(n-2)$ implies that $(b-3,w+1)$ is reachable.  Since $(b-3,w+1)$
transitions to $(b,w)$ in one step, we conclude that $(b,w)$ is
reachable.

Alternatively, if $w \geq b$, then, by the same reasoning, $(b+1,
w-3)$ is reachable, and therefore $(b,w)$ is reachable.

So in any case, $(b,w)$ is reachable, as required.
\end{proof}
\end{solution}

\eparts

\end{problem}

\endinput

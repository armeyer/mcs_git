\documentclass[problem]{mcs}

\begin{pcomments}
  \pcomment{CP_de_Bruijn_graphs}
  \pcomment{from: S09.cp7r}
  \pcomment{commented out in S09, last part needs some revision}
\end{pcomments}

\pkeywords{
  digraphs
  Euler_circuit
  binary
  strings
}

%%%%%%%%%%%%%%%%%%%%%%%%%%%%%%%%%%%%%%%%%%%%%%%%%%%%%%%%%%%%%%%%%%%%%
% Problem starts here
%%%%%%%%%%%%%%%%%%%%%%%%%%%%%%%%%%%%%%%%%%%%%%%%%%%%%%%%%%%%%%%%%%%%%

\begin{problem}

  A $3$-bit string is a string made up of $3$ characters, each a $0$
  or a $1$.  Suppose you'd like to write out, in one string, all eight
  of the 3-bit strings in any convenient order.  For example, if you
  wrote out the $3$-bit strings in the usual order starting with
  000 001 010\dots, you could concatenate them together to get a
  length $3\cdot 8 = 24$ string that started 000001010\dots.

  But you can get a shorter string containing all eight $3$-bit
  strings by starting with 00010\dots.  Now $000$ is present as bits
  $1$ through $3$, and $001$ is present as bits $2$ through $4$, and
  $010$ is present as bits $3$ through $5$, \dots.

\bparts

\ppart\label{gf3str} Say a string is \emph{3-good} if it contains every
$3$-bit string as $3$ consecutive bits somewhere in it.  Find a 3-good
string of length 10, and explain why this is the minimum length
for any string that is 3-good.

\begin{solution}
The string $0001110100$ is a length 10 string that is 3-good.  You
can't do better: there must be two bits to start and each additional
bit can yield at most one new $3$-bit string.
\end{solution}

\ppart\label{all10edges} Explain how any walk that includes every edge
in the graph shown in Figure~\ref{3-debruijn} determines a string that
is 3-good.  Find the walk in this graph that determines your
3-good string from part~\eqref{gf3str}.

\begin{solution}
A string can be built up from any walk by starting with the $2$ bits
in the vertex at the start of the walk and successively adding the bit
that labels the edge to the end of the string being built.  If the
walk includes every edge, then any string $b_1 b_2 b_3$ will appear as
a substring when the edge $\diredge{b_1 b_2}{b_2 b_3}$ appears in the
walk.

In particular, the string $0001110100$ is determined by the walk that
goes through the following sequence of edges:
\[
\diredge{00}{00} \diredge{00}{01} \diredge{01}{11} \diredge{11}{11}
\diredge{11}{10} \diredge{10}{01} \diredge{01}{10} \diredge{10}{00}.
\]

\iffalse
$00,00,01,11,11,10,01,10,00$
\fi

\end{solution}

\ppart Explain why a walk in the graph of Figure~\ref{3-debruijn} that
includes every every edge \emph{exactly once} provides a
minimum-length 3-good string.\footnote{The 3-good strings explained
  here generalize to $n$-good strings for $n \ge 3$.  They were
  studied by the great Dutch mathematician/logician Nicolaas de
  Bruijn, \index{de Bruijn, Nicolaas} and are known as \emph{de Bruijn
    sequences}.  de Bruijn died in February, 2012 at the age of 94.}

\begin{solution}
  Since there are $8$ edges, the string determined by the walk will be
  of length $10$, which is the minimum possible as observed in
  part~\eqref{gf3str}.  Since the walk includes every edge, it will
  determine a 3-good string by part~\eqref{all10edges}.
\end{solution}

\ppart Generalize the 2-bit graph to a $k$-bit digraph $B_k$ for $k
\ge 2$, where $\vertices{B_k} \eqdef \set{0,1}^k$, and any walk
through $B_k$ that contains every edge exactly once determines a
minimum length $(k+1)$-good bit-string.\footnote{Problem~\bref{PS_directed_Euler_circuits}
  explains why such ``Eulerian'' paths exist.  \iffalse that if the in-degree of
  every vertex of a digraph is equal to its out-degree, and there are
  paths between any two vertices, then there is a closed walk that
  includes every edge exactly once.  So the graph $B_k$ implies that
  there always is a length $2^{k+1} + k$ bit-string in which every
  length $k+1$ bit-string appears as a substring. \fi }

What is this minimum length?

Define the transitions of $B_k$.  Verify that the in-degree of each
vertex is the same as its out-degree and that there is a positive
length path from any vertex to any other vertex (including itself) of
length at most $k$.

\begin{solution}
$2^{k+1} + k$.

A bit-string of length $n$ has exactly $n-k$ locations where a length
$k+1$ subsequence can begin.  Since there are $2^{k+1}$ length-$(k+1)$
bit-strings, the minimum length $n$ of any $(k+1)$-good bit-string
must satisfy $n-k \ge 2^{k+1}$, so the mininum length is $2^{k+1} +
k$.  This is exactly the length of the bit-string that would be
determined by a walk containing all $2 \cdot 2^k$ edges in
$\edges{B_k}$.
\[
\edges{B_k} \eqdef \set{\diredge{ax}{xb} \suchthat x \in
  \set{0,1}^{k-1} \QAND a,b \in \set{0,1}}
\]

If $y \in \set{0,1}^k$, then $y = ax$ and $y = zb$ for unique strings
$x,z \in \set{0,1}^{k-1}$ and bits $a,b \in \set{0,1}$.  Then by
definition of $\edges{B_k}$, there are exactly two edges out of $y$,
one going to $x0$ and the other to $x1$, so $\outdegr{y} = 2$.
Likewise, there are exactly two edges into $y$, one from $0z$ and the
other from $1z$, so $\indegr{y} = 2$.

To get from vertex $b_1b_2\dots b_k$ to $c_1c_2\dots c_k$ with a
length $k$ walk, proceed as follows:
\begin{align*}
b_1 b_2 b_3\dots b_k & \to b_2 b_3 \dots b_k c_1 \to b_3 \dots b_k c_1 c_2\\
                & \to \dots \to b_{k} c_1 c_2\dots c_{k-1} \to c_1 c_2\dots c_k.
\end{align*}

Since a walk of length $k$ exists, a path of length at most $k$ can be
obtained by removing the cycles in the walk.
\end{solution}

\eparts

\begin{figure}
\graphic[height=3in]{debruijn}
\caption{The 2-bit graph.}
\label{3-debruijn}
\end{figure}


\end{problem}

%%%%%%%%%%%%%%%%%%%%%%%%%%%%%%%%%%%%%%%%%%%%%%%%%%%%%%%%%%%%%%%%%%%%%
% Problem ends here
%%%%%%%%%%%%%%%%%%%%%%%%%%%%%%%%%%%%%%%%%%%%%%%%%%%%%%%%%%%%%%%%%%%%%

\endinput

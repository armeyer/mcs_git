\documentclass[problem]{mcs}

\begin{pcomments}
  \pcomment{from: S09.cp6t}
  \pcomment{remove hard ref to thm in notes}
  \pcomment{was commented out in S09}
\end{pcomments}

\pkeywords{
  trees
  graphs
  connectivity
  cycles
}

%%%%%%%%%%%%%%%%%%%%%%%%%%%%%%%%%%%%%%%%%%%%%%%%%%%%%%%%%%%%%%%%%%%%%
% Problem starts here
%%%%%%%%%%%%%%%%%%%%%%%%%%%%%%%%%%%%%%%%%%%%%%%%%%%%%%%%%%%%%%%%%%%%%

\begin{problem}
Prove that a graph is a tree iff it has a unique simple path between any
two vertices.

\solution{The proof in Week 6 Notes of Theorem 4.1.2 shows that in a tree
there are unique simple paths between any two vertices:

\begin{quote}
There is at least one path, and hence one simple path, between every pair
of vertices, because the tree is connected.  Suppose that there are two
different simple paths between vertices $u$ and $v$.  Beginning at $u$,
let $x$ be the first vertex where the paths diverge, and let $y$ be the
next vertex they share, illustrated in the figure below.  Then there are
two simple paths from $x$ to $y$ with no common edges, and this defines a
simple cycle.  This is a contradiction, since trees are acyclic.
Therefore, there is exactly one simple path between every pair of
vertices.

\mfigure{!}{1in}{figures/unique-path}
\end{quote}

Conversely, suppose we have a graph, $G$, with unique paths.  Now $G$ is
connected since there is a path between any two vertices.  So we need only
show that $G$ has no simple cycles.  But if there was a simple cycle in
$G$, there are two paths between any two vertices on the cycle (going one
way around the cycle or the other way around), a violation of uniqueness.
So $G$ must cannot have any simple cycles.}
\end{problem}

%%%%%%%%%%%%%%%%%%%%%%%%%%%%%%%%%%%%%%%%%%%%%%%%%%%%%%%%%%%%%%%%%%%%%
% Problem ends here
%%%%%%%%%%%%%%%%%%%%%%%%%%%%%%%%%%%%%%%%%%%%%%%%%%%%%%%%%%%%%%%%%%%%%

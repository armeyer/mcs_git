\documentclass[problem]{mcs}

\begin{pcomments}
  \pcomment{CP_conquering_the_galaxy}
  \pcomment{from: S09.cp4m, S14.cp7m}
\end{pcomments}

\pkeywords{
  partial_order
  scheduling
  chain
  antichain
  critical_path
}

\newcommand{\Jay}{Lisa}
\newcommand{\Rongrong}{Annie}

%%%%%%%%%%%%%%%%%%%%%%%%%%%%%%%%%%%%%%%%%%%%%%%%%%%%%%%%%%%%%%%%%%%%%
% Problem starts here
%%%%%%%%%%%%%%%%%%%%%%%%%%%%%%%%%%%%%%%%%%%%%%%%%%%%%%%%%%%%%%%%%%%%%

\begin{problem}
A pair of Math for Computer Science Teaching Assistants, \Jay\ and
\Rongrong, have decided to devote some of their spare time this term
to establishing dominion over the entire galaxy.  Recognizing this as
an ambitious project, they worked out the following table of tasks on
the back of \Rongrong's copy of the lecture notes.

\begin{enumerate}
\item \textbf{Devise a logo} and cool imperial theme music - 8 days.
\item \textbf{Build a fleet} of Hyperwarp Stardestroyers out of eating
  paraphernalia swiped from Lobdell - 18 days.
\item \textbf{Seize control} of the United Nations - 9 days, after task \#1.
\item \textbf{Get shots} for \Jay's cat, Tailspin - 11 days, after task \#1.
\item \textbf{Open a Starbucks chain} for the army to get their caffeine - 10 
days, after task \#3.
\item \textbf{Train an army} of elite interstellar warriors by dragging
people to see \emph{The Phantom Menace} dozens of times - 4 days, after
tasks \#3, \#4, and \#5.
\item \textbf{Launch the fleet} of Stardestroyers, crush all sentient
alien species, and establish a Galactic Empire - 6 days, after tasks \#2 and
\#6.
\item \textbf{Defeat Microsoft} - 8 days, after tasks \#2 and \#6.
\end{enumerate}

We picture this information in Figure~\ref{fig:tasks} below by drawing a
point for each task, and labelling it with the name and weight of the
task.  An edge between two points indicates that the task for the higher
point must be completed before beginning the task for the lower one.
    \begin{figure}[htbp]
    \begin{center}
%    \unitlength=0.047pt
    \unitlength=0.04pt
    \input{ps4-1.latex}
    \end{center}
    \caption{Graph representing the task precedence constraints.}
    \label{fig:tasks}
    \end{figure}

\bparts

\ppart Give some valid order in which the tasks might be completed.

\begin{solution}
We can easily find several of them. The most natural one is valid, too:
\#1, \#2, \#3, \#4, \#5, \#6, \#7, \#8.

\end{solution}
\eparts

\Jay\  and \Rongrong\  want to complete all these tasks in the
shortest possible time. However, they have agreed on some constraining
work rules.
\begin{itemize}

\item Only one person can be assigned to a particular task; they cannot
work together on a single task.

\item Once a person is assigned to a task, that person must work
exclusively on the assignment until it is completed.  So, for example,
\Jay\  cannot work on building a fleet for a few days, run to get shots
for Tailspin, and then return to building the fleet.

\end{itemize}

% For each task, the table above lists the time required for completion
% and which other tasks must be finished before the given task is begun.

\bparts

\ppart \Jay\  and \Rongrong\  want to know how long conquering the
galaxy will take.  \Rongrong\  suggests dividing the total number of days of
work by the number of workers, which is two.  What lower bound on the time
to conquer the galaxy does this give, and why might the actual time
required be greater?

\begin{solution}
\begin{equation*}
\frac{8 + 18 + 9 + 11 + 10 + 4 + 6 + 8}{2} = 37 \text{ days}
\end{equation*}

If working together and interrupting work on a task were permitted, then
this answer would be correct.  However, the rules may prevent \Jay\  and
\Rongrong\  from both working all the time.  For example, suppose the only task
was building the fleet.  It will take 18 days, not 18/2 days, to complete,
because only one person can work on it and the other must sit idle.
\end{solution}

\ppart \Jay\  proposes a different method for determining the
duration of their project.  She suggests looking at the duration of
the \emph{\idx{critical path}}, the most time-consuming sequence of tasks
such that each depends on the one before.  What lower bound does this
give, and why might it also be too low?

\begin{solution}
The longest sequence of tasks is devising a logo (8 days),
seizing the U.N. (9 days), opening a Starbucks (10 days), training
the army (4 days), and then defeating Microsoft (8 days).  Since these
tasks must be done sequentially, galactic conquest will require at
least 39 days.

If there were enough workers, this answer would be correct; however, with
only two workers, \Jay\  and \Rongrong\  may be unable to make progress on the
critical path every day.  For example, suppose there were only four
tasks: devise logo, build fleet, seize control, get shots.  Now the
critical path consists of two critical tasks: devise logo, get shots,
which take 19 days.  But to get through this path in 19 days, some worker
must be working on a critical task at all times for the 19 days.  This
leaves only one worker free to complete building the fleet and seizing
control, which will take at least 27 days.  So in fact, 27 days is the
minimum time for two workers to complete these four tasks.
\end{solution}

\ppart What is the minimum number of days that \Jay\  and \Rongrong\ 
need to conquer the galaxy?  No proof is required.

\begin{solution}
40 days.  Tasks could be divided as follows:

\Rongrong: \#1 (days 1-8), \#3 (days 9-17), \#4 (days 18-28), \#8 (days
33-40).

\Jay: \#2 (days 1-18), \#5 (days 19-28), \#6 (days 29-32), \#7 (days
33-38).  

It takes some care to verify that 40 days is the best you can do.  If
someone comes up with a simple proof of this, tell the course staff.

\end{solution}
\eparts
\end{problem} 

%%%%%%%%%%%%%%%%%%%%%%%%%%%%%%%%%%%%%%%%%%%%%%%%%%%%%%%%%%%%%%%%%%%%%
% Problem ends here
%%%%%%%%%%%%%%%%%%%%%%%%%%%%%%%%%%%%%%%%%%%%%%%%%%%%%%%%%%%%%%%%%%%%%

\endinput

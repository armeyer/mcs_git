\documentclass[problem]{mcs}

\begin{pcomments}
  \pcomment{TP_composition_of_jections}
  \pcomment{renamed from PS_composition_of_jections}
  \pcomment{from: S09 ln3 notesproblem; mostly subsumed by CP_surj_relation}
  \pcomment{was called PS_composition-of-jections}
  \pcomment{overlaps CP_surj_relation}
\end{pcomments}

\pkeywords{
  composition
  surjection
  bijection
  inverse}

%%%%%%%%%%%%%%%%%%%%%%%%%%%%%%%%%%%%%%%%%%%%%%%%%%%%%%%%%%%%%%%%%%%%%
% Problem starts here
%%%%%%%%%%%%%%%%%%%%%%%%%%%%%%%%%%%%%%%%%%%%%%%%%%%%%%%%%%%%%%%%%%%%%

\begin{problem}
  Let $f:A \to B$ and $g: B \to C$ be functions and $h:A \to C$ be their
  composition, namely, $h(a) \eqdef g(f(a))$ for all $a \in A$.
\bparts
  \ppart Prove that if $f$ and $g$ are surjections, then so is $h$.

  \ppart Prove that if $f$ and $g$ are bijections, then so is $h$.

  \ppart If $f$ is a bijection, then so is $\inv{f}$.

\eparts

\begin{solution}
TBA
\end{solution}

\end{problem}

%%%%%%%%%%%%%%%%%%%%%%%%%%%%%%%%%%%%%%%%%%%%%%%%%%%%%%%%%%%%%%%%%%%%%
% Problem ends here
%%%%%%%%%%%%%%%%%%%%%%%%%%%%%%%%%%%%%%%%%%%%%%%%%%%%%%%%%%%%%%%%%%%%%

\endinput

\documentclass[problem]{mcs}

\begin{pcomments}
  \pcomment{FP_counting_various_s18}
  \pcomment{combined from FP_counting_various_f15 and PS_alphabet for s18 mid4}

  \pcomment{Comments copied over from FP_counting_various_f15:}
  \pcomment{S08.ps9, revised by ARM 11/14/09}
  \pcomment{soln to 2n-into-pairs part revised 5/12/13 ARM}
  \pcomment{subsumes PS_misc_counting}
\end{pcomments}

\pkeywords{
  binomial_coefficient
  factorial
  string
  digits
}

%%%%%%%%%%%%%%%%%%%%%%%%%%%%%%%%%%%%%%%%%%%%%%%%%%%%%%%%%%%%%%%%%%%%%
% Problem starts here
%%%%%%%%%%%%%%%%%%%%%%%%%%%%%%%%%%%%%%%%%%%%%%%%%%%%%%%%%%%%%%%%%%%%%

\begin{problem}
Answer the following questions with a number or a simple formula
involving factorials and binomial coefficients.  Briefly explain your answers.

\bparts

%%% PART %%%
\ppart
There is a robot that steps between integer positions in 2-dimensional
space.  Each step of the robot increments one coordinate and leaves the
other one unchanged. Now, the robot got special gear that allows
him to also make a limited number of ``diagonal'' steps, in which both
coordinates are incremented.

We would like to calculate the number of paths the robot can follow going
from the origin $(0,0)$ to the position $(M, N)$ if he makes exactly $K$ diagonal
steps.  Assume that $K \leq \min(M,N)$.

\begin{enumerate}[(i)]
\item
Let 0 correspond to a diagonal step, 1 to a step along the first
coordinate, and 2 to a step along the second coordinate.  Demonstrate
a set of strings of 0's, 1's, and 2's that has a bijection to the set
of possible robot paths, and describe this bijection.

\examspace[1.5in]

\begin{solution}
A set of strings of length $M+N-K$ with exactly $K$ 0's, $M-K$ 1's and
$N-K$ 2's.  The robot will make exactly $K$ diagonal steps, which
implies that it will make exactly $M-K$ steps along the first
coordinate, and $N-K$ steps along the second coordinate.

Let 0 correspond to a diagonal step, 1 to a step along the first
coordinate, and 2 to a step along the second coordinate.  The robot
will make exactly $K$ diagonal steps, which implies that it will make
exactly $M-K$ steps along the first coordinate, and $N-K$ steps along
the second coordinate.  Note that different sequences of steps will
result in different paths.  Therefore, a string of length $M+N-K$ with
$K$ 0's, $M-K$ 1's and $N-K$ 2's maps to a unique path obtained by
taking the steps that correspond to digits in the string (e.g., from
left to right).
\end{solution}

\item
How many possible paths can the robot take?

\examspace[1in]

\begin{solution}
\[
\binom{M+N-K}{M-K,N-K,K} = (M+N-K)! / ((M-K)!(N-K)!(K)!),
\]
which is the number of strings of $K$ 0's, $M-K$ 1's and $N-K$ 2's.
\end{solution}
\end{enumerate}

\ppart How many ways are there to order the 26 letters of the alphabet
(with each letter used exactly once) so that no two of the vowels
\STR{a}, \STR{e}, \STR{i}, \STR{o}, \STR{u} appear consecutively and
the last letter in the ordering is not a vowel?

\hint  Every vowel appears to the left of a consonant.

\examspace[1in]

\begin{solution}
The constraint on where vowels can appear is equivalent to the
requirement that every vowel appears to the left of a consonant.  So given
a sequence of the 21 consonants, there are $\binom{21}{5}$ positions where
the 5 vowels can be placed.  After determining such a placement, we can
reorder the consonants and vowels in any order.  Thus, the number is:
\[
\binom{21}{5}\cdot 21! \cdot 5!.
\]
\end{solution}

\eparts

\end{problem}

%%%%%%%%%%%%%%%%%%%%%%%%%%%%%%%%%%%%%%%%%%%%%%%%%%%%%%%%%%%%%%%%%%%%%
% Problem ends here
%%%%%%%%%%%%%%%%%%%%%%%%%%%%%%%%%%%%%%%%%%%%%%%%%%%%%%%%%%%%%%%%%%%%%

\endinput

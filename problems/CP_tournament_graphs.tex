\documentclass[problem]{mcs}

\begin{pcomments}
  \pcomment{from: S09.cp7r}
  \pcomment{Could use some revision (one of the parts was redundant in the version used in class).}
\end{pcomments}

\pkeywords{
  digraphs
  relations
  relational_properties
  partial_orders
  cycles
}

%%%%%%%%%%%%%%%%%%%%%%%%%%%%%%%%%%%%%%%%%%%%%%%%%%%%%%%%%%%%%%%%%%%%%
% Problem starts here
%%%%%%%%%%%%%%%%%%%%%%%%%%%%%%%%%%%%%%%%%%%%%%%%%%%%%%%%%%%%%%%%%%%%%

\begin{problem} {\bf Tournament Graphs}

  \newcommand{\beats}{\!\rightarrow\!}

  Consider a $n$-player round-robin tournament where every pair of 
  distinct players compete in a single game that doesn't allow for a tie. 
  We can model the results of such a tournament using either a ``beats'' 
  relation or a digraph (called a \emph{tournament graph}).  The players 
  are represented by vertices and there is an edge $x \beats y$ if $x$ 
  beat $y$.

  \bparts

  \ppart Explain why there can be no cycles of length 2 or less.

  \solution{There are no self-loops in a tournament graph since no player 
  plays himself. Since every pair competes exactly once and there are no 
  ties, there are no cycles\footnote{Since there are no self-loops, 
  any cycle of length 2 or 3 must necessarily be simple.} of length 2 
  (it cannot be that $x$ beats $y$ and $y$ beats $x$).}

  \ppart Explain whether the ``beats'' relation for a tournament 
  graph is always/sometimes/never
  \begin{itemize}
  \item symmetric,
  \item asymmetric,
  \item reflexive,
  \item irreflexive,
  \item transitive.
  \end{itemize}

  \solution{No self-loops implies the relation is irreflexive.  It 
  is also asymmetric since it is irreflexive and for every pair of 
  distinct players, exactly one game is played and results in a win
  for one of the players.  Some tournament graphs represent 
  transitive relations and others don't.}

  \ppart Show that a tournament graph represents a total order iff
  there are no cycles of length 3.

  \solution{As observed in the previous part, the ``beats'' relation 
  whose graph is a tournament is asymmetric and irreflexive. Since 
  every pair of players is comparable, the relation is a total order 
  iff it is transitive.

  ``Beats'' is transitive iff for any players $x$, $y$ and $z$, 
  $x \beats y$ and $y \beats z$ implies that $x \beats z$ 
  (and consequently that there is no edge $z \beats x$).  Therefore, 
  ``beats'' is transitive iff there are no cycles of length 3.}

  \eparts
\end{problem}

%%%%%%%%%%%%%%%%%%%%%%%%%%%%%%%%%%%%%%%%%%%%%%%%%%%%%%%%%%%%%%%%%%%%%
% Problem ends here
%%%%%%%%%%%%%%%%%%%%%%%%%%%%%%%%%%%%%%%%%%%%%%%%%%%%%%%%%%%%%%%%%%%%%

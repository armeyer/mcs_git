\documentclass[problem]{mcs}

\begin{pcomments}
  \pcomment{MQ_identify_jections_S15}
  \pcomment{by: Shen, Elizabeth, S15}
\end{pcomments}

\pkeywords{
  bijection
  injection
  surjection
  domain
}

%%%%%%%%%%%%%%%%%%%%%%%%%%%%%%%%%%%%%%%%%%%%%%%%
% Problem starts here %%%%%%%%%%%%%%%%%%%%%%%%%%
%%%%%%%%%%%%%%%%%%%%%%%%%%%%%%%%%%%%%%%%%%%%%%%%

\begin{problem}

%LOOKING TO ADD HORIZONTAL LINES FOR STUDENT ANSWERS -emshen 2/18/15 3AM 

Formulas defining functions from integers to integers are listed
below.  For each function, indicate whether it is
\begin{itemize}
\item \textbf{B}, a bijection $[ =1\text{ out}, =1 \text{ in}]$,
\item \textbf{S}, surjection $[ \geq 1 \text{ in}]$, but not a bijection,
\item \textbf{I}, an injection $[ \leq 1 \text{ in}]$, but not a bijection,
\item \textbf{N}, neither an injection nor a surjection.
\end{itemize}

\inbook{Briefly explain your answers.}

\bparts

\ppart $a(x) \eqdef x^2$.\hfill\examrule

\begin{solution}

\textbf{N}. $a(-1) = a(1)$, so $f$ is not an injection.  $a$ is not a
surjection because $-1 \notin \range{a}$.
\end{solution}

%\examspace[0.5in]

\ppart $b(x) \eqdef x+2$.\hfill\examrule
\begin{solution}
\textbf{B}.  The function $x-2$ reverses the arrows to get back where
you started.
\end{solution}

%\examspace[0.5in]

\ppart  $c(x) \eqdef 2x$.\hfill\examrule

\begin{solution}
\textbf{I}. $c$ is not surjective because there no odd number in in
$\range{c}$.
\end{solution}

%\examspace[0.5in]

\ppart $d(x) \eqdef -x$.\hfill\examrule

\begin{solution}
\textbf{B}.  Every integer has one and only one negative.
\end{solution}

%\examspace[0.5in]

\ppart $e(x) \eqdef \floor{x/2}$, that is, the quotient of $x$ divided by 2.\hfill\examrule

%\examspace[0.5in]

\begin{solution}
\textbf{S}.  $e$ is not injective because $h(4) = h(5) = 2$.
\end{solution}

\iffalse  %trivial answers: let domain and codomain be empty, or have 1 element or be {2, 4} and {1,2}....
          %Also problem statement was revised to say all functions go $\integers \to \integers$

\examspace[0.5in]

\ppart
Give an example of a domain and codomain pairing that would make
$j(x)$ a bijection.

\textbf{Domain:} \examspace[.5in]

\textbf{Codomain:} \examspace[.5in]

\begin{solution}
One possible solution is the domain of even integers and the codomain of  all integers. 
\end{solution}
\fi

\eparts

\end{problem}

\endinput

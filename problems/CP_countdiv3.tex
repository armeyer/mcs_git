\documentclass[problem]{mcs}

\begin{pcomments}
  \pcomment{CP_countdiv3}
  \pcomment{ARM from 'divisibility' MITx, April 26, 2018}
\end{pcomments}

\pkeywords{
  counting
 }

%%%%%%%%%%%%%%%%%%%%%%%%%%%%%%%%%%%%%%%%%%%%%%%%%%%%%%%%%%%%%%%%%%%%%
% Problem starts here
%%%%%%%%%%%%%%%%%%%%%%%%%%%%%%%%%%%%%%%%%%%%%%%%%%%%%%%%%%%%%%%%%%%%%

\begin{problem}
How many ways are there to pick three distinct numbers from the
integer interval $\Zintv{1}{15}$ such that the sum of the numbers is
divisible by three?

\begin{solution}
\textbf{155}

Divide each of the fifteen numbers in the interval by three.  Five of
them leave remainder zero, five remainder one, and five remainder two.
For the sum of three numbers in the interval to be divisible by three,
they must all leave the same remainder, or each must leave a different
remainder.  The are $\binom{5}{3}$ ways to pick three numbers with any
given remainder, and there are three remainders, for a total of $3
\cdot \binom{5}{3}$ ways to pick three numbers with the same
remainder.  To pick three numbers in different classes, there are 5
choices for each class and 3 classes, yielding $5^3$ possible picks.
So the total number of choices is
\[
3 \cdot \binom{5}{3} + 5^3 = 155.
\]
\end{solution}
\end{problem}

\endinput

\documentclass[problem]{mcs}

\begin{pcomments}
  \pcomment{FP_order_modn}
  \pcomment{first part of FP_Fermat_primes}
  \pcomment{ARM 5/19/17}
\end{pcomments}

\pkeywords{
  number_theory
  modular_arithmetic
  Z_n
}

%%%%%%%%%%%%%%%%%%%%%%%%%%%%%%%%%%%%%%%%%%%%%%%%%%%%%%%%%%%%%%%%%%%%%
% Problem starts here
%%%%%%%%%%%%%%%%%%%%%%%%%%%%%%%%%%%%%%%%%%%%%%%%%%%%%%%%%%%%%%%%%%%%%

\begin{problem}
The \emph{order} of $k \in \Zmod{n}$ is the smallest positive $m$ such
that $k^m = 1 \inzmod{n}$.

Prove that
\[
k^m = 1 \inzmod{n}\ \QIMPLIES\ \ordmod{k}{n} \divides m.
\]

\hint Take the remainder of $m$ divided by the order.

\examspace[2.5in]

\begin{solution}
Let $d$ be the order of $k$, and $q,r$ be the quotient and remainder
of $m$ divided by $d$, so
\[
m = qd + r
\]
where $r \in [0,k)$.  Now,
\begin{align*}
1 = k^m & = k^{qk+r} \inzmod{n}\\
        & = \paren{k^d}^q \cdot k^r\\
        & = 1^q \cdot k^r  \inzmod{n}\\
        & = k^r.
\end{align*}
But $r \in \Zintvco{0}{d}$, and since $d$ is the smallest positive power of $k$
  equal to 1 in $\Zmod{n}$, we must have $r = 0$, that is, $d \divides
  m$.
\end{solution}

\end{problem}

%%%%%%%%%%%%%%%%%%%%%%%%%%%%%%%%%%%%%%%%%%%%%%%%%%%%%%%%%%%%%%%%%%%%%
% Problem ends here
%%%%%%%%%%%%%%%%%%%%%%%%%%%%%%%%%%%%%%%%%%%%%%%%%%%%%%%%%%%%%%%%%%%%%

\endinput

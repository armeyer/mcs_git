\documentclass[problem]{mcs}

\begin{pcomments}
  \pcomment{PS_eligible_token_state_machine}
  \pcomment{extension of FP_eligible_token_state_machine from S18, final, S16.mid2}
  \pcomment{ARM revised 5/21/18}
  \pcomment{last two parts by Richard Spence in 2016 not in exams}
  \pcomment{leaves unresolved the reachability of eligible states with min b,w <= 2.}
\end{pcomments}

\pkeywords{
  state_machine
  invariant
  preserved_invariant
  reachable
  induction
  remainder
  derived_variable
  constant
  increasing
}

%%%%%%%%%%%%%%%%%%%%%%%%%%%%%%%%%%%%%%%%%%%%%%%%%%%%%%%%%%%%%%%%%%%%%
% Problem starts here
%%%%%%%%%%%%%%%%%%%%%%%%%%%%%%%%%%%%%%%%%%%%%%%%%%%%%%%%%%%%%%%%%%%%%

\newcommand{\BE}{\text{BE}}

\begin{problem}

\emph{Token replacing-1-3} is a single player game using a set of tokens,
each colored black or white.  In each move, a player can replace a
black token with three white tokens, or replace a white token with
three black tokens.  We can model this game as a state machine whose
states are pairs $(b,w)$ of nonnegative integers, where $b$ is the
number of black tokens and $w$ the number of white ones.

The game has two possible start states: $(5,4)$ or $(4,3)$.

We call a state $(b,w)$ \emph{eligible} when
\begin{align}
\rem{b-w}{4} & = 1
\end{align}
This problem examines the connection between eligible states and
states that are \emph{reachable} from either of the possible start
states.

\bparts

\ppart Prove that the predicate
\[
P(b,w)\eqdef [\rem{b-w}{4} = 1],
\]
is a preserved invariant of this state machine.

\examspace[1.0in]

\begin{solution}
$P$ is a preserved invariant, because if state $(b,w)$ transitions to
  $(b',w')$, then by definition $(b',w') = (b+3,w-1)\text{ or }(b-1,
  w+3)$.  But
\[
\rem{(b+3)-(w-1)}{4} = \rem{(b-1)-(w+3)}{4} = \rem{b-w}{4}.
\]
So if $\rem{b-w}{4} = 1$ then $\rem{b'-w'}{4} = 1$.
\end{solution} 

\ppart\label{reachisel} Conlude that every reachable state is eligible.

\examspace[0.6in]

\begin{solution}
$P$ is a preserved invariant that is true for both start
  states, so by the Invariant Principle, $P$ is true for all reachable
  states.
\end{solution}

\ppart Prove that the eligible state $(3,2)$ is not reachable.  \hint
$b+w$ is strictly increasing.

\examspace[0.8in]
    
\begin{solution}
Each transition increases $b+w$ by $2$, so $b+w$ is a strictly
increasing derived variable.  Since $b+w$ for both start states is
$\geq 7$, no state with $b+w <7$ is reachable.  In particular, $3+2 <
7$, so $(3,2)$ is not reachable.
\end{solution}

\eparts

\medskip

Say that a state $(b,w)$ is \emph{big enough} when $\min(b,w) > 2$.
We now prove the
\begin{claim*}
A big enough state is reachable iff it is eligible.
\end{claim*}

Reachable states must be eligible by part~\eqref{reachisel}, so we
just have to show that eligibility implies reachability for big enough
states.  Let
\[
\BE(b,w) \eqdef (b,w)\ \text{is big enough and eligible}.
\]

\bparts

\ppart\label{fact5} Prove that if $\max(b,w) \leq 5$, then $\BE(b,w)$ implies
$(b,w)$ is reachable.

\hint There are only nine states with $b,w \in \set{3,4,5}$.

\begin{solution}
There are only nine states with $3 \leq \min(b,w)$ and $\max(b,w) \leq
  5$, and it is not hard to check that among these, only the start
  state $(5,4)$ satisfies the preserved invariant.  The start state is
  reachable by definition, so all big-enough eligible states with
  $\max(b,w) \leq 5$ are reachable.
\end{solution}


\ppart Prove the Claim by strong induction using the induction
hypothesis:
\begin{equation}\label{indhypP}
Q(n) \eqdef \forall b,w.\, [b+w = n\ \QIMPLIES\
    [\BE(b,w) \QIMPLIES (b,w)\ \text{is reachable}].
\end{equation}

\hint $Q(n-2) \QIMPLIES Q(n)$.

\examspace[3.0in]

\begin{solution}
\begin{proof}
\inductioncase{Base cases}: ($n \leq 5$).  The base cases follow from
part~\eqref{fact5}.

\inductioncase{Induction step}.  Given $b+w = n \geq 6$ and
$\BE(b,w)$, we must prove that $(b,w)$ is reachable.  By
part~\eqref{fact5}, we may assume that $\max(b,w) \geq 6$.

Now suppose $b \geq w$.  Since $(b,w)$ is big enough, $\min(b-3,w+1)
\geq 3$, so $(b-3,w+1)$ is big enough.  Also, $\rem{(b-3,w+1)}{4} =
\rem{(b,w)}{4}$ and $(b,w)$ is eligible, so $(b-3,w+1)$ is also
eligible.  But $(b-3)+(w+1) = n-2 > 0$, so $(b-3,w+1)$ is reachable by
induction hypothesis $Q(n-2)$.  Since $(b-3,w+1)$ transitions to
$(b,w)$ in one step, we conclude that $(b,w)$ is reachable.

Alternatively, if $w \geq b$, then, by the same reasoning, $(b+1,
w-3)$ is reachable, and therefore $(b,w)$ is reachable.

So in any case, $(b,w)$ is reachable, as required.
\end{proof}
\end{solution}


\iffalse
\ppart Prove that the derived variable $\rem{b-w}{4}$ is a constant.
Conclude that $(4^7 +1 , 4^5 +2)$ is \emph{not} reachable.

\examspace[1.0in]

\begin{solution}
If there is a transition from $(b,w)$ to $(b',w')$, then $(b'-w') =
(b-w) \pm 4$, which implies that $\rem{b'-w'}{4} = \rem{b-w}{4}$, that
is the value of the derived variable is the same.

This derived variable equals 1 for both start states and therefore
equals 1 for all states reachable from either of these start states.

On the other hand, this derived variable equals 3 for the target state
$(4^7 + 1 , 4^5 + 2)$:
\[
\rem{(4^7 + 1) - (4^5 +2)}{4} = \rem{4(4^6 - 4^4) - 4 + 3}{4} = 3h
\]
which means $(4^7 + 1 , 4^5 + 2)$ is not reachable.
\end{solution}
\fi

\ppart Verify that $\rem{3b-w}{8}$ is also a derived variable that is
constant.  Conclude that no state is reachable from both start states.

\examspace[1.0in]

\begin{solution}
The variable equals 3 at start state $(5,4)$ while it equals 1 at
start state $(4,3)$.
\end{solution}

\eparts

\end{problem}

\endinput

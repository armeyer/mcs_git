\documentclass[problem]{mcs}

\begin{pcomments}
  \pcomment{from: S09.cp4t}
\end{pcomments}

\pkeywords{
  well-ordering
  WOP
  lexicographic
  fun_game
}

%%%%%%%%%%%%%%%%%%%%%%%%%%%%%%%%%%%%%%%%%%%%%%%%%%%%%%%%%%%%%%%%%%%%%
% Problem starts here
%%%%%%%%%%%%%%%%%%%%%%%%%%%%%%%%%%%%%%%%%%%%%%%%%%%%%%%%%%%%%%%%%%%%%

\begin{problem}\label{nngame}
  In the \term{2D Origin Game}, two players make alternate moves in a
  game played on nonnegative integer points in the plane.  A move
  consists of a pair $(x,y)$ of nonnegative integers, subject to the
  constraint that none of the previous moves may simultaneously be
  below and to the left of the current move.  That is, if $(x,y)$ is
  the current move and $(x',y')$ is any previous move, then either
  $x<x'$ (so the previous move is to the right of the current one) or
  $y<y'$ (so the previous move is above the previous one).  A player
  who moves to the origin $(0,0)$ is the loser.

  For example, the Player 1 might choose (5,6), after which Player 2
  can move to any point $(n,m)$ such that $n<5$ or $m<6$, for example,
  (4,12).  Now the players might move successsively to (4,11), (29,5),
  (1,1), (0,54), (0,1).  At this point it's Player 2's turn, and he
  can move to (1,0).  Then it is Player 1's move, and the only
  available move is to the origin (0,0), so Player 1 loses this play
  of the game.

\bparts 

\ppart\label{winstrat} Identify a winning strategy for the first
player, and argue its correctness.

\solution{The first player essentially has two winning strategies:

{\bf Strategy 1:} The first player starts with $(1,1)$.  Then the
second player can only pick $(0,n)$ or $(n,0)$ for some $n$.  The
first player then responds with, $(n,0)$ or $(0,n)$, respectively.  By
symmetry, the first player will always have a move whenever the second
player had a move, except when the second player picks $(0,0)$ and
loses.

{\bf Strategy 2:} The second winning strategy is slightly more tricky:
The first player starts by picking $(2,0)$,\footnote{Of course,
  picking $(0,2)$ would be equivalent, and the following argument will
  hold for it as well, with x and y coordinates replaced.} and then:

\begin{itemize}
\item
If the 2nd player picks $(0,n)$ for any $n\geq{1}$, 1st player
responds with $(1,n-1)$.

\item
If the 2nd player picks $(1,n)$ for any $n\geq{0}$, 1st player
responds with $(0,n+1)$.
\end{itemize}
The above procedure should be repeated for all consecutive moves.
Notice that if Player 2 had a valid move, so does Player 1, unless
Player 2 picked $(0,0)$ and lost.  }

\ppart Does your strategy guarantee any bound on the number of game
moves?

\solution{No.  For the first strategy, for example, a possible length
  $2n+1$ play using the first strategy is $(1,1)
  (0,n),(n,0),(0,n-1),(n-1,0)\dots,(0,0)$.}

\eparts
\end{problem}

%%%%%%%%%%%%%%%%%%%%%%%%%%%%%%%%%%%%%%%%%%%%%%%%%%%%%%%%%%%%%%%%%%%%%
% Problem ends here
%%%%%%%%%%%%%%%%%%%%%%%%%%%%%%%%%%%%%%%%%%%%%%%%%%%%%%%%%%%%%%%%%%%%%

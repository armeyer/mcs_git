\documentclass[problem]{mcs}

\begin{pcomments}
  \pcomment{FP_3fib}
  \pcomment{ARM 5/22/13}
\end{pcomments}

\pkeywords{
  induction
  fibonacci
  recurrence
  gcd
}

%%%%%%%%%%%%%%%%%%%%%%%%%%%%%%%%%%%%%%%%%%%%%%%%%%%%%%%%%%%%%%%%%%%%%
% Problem starts here
%%%%%%%%%%%%%%%%%%%%%%%%%%%%%%%%%%%%%%%%%%%%%%%%%%%%%%%%%%%%%%%%%%%%%

\begin{problem}

\begin{staffnotes}
(a) 4 pts, (b) 5, (c) 3
\end{staffnotes}
  
Define the \emph{Triple Fibonacci} numbers $T_0,T_1,\dots$ recursively
by the rules
\begin{align}
T_0 & = T_1 \eqdef 3,\notag\\
T_n & \eqdef T_{n-1} + T_{n-2} & \text{(for $n \geq 2$)}.\label{3n23n1}
\end{align}

\bparts

\ppart\label{3DF} Prove that all Triple Fibonacci numbers are
divisible by 3.

\examspace[2.5in]

\begin{solution}
\begin{proof}
We use strong induction on $n$ with induction hypothesis
\[
P(n) \eqdef\ 3 \divides T_n.
\]

\inductioncase{Base case:} $(n=0,1)$.  $P(0)$ and $P(1)$ are true
since $T_0=T_1=3$ are both divisible by 3.

\inductioncase{Inductive step:} For $n \geq 2$, $T_n$ is defined
by~\eqref{3n23n1}.  By strong induction, we may assume that $T_{n-1}$
  and $T_{n-2}$ are divisible by 3, and hence $T_n$ is the sum of two
  numbers that are divisible by three, and therefore is itself
  divisible by 3.
\end{proof}
\end{solution}

\ppart Prove that the gcd of every pair of consecutive Triple
Fibonacci numbers is 3.

\examspace[3.5in]

\begin{solution}

\begin{proof}
By induction on $n$ with induction hypothesis
\[
Q(n) \eqdef\  [\gcd(T_{n},T_{n-1}) = 3].
\]

\inductioncase{Base case}: $(n = 1)$.  $P(1)$ holds since
$\gcd(3,3)=3$.

\inductioncase{Inductive step:} To prove that $\gcd(T_{n+1}, T_n) = 3$,
we use the recursive definition~\ref{3n23n1},
\begin{align*}
\gcd(T_{n+1}, T_{n})
& = \gcd(T_{n}, T_{n+1} - T_{n})  & (\gcd(a,b) = \gcd(b, a-b))\\
& = \gcd(T_{n}, T_{n-1})   & \text{(by~\ref{3n23n1})}\\
& = 3    & \text{(by induction hypothesis $Q(n)$)}.
\end{align*}
\end{proof}

\end{solution}

\examspace
\ppart Express the generating function $T(x)$ for the Triple Fibonacci
as a quotient of polynomials.  (You do \emph{not} have to find a
formula for $[x^n]T(x)$.)

\begin{solution}

\[
T(x)(1-x-x^2) = T_0 + T_1x -2T_0x = 3
\]
so
\[
T(x) = \frac{3}{1-x-x^2}
\]

\end{solution}

\eparts

\end{problem}


%%%%%%%%%%%%%%%%%%%%%%%%%%%%%%%%%%%%%%%%%%%%%%%%%%%%%%%%%%%%%%%%%%%%%
% Problem ends here
%%%%%%%%%%%%%%%%%%%%%%%%%%%%%%%%%%%%%%%%%%%%%%%%%%%%%%%%%%%%%%%%%%%%%

\endinput

\documentclass[problem]{mcs}

\begin{pcomments}
  \pcomment{from: F07.ps2}
\end{pcomments}

\pkeywords{
  partial_order
  weak_partial_order
  reflexive
  antisymmetric
  transitive
  subset
  isomorphic
  inverse_image
}

%%%%%%%%%%%%%%%%%%%%%%%%%%%%%%%%%%%%%%%%%%%%%%%%%%%%%%%%%%%%%%%%%%%%%
% Problem starts here
%%%%%%%%%%%%%%%%%%%%%%%%%%%%%%%%%%%%%%%%%%%%%%%%%%%%%%%%%%%%%%%%%%%%%

\begin{problem}
This problem asks for a proof of Notes Lemma~\ref{rgb}:

Let $\preceq$ be a weak partial order on a set, $A$.  Then $\preceq$ is
isomorphic to the subset relation, $\subseteq$, on the the collection of
inverse images, $\preceq\set{a}$, of elements $a \in A$.

\bparts

\ppart Prove that the function taking $a \in A$ to its inverse image,
$\preceq\set{a}$, is a bijection to the set of such inverse images.

\begin{solution}
  The mapping from $a \in A$ to $\preceq\set{a}$ is a surjective function
  onto the inverse images by definition.  It is also injective, and hence
  a bijection, because if $\preceq\set{a} =\ \preceq\set{b}$, then since $a
  \preceq\set{a}$ by reflexivity, we have $a \in \preceq\set{b}$, which
  means $a \preceq b$.  Symmetrically, $b \preceq a$, and hence $a=b$ by
  antisymmetry.
\end{solution}

\ppart Complete the proof by showing that
\begin{equation}\label{apbiff}
a \preceq b  \qiff  (\preceq\set{a})\ \subseteq\ (\preceq\set{b})
\end{equation}
for all $a,b \in A$.
\begin{solution}
%this should be revised into an "iff" proof

For the left-to-right direction,suppose $a \preceq b$.  To prove that
$(\preceq\set{a})\ \subseteq\ (\preceq\set{b})$, suppose $c \in \preceq\set{a}$,
which means that $c\preceq a$.  So by transitivity, $c \preceq b$.  This
meand that $c\in \preceq\set{b}$.  Hence every $c \in \preceq\set{a}$ is
also in $\preceq\set{b}$, which proves containment.

For the right-to-left, suppose $(\preceq\set{a})\ \subseteq\
(\preceq\set{b})$.  But $a \in (\preceq\set{a})$ by reflexivity, so $a\in
(\preceq\set{b})$, which means that $a\preceq b$.

\end{solution}

\eparts
\end{problem}


%%%%%%%%%%%%%%%%%%%%%%%%%%%%%%%%%%%%%%%%%%%%%%%%%%%%%%%%%%%%%%%%%%%%%
% Problem ends here
%%%%%%%%%%%%%%%%%%%%%%%%%%%%%%%%%%%%%%%%%%%%%%%%%%%%%%%%%%%%%%%%%%%%%

\endinput

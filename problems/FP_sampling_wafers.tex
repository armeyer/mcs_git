\documentclass[problem]{mcs}

\begin{pcomments}
  \pcomment{FP_sampling_wafers}
  \pcomment{by kazerani on 12/5/11 verbatim from F07 Final P11}
\end{pcomments}


\pkeywords{
binomial
sampling
probability
}

%%%%%%%%%%%%%%%%%%%%%%%%%%%%%%%%%%%%%%%%%%%%%%%%%%%%%%%%%%%%%%%%%%%%%
% Problem starts here
%%%%%%%%%%%%%%%%%%%%%%%%%%%%%%%%%%%%%%%%%%%%%%%%%%%%%%%%%%%%%%%%%%%%%

\begin{problem}

\begin{staffnotes}
\textbf{WARNING: I believe this appears in the F05 final, which I think is OCW
-accessible.  Similar problems appear on other finals.  Ones involving 
donuts or cakes or lines of code.}
\end{staffnotes}

  On December 13, 2007, the MIT fabrication facility produced a long
  run of silicon wafers.  To estimate the fraction $d$ of defective
  wafers in this run, we will take a small sample of $n$ wafers from
  the run, chosen randomly and independently.  Then we will test the
  chosen wafers for defects, and will estimate that $d$ is
  approximately the same as the fraction of defective wafers in the
  sample.

A calculation based on the Binomial Sampling Theorem (given below)
will yield a near-minimal number $n_0$, and such that with a sample of
size $n = n_0$, the estimated fraction will be within 0.006 of the
actual fraction $d$ with 97\% confidence.

\begin{theorem*} [Binomial Sampling]
Let $K_1, K_2, \dots$, be a sequence of mutually independent 0-1-valued
random variables with the same expectation $p$ and let
\[
S_n \eqdef \sum_{i=1}^n K_i.
\]
Then, for $1/2 > \epsilon > 0$,
\begin{equation}\label{eps}
\pr{\abs{\frac{S_n}{n} - p} \geq \epsilon}
\leq
\frac{1 + 2\epsilon}{2\epsilon} \cdot
                \frac{2^{-n (1- H((1/2) - \epsilon))}}{\sqrt{2 \pi (1/4 - \epsilon^2) n}}
\end{equation}
where
\[
H(\alpha) \eqdef - \alpha\log_2 \alpha - (1-\alpha)\log_2 (1-\alpha).
\]
\end{theorem*}

\begin{problemparts}

\problempart \label{sample-thm} Briefly explain how to use the Binomial 
Sampling
Theorem to find $n_0$, explicitly indicating which values should be
plugged into what formulas.  You are not expected to calculate any 
actual
values.

\begin{solution}
To find $n_0$, let $\epsilon = 0.006$, and
search for the smallest $n$ such that the right-hand side of~\eqref{eps} 
is
$\leq 0.03$.
\end{solution}
\examspace[2in]

\instatements{\newpage} \problempart The calculations in
part~\eqref{sample-thm} depend on some facts about the run and how the 
$n$
wafers in the sample are chosen.  Mathematically, the fabrication
\emph{run} is an actual outcome that happened a week ago.  The
\emph{sample} is a random variable defined by the process for randomly
choosing $n$ wafers from the run.

Write \textbf{T} or \textbf{F} next to each of the following statements 
to
indicate whether it is True or False.

\[\begin{array}{lccr}
\parbox{5.5in}{
  The probability that the ninth wafer in the \emph{sample} is
  defective is $d$.
}&& \fbox{\makebox[\totalheight]{\rule[-0.1in]{0in}{0.3in}
\insolutions{\textbf{T}}}}&\end{array}\]

\begin{solution}
    The ninth wafer in the sample is equally likely to be any wafer in
    the run, so the probability it is defective is the same as the
    fraction, $d$, of defective wafers in the fabrication run.
\end{solution}

\[\begin{array}{lccr}
\parbox{5.5in}{
  The probability that the ninth wafer in the \emph{run} is
  defective is $d$.
}&& \fbox{\makebox[\totalheight]{\rule[-0.1in]{0in}{0.3in}
\insolutions{\textbf{F}}}}&\end{array}\]

\begin{solution}

    The fabrication run was completed last week, so there's nothing
    probabilistic about the defectiveness of the ninth (or any other)
    wafer in the run: either it is or it isn't defective, though we
    don't know which.  You could argue that this means it is defective
    with probability zero or one, but in any case, it certainly isn't
    $d$.
\end{solution}

\[\begin{array}{lccr}
\parbox{5.5in}{
  All wafers are equally as likely to be selected as the third wafer in
  the \emph{sample}.
}&& \fbox{\makebox[\totalheight]{\rule[-0.1in]{0in}{0.3in}
\insolutions{\textbf{T}}}}&\end{array}\]

\begin{solution}

    The meaning of ``random choices of wafers from the run'' is
    precisely that at each of the $n$ choices in the sample, in
    particular at the third choice, each wafer in the run is equally
    likely to be chosen.
\end{solution}

\[\begin{array}{lccr}
\parbox{5.5in}{
  The expectation of the indicator variable for the last wafer in
  the \emph{sample} being \text{defective} is~$d$.
}&& \fbox{\makebox[\totalheight]{\rule[-0.1in]{0in}{0.3in}
\insolutions{\textbf{T}}}}&\end{array}\]

\begin{solution}
  The expectation of the indicator variable is the same as the
  probability that it is 1, namely, it is the probability that the
  $n$th wafer chosen is defective, which, by the reasoning we used
  in the first part of this question, is $d$.
\end{solution}

\[\begin{array}{lccr}
\parbox{5.5in}{
  Given that the first wafer in the \emph{sample} is defective, the
  probability that the second wafer will also be defective is greater
  than than~$d$.
}&& \fbox{\makebox[\totalheight]{\rule[-0.1in]{0in}{0.3in}
\insolutions{\textbf{F}}}}&\end{array}\]

\begin{solution}
  The meaning of ``\emph{independent} random choices of wafers
  from the run'' is precisely that at each of the $n$ choices in the
  sample, in particular at the second choice, each wafer in the run
  is equally likely to be chosen, independent of what the first or
  any other choice happened to be.
\end{solution}

\[\begin{array}{lccr}
\parbox{5.5in}{
  Given that the last wafer in the \emph{run} is defective, the
  probability that the next-to-last wafer in the run will also be
  defective is greater than than~$d$.
}&& \fbox{\makebox[\totalheight]{\rule[-0.1in]{0in}{0.3in}
\insolutions{\textbf{F}}}}&\end{array}\]

\begin{solution}
  As noted above, it's zero or one.
\end{solution}

\[\begin{array}{lccr}
  \parbox{5.5in}{
    It turns out that there are several different possible wafer
    colors. Given that the first two wafers in the \emph{sample} are
    the same color, the probability that the first wafer is defective
    may be greater than $d$.
  }&& \fbox{\makebox[\totalheight]{\rule[-0.1in]{0in}{0.3in}
      \insolutions{\textbf{T}}}}&\end{array}\]

\begin{solution}
  We don't know how color correlates to defectiveness.  It could be
  for example, that most wafers in the run are white, and most white
  wafers are defective.  Then given that two randomly chosen wafers
  in the sample are the same color, their most likely color is white.
  This makes them more likely to be defective than usual, that is,
  the conditional probability that they will be defective would be
  greater than $d$.
\end{solution}

\[\begin{array}{lccr}
\parbox{5.5in}{
  The probability that all the wafers in the \emph{sample} will be
  different is nonzero.
}&& \fbox{\makebox[\totalheight]{\rule[-0.1in]{0in}{0.3in}
\insolutions{\textbf{T}}}}&\end{array}\]

\begin{solution}
  We know the length $r$ of the fabrication run is larger than the
  sample size $n$ in which case the probability that all the wafers
  in the sample are different is
  \[
  \frac{r}{r}\cdot \frac{r-1}{r}\cdot \frac{r-2}{r} \cdots 
\frac{r-(n-1)}{r}
  = \frac{r!}{(r-n)!r^n} > 0.
  \]
\end{solution}

\end{problemparts}

\end{problem}

%%%%%%%%%%%%%%%%%%%%%%%%%%%%%%%%%%%%%%%%%%%%%%%%%%%%%%%%%%%%%%%%%%%%%
% Problem ends here
%%%%%%%%%%%%%%%%%%%%%%%%%%%%%%%%%%%%%%%%%%%%%%%%%%%%%%%%%%%%%%%%%%%%%

\endinput

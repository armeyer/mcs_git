%PS_card_shuffle_state_machine.tex

\documentclass[problem]{mcs}

\begin{pcomments}
  \pcomment{F03.ps4}

\end{pcomments}

\pkeywords{
  state_machines
  termination
  partial_correctness
  invariant
}

%%%%%%%%%%%%%%%%%%%%%%%%%%%%%%%%%%%%%%%%%%%%%%%%%%%%%%%%%%%%%%%%%%%%%
% Problem starts here
%%%%%%%%%%%%%%%%%%%%%%%%%%%%%%%%%%%%%%%%%%%%%%%%%%%%%%%%%%%%%%%%%%%%%

\begin{problem}
Start with 102 coins on a table, 98 showing ``heads'' and 4 showing
``tails.''  There are two ways to change the coins:
\begin{enumerate}
\item[(i)] flip over any ten coins, or

\item[(ii)] let $n$ be the number of coins on the table showing heads.  Place
$n+1$ additional coins on the table, with all the new coins initially
showing tails.
\end{enumerate}

For example, you might begin by flipping nine heads and one tail,
yielding 90 heads and 12 tails, then add 91 tails, yielding 90 heads and
103 tails.

\bparts
\ppart
Model this situation as a state machine, carefully defining the set of
states, the start state and the possible state transitions.

\begin{solution}
 This can be modeled by a state machine $M \eqdef
\set{Q,Q_0,\delta}$, where:

\begin{itemize}
\item $Q\eqdef \naturals\times\naturals$, the first integer is the number of
heads and the second integer is the number of tails;

\item  $Q_0 \eqdef \set{(98,4)}$

\item  $\delta$ is given by the rule
\[
(h,t) \to 
\begin{cases}
(h-a+(10-a),t+a-(10-a)), &\mbox{ for } 0\leq a\leq 10,\ a\leq h,\ 10-a\leq t,\\
(h,t+h+1).
\end{cases}
\]
\end{itemize}

%\vskip0.5in
%{\bf Comment:} Most students forgot to specify the range of $a$ precisely
%for the first transition.

\end{solution}

\ppart Explain how to reach a state with exactly one tail showing.

\begin{solution}

\begin{enumerate}

\item Do operation 2 three times, yielding $(98, 4+3\cdot99)= (98,301)$.

\item Repeat 30 times: Do operation 1 to flip 10 tails into heads. This
will result in the state $(398,1)$, which is the desired state.
\end{enumerate}

\end{solution}


\ppart
Define the following derived variables of a state $q$:
\begin{eqnarray*}
 C(q) & \eqdef & \mbox{ the number of coins on the table},\\
 H(q) & \eqdef & \mbox{ the number of heads},\\
 T(q) & \eqdef & \mbox{ the number of tails},\\
 C_2(q) & \eqdef & \mbox{remainder}(C(q)/ 2),\\
 H_2(q) & \eqdef & \mbox{remainder}(H(q)/ 2),\\
 T_2(q) & \eqdef & \mbox{remainder}(T(q)/ 2).
\end{eqnarray*}

Which of these variables is
\begin{enumerate}

\item strictly increasing \begin{solution}
NONE
\end{solution}
\item weakly increasing \begin{solution}
$C$, $H_2$
\end{solution}
\item strictly decreasing \begin{solution}
NONE
\end{solution}
\item weakly decreasing \begin{solution}
$H_2$
\end{solution}
\item constant \begin{solution}
$H_2$
\end{solution}
\end{enumerate}

\begin{solution}
{\bf Comment:} Notice that a constant variable like $H_2$ is
also weakly decreasing and weakly increasing, by definition.
\end{solution}


\ppart Prove that it is not possible to reach a state in which there is
exactly one head showing.

\begin{solution}

We claimed above that $H_2(q)$ is an invariant value, that is, it does not
change under state tramsitions.  To prove this, let $(h,t)$ be a state
with $h$ even.  For the next state, we have two cases to consider:
\begin{enumerate}
\item The first operation is executed:
$(h,t)\rightarrow(h-2a+10,t+2a-10)$.  Since $-2a+10$ is even, $H_2((h,t))=
H_2(h-2a+10, t+2a-10)$.

\item The second operation is executed: $(h,t)\rightarrow(h,t+h+1)$. The
number of heads does not change in this case, so $H_2$ does not change.
\end{enumerate}
Since the initial number of heads, 98, is even, that is, $H_2((98,4))=0$,
the Invariant Method now implies that the number of heads in a reachable
state is always even.  But since one is odd, it is not possible to reach a
state in which there is exactly one head showing.

\end{solution}

\eparts
\end{problem}

\endinput

\documentclass[problem]{mcs}

\begin{pcomments}
    \pcomment{TP_Inverse_Relations}
    \pcomment{subsumed by TP_inverse_relation_table}
    \pcomment{Converted from inverse-relations.scm
              by scmtotex and dmj
              on Sun 13 Jun 2010 10:52:29 AM EDT}
    \pcomment{edited by ARM 6/17/10}
\end{pcomments}

\pkeywords{
  relation
  injection
  bijection
  surjection
  function
  total
  inverse
}

\begin{problem}
Fill in the statements below with the correct phrase among the following:
\begin{itemize}
\item a function
\item total
\item a surjection
\item an injection
\item a bijection
\end{itemize}

\begin{editingnotes}
In other words, you get the diagram for $\inv{R}$ from $R$ by
``reversing the arrows'' in the diagram describing $R$.  Now many of
the relational properties of $\inv{R}$ correspond to different
properties of $R$.  For example, $R$ is total iff $\inv{R}$ is a
surjection.
\end{editingnotes}

\bparts

\ppart $R$ is a function iff $\inv{R}$ is \underline{\ \ \ \ \ \ \ \ }.

\begin{solution}
an injection.
\end{solution}

\ppart $R$ is a surjection iff $\inv{R}$ is \underline{\ \ \ \ \ \ \ \ }.

\begin{solution}
total.
\end{solution}

\ppart
$R$ is an injection iff $\inv{R}$ is \underline{\ \ \ \ \ \ \ \ }.

\begin{solution}
a function.
\end{solution}

\ppart $R$ is a total iff $\inv{R}$ is \underline{\ \ \ \ \ \ \ \ }.

\begin{solution}
a surjection.
\end{solution}

\ppart
$R$ is a function iff $\inv{R}$ is \underline{\ \ \ \ \ \ \ \ }.

\begin{solution}
an injection.
\end{solution}

\ppart
$R$ is a bijection iff $\inv{R}$ is \underline{\ \ \ \ \ \ \ \ }.

\begin{solution}
a bijection.
\end{solution}

\eparts

\end{problem}

\endinput

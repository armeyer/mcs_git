\documentclass[problem]{mcs}

\begin{pcomments}
  \pcomment{FP_chain_list_conflict}
  \pcomment{part (a) of FP_chain_list}
  \pcomment{ARM 4/5/16}
\end{pcomments}

\pkeywords{
  partial_order
  maximum
  linear
  chain
  comparable
}

%%%%%%%%%%%%%%%%%%%%%%%%%%%%%%%%%%%%%%%%%%%%%%%%%%%%%%%%%%%%%%%%%%%%%
% Problem starts here
%%%%%%%%%%%%%%%%%%%%%%%%%%%%%%%%%%%%%%%%%%%%%%%%%%%%%%%%%%%%%%%%%%%%%

\begin{problem} 
Let $R$ be a weak partial order on a set $A$.  Suppose $C$ is a
finite chain.\footnote{A set $C$ is a \emph{chain} when it is
  nonempty, and all elements $c,d \in C$ are comparable.  Elements $c$
  and $d$ are \emph{comparable} iff $[c \mrel{R} d\ \QOR\ d \mrel{R}
    c]$.  \iffalse A partial order for which every two different
  elements are comparable is called a \emph{linear order}\fi}

that $C$ has a maximum element.  \hint Induction on the
size of $C$.

\begin{solution}
As hinted, we give a proof by induction on the size of $C$.

\begin{proof}
The induction hypothesis is:
\begin{quote}
$P(n) :=$ If $C$ is a chain of size $n$, then $C$ has a maximum
  element.
\end{quote}

\inductioncase{Base case}: ($n=1$).  The one element is $C$ is the
maximum (ands also minimum) element, bym definition of maximum.

\inductioncase{Induction step}: To prove $P(n+1)$ for $n \geq 1$, let
$C_{n+1}$ be a chain of size $n+1$ and let $x$ be an arbitrary element
in $C_{n+1}$.  Then $C_{n+1}-\set{x}$ is a chain of size $n$, so it
has a maximum element $m$ by induction hypothesis.  Now compare $x$
and $m$.  If $x \mrel{R} m$, then $m$ is the3 maximum element in
$C_{n+1}$.  On the other hand, $m \mrel{R} x$, then (by transitivity
of $R$), $x$ is a maximum element of $C_{n+1}$.  In any case, 
$C_{n+1}$ has a maximum element, which proves $P(n+1)$.
\end{proof}

\end{solution}
\end{problem} 


%%%%%%%%%%%%%%%%%%%%%%%%%%%%%%%%%%%%%%%%%%%%%%%%%%%%%%%%%%%%%%%%%%%%%
% Problem ends here
%%%%%%%%%%%%%%%%%%%%%%%%%%%%%%%%%%%%%%%%%%%%%%%%%%%%%%%%%%%%%%%%%%%%%

\endinput

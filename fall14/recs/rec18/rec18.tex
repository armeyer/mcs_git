\documentclass[12pt]{article}
\usepackage{../light}
\usepackage{enumerate}

\hidesolutions
\showsolutions

\begin{document}

\recitation{18}{November 12, 2014}

%%%%%%%%%%%%%%%%%%%%%%%%%%%%%%%%%%%%%%%%%%%%%%%%%%%%%%%%%%%%%%%%%%%%%%%%%%%%%55
\subsection*{Problem 1}

Write a formula for the generating function whose successive
coefficients are given by the sequence:


\begin{enumerate}
    \item $0, 0, 1, 1, 1,\dots$
    \solution[\vspace{1in}]{
        $$\frac{x^2}{1-x}$$
    }

    \item $1, 1, 0, 0, 0,\dots$
    \solution[\vspace{1in}]{
        $$ 1+x $$
    }

    \item $1, 0, 1, 0, 1, 0, 1,\dots$
    \solution[\vspace{1in}]{
        $$ \frac{1}{1-x^2} $$
    }

    \item $1, 4, 6, 4, 1, 0, 0, 0,\dots$
    \solution[\vspace{1in}]{
        $$(1+x)^4$$
    }

    \item $1, 2, 3, 4, 5,\dots$
    \solution[\vspace{1in}]{
        $1/(1-x)^2$, the derivative of $1/(1-x)$.
    }

    \item $1, 4, 9, 16, 25,\dots$
    \solution[\vspace{1in}]{
        $(1+x)/(1-x)^3$, the derivative of $x/(1-x)^2$.
    }

    \item $1, 1, 1/2, 1/6, 1/24, 1/120,\dots$
    \solution[\vspace{1in}]{
        $$ e^{x} $$
    }

    \end{enumerate}



%%%%%%%%%%%%%%%%%%%%%%%%%%%%%%%%%%%%%%%%%%%%%%%%
\newpage
\subsection*{Problem 2}


T-Pain is planning an epic boat trip and he needs to decide what to
bring with him.

\begin{itemize}

\item He must bring some burgers, but they only come in packs of 6.

\item He and his two friends can't decide whether they want to dress formally or
casually.  He'll either bring 0 pairs of flip flops or 3 pairs.

\item He doesn't have very much room in his suitcase for towels, so he can
  bring at most 2.

\item In order for the boat trip to be truly epic, he has to bring at least 1
nautical-themed pashmina afghan.

\end{itemize}

\begin{enumerate}

    \item Let $B(x)$ be the generating function for the number of ways to
bring $n$ burgers, $F(x)$ for the number of ways to bring $n$ pairs of
flip flops, $T(x)$ for towels, and $A(x)$ for Afghans.  Write simple
formulas for each of these.

\solution[\vspace{2in}]{
\begin{align*}
    B(x) & = \frac{x^6}{1-x^6},\\
    F(x) & = (1+x^3),\\
    T(x) & = 1+x+x^2 = \frac{1-x^3}{1-x}\\
    A(x) & = \frac{x}{1-x}.
\end{align*}
}

    \item Let $g_n$ be the the number of different ways for T-Pain to
bring $n$ items (burgers, pairs of flip flops, towels, and/or afghans)
on his boat trip.  Let $G(x)$ be the generating function
$\sum_{n=0}^{\infty} g_nx^n$.  Verify that
\[
G(x) = \frac{x^7}{\paren{1-x}^2}.
\]

\solution[\vspace{2in}]{
By the Convolution Rule,
\begin{align*}
G(x) & = B(x)F(x)T(x)A(x)\\
     & = \frac{x^6}{1-x^6}(1+x^3)\frac{1-x^3}{1-x}\frac{x}{1-x}\\
     & = \frac{x^6(1+x^3)(1-x^3)x}{(1-x^6)(1-x)^2}\\
     & = \frac{x^7}{(1-x)^2}
\end{align*}
}

\item Find a simple formula for $g_n$.

\solution[\vspace{4in}]{
\begin{equation}\label{gnn-6}
g_n  = \begin{cases} 0 & \text{for $n < 7$}\\
                     n-6 & \text{for $n \geq 7$}.
\end{cases}
\end{equation}

Let
\[
H(x) := \frac{1}{(1-x)^2},
\]
so $G(x) = x^7H(x)$.  We know that the coefficient, $h_n$, of $x^n$ in
the series for $H(x)$ is, by the Convolution Rule, the number of ways
to select $n$ items of two different kinds, namely, $h_n =
\binom{n+1}{1}=n+1$.  So we conclude that for $n \geq 7$, the $n$th
coefficient in the series for $G(x)$ is $h_{n-7}$
namely~\eqref{gnn-6}.
}

\end{enumerate}





%%%%%%%%%%%%%%%%%%%%%%%%%%%%%%%%%%%%%%%%%%%%%%%%%%%%%%%%%5
\newpage
\subsection*{Problem 3}


Let $a_{n}$ be the number of ways to make change for \$$n$ using \$2
and \$3 coins.  For example, $a_{5} = 1$ because the only way to make
change for \$5 is with one \$2 coin and one \$3 coin, but $a_{6} = 2$
because there are two ways to make change for \$6, namely using three
\$2 coins or using two \$3 coins.

Express the generating function for the sequence of $a_{n}$'s as a
rational function (quotient of products of polynomials).  You need not
simplify your formula or solve for $a_n$.

\solution[\vspace{5in}]{
\[
1/(1-x^2)(1-x^3)
\]

Using \$2 coins, there is only one way to make change for \$$n$ when
$n$ is even, and no way to do it when $n$ is odd.  So the generating
function for the number of ways to make change for \$$n$ using only
\$2 coins is
\[
    1+x^2+x^4+x^6+\cdots = \frac{1}{1-x^2}
\]
Similarly, the generating function for the number of ways to make
change for \$$n$ using only \$3 coins is
\[
\frac{1}{1-x^3}
\]

The generating function for the number of ways to make change using
both kinds of coins is the product of the generating functions for
each kind of coin.
}







%%%%%%%%%%%%%%%%%%%%%%%%%%%%%%%%%%%%%%%%%%%%%%%%%%%%%%%%%
\newpage
\subsection*{Problem 4}


You would like to buy a bouquet of flowers.  You find an
online service that will make bouquets of \textbf{lilies}, \textbf{roses}
and \textbf{tulips}, subject to the following constraints:
\begin{itemize}
\item there must be at most 1 lily,
\item there must be an odd number of tulips,
\item there must be at least two roses.
\end{itemize}

Example: A bouquet of no lilies, 3 tulips, and 5 roses satisfies the
constraints.

Express $B(x)$, the generating function for the number of ways
to select a bouquet of $n$ flowers, as a quotient of polynomials (or
products of polynomials).  You do not need to simplify this
expression.

\solution[\vspace{5in}]{
Generating function for the number of ways to choose
lilies:
\[
L(x) = 1+ x
\]

Generating function for the number of ways to choose tulips:
\[
T(x) = x + x^3 + x^5 + \cdots = \frac{x}{1-x^2}
\]

Generating function for the number of ways to choose roses:
\[
R (x) = x^2+x^3 + x^4 + \cdots = \frac{x^2}{1-x}
\]

By the \index{convolution} Convolution Property, the generating function
$B(x)$ is the product of these functions, namely,
\begin{align*}
B(x) & = L(x) R(x) T(x) \\
     & = (1+ x)\frac{x}{1-x^2} \frac{x^2}{1-x}\\
     & = \frac{x^3(1+ x)}{(1+x)(1-x)^2}\\
     & = \frac{x^3}{(1-x)^2}
\end{align*}
}


\end{document}

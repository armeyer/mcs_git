%CP_power_set-tower

\documentclass[problem]{mcs}

\begin{pcomments}
  \pcomment{9/23/09 by ARM, from logic notesproblem}
\end{pcomments}

\pkeywords{
 powerset
 cardinality
 Strictly bigger
 infinite
}

%%%%%%%%%%%%%%%%%%%%%%%%%%%%%%%%%%%%%%%%%%%%%%%%%%%%%%%%%%%%%%%%%%%%%
% Problem starts here
%%%%%%%%%%%%%%%%%%%%%%%%%%%%%%%%%%%%%%%%%%%%%%%%%%%%%%%%%%%%%%%%%%%%%

\begin{problem} 
There are lots of different sizes of infinite sets.  For example, starting
with the infinite set, $\naturals$, of nonnegative integers, we can build
the infinite sequence of sets
\[
\naturals,\ \power(\naturals),\ \power(\power(\naturals)),\
\power(\power(\power(\naturals))),\ \dots.
\]
By Theorem~\ref{powbig} from the Notes, each of these sets is
\emph{strictly bigger}\footnote{Reminder: set $A$ is \term{strictly bigger than}
  set $B$ just means that $A \surj B$, but $\QNOT(B \surj A)$.}
than all the preceding ones.  But that's not all:
if we let $U$ be the union of the sequence of sets above, then $U$ is
strictly bigger than every set in the sequence!  Prove this:
\begin{lemma*}
  Let $A \surj B$ mean that there is a surjective function from
  $A$ to $B$.  Let $\power^n(\naturals)$ be the $n$th set in the sequence,
  and
  \[
  U \eqdef \lgunion_{n=0}^\infty \power^n(\naturals).
  \]
  Then
\begin{enumerate}
\item\label{Usp} $U \surj \power^n(\naturals)$ for every $n \in \naturals$, but

\item\label{nopsU} there is no $n \in \naturals$ for which
  $\power^n(\naturals) \surj U$.
\end{enumerate}
\end{lemma*}

Now of course, we could take $U, \power(U), \power(\power(U)), \dots$ and can
keep on indefinitely building still bigger infinities.

\begin{solution}
Everything follows from a trivial observation: if $A \supseteq B$, then $A
\surj B$.  (Why is this trivial?)

So since $U \supseteq \power^n(\naturals)$, we have $U \surj
\power^n(\naturals)$, which proves~\ref{Usp}.

To prove~\ref{nopsU}, assume to the contrary that $\power^m(\naturals)
\surj U$.  Now we know from~\ref{Usp} that $U \surj
\power^{m+1}(\naturals)$.  But this implies that
\[
\power^m(\naturals)
\surj \power^{m+1}(\naturals) = \power(\power^m(\naturals)),
\]
contradicting the fact that, by Theorem~\ref{powbig}, a power set of
$\power^m(\naturals))$ is ``strictly bigger'' than $\power^m(\naturals))$.
\end{solution}

\end{problem}

%%%%%%%%%%%%%%%%%%%%%%%%%%%%%%%%%%%%%%%%%%%%%%%%%%%%%%%%%%%%%%%%%%%%%
% Problem ends here
%%%%%%%%%%%%%%%%%%%%%%%%%%%%%%%%%%%%%%%%%%%%%%%%%%%%%%%%%%%%%%%%%%%%%

\endinput

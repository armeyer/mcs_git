\documentclass[problem]{mcs}

\begin{pcomments}
  \pcomment{FP_towers_of_Sheboygan_fall13} \pcomment{conflicts with
    CP_towers_of_Sheboygan}
\end{pcomments}

\pkeywords{
  generating_function
  Sheboygan
  linear_recurrence
  towers_of_Hanoi
}

%%%%%%%%%%%%%%%%%%%%%%%%%%%%%%%%%%%%%%%%%%%%%%%%%%%%%%%%%%%%%%%%%%%%%
% Problem starts here
%%%%%%%%%%%%%%%%%%%%%%%%%%%%%%%%%%%%%%%%%%%%%%%%%%%%%%%%%%%%%%%%%%%%%

\begin{problem}
 The Towers of Sheboygan puzzle involves 3 posts and $n$ rings of
 different sizes.  The rings are placed on post \#1 in order of size
 with the smallest ring on top and largest on bottom as in
 Figure~\ref{sheboygan-discs}.

\begin{figure}

\graphic{Fig_10-1}

\caption{The initial configuration of the rings in the Towers of Sheboygan
  puzzle.}

\label{sheboygan-discs}

\end{figure}

The objective is to transfer all $n$ rings to post \#2 via a sequence
of moves.  A \emph{move} consists of moving the top ring from either
post \#1 to post \#2, from post \#2 to post \#3, or from post \#3 to
post \#1.  For example, moving a ring directly from post \#1 to post
\#3 is not permitted.  Moreover, a larger ring can never be placed on
top of a smaller ring.

The best procedure for moving the rings can be defined in terms of two
\idx{mutually recursive} procedures, procedure $P_1(n)$ for moving a
whole stack of $n$ rings 1 pole forward, and $P_2(n)$ for moving a
whole stack of $n$ rings 2 poles forward.  This is trivial for $n=0$.
For $n>0$, define:

$P_1(n)$: Apply $P_2(n-1)$ to move the top $n-1$ rings two poles
forward.  Then move the remaining big ring.  Next apply $P_2(n-1)$
again to move the stack of $n-1$ rings two poles forward to land on
top of the big ring.

$P_2(n)$: Apply $P_2(n-1)$ to move the top $n-1$ rings two poles
forward.  Then move the remaining big ring.  Next apply $P_1(n-1)$ to
move the stack of $n-1$ rings one pole forward.  Now move the big ring
again.  Finally, apply $P_2(n-1)$ again to move the stack of $n-1$
rings two poles forward to land on the big ring.

Let $t_n$ be the number of moves needed to solve the Sheboygan puzzle
using procedure $P_1(n)$.
\bparts

\examspace

\ppart Show that
\begin{equation}\label{TnTT}
t_n = 2t_{n-1} +  2t_{n-2} + 3,
\end{equation}
for $n > 1$.

\hint Let $u_n$ be the number of moves used by procedure $P_2(n)$.
Express and $t_n$ and $u_n$ on terms of $t_{n-1}$ and $u_{n-1}$.

\examspace[2.0in]

\begin{solution}
From the definitions of procedures $P_1$ and $P_2$ we have
\begin{align}
t_0 & = 0, \notag\\
u_0 & = 0, \notag\\
t_n & = u_{n-1} + 1 + u_{n-1} & \text{for } n > 0,\label{ST}\\
u_n & = u_{n-1} + 1 + t_{n-1} + 1 + u_{n-1} & \text{for } n > 0.\label{TT}
\end{align}

Using~\eqref{ST} to get $u_{n-1} = (t_n - 1)/2$ and then expressing
$u$'s in~\eqref{TT} in terms of $t$'s, we conclude that for $n > 0$,
\[
\frac{t_{n+1} -1}{2} =  (t_n -1) + t_{n-1} + 2
\]
so
\[
t_{n+1} = 2t_n + 2t_{n-1} + 3,
\]
which is the same as the given recurrence~\eqref{TnTT} with $n+1$
replacing $n$.
\end{solution}

%\examspace
\ppart Let $T(x) \eqdef t_0 + t_1 x + t_2 x^2 + t_3 x^3 + \cdots$ be
the generating function for the $t_n$'s.  Show that
\[
T(x) = \frac{2x^2+x}{(1-x)(1-2x-2x^2)}
\]

\begin{solution}
\[
\begin{array}{rcccrcrcr}
T(x)      & = & t_0 & + &  t_1 x & + &  t_2 x^2 & + &  t_3 x^3  + \cdots,\\
-2xT(x)   & = &     & - & 2t_0 x & - & 2t_1 x^2 & - & 2t_2 x^3  - \cdots,\\
-2x^2T(x) & = &     &   &        & - & 2t_0 x^2 & - & 2t_1 x^3  - \cdots,\\
-3/(1-x)  & = & - 3 & - &    3 x & - &    3 x^2 & - &    3 x^3  - \cdots,
\end{array}
\]
so
\[
\begin{array}{rcccrcrcr}
T(x)(1-2x-2x^2)-\frac{3}{1-x}
       & = & t_0-3  & + & \paren{t_1-2t_0 -3} x
                                 & + &    0 x^2 & + &    0 x^3  + \cdots\\
       & = & -3   & + & (-2) x,
\end{array}
\]
so
\[
T(x)(1-2x-2x^2)  = \frac{3}{1-x} -3 -2x  = \frac{2x^2 + x}{1-x},
\]
and
\begin{equation}\label{Tfrac}
T(x) = \frac{2x^2+x}{(1-x)(1-2x-2x^2)}.
\end{equation}

For completeness, we derive a closed form for $t_n$, which was
\textbf{not asked for to answer this question}.

From the quadratic formula we derive
\[
T(x) = \frac{2x^2+x}{(1-x)(1-\alpha x)(1-\beta x)}
\]
where $\alpha = 1+\sqrt{3}$, $\beta = 1-\sqrt{3}$.  This implies that
$T(x)$ can be expressed using partial fractions as
\begin{equation}\label{Tpfrac}
T(x) = \frac{a}{1 - \alpha x} + \frac{b}{1 - \beta x} + \frac{c}{1 - x}
\end{equation}
To find $a,b,c$, multiply both sides of~\eqref{Tpfrac} by
$(1-\alpha x)(1-\beta x)(1-x)$ to get
\begin{equation}\label{Tpfrac2}
2x^2+x = a(1-\beta x)(1-x) +
         b(1-\alpha x)(1- x) +
         c(1-\alpha x)(1-\beta x).
\end{equation}
Letting $x=1$ gives
\[
3 = c (1-\alpha)(1-\beta) = c (-3)
\]
so $c = -1$.  Similarly, letting $x=1/\alpha$ gives (after a little
calculation) $a = 1/(3-\sqrt{3})$, and letting $x=1/\beta$ gives $b =
1/(3+\sqrt{3})$.

Finally, since
\[
[x^n]\paren{d/(1-\delta x)} = d \delta^n,
\]
we conclude that
\begin{align*}
t_n & = a \alpha^n + b \beta n + c 1^n\\
    & = \frac{1}{3-\sqrt{3}}(1+\sqrt{3})^n+\frac{1}{3+\sqrt{3}}(1-\sqrt{3})^n - 1
\end{align*}

In particular, we conclude that $t_n = \Theta((1+ \sqrt{3})^n)$.
Noting that the naive strategy for this problem described in the first
part of Problem~\bref{CP_towers_of_Sheboygan} required $s_n =
\Theta(4^n)$ moves, we conclude that $t_n = o(s_n)$, so our mutually
recursive strategy is asymptotically more efficient that the naive
one.
\end{solution}

\begin{staffnotes}
Rubric from Stephanie Wang:

part a (5 points)

+2 for getting the two equations relating $t_n$ and $u_n$

+3 for manipulating the $t_n$ and $u_n$ equations to get a recurrence for $t_n$ in terms of $t_n$
or +5 for showing the recurrence in some other way - I don't think I ever saw this done correctly

part b (9 points)

+3 for writing down the generating function to cancel away the
$+t_{n-1}$ and $+t_{n-2}$ terms (writing $-2xT(x)$, etc.)

+2 for writing down the generating function to cancel away the +3 term
(writing $-3/(1-x)$ or similar)

+2 for correctly cancelling away the terms that evaluate to 0 (writing
$T(x) - 2xT(x) - 2x^2T(x) - 3/(1-x) =$ the nonzero terms)

+2 for using the above equation to get to the final generating
function (this was just arithmetic from the above step)

or +9 for correctly coming to the final answer in some other way (not common, but a couple people had a variation that didn't fit in this sequence of steps)
\end{staffnotes}

\eparts

\end{problem}

\endinput

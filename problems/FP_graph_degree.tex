\documentclass[problem]{mcs}

\begin{pcomments}
  \pcomment{FP_graph_degree} 
  \pcomment{by Peter 12/04/11}
  \pcomment{ARM 12/15/11: poorly written numerical variant of
    FP_bipartite_matching_sex; DO NOT USE}
\end{pcomments}

\pkeywords{
  graphs
  bipartite
  matching
  Halls_theorem
}

%%%%%%%%%%%%%%%%%%%%%%%%%%%%%%%%%%%%%%%%%%%%%%%%%%%%%%%%%%%%%%%%%%%%%
% Problem starts here
%%%%%%%%%%%%%%%%%%%%%%%%%%%%%%%%%%%%%%%%%%%%%%%%%%%%%%%%%%%%%%%%%%%%%

\begin{problem}
  There are $n$ students. They go to a donut shop that sells $d$ types of donuts.
Each student buys some donuts, but no more than 1 of each type. (multiple students 
can still buy the same type of donut). The average number of donuts bought by 
each student is $x$, and the average number of students that bought each donut is $y$. 
Suppose that the value of $x$ is exactly half of $y$.

  \bparts

  \ppart What is the ratio $\frac{n}{d}$?

\begin{solution}
This is similar to the sexual demographics problem, so according to equation~(\bref{avgsexMF}),
\[
y = \frac{n}{d} \cdot x,
\]
so
\[
\frac{n}{d} = \frac{y}{x} = \frac{2}{1} = 2
\]
\end{solution}

\ppart Suppose that 30\% of the students bought no donuts and 15\% of
the donuts were not purchased by any students.  Excluding these
students and donuts would change the ratio in the previous part from
$(n/d)$ to $c(n/d)$. What is $c$?

\begin{solution}
The new ratio will be
\[
\frac{n - .3n}{d - .15d} =  \frac{.7n}{.85d} = \frac{14}{17} \cdot \frac{f}{m},
\]
so $c = 14/17$.
\end{solution}
\eparts
\end{problem}

%%%%%%%%%%%%%%%%%%%%%%%%%%%%%%%%%%%%%%%%%%%%%%%%%%%%%%%%%%%%%%%%%%%%%
% Problem ends here
%%%%%%%%%%%%%%%%%%%%%%%%%%%%%%%%%%%%%%%%%%%%%%%%%%%%%%%%%%%%%%%%%%%%%

\endinput

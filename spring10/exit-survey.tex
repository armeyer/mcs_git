\documentclass[handout]{mcs}

\makeatletter

\def\ps@myheadings{%
    \let\@oddfoot\@empty\let\@evenfoot\@empty
    \def\@evenhead{\thepage\hfil\slshape\leftmark}%
    \def\@oddhead{\rlap{\slshape\rightmark}\hfil\thepage\hfil \llap{Name:\brule{2in}}}%
    \let\@mkboth\@gobbletwo
    \let\sectionmark\@gobble
    \let\subsectionmark\@gobble
    }

\makeatother

\begin{document}

\paper{exit}{}{End-of-term Survey}

\iffalse
\section{Purpose}

The EECS Department is trying to develop a system to monitor and
improve teaching effectivness.  Part of this effort involves formulating
and checking an Educational Objectives and Outcomes statement like the one
for 6.042 on the course web page.
%; this statement is repeated below.
\fi

This survey asks for your feedback on how well 6.042J/18.062J worked for
you.  Comparing student self-assessments given in this survey to student
grades helps us determine how to improve the course.  We would also be
grateful for any improvements you care to suggest.

\textbf{You may submit this form anonymously}.  Even if you remain
anonymous, it would be helpful if indicated your TA or LA.

\large{Your Name: (optional)} \brule{3in}

\large{Circle your TA/LA}:
\begin{center}
  \begin{tabular}{c}
        \courseassistants\\
\\
Cordelia \qquad David \qquad  Liz \qquad  Oscar \qquad Stav \qquad Stephanie 
  \end{tabular}
\end{center}

\newpage
\section*{Learning Outcomes}
In the indicated space next to each item, please enter a digit from five
(5) to one (1) where

\begin{center}
\begin{tabular}{rcll}
\hline
\textbf{5} & means &  ``this outcome was & \textbf{thoroughly}
achieved for \textbf{me personally}.''\\
\textbf{4} & means &  ``\dots & \textbf{adequately} \dots''\\
\textbf{3} & means &  ``\dots & \textbf{somewhat} \dots''\\
\textbf{2} & means &  ``\dots & \textbf{barely} \dots''\\
\textbf{1} & means &  ``\dots & \textbf{not} \dots''\\
\hline
\end{tabular}
\end{center}


\iffalse

\subsection{Objectives}
On completion of 6.042, students will be able to
\begin{enumerate}
\item
\label{Basic Discrete Mathematics Concepts}
\textbf{reason mathematically about basic data types and structures} (such
as numbers, sets, graphs, and trees) used in computer algorithms and
systems; distinguish rigorous definitions and conclusions from merely
plausible ones; synthesize elementary proofs, especially proofs by
induction.\brule{0.5in}

\item
\label{Computational Processes} 
\textbf{model and analyze computational processes} using state machine
methods.\brule{0.5in}

\item \label{Discrete Probability} \textbf{apply principles of discrete
probability} to calculate probabilities and expectations of simple random
processes.\brule{0.5in}

\item 
\label{teams} 
\textbf{work in small teams} to accomplish all the objectives above.\brule{0.5in}
\end{enumerate}

\section*{Learning Outcomes}

In the indicated space next to each item, please enter a digit from five
(5) to one (1) where

\begin{center}
\begin{tabular}{rcll}
\hline
\textbf{5} & means &  ``this objective/outcome was & \textbf{thoroughly}
achieved for me personally.''\\
\textbf{4} & means &  ``\dots & \textbf{adequately} \dots''\\
\textbf{3} & means &  ``\dots & \textbf{somewhat} \dots''\\
\textbf{2} & means &  ``\dots & \textbf{barely} \dots''\\
\textbf{1} & means &  ``\dots & \textbf{not} \dots''\\
\hline
\end{tabular}
\end{center}
\fi

Students will be able to:
\begin{enumerate}

\item \label{basic} use \textbf{logical notation} to define and reason
  about fundamental mathematical concepts such as \textbf{sets,
    relations, functions}, and integers.\hfill \brule{0.5in}

%(enter 5 ``thoroughly achieved, \dots, 1 ``not achieved'') \hfill \brule{0.5in}

\item \label{proofs} create elementary \textbf{mathematical
proofs} and identify fallacious \emph{reasoning} (not just fallacious
conclusions).  \hfill \brule{0.5in}

\item \label{induction} synthesize \textbf{induction hypotheses} and
  simple proofs using \textbf{ordinary induction}.  \hfill
  \brule{0.5in}

\item \label{structural induction} find simple proofs using
  \textbf{structural induction}.  \hfill \brule{0.5in}

\item \label{arithmetic} prove elementary properties of \textbf{modular
arithmetic} and explain their applications in Computer Science,
for example in \textbf{cryptography}.\hfill \brule{0.5in}

\item \label{graphs} apply \textbf{graph theory} models of data
  structures to solve problems of scheduling and connectivity.  \hfill
  \brule{0.5in}

\item \label{invariants} apply the method of invariants and
  well-ordering to prove correctness and termination of \textbf{state
    machines} and processes.  \hfill \brule{0.5in}

\item \label{asymptotics} compare asymptotic growth rates of
  elementary functions and explain properties \textbf{asymptotic
    relations} such as O() and o().

\item \label{counting} find \textbf{counting formulas} for numbers of
  outcomes of elementary combinatorial processes such as
  \textbf{permutations and combinations}. \hfill \brule{0.5in}

\item\label{generating functions} use \textbf{generating functions} to solve
  linear recurrences and elementary counting problems.
  \hfill \brule{0.5in}

\item \label{probability} calculate \textbf{probabilities} and
  discrete distributions for simple combinatorial processes; calculate
  \textbf{expectations}; explain \textbf{confidence levels} for
  samples \hfill \brule{0.5in}

\item \label{student teams} problem-solve in a \textbf{small team} with
fellow students.  \hfill \brule{0.5in}

\end{enumerate}

\newpage
\section*{Course Activities}

In the indicated space next to each item, please enter a digit from five
(5) to one (1) where

\begin{center}
\begin{tabular}{rcll}
\hline
\textbf{5} & means &  ``\textbf{very} helpful.''\\
\textbf{4} & means &  ``\textbf{moderately} helpful''\\
\textbf{3} & means &  ``\textbf{somewhat}  helpful''''\\
\textbf{2} & means &  ``\textbf{barely}  helpful''\\
\textbf{1} & means &  ``\textbf{not}  helpful''\\
\hline
\end{tabular}
\end{center}

\begin{itemize}

\item How helpful have the following aspects of the course been in
achieving the course outcomes for you personally:
\begin{center}

\begin{tabular}{| l | c |}
%| c |}
%                & \textbf{understanding the big ideas}
%                & \textbf{preparing for exams}
materials & value\\
\hline  \hline
   course notes          & \hspace{0.7in}\\  \hline
   problem sets          & \\  \hline
   problem set solutions & \\  \hline
   online tutor problems & \\  \hline
   NB annotation system  & \\  \hline
   lectures              & \\  \hline
%   lectures by Smyth   & \\  \hline
   reading lecture slides during class    & \\  \hline
   reading lecture slides after class     & \\  \hline
   team problem-solving in class          & \\  \hline
   TA/LA in class        & \\  \hline
   staff outside class (eg, office hours) & \\  \hline
   staff email           &\\ \hline
   collaborating on psets                 & \\  \hline
   reviewing team problem solutions       & \\  \hline
   biweekly miniquizzes                   & \\  \hline
   online grade reports                   & \\  \hline
\end{tabular}
\end{center}

\item How helpful overall has the course been in helping you learn the
material?  \hfill \brule{0.5in}

\item How helpful has the course been in leading you to appreciate the role of
Discrete Math in Computer Science? \hfill \brule{0.5in}

\end{itemize}

\newpage
\section*{Further Comments}

How interested would you be in having another class in the
``lecture/team-problems'' style of 6.042?
\begin{center}
\begin{tabular}{ccccc}
enthusiastic &  interested &  somewhat interested  &  uninterested &  unwilling
\end{tabular}
\end{center}

\vspace{0.5in}
We would be pleased to hear any other comments or suggestions you may have
about the course:

\textbox{\hspace{7in}
\vspace{6in}}

\end{document}

\documentclass[problem]{mcs}

\begin{pcomments}
  \pcomment{CP_graph_logic_probability}
  \pcomment{S08.final, F15.cp12m}
  \pcomment{rewritten ARM 12/01/15}
\end{pcomments}

\pkeywords{
  graph
  logic
  probability
  independence
}

%%%%%%%%%%%%%%%%%%%%%%%%%%%%%%%%%%%%%%%%%%%%%%%%%%%%%%%%%%%%%%%%%%%%%
% Problem starts here
%%%%%%%%%%%%%%%%%%%%%%%%%%%%%%%%%%%%%%%%%%%%%%%%%%%%%%%%%%%%%%%%%%%%%

\begin{problem}
Let $G$ be a simple graph and let $B(u,v)$ mean there is an edge
between vertices $u$ and $v$.  That is,
\[
B(u,v)\ \text{means}\ \edge{u}{v}\in \edge{G}.
\]
Since $G$ is simple, we have $B(u,v)$ iff $B(v,u)$.  Also
$\QNOT(B(u,u)$ is valid since vertices in simple graphs do not have
self-loops.

Let $W(u,v)$ mean that there is a length-two walk in $G$ between $u$
and $v$.

\bparts

\ppart Explain why $E(x,y)$ implies $W(x,x)$.

\examspace[0.5in]

\begin{solution}
Going back and forth on edge $\edge{x}{y}$ is a length-two walk from
$x$ to $x$.
\end{solution}

\ppart Write a predicate-logic formula defining $W(u,v)$ in terms of
the predicate $B(.,.)$

\examspace[0.6in]

\begin{solution}
\[
W(x,y) \eqdef \exists z.\, B(x,z) \QAND\ B(z,y).
\iffalse \QAND z \neq x \QAND\ z \neq y \QAND \fi
\]
\end{solution}



\eparts
\medskip

We assume below that $w$, $x$, $y$ and $z$ are distinct
vertices of $G$.

Now let $V$ be a fixed set of vertices and $G$ be a graph with
$\vertices{G} = V$ constructed randomly as follows: for every two
distinct vertices $u,v \in V$, independently include edge
$\edge{u}{v}$ in $\edges{G}$ with probability $p$.

So $\pr{B(u,v)} = p$, and $B(u,v)$ is independent of $B(r,s)$ when
$\edge{u}{v} \neq \edge{r}{s}$.

\begin{staffnotes}
Formally, our sample space is the set of simple graphs $G$ with
$\vertices{G} = V$ and $\pr{G} = p^k(1-p)^m$, where $k =
\card{\edges{G}}$ and $m = \binom{n}{2} - k$.

An event $E$ \emph{depends on an edge $\edge{u}{v}$} iff $E$ and
$B(u,v)$ are not independent.  If two events \textbf{depend on
  disjoint sets of edges}, then they are independent.
\end{staffnotes}

\bparts

\ppart Let $g$ be the number of edges of $G$.  Express $\pr{G}$ in
terms of $g, n$, and $p$.

\examspace[0.5in]

\begin{solution}
\[
p^g(1-p)^{\binom{n}{2}-g},
\]
\end{solution}
\eparts
\medskip

Events that depends on disjoint sets of edges will clearly be
independent, so
\begin{equation}\label{EandE2}
[B(x,w) \QAND\ B(w,y)]\ \text{is independent of}\ [B(x,z) \QAND\ B(z,y)]
\end{equation}

%%\examspace[0.25in]
%% \begin{solution}
%% \True.  Since edges are chosen independently and since all the
%% following edges are distinct, the presence of edges $\edge{x}{w},
%% \edge{w}{y}$, is independent of the presence of edges $\edge{x}{z},
%% \edge{z}{y}$.
%% \end{solution}

\bparts

\ppart\label{indeppairscircle}
\begin{enumerate}[i.]

\item\label{ExyW} Briefly explain why $B(x,y)$ is independent of
  $W(x,y)$ when $x \neq y$.

\examspace[0.25in]

\begin{solution}
Since there are no self-loops, the edge $\edge{x}{y}$ cannot
be part of any length-two walk from $x$ to $y$.  So the existence of a
length-two walk from $x$ to $y$ does not depend on whether this edge
is present in $G$.
\end{solution}

\iffalse
\ppart $W(w,x)$ versus $W(x,y)$ 

  \begin{solution}
\False, as demonstrated by the counterexample of
  $\card{V} = 4$ and $p = \frac{1}{2}$.  \TBA{explanation.}
\end{solution}
\fi

\item\label{Wwxyz} Briefly suggest why
$W(w,x)$ is \emph{not} independent $W(y,z)$ for distinct vertices $w$,
  $x$, $y$ and $z$.

\examspace[0.25in]

\begin{solution}
The more randomly chosen edges there are in $G$, the more likely
length two walks become.  Now $W(w,x)$ implies the existence of two
edges---which might be incident to $y$ or $z$---and this makes
$W(y,z)$ more likely given $W(w,x)$.
\end{solution}

\end{enumerate}

%\examspace

\ppart \label{nottwopath} \; Write a simple formula in terms of $n$
and $p$ for $\prob{\QNOT W(x,y)}$, for distinct vertices $x$ and $y$.

\examspace[0.5in]

\hint Use part~\eqref{indeppairscircle}.\eqref{EandE2}.
 
\begin{solution}
\[
(1-p^2)^{n-2}.
\]

Let $Z \eqdef V - \set{x,y}$ be the set of all the vertices other than
$x$ and $y$.
\begin{align*}
\lefteqn{\pr{\QNOT(W(x,y))}}\\
  & = \pr{\QAND_{z \in Z} \QNOT(B(x,z) \QAND\ B(z,y))},\\
  & = \prod_{z \in Z} \pr{\QNOT(B(x,z) \QAND\ B(z,y))},
          & \text{(indep. by part~\eqref{indeppairscircle}.\eqref{EandE2})}\\
  & = \prod_{z \in Z} (1 - \pr{B(x,z) \QAND\ B(z,y)}),\\
  & = \prod_{z \in Z} (1-\pr{B(x,z)}\cdot \pr{B(y,z)}),
          & \text{(edges are independent)}\\
  & = \prod_{z \in Z} (1-p^2), \\
  & = (1-p^2)^{n-2}.
\end{align*}
\end{solution}

\ppart \label{3cycle} \; Express $x$ and $y$ being on a three-cycle as
a simple predicate formula involving $B(x,y)$ and $W(x,y)$.

\examspace[0.5in]

\begin{solution}
$x$ and $y$ lie on a three-cycle iff $B(x,y) \QAND\ W(x,y)$.
\end{solution}

\examspace[0.15in]

\ppart Let $r \eqdef \pr{\QNOT(W(x,y))}$.  Write a simple expression
in terms of $p$ and $r$ for the probability that $x$ and $y$ lie on a
three-cycle in $G$.

\hint Parts~\eqref{indeppairscircle}.\eqref{ExyW} and~\eqref{3cycle}.

\examspace[0.5in]

\begin{solution}
\[
p(1-r).
\]

Since $B(x,y)$ and $W(x,y)$ are independent,
\begin{align*} 
\pr{B(x,y) \QAND\ W(x,y)}
   & = \pr{B(x,y)} \cdot \pr{W(x,y)} \\
   & = p (1-r) .
\end{align*}
\end{solution}

\eparts
\end{problem}

%%%%%%%%%%%%%%%%%%%%%%%%%%%%%%%%%%%%%%%%%%%%%%%%%%%%%%%%%%%%%%%%%%%%%
% Problem ends here
%%%%%%%%%%%%%%%%%%%%%%%%%%%%%%%%%%%%%%%%%%%%%%%%%%%%%%%%%%%%%%%%%%%%%

\endinput


Original Solution: In this counterexample,
\[
W(w,x) \equiv (B(w,y) \QAND{} B(y,x)) \QOR{} (B(w,z) \QAND{} B(z,x))
\]
and
\[
W(x,y) \equiv (B(x,w) \QAND{} B(w,y)) \QOR{} (B(x,z) \QAND{} B(z,y)).
\]

By symmetry, we apply inclusion-exclusion to calculate the probability
for either of these events:
\[
\paren{\frac{1}{2}}^2 + \paren{\frac{1}{2}}^2 - \paren{ \frac{1}{2}}^4 = \frac{7}{16}.
\]

Now consider $\prcond{W(w,x)}{W(x,y)}$, the fraction of outcomes
satisfying $W(x, y)$ that also satisfy $W(w, x)$.  Partition the
outcomes satisfying $W(x, y)$ by whether they also satisfy $B(w, y)$.
Both sides of the partition are independent of $B(y, x)$ in the sense
formalized above, since $B(y, x)$ doesn't appear in the definition of
$W(x, y)$.  That means that the outcomes in the subcase for $B(w, y)$
can be partitioned into equally sized sets, one with $B(y, x)$ and the
other with $\bar{B(y, x)}$.  Clearly every element of the first set
satisfies $W(w, x)$, so
\[
\prcond{W(w, x)}{W(x, y) \QAND{} B(y, x)} \geq \frac{1}{2}.
\]

--------------
  For
  $|V| = 4$ and $p = \frac{1}{2}$, we have
\[
W(w,x) \equiv (B(w,y) \QAND\ B(y,x)) \QOR\ (B(w,z) \QAND\ B(z,x))
\]
 and
\[
W(y,z) \equiv (B(y,x) \QAND\ B(x,z)) \QOR\ (B(y,w) \QAND\ B(w,z)).
\]

\[
\prcond{W(w, x)}{W(y, z) \QAND{} (B(y,x) \QAND{} B(x,z))} = \frac{3}{4},
\]
since under these conditions the formula for $W(w,x)$ simplifies to
$B(w,y) \QOR B(w,z)$.
\[
\prcond{W(w, x)}{W(y, z) \QAND\ \QNOT (B(y,x) \QAND\ B(x,z))} = \frac{1}{2}
\]
since under these conditions we know $B(y,w) \QAND\ B(w,z)$, and the
formula for $W(w,x)$ simplifies to 
\[
(B(y,x) \QOR\ B(z,x)) \QAND\ \QNOT(B(y,x) \QAND\ B(z,x)).
\]
Both sides of the partition have probabilities no less than
$\frac{1}{2}$, so $\prcond{W(w, x)}{W(y, z)} \geq \frac{1}{2}$, which
again is above $\pr{W(w, x)}$, which can be computed as $\frac{7}{16}$
as in the prior case.
-----------

The outcomes in the subcase for $\bar{B(w, y)}$ must all have $B(x,
z)$, so, like above partitioning them based on $B(w, z)$, we get two
equal-size sets, where the set with $B(w, z)$ all satisfy $W(w, x)$,
and
\[
\prcond{W(w, x)}{W(x, y) \QAND\ \bar{B(y, x)}} \geq \frac{1}{2}.
\]
The true value of $\prcond{W(w, x)}{W(x, y)}$ must lie somewhere
between these two values, so it also must be no less than
$\frac{1}{2}$, and thus it must be greater than $\pr{W(w, x)} =
\frac{7}{16}$.

\vspace{0.05in} Less complicated explanation: We want to show that
$\prcond{W(w,x)}{W(x,y)} \neq \Pr{W(w,x)}$.  If $W(x,y)$, this
increases the probability of $B(z,x)$ and $B(w,y)$, which can be used
for $W(w,x)$.

\fi



% Exam version commented out for use as a class problem.
\iffalse
\ppart \inbook{Indicate}\inhandout{Circle} the mathematical formula that best expresses the
definition of $W(x,y)$.

\begin{itemize}
\item $W(x,y) \eqdef \exists z.\ B(x,z) \QAND\ B(y,z)$
\examspace[0.15in]
\item $W(x,y) \eqdef x \neq y \QAND\ \exists z.\ B(x,z) \QAND\ B(y,z)$
\examspace[0.15in]
\item $W(x,y) \eqdef \forall z.\ B(x,z) \QOR\ B(y,z)$
\examspace[0.15in]
\item $W(x,y) \eqdef \forall z.\ x \neq y \QIMPLIES\ [B(x,z) \QOR\ B(y,z)]$
\end{itemize}
\fi

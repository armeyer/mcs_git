\documentclass[problem]{mcs}

\begin{pcomments}
  \pcomment{PS_gcd_union}
  \pcomment{subsumes mistaken FP/TP_gcd_linear_combination_induction/wop, 3/16/16}
  \pcomment{ARM 3/16/16}
\end{pcomments}

\pkeywords{
  induction
  gcd
  linear_combination
}

%%%%%%%%%%%%%%%%%%%%%%%%%%%%%%%%%%%%%%%%%%%%%%%%%%%%%%%%%%%%%%%%%%%%%
% Problem starts here
%%%%%%%%%%%%%%%%%%%%%%%%%%%%%%%%%%%%%%%%%%%%%%%%%%%%%%%%%%%%%%%%%%%%%

\begin{problem}
For any set $A$ of integers,
\[
\gcd(A) \eqdef \text{the greatest common divisor of the elements of $A$.}
\]

The following useful property of gcd's of sets is easy to take for
granted:
\begin{theorem*}
\begin{equation}\tag{AuB}
\gcd(A \union B) = \gcd(\gcd(A),\gcd(B)),
\end{equation}
for all finite sets $A,B \subset \integers$.
\end{theorem*}

Theorem~(AuB) has an easy proof as a Corollary of the Unique
Factorization Theorem.  In this problem we develop a proof by
induction just making repeated use of
\inbook{Lemma~\bref{lem:gcd-hold}.\bref{gcd1}} \inhandout{the fact
  that}:
\begin{equation}\tag{gcddiv}
(d \divides a \QAND d \divides b)\ \QIFF\ d \divides \gcd(a,b). 
\end{equation}

The key to proving~(AuB) will be generalizing~(gcddiv) to finite sets.
\begin{definition*}
For any subset $A\subseteq \integers$,
\begin{equation}\tag{divdef}
d \divides A \eqdef\ \forall a \in A.\ d \divides a.
\end{equation}
\end{definition*}

\begin{lemma*}
\begin{equation}\tag{A-iff-gcdA}
d \divides A \QIFF\ d \divides \gcd(A).  %dAdgA
\end{equation}
for all $d \in \integers$ and finite sets $A \subset \integers$.
\end{lemma*}

\bparts

\ppart\label{gcdassocpart} Prove that

\begin{equation}\tag{gcd-assoc}
\gcd(a,\gcd(b,c)) = \gcd(\gcd(a,b),c)
\end{equation}
for all integers $a,b,c$.

\begin{solution}
It is sufficient to prove fact that both sides of the equality have
the same divisors.  This follows by repeated use of~(gcddiv):
\begin{align*}
d \divides \gcd(a, \gcd(b,c))
  & \QIFF\ d \divides a  \QAND d \divides \gcd(b,c)\\
  & \QIFF  d \divides a  \QAND (d \divides b \QAND d \divides c)\\
  & \QIFF (d \divides a \QAND d \divides b)  \QAND d \divides c\\
  & \QIFF d \divides \gcd(a,b) \QAND d \divides c\\
  & \QIFF d \divides \gcd(\gcd(a,b), c).\\
\end{align*} 
\end{solution}
\eparts

\medskip
From here on we write ``$a \union A$'' as an abbreviation for
``$\set{a} \union A$.''  

\bparts
\ppart\label{abCpart}  Prove that
\begin{equation}\tag{abCgcd}
d \divides (a \union b \union C) \QIFF\ d \divides (\gcd(a,b) \union C)
\end{equation}
for all $a,b,d \in \integers$, and $C\subseteq \integers$.

\begin{proof}
\begin{align*}
d \divides (a \union b \union C)
  & \QIFF (d \divides a) \QAND (d \divides b) \QAND (d \divides C)
        & \text{(def~(divdef) of divides)}\\
  & \QIFF (d \divides \gcd(a,b)) \QAND (d \divides C)
        & \text{by~(gcddiv)}\\
  & \QIFF d \divides (\gcd(a,b) \union C)
        & \text{(def~(divdef) of divides)}.
\end{align*}
\end{proof}

\ppart Using parts~\eqref{gcdassocpart} and~\eqref{abCpart}, prove by
induction on the size of $A$, that
\begin{equation}\tag{divauA}
d \divides (a \union A)\quad  \QIFF\quad d \divides \gcd(a,\gcd(A)),
\end{equation}
for all integers $a,d$ and finite sets $A \subset \integers$.
Explain why this proves~(A-iff-gcdA).    % Lemma~(dAdgA).

\begin{solution}
\begin{proof}

The induction hypothesis will be
\[
P(n) \eqdef \card{A} = n \QIMPLIES [d \divides (a \union A) \QIFF d \divides \gcd(a,\gcd(A))],
\]
for all integers $n,a,d$ with $n>0$ and all finite sets $A \subset \integers$.

\inductioncase{Base cases}: ($n = 1,2$).  The case ($n=1$) follows
from the fact that $\gcd(a,b) = \gcd(\set{a,b})$.  The case ($n=2$)
follows from~(gcd-assoc).

\inductioncase{Induction step}: Suppose $\card{A} = n+1$ and $A = b
\union B$ where $\card{B} = n$.  Then
\begin{align*}
d \divides (a \union A)
   & \QIFF (d \divides a) \QAND (d \divides b) \QAND (d \divides B)
        & \text{(by def~(divdef)}\\
   & \QIFF ((d \divides a) \QAND (d \divides b)) \QAND (d \divides B)\\
   & \QIFF (d \divides \gcd(a,b)) \QAND (d \divides B)
            & \text{(by~(gcddiv))}\\
   & \QIFF d \divides (\gcd(a,b) \union B)
        & \text{(by def~(divdef)}\\
   & \QIFF d \divides \gcd(\gcd(a,b),\gcd(B))
        & \text{(by $P(n)\ \text{for}\ B$)}\\
   & \QIFF (d \divides \gcd(a,\gcd(b,\gcd(B)))
           & \text{(by~(gcd-assoc))}\\
   & \QIFF (d \divides a) \QAND (d \divides \gcd(b,\gcd(B)))
           & \text{(by~(gcddiv))}\\
   & \QIFF (d \divides a) \QAND (d \divides \gcd(b \union B))
        & \text{(by $P(n)\ \text{for}\ B$)}\\
   & \QIFF (d \divides a) \QAND (d \divides \gcd(A))\\
   & \QIFF d \divides \gcd(a,\gcd(A))
           & \text{(by~(gcddiv))}.
\end{align*}
Equation~(A-iff-gcdA)  %Lemma~(dAdgA)
follows from~(divauA) by choosing $a \in A$.
\end{proof}

\end{solution}

\ppart Prove Theorem~(AuB).

\begin{solution}
We prove~(AuB) by showing that
\[
d \divides (A \union B) \QIFF\ d \divides \set{\gcd(A),\gcd(B)}.
\]
\begin{proof}
\begin{align*}
d \divides (A \union B)
  & \QIFF d \divides A \QAND d \divides B
       & \text{(by def~(divdef))}\\
  & \QIFF d \divides \gcd(A) \QAND d \divides \gcd(B)
        & \text{(by Theorem~(AuB))}\\
  & \QIFF d \divides \set{\gcd(A),\gcd(B)}
       & \text{(by def~(divdef))}.
\end{align*}
\end{proof}

\end{solution}

\ppart Conclude that $\gcd(A)$ is an integer linear combination of the
elements in $A$.

\begin{solution}
Proof sketch: by induction on the size of $A$.  The one-element case
is immediate.  Otherwise $A = b \union B$ with $\card{B} < \card{A}$.
Now $\gcd(A) = \gcd(b,\gcd(B))$ by~(divauA), and by induction
$\gcd(B)$ is a linear combination of $B$ so $\gcd(A)$ is a linear
combination of $b$ and a linear combination of $B$, which is a linear
combination of $A$.
\end{solution}

\eparts

\end{problem}

%%%%%%%%%%%%%%%%%%%%%%%%%%%%%%%%%%%%%%%%%%%%%%%%%%%%%%%%%%%%%%%%%%%%%
% Problem ends here
%%%%%%%%%%%%%%%%%%%%%%%%%%%%%%%%%%%%%%%%%%%%%%%%%%%%%%%%%%%%%%%%%%%%%

\endinput

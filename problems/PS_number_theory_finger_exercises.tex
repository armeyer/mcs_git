\documentclass[problem]{mcs}

\begin{pcomments}
  \pcomment{from: S09.ps7}
  \pcomment{from: F08}
\end{pcomments}

\pkeywords{
  number_theory
  Pulverizer
  modular_arithmetic
}

%%%%%%%%%%%%%%%%%%%%%%%%%%%%%%%%%%%%%%%%%%%%%%%%%%%%%%%%%%%%%%%%%%%%%
% Problem starts here
%%%%%%%%%%%%%%%%%%%%%%%%%%%%%%%%%%%%%%%%%%%%%%%%%%%%%%%%%%%%%%%%%%%%%

\begin{problem}

For the following parts, a correct numerical answer will only earn
credit if accompanied by it's derivation.  Show your work.

\bparts

\ppart\label{pulverizer} Use the Pulverizer to find integers
$s$ and $t$ such that $95 s + 52 t = \gcd(95,52)$.

\solution{
\[
\begin{array}{ccccrcl}
x & \quad & y & \quad & \rem(x,y) & = & x - q \cdot y \\ \hline
95 && 52 && 43  & = &   95 - 1 \cdot 52 \\
52 && 43 && 9   & = &   52 - 1 \cdot 43 \\
&&&&            & = &   52 - 1 \cdot (95 - 1 \cdot 52) \\
&&&&            & = &   -1 \cdot 95 + 2 \cdot 52 \\
43 && 9  && 7   & = &   43 - 4 \cdot 9 \\
&&&&            & = &   (95 - 1 \cdot 52) -
                               4 \cdot (-1 \cdot 95 + 2 \cdot 52) \\
&&&&            & = &   5 \cdot 95 - 9 \cdot 52 \\
9  && 7  && 2   & = &   9 - 1 \cdot 7 \\
&&&&            & = &   (-1 \cdot 95 + 2 \cdot 52) - 
				1 \cdot (5 \cdot 95 - 9 \cdot 52) \\
&&&&            & = &   -6 \cdot 95 + 11 \cdot 52 \\
7  && 2  && 1   & = &   7 - 3 \cdot 2 \\
&&&&            & = &   (5 \cdot 95 - 9 \cdot 52) -		
				3 \cdot (-6 \cdot 95 + 11 \cdot 52) \\
&&&&            & = &   \fbox{$23 \cdot 95 - 42 \cdot 52$} \\
2  && 1  && 0
\end{array}
\]

{\it {\bf Exam tip:} each time $rem(x,y)$ is calculated,
substitutions are immediately made to then express it as a
linear combination of 95 and 52 (using the remainders
calculated on previous lines). Simplifying at each step
leads to a much faster computation of $s$ and $t$.}
}

\ppart Use the previous part to find the inverse of 52 modulo 95
in the range $\{1,\ldots,94\}$.

\solution{53

From part \eqref{pulverizer}, $1 = 23 \cdot 95 - 42 \cdot 52$
and so $1 \equiv -42 \cdot 52 \pmod{95}$. Therefore -42 is \emph{an}
inverse of 52. However, it is not \emph{the} unique inverse of
52 in the range $\{1,\ldots,94\}$, which is given by $\rem{-42}{95}=53$.
}


\ppart Use Fermat's theorem to find the inverse of 13 modulo 23 in
the range $\{1,\ldots,22\}$.

\solution{16

Since 23 is prime, Fermat's theorem implies
$13^{23-2} \cdot 13 \equiv 1 \pmod{23}$ and so $\rem{13^{23-2}}{23}$ is
the inverse of 13 in the range $\{1,\ldots,22\}$. Using the method
of repeated squaring,

%
\[
\begin{array}{lcl}
13^{2}  & =      & 169\\
	& =      & 7  \cdot 23 + 8\\
	& \equiv & 8\\ &&\\
13^{4}  & \equiv & 8^2\\
	& =      & 64\\
	& =      & 2  \cdot 23 + 18\\
	& \equiv & 18\\ &&\\
13^{8}  & \equiv & 18^2\\
	& =      & 324\\
	& =      & 14 \cdot 23 + 2\\
	& \equiv & 2\\ &&\\
13^{16} & \equiv & 2^2\\
	& =      & 4\\ &&\\
13^{21} & =      & 13^{16} \cdot 13^{4} \cdot 13\\
	& \equiv & 4 \cdot (6 \cdot 3) \cdot 13\\
	& =      & (4 \cdot 6) \cdot (3 \cdot 13)\\
	& =      & 24 \cdot 39\\
	& \equiv & 1 \cdot 39\\
	& \equiv & \fbox{$16$}
\end{array}
\]
%
where the modulus for each of the congruences is 23.
}

\eparts
\end{problem}

%%%%%%%%%%%%%%%%%%%%%%%%%%%%%%%%%%%%%%%%%%%%%%%%%%%%%%%%%%%%%%%%%%%%%
% Problem ends here
%%%%%%%%%%%%%%%%%%%%%%%%%%%%%%%%%%%%%%%%%%%%%%%%%%%%%%%%%%%%%%%%%%%%%

\documentclass[problem]{mcs}

\begin{pcomments}
  \pcomment{TP_cardinality_class + TP_uncountable_example}
  \pcomment{prepared for midterm 2, Fall15}
  \pcomment{Zoran Dzunic 10/10/15}
\end{pcomments}

%%%%%%%%%%%%%%%%%%%%%%%%%%%%%%%%%%%%%%%%%%%%%%%%%%%%%%%%%%%%%%%%%%%%%
% Problem starts here
%%%%%%%%%%%%%%%%%%%%%%%%%%%%%%%%%%%%%%%%%%%%%%%%%%%%%%%%%%%%%%%%%%%%%
\begin{problem}

\bparts

\ppart
For each of the following sets, indicate whether it is
finite\inhandout{ (\textbf{F})},
countably infinite\inhandout{ (\textbf{C})},
or uncountable\inhandout{ (\textbf{U})}.

\renewcommand{\theenumi}{\roman{enumi}}
\renewcommand{\labelenumi}{(\theenumi)}

\begin{enumerate}%{i}

%\item \label{itm:roots} The set of solutions to the equation $x^3 - x
%  = -0.1$. \hfill\examrule[0.4in]

%\item \label{itm:naturals} The set of natural numbers $\naturals$. \instatements{\hfill\examrule[0.4in]}

%\item \label{itm:rationals} The set of rational numbers $\rationals$. \instatements{\hfill\examrule[0.4in]}

%\item \label{itm:reals} The set of real numbers $\reals$. \instatements{\hfill\examrule[0.4in]}

\item %\label{itm:ints}
%The set of even integers greater than $10^{100}$.\hfill\examrule[0.4in]
$\integers^2$ \hfill\examrule[0.4in]

%\item \label{itm:char} The set of words in the English language no more
%  than 20 characters long. \instatements{\hfill\examrule[0.4in]}

\item %\label{itm:self_bij}
The powerset of the integer interval $\Zintvcc{10}{10^{10}}$.\hfill\examrule[0.4in]
%  bijections from $\set{1,2,\dots,10}$ to itself.

\item %\label{itm:roots}
% The complex numbers $c$ such that $\exists m,n \in \integers.\,
% (m+nc)c = 0$. \hfill\examrule[0.4in]
The rational numbers in the interval $\Zintvcc{0}{1}$. \hfill\examrule[0.4in]

\item %\label{itm:complex}
%The set of ``pure'' complex numbers of the form $ri$ for nonzero real
%numbers $r$.\hfill\examrule[0.4in]
The set of rational numbers of the form $\frac{m}{n}$, where $m$ and $n$ are integers
whose product is a 4-digit positive number.\hfill\examrule[0.4in]

\medskip Let $\mathcal{F}$ be a finite set and $\mathcal{U}$ be an uncountable set.

\item %\label{itm:unionuc}
 $\mathcal{F} \union \mathcal{U}$. \hfill\examrule[0.4in]

\item %\label{itm:interuc}
$\mathcal{F} \intersect \mathcal{U}$\hfill\examrule[0.4in]

\item %\label{itm:diffuc}
$\mathcal{F} - \mathcal{U}$ \hfill\examrule[0.4in]

%\item \label{itm:union_count_uncount} A set $A \union B$ where $A$ is
%  countable and $B$ is uncountable.  \instatements{\hfill\examrule[0.4in]}

\end{enumerate}

\begin{solution}

\begin{enumerate}

\item %\label{itm:ints}
\inhandout{ (\textbf{C})}

\item %\label{itm:complex}
\inhandout{ (\textbf{F})}

\item %\label{itm:self_bij}
\inhandout{ (\textbf{C})}

\item %\label{itm:roots}
\inhandout{ (\textbf{F})}

\item %\label{itm:unionuc}
\inhandout{ (\textbf{U})}

\item %\label{itm:interuc}
\inhandout{ (\textbf{F})}

\item %\label{itm:diffuc}
\inhandout{ (\textbf{F})}

\end{enumerate}

\iffalse
\begin{center}
\begin{tabular}{c |c | c}
Finite & Countably infinite & Uncountable \\ \hline
\ref{itm:roots}  &  \ref{itm:ints}  & \ref{itm:complex}\\
\ref{itm:self_bij} & \ref{itm:total_surj} &
\end{tabular}
\end{center}
\fi

\end{solution}

\examspace[0.1in]

\ppart
Describe sets $A$ and $B$ such that
\[
A \strict \reals \strict B.
\]
%Recall that $A \strict B$ means that $A$ is not ``as big as'' $B$.

\begin{solution}
Let $A$ be $\integers$ and $B$ be $\power(\reals)$.
\end{solution}

%\examspace[2.0in]

\eparts

\end{problem}

%%%%%%%%%%%%%%%%%%%%%%%%%%%%%%%%%%%%%%%%%%%%%%%%%%%%%%%%%%%%%%%%%%%%%
% Problem ends here
%%%%%%%%%%%%%%%%%%%%%%%%%%%%%%%%%%%%%%%%%%%%%%%%%%%%%%%%%%%%%%%%%%%%%
\endinput

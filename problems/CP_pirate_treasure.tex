\documentclass[problem]{mcs}

\begin{pcomments}
  \pcomment{CP_pirate_treasure}
  \pcomment{first used SP12}
  \pcomment{revised by drewe 2/21/2012}
  \pcomment{soln revised by ARM 3/6/12}
  \pcomment{part (b) is a divisibility state machine problem ala Die
    Hard that does not depend on congruences  --ARM 3/6/12}
  \pcomment{without a hint, part(a) requires some pigeon-hole
    ingenuity.  Makes it better suited for a pset. --ARM 3/6/12}
\end{pcomments}

\pkeywords{
  number_theory
  modular_arithmetic
  invariant
  divisible
  prime
}

%%%%%%%%%%%%%%%%%%%%%%%%%%%%%%%%%%%%%%%%%%%%%%%%%%%%%%%%%%%%%%%%%%%%%
% Problem starts here
%%%%%%%%%%%%%%%%%%%%%%%%%%%%%%%%%%%%%%%%%%%%%%%%%%%%%%%%%%%%%%%%%%%%%

\begin{problem}

\bparts

\ppart Ten pirates find a chest filled with gold and silver coins.
There are twice as many silver coins in the chest as there are gold.
They divide the gold coins in such a way that the difference in the
number of coins given to any two pirates is not divisible by $10$.
They will only take the silver coins if it is possible to divide them
the same way.  Is this possible, or will they have to leave the silver
behind?  Prove your answer.

\begin{staffnotes}
Give hint: Let $g_1,g_2,\dots, g_{10}$ be the number of gold coins
given to the 10 pirates.  Then no two of the $g_i$'s are $\equiv
\pmod{10}$.
\end{staffnotes}

\begin{solution}
Let $g_1,g_2,\dots, g_{10}$ be the number of gold coins given to the
10 pirates.  No two of the $g_i$'s are $\equiv \pmod{10}$, so these
numbers must be $\equiv \pmod{10}$ to all ten values in $[0,10)$.  So
  the number of gold coins is $\equiv 0+1+2+\ldots+9 = 45 \equiv 5
  \pmod{10}$.  The same reasoning implies that if the silver coins
  are to be divided up under the same constraints, then the number of
  silver coins must be $\equiv 5 \pmod {10}$.  But the number of
  silver coins is twice the number of gold, and so is $\equiv 2 \cdot
  5 \equiv 0 \not\equiv 5 \pmod{10}$, so the silver coins cannot be
  divided up in the same way.
\end{solution}

\ppart There are also 3 sacks in the chest, containing 5, 49, and 51
rubies respectively.  The treasurer of the pirate ship is bored and
decides to play a game with the following rules:
\begin{itemize}
\item He can merge any two piles together into one pile, and
\item he can divide a pile with an even number of rubies into two
  piles of equal size.
\end{itemize}
He makes one move every day, and he will finish the game when he has
divided the rubies into 105 piles of one.  Is it possible for him to
finish the game?

\begin{solution}
It isn't possible for him to create a pile of size 1, and so he can't
finish.

The initial piles are of odd sizes $(5,49,51)$, so the only possible
move is to merge two piles.  There are three choices for the pair of
piles to merge, leading to two piles that are of sizes $(54, 51)$,
$(49, 56)$, or $(5,100)$.

Now suppose there are a bunch of piles whose sizes are all divisible
by some odd prime $p$.  Then merging two piles yields a pile whose
size is still divisible by $p$.  Also, splitting a pile whose size is
even leaves two half-size piles of sizes divisible by $p$.  In other
words, having all piles of sizes divisible by $p$ is a \idx{preserved
  invariant} of moves in the game.

But the sizes $54$ and $51$ in the first pair above are both divisible
by 3, and likewise the second pair are divisible by 7 and the third by
5.  So from each of these pairs, no sequence of moves can yield a pile
of size 1.

\end{solution}

\eparts
\end{problem}

%%%%%%%%%%%%%%%%%%%%%%%%%%%%%%%%%%%%%%%%%%%%%%%%%%%%%%%%%%%%%%%%%%%%%
% Problem ends here
%%%%%%%%%%%%%%%%%%%%%%%%%%%%%%%%%%%%%%%%%%%%%%%%%%%%%%%%%%%%%%%%%%%%%


\endinput

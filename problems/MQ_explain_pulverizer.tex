\documentclass[problem]{mcs}

\begin{pcomments}
  \pcomment{MQ_explain_pulverizer}
  \pcomment{ARM 3/3/14}
\end{pcomments}

\pkeywords{
  gcd
  linear_combinations
  Pulverizer
  number_theory
}

%%%%%%%%%%%%%%%%%%%%%%%%%%%%%%%%%%%%%%%%%%%%%%%%%%%%%%%%%%%%%%%%%%%%%
% Problem starts here
%%%%%%%%%%%%%%%%%%%%%%%%%%%%%%%%%%%%%%%%%%%%%%%%%%%%%%%%%%%%%%%%%%%%%

\begin{problem}

\mbox{}

\bparts

\ppart Given inputs $m,n \in \integers^+$, the Pulverizer will produce
$x,y \in \integers$ such that:

\begin{solution}
\[
xm+yn = \gcd(m,n)
\]

\end{solution}

\examspace[1.5in]

\ppart Assume $n>1$.  Explain how to use the numbers $x,y$ to find
the inverse of $m$ modulo $n$ when there is an inverse.

\begin{solution}
There is an inverse for $m$ iff $\gcd(m,n)=1$, in which case $x$ is an
inverse of $m$ modulo $n$, and $\rem{x}{n}$ is the inverse in
the interval $\Zintvco{1}{n}$.
\end{solution}

\examspace[1.0in]

\eparts

\end{problem}


%%%%%%%%%%%%%%%%%%%%%%%%%%%%%%%%%%%%%%%%%%%%%%%%%%%%%%%%%%%%%%%%%%%%%
% Problem ends here
%%%%%%%%%%%%%%%%%%%%%%%%%%%%%%%%%%%%%%%%%%%%%%%%%%%%%%%%%%%%%%%%%%%%%

\endinput

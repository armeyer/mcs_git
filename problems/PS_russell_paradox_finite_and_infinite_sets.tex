\documentclass[problem]{mcs}

\begin{pcomments}
  \pcomment{from: S06.ps1}
\end{pcomments}

\pkeywords{
  surjection
  russell's paradox
}

%%%%%%%%%%%%%%%%%%%%%%%%%%%%%%%%%%%%%%%%%%%%%%%%%%%%%%%%%%%%%%%%%%%%%
% Problem starts here
%%%%%%%%%%%%%%%%%%%%%%%%%%%%%%%%%%%%%%%%%%%%%%%%%%%%%%%%%%%%%%%%%%%%%

\begin{problem}  %verbatim from fall 05 ps1:

If a set, $A$, is finite, then $\card{A} < 2^{\card{A}} =
\card{\power(A)}$, and so there is no surjection from set $A$ to its
powerset.  Show that this is still true if $A$ is infinite.  \hint Remember
Russell's paradox and consider $\set{x \in A \suchthat x
\notin f(x)}$ where $f$ is such a surjection.

\solution{ We prove there is no surjection by contradiction: suppose there
was a surjection $f:A \to \power(A)$ for some set $A$.  Consider the
set of all elements of $A$ that do \emph{not} belong to their image under $f$:
\[
W \eqdef \set{x \in A \suchthat x \notin f(x)} .
\]  
By the definition of this set, we know that for all $x \in A$:
\begin{equation}\label{xw}
(x \in W) \iff (x \notin f(x)) .
\end{equation}
But $W \subseteq A$ (by the definition of $W$), and hence $W$ is a member
of $\power(A)$.  Since $f$ is a surjection onto $\power(A)$,  
this means that $W = f(a)$ for some $a \in A$.  So we have from~\eqref{xw}, that
\begin{equation}\label{xf}
(x \in f(a)) \iff (x \notin f(x))
\end{equation}
for all $x \in A$.  Substituting $a$ for $x$ in~\eqref{xf} yields a
contradiction, proving that there cannot be such an $f$.}

\end{problem}

%%%%%%%%%%%%%%%%%%%%%%%%%%%%%%%%%%%%%%%%%%%%%%%%%%%%%%%%%%%%%%%%%%%%%
% Problem ends here
%%%%%%%%%%%%%%%%%%%%%%%%%%%%%%%%%%%%%%%%%%%%%%%%%%%%%%%%%%%%%%%%%%%%%
\documentclass[problem]{mcs}

\begin{pcomments}
  \pcomment{CP_inclusion-exclusion_algebra_proof}
  \pcomment{from: ARM new 11/15/09 based on S09 slides}
  \pcomment{revised ARM 5/16/18}
\end{pcomments}

\pkeywords{
  inclusion-exclusion
  product
  sum
  subset
  algebra
}

%%%%%%%%%%%%%%%%%%%%%%%%%%%%%%%%%%%%%%%%%%%%%%%%%%%%%%%%%%%%%%%%%%%%%
% Problem starts here
%%%%%%%%%%%%%%%%%%%%%%%%%%%%%%%%%%%%%%%%%%%%%%%%%%%%%%%%%%%%%%%%%%%%%

\begin{problem}
Let's develop a proof of the Inclusion-Exclusion formula using high school
algebra.

\bparts

\ppart Most high school students will get freaked by the following formula,
even though they actually know the rule it expresses.  How would you
explain it to them?

\begin{equation}\label{1+xproda}
\prod_{i=1}^n \paren{1+x_i} = \sum_{I \subseteq \set{1,\dots,n}} \prod_{j \in I}x_j.
\end{equation}
\hint Show them an example.

\begin{solution}
Let's do an example.  To ``multiply out''
%\begin{equation}\label{x3prod}
\[
(1+x_1)(1+x_2)(1+x_3),
\]
%\end{equation}
you would form \emph{\idx{monomial}} products by selecting some of the
$x_i$'s to multiply together.  For example, selecting $x_i$'s with
\begin{itemize}
\item $i \in \set{1,3}$ leads to the monomial $x_1x_3$,
%\item $i \in \set{2}$ leads to the monomial $x_2$,
\item $i \in \set{1,2,3}$ leads to the monomial $x_1x_2x_3$, and
\item $i \in \emptyset$ leads (by convention) to the monomial $1$.
\end{itemize}
Then you sum up the monomials from \emph{all possible} selections to get
\[
(1+x_1)(1+x_2)(1+x_3) = 1 + x_1 + x_2 + x_3 + x_1x_2 + x_1x_3 + x_2x_3 + x_1x_2x_3.
\]

Now we can decipher~\eqref{1+xproda} as saying to do the same thing for
the product of $n$ different $(1+x_i)$'s:  for any selection of $x_i$'s
with $i$ in some subset $I \subseteq \set{1,\dots,n}$, multiply the
$x_i$'s to get the monomial
\[
\prod_{i \in I} x_i,
\]
and sum up all such monomials obtained by every possible selection $I$
to get the right-hand side of equation~\eqref{1+xproda}.
\end{solution}

\eparts

\bigskip

Now let $S_1,S_2,\dots,S_n$ be a sequence of finite sets, and let $U
\def \lgunion_{i=1}^n S_i$ be their union.  The Inclusion-Exclusion rule~\bref{incexII} says
\begin{equation}\tag{I-E}
\Card{U}
   =  \sum_{\emptyset \neq I \subseteq \set{1,\dots,n}} (-1)^{\card{I}+1} \Card{\lgintersect_{i \in I} S_i}
\end{equation}

Now to start proving, let $M_S$ be the
\emph{membership} function for any set $S$:
\[
M_S(x) = \begin{cases}
         1 & \text{if $x \in S$},\\
         0 & \text{if $x \notin S$},
         \end{cases}
\]
and abbreviate $M_{S_i}$ as $M_i$.  Set complements are defined relative to $U$, that is,
\[
\setcomp{T} \eqdef U - T,
\]
for $T \subseteq U$.

\bparts

\ppart\label{Mprops}
Verify that for $T \subseteq U$ and $I \subseteq \Zintv{1}{n}$
\begin{align}
M_{\setcomp{T}}  & = 1- M_T, \label{Tbar1-T}\\
M_{\paren{\lgintersect_{i \in I} S_i}} & = \prod_{i \in I}M_i, \label{intersectSs}\\
M_{\paren{\lgunion_{i \in I} S_i}} & = 1- \prod_{i\in I}(1-M_i). \label{1-Mprod}
\end{align}
Note that~\eqref{intersectSs} holds when $I$ is empty because, by
convention, an empty product equals 1, and an empty intersection
$\lgintersect_{i \in \emptyset} S_i$ by definition equals $U$.

\begin{solution}
For all $u \in U$,
\[
M_{\setcomp{T}}(u) = 1 \QIFF u \in \setcomp{T} \QIFF M_T(u) = 0 \QIFF 1-M_T(u) = 1,
\]
which proves~\eqref{Tbar1-T}.

\iffalse
M_{\setcomp{T}}(u) = 0 & \QIFF u \notin \setcomp{T} \QIFF u \in T \QIFF M_T(u) = 1 \QIFF 1-M_T(u) = 0
\fi

Similarly, to prove~\eqref{intersectSs},
\begin{align*}
\lefteqn{M_{\paren{\lgintersect_{i \in I} S_i}}(u) = 1}\\
&  \QIFF u \in \lgintersect_{i \in I} S_i\\
&  \QIFF \LGQAND_{i \in I} [u \in S_i]\\
&  \QIFF \LGQAND_{i \in I} [M_i(u) = 1]\\
&  \QIFF \prod_{i \in I} M_i(u) = 1.
\end{align*}

Finally,~\eqref{1-Mprod} follows from~\eqref{Tbar1-T}
and~\eqref{intersectSs} by DeMorgan's Law.
\end{solution}

\ppart Use~\eqref{1+xproda} and~\eqref{1-Mprod} to prove
\begin{equation}\label{MUsum}
M_U = \sum_{\emptyset \neq I \subseteq \set{1,\dots,n}}
             (-1)^{\card{I}+1} \prod_{j \in I}M_j.
\end{equation}
\begin{solution}

\begin{align*}
M_U & = M_{\paren{\lgunion_{i=1}^n S_i}}\\
    & = 1 - \prod_{i=1}^n (1-M_i)
         &\text{by~\eqref{1-Mprod}}\\
    & = 1 - \sum_{I \subseteq \set{1,\dots,n}}\prod_{j \in I}(-M_j)
        & \text{by~\eqref{1+xproda}}\\
    & = 1 - \sum_{I \subseteq \set{1,\dots,n}} (-1)^{\card{I}}\prod_{j \in I} M_j
        & \text{(factoring $-1$)}\\
    & = 1 - \paren{1 + \sum_{\emptyset \neq I \subseteq \set{1,\dots,n}}
             (-1)^{\card{I}} \prod_{j \in I} M_j}
         & \text{(empty product equals 1)}\\
    & = \sum_{\emptyset \neq I \subseteq \set{1,\dots,n}}
             (-1)^{\card{I}+1} \prod_{j \in I} M_j.
\end{align*}

\end{solution}

\ppart Prove that
\begin{equation}\label{}\label{cardTsum}
\card{T}  = \sum_{u \in U} M_T(u),
\end{equation}
for all $T \subseteq U$.

\begin{solution}
\[
\sum_{u \in U} M_T(u) = \sum_{u \in T} M_T(u) + \sum_{u \in \setcomp{T}} M_T(u) 
= \paren{\sum_{u \in T} 1} +  \paren{\sum_{u \in \setcomp{T}} 0}
= \card{T}+0 = \card{T}.
\]
\end{solution}

\ppart Now use
\iffalse~\eqref{intersectSs},~\eqref{MUsum} and~\eqref{cardTsum} \fi
the previous parts to prove equation~(I-E).
\iffalse
\begin{equation}\label{incexc-subsets}
\card{D}  = \sum_{\emptyset \neq I \subseteq \set{1,\dots,n}}
                  (-1)^{\card{I}+1} \Card{\lgintersect_{i \in I} S_i}
\end{equation}
\fi

\begin{solution}
Summing both sides of~\eqref{MUsum} over $u \in U$, we have
\begin{align*}
\card{U} & = \sum_{u \in U} M_U(u)
              & \text{(by~\eqref{cardTsum})}\\
         & = \sum_{u \in U} \paren{\sum_{\emptyset \neq I \subseteq \set{1,\dots,n}}
             (-1)^{\card{I}+1} \prod_{j \in I}M_j(u)}
              & \text{(by~\eqref{MUsum})}\\
         & = \sum_{u \in U}
                 \paren{\sum_{\emptyset \neq I \subseteq \set{1,\dots,n}}
                       (-1)^{\card{I}+1} M_{\lgintersect_{i \in I} S_i}(u)}
             & \text{(by~\eqref{intersectSs})}\\
         & = \sum_{\emptyset \neq I \subseteq \set{1,\dots,n}}
                  (-1)^{\card{I}+1} \paren{\sum_{u \in U}
                             M_{\lgintersect_{i \in I} S_i}(u)}
              & \text{(reversing the order of sums)}\\
         & = \sum_{\emptyset \neq I \subseteq \set{1,\dots,n}}
                  (-1)^{\card{I}+1} \Card{\lgintersect_{i \in I} S_i}
              & \text{(by~\eqref{cardTsum})}.
\end{align*}
\end{solution}

\ppart Finally, explain why~(I-E) immediately implies the
usual form of the Inclusion-Exclusion Principle:

\begin{equation}\label{incexc-n}
  \card{U} = \sum_{i=1}^n (-1)^{i+1}
             \sum_{\substack{I \subseteq \Zintv{1}{n}\\ \card{I}=i}}
                 \Card{\lgintersect_{j \in I} S_j}.
\end{equation}

\begin{solution}
  We obtain~\eqref{incexc-n} from~(I-E) by breaking up
  the sum over nonempty subsets $I \subseteq \Zintv{1}{n}$ into
  separate sums over all the subsets of size $i$, for $1 \leq i \leq n$.
\end{solution}

\eparts
\end{problem}


%%%%%%%%%%%%%%%%%%%%%%%%%%%%%%%%%%%%%%%%%%%%%%%%%%%%%%%%%%%%%%%%%%%%%
% Problem ends here
%%%%%%%%%%%%%%%%%%%%%%%%%%%%%%%%%%%%%%%%%%%%%%%%%%%%%%%%%%%%%%%%%%%%%

\endinput

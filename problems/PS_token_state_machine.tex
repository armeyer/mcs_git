\documentclass[problem]{mcs}

\begin{pcomments}
  \pcomment{PS_token_state_machine}
  \pcomment{Mild rewording of FP_token_state_machine by zabel, 9/29/17}
  \pcomment{Original problem: author: Zoran Dzunic, edited ARM 10/15/15}
\end{pcomments}

\pkeywords{
  state_machine
  invariant
  preserved_invariant
  induction
  congruence
}

%%%%%%%%%%%%%%%%%%%%%%%%%%%%%%%%%%%%%%%%%%%%%%%%%%%%%%%%%%%%%%%%%%%%%
% Problem starts here
%%%%%%%%%%%%%%%%%%%%%%%%%%%%%%%%%%%%%%%%%%%%%%%%%%%%%%%%%%%%%%%%%%%%%

\begin{problem}
\emph{Token replacing-1-2} is a single player game using a set of
tokens, each colored black or white.  Except for color, the tokens are
indistinguishable. \iffalse The game starts with one black token.\fi
In each move, a player can replace one black token with two white
tokens, or replace one white token with two black tokens.

We can model this game as a state machine whose states are pairs
$(n_b,n_w)$ where $n_b \geq 0$ equals the number of black tokens, and
$n_w \geq 0$ equals the number of white tokens.

\bparts
  
\ppart Which of the following predicates are preserved invariants? Why or why not?
\begin{align}
\rem{n_b + n_w}{3} &\neq 2\\
\rem{n_w - n_b}{3} &= 2\\
\rem{n_b - n_w}{3} &= 2\\
n_b+n_w & > 5 \\
n_b+n_w & < 5
\end{align}

\begin{solution}
Predicates 2, 3, and 4 are preserved invariants.

Detailed explanation:
\begin{enumerate}
\item \false.  True for $(1, 0)$, but not for the state $(0, 2)$
  that it transitions to.

\item \true.
Let $(n_b, n_w)$ be an arbitrary state for which $\rem{n_w - n_b}{3} = 2$.
The only two possible transitions are into state $(n'_b, n'_w) = (n_b - 1, n_w + 2)$
or state $(n'_b, n'_w) = (n_b + 2, n_w - 1)$.  Note that
$\rem{n'_w - n'_b}{3} = \rem{n_w - n_b \pm 3}{3} = \rem{n_w - n_b}{3} = 2$,
and thus the property is preserved.

\item \true.
Same reasoning as for predicate~(2).

\item \true.  Note that $n_b + n_w$ increases by 1 at each move.
  Thus, if $n_b + n_w > 5$, the same must hold for any state it can
  move to.

\item \false.  Predicate is true for state $(2, 2)$, but not for
  state $(1, 4)$ that it can move to.
\end{enumerate}
\end{solution}

\medskip Now assume the game starts with a single black token, that
is, the start state is $(1,0)$.

\ppart Which of the above predicates are true for all reachable states? Why or why not?

\begin{solution}
Predicate 2 is true for all reachable states.

Detailed explanation:
\begin{enumerate}

\item \false.
Not true for state $(0, 2)$ that is reachable:
$(1, 0) \rightarrow (0, 2)$.

\item \true.
True in the start state and is also a preserved invariant,
and is therefore true for all reachable states by the invariant principle.

\item \false.
Not true in the start state.

\item \false.
Not true in the start state.

\item \false.  Note that $n_b + n_w$ increases by 1 at each move,
so it is always false after the fourth move of any kind.
\end{enumerate}

\end{solution}

\ppart
Define the predicate $T(n_b,n_w)$ by the rule:
\[
T(n_b,n_w) \eqdef\ \rem{n_w - n_b}{3} = 2.
\]
We will now prove the following:

\begin{claim*}
If $T(n_b, n_w)$, then state $(n_b, n_w)$ is reachable.
\end{claim*}

Note that this claim is different from the claim that $T$ is a
preserved invariant.

Prove the claim by induction in $n$ using induction
hypothesis $P(n) \eqdef$
\[
  \forall (n_b, n_w).\,
  [(n_b + n_w = n) \QAND  T(n_b,n_w)]
  \QIMPLIES (n_b,n_w) \text{ is reachable}.
\]
%Use the states with $n \leq 2$ as base cases.

\begin{solution}
  \begin{proof} We follow the suggested induction, with the $n\le 2$ states as our base cases.

    \textbf{Base cases:} $n \le 2$. There are only six states with $n \leq 2$:
    \[
      (0,0), (1,0), (0,1), (1,1), (0,2), (2,0).
    \]
    Of these, only $T(1,0)$ and $T(0,2)$ hold. $(1,0)$ is reachable
    since it is the start state, and $(0,2)$ is reachable in one step from
    the start state.  So all the states with $n \leq 2$ and satisfying
    property $T$ are reachable.

    \textbf{Inductive Step:} 
    Assume that the induction hypothesis holds for some $n \ge 2$.
    Suppose $n_b + n_w = n+1$ and $T(n_b,n_w)$ holds.  We want to show
    that $(n_b, n_w)$ is reachable.

    Since $n+1 \geq 3$, either $n_b \geq 2$ or $n_w \geq 2$.

    In the case that $n_b \geq 2$, we have $n_b-2 \geq 0$, so
    $(n_b-2,n_w+1)$ is a state. Also, $T(n_b-2,n_w+1)$ holds because
    \[
      \rem{(n_b-2)-(n_w+1)}{3} = \rem{n_b-n_w -3}{3} = \rem{n_b-n_w}{3} = 2.
    \]
    Since
    \[
      (n_b-2)+(n_w+1) = n_b+n_w - 1 = n,
    \]
    we conclude by induction hypothesis $P(n)$ that $(n_b-2,n_w+1)$ is
    reachable.  But $(n_b-2,n_w+1)$ transitions in one step to
    $(n_b,n_w)$, which proves that $(n_b,n_w)$ is reachable.

    The same argument applies in the case that $n_w \geq 2$.

    We conclude that in any case $(n_b,n_w)$ is reachable, which completes
    the induction step.
  \end{proof}

  To understand the difference between this problem and proving that $T$
  is a preserved invariant, it may be helpful to think about the problem
  of making different postage amounts, for example using 6 and 15 cent
  stamps\inbook{ (Problem~\bref{FP_6_and_15_cent_stamps_by_WOP})}.  An
  easy invariant proof confirms the fact that every makeable postage
  amount is divisible by three.  This preserved invariant for postage
  corresponds to invariance of $T$ for reachable token machine states.

  A very different fact is that \emph{every} (large enough) amount of
  postage that is divisible by three is actually makeable from 6 and 15
  cent stamps.  Postage \emph{makeability} for amounts divisible by
  three corresponds to \emph{reachability} of token machine states
  satisfying $T$.
\end{solution}



\eparts

\end{problem}

%%%%%%%%%%%%%%%%%%%%%%%%%%%%%%%%%%%%%%%%%%%%%%%%%%%%%%%%%%%%%%%%%%%%% 
% Problem ends here
%%%%%%%%%%%%%%%%%%%%%%%%%%%%%%%%%%%%%%%%%%%%%%%%%%%%%%%%%%%%%%%%%%%%% 

\endinput

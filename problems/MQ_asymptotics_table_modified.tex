\documentclass[problem]{mcs}

\begin{pcomments}
  \pcomment{MQ_asymptotics_table_modified}
  \pcomment{same idea as: MQ_asymptotics_table}
  \pcomment{from: S09.mq5}
\end{pcomments}

\pkeywords{
  asymptotics
}

%%%%%%%%%%%%%%%%%%%%%%%%%%%%%%%%%%%%%%%%%%%%%%%%%%%%%%%%%%%%%%%%%%%%%
% Problem starts here
%%%%%%%%%%%%%%%%%%%%%%%%%%%%%%%%%%%%%%%%%%%%%%%%%%%%%%%%%%%%%%%%%%%%%

\begin{problem}

For each pair of functions, $f:\nngint^+\to\nngint$ and
$g:\nngint^+\to\nngint$, in the table below, indicate which of the
listed \idx{asymptotic relations} hold \textbf{and} which do not.
(Fill \textbf{every} cell in the table.  You may use checkmarks and
crosses, ``T'' and ``F'', ``TRUE'' and ``FALSE'', ``Y'' and ``N'', or
``YES'' and ``NO''.)

\[
\begin{array}{|c|c|c|c|c|c|}
\hline
f(n) & g(n) & f = O(g) & f = o(g) & g = O(f) & g = o(f) \\
\hline\hline
\log_4{n} & \sqrt[3]{n} & & & & \\ \hline
n^2 + 3^n & n^3 + 2^n & & & & \\ \hline
n\ln{n!} & n^2 \log_{10}{n^2} & & & & \\ \hline
n^{2\cos{(\pi n/2)}+3} & 5n^5+3n^3+n & & & & \\ \hline
\end{array}
\]

\begin{solution}

%[\vspace{3in}]
\[
\begin{array}{|c|c|c|c|c|c|}
\hline
f(n) & g(n) & f = O(g) & f = o(g) & g = O(f) & g = o(f) \\
\hline\hline
\log_4{n} & \sqrt[3]{n} & YES & YES & NO & NO \\ \hline
n^2 + 3^n & n^3 + 2^n & NO & NO & YES & YES \\ \hline
n\ln{n!} & n^2 \log_{10}{n^2} & YES & NO & YES & NO \\ \hline
n^{2\cos{(\pi n/2)}+3} & 5n^5+3n^3+n & YES & NO & NO & NO \\ \hline
\end{array}
\]

\textbf{Justification:}
\[
\begin{array}{|c|c|c|c|c|c|}
\hline
f(n) & g(n) & f = O(g) & f = o(g) & g = O(f) & g = o(f) \\
\hline\hline
\log_4{n} & \sqrt[3]{n} & YES & YES & NO & NO \\ \hline
\end{array}
\]

Using either (1) l'H\^{o}pital's Rule or (2) the fact that 
$\log{n}=o\paren{n^\epsilon}$ for all $\epsilon > 0$ (see the Notes), conclude that
$f = o(g)$.  This implies that $f = O(g)$, $g \neq o(f)$, and $g \neq O(f)$.

\[
\begin{array}{|c|c|c|c|c|c|}
\hline
f(n) & g(n) & f = O(g) & f = o(g) & g = O(f) & g = o(f) \\
\hline\hline
n^2 + 3^n & n^3 + 2^n & NO & NO & YES & YES \\ \hline
\end{array}
\]

Intuitively, $3^n$ grows far faster than $n^2$ and $2^n$ grows far faster
than $n^3$, as $n$ grows large.  (Any power of $n$ is asymptotically
smaller than any increasing exponential in $n$.)  Also, $3^n$ grows far faster than
$2^n$.  (Given two increasing exponentials, the one with the smaller base will
be asymptotically smaller.)  A bit more rigorously,
\begin{align*}
\lim_{n\to\infty}{\frac{g(n)}{f(n)}} &= \lim_{n\to\infty}{\frac{n^3 + 2^n}{n^2 + 3^n}} \\
                                     &= \lim_{n\to\infty}{\frac{\frac{n^3}{3^n} + \paren{\frac{2}{3}}^n}{\frac{n^2}{3^n} + 1}} \\
                                     &= \frac{\lim_{n\to\infty}{\frac{n^3}{3^n}} + \lim_{n\to\infty}{\paren{\frac{2}{3}}^n}}{\lim_{n\to\infty}{\frac{n^2}{3^n}} + \lim_{n\to\infty}{1}} \\
                                     &= \frac{0 + 0}{0 + 1} \\
                                     &= 0
\end{align*}
Where $\displaystyle \lim_{n\to\infty}{\frac{n^3}{3^n}}$ and  
$\displaystyle \lim_{n\to\infty}{\frac{n^2}{3^n}}$ can be found to be zero by l'H\^{o}pital's Rule, and $\displaystyle \lim_{n\to\infty}{\paren{\frac{2}{3}}^n}$ is 
zero because $\abs{\frac{2}{3}}<1$.  Thus $g = o(f)$, which implies $g = O(f)$, $f \neq o(g)$, and $f \neq O(g)$.

\[
\begin{array}{|c|c|c|c|c|c|}
\hline
f(n) & g(n) & f = O(g) & f = o(g) & g = O(f) & g = o(f) \\
\hline\hline
n\ln{n!} & n^2 \log_{10}{n^2} & YES & NO & YES & NO \\ \hline
\end{array}
\]

Using Stirling's formula, $\displaystyle n! \sim \sqrt{2\pi n} 
\paren{\frac{n}{e}}^n$, it is easy to show that $\ln{n!} \sim n\ln{n}$ 
and hence that $f(n) \sim n^2\ln{n}$.  Now, 
\begin{align*}
n^2 \log_{10}{n^2}&=2n^2 \log_{10}{n}\\
                  &=2n^2\frac{\ln{n}}{\ln{10}}  
\end{align*}
It should be evident now that $g(n) \sim \frac{2}{\ln{10}}f(n)$.
Hence $f \neq o(g)$ and $g \neq o(f)$, but $f = O(g)$ and $g = O(f)$.

\[
\begin{array}{|c|c|c|c|c|c|}
\hline
f(n) & g(n) & f = O(g) & f = o(g) & g = O(f) & g = o(f) \\
\hline\hline
n^{2\cos{(\pi n/2)}+3} & 5n^5+3n^3+n & YES & NO & NO & NO \\ \hline
\end{array}
\]

Notice that
\[ f(n) = \left\{ \begin{array}{ll}
         n^5 & \mbox{if $n \equiv 0 \pmod{4}$}\\
         n^3 & \mbox{if $n \equiv 1 \pmod{4}$ or $n \equiv 3 \pmod{4}$}\\
         n   & \mbox{if $n \equiv 2 \pmod{4}$}.\end{array} \right.
\] 
Because $f(n)$ is thus clearly bounded above by $n^5$ and $g(n)$ is a
polynomial of degree 5, have $f=O(g)$.  The behavior of $f(n)$ when
$n$ is not a multiple of 4 leads to $g \neq O(f)$.  But
$\lim_{n\to\infty}{\frac{f(n)}{g(n)}}$ and
$\lim_{n\to\infty}{\frac{g(n)}{f(n)}}$ are both nonzero, so $f \neq
o(g)$ and $g \neq o(f)$.
\end{solution}


\end{problem}


%%%%%%%%%%%%%%%%%%%%%%%%%%%%%%%%%%%%%%%%%%%%%%%%%%%%%%%%%%%%%%%%%%%%%
% Problem ends here
%%%%%%%%%%%%%%%%%%%%%%%%%%%%%%%%%%%%%%%%%%%%%%%%%%%%%%%%%%%%%%%%%%%%%

\endinput

\documentclass[problem]{mcs}

\begin{pcomments}
  \pcomment{PS_probabilistic_proof_definite}
  \pcomment{Specialization of PS_probabilistic_proof to a fixed $k$ value}
  \pcomment{\textbf{Random Tournaments}}
  \pcomment{from S01.tut12}
  \pcomment{added to F02 repository by Tina Wang, S02}
  \pcomment{edited by ARM 5/18/12}
\end{pcomments}

\pkeywords{
  probability
  probabilistic_method
  Boole
  Booles_inequality
  union_bound
  tournament
}

%%%%%%%%%%%%%%%%%%%%%%%%%%%%%%%%%%%%%%%%%%%%%%%%%%%%%%%%%%%%%%%%%%%%%
% Problem starts here
%%%%%%%%%%%%%%%%%%%%%%%%%%%%%%%%%%%%%%%%%%%%%%%%%%%%%%%%%%%%%%%%%%%%%

                                                                         
\begin{problem}           
The results of a round robin tournament in which every two people play
each other and one of them wins can be modelled a \term{tournament
  digraph}---a digraph with exactly one directed edge between each pair of
distinct vertices. We'll draw a directed edge $\diredge{v}{w}$ if player $v$ beats player $w$, and otherwise we'll include directed edge $\diredge{w}{v}$.

An $n$-player tournament is \emph{$k$-neutral} for some $k \in [0,n)$,
  when, for every set of $k$ players, there is another player who
  beats them all.  For example, being 1-neutral is the same as not
  having a ``best'' player who beats everyone else.

  This problem will prove the existence of an $n$-player tournament that is $10$-neutral, if $n$ is large enough. We will do
  this by reformulating the question in terms of probabilities.  In
  particular, for any fixed $n$, we assign probabilities to each
  $n$-vertex tournament digraph by choosing a direction for the edge
  between any two vertices, independently and with equal probability for
  each edge.

\begin{problemparts}

\problempart\label{prbsn} For any set $S$ of $10$ players, let $B_S$ be
the event that no contestant beats everyone in $S$.  Express
$\prob{B_S}$ in terms of $n$.

\examspace[1in]

\begin{solution}
The probability that a player outside $S$ beats everyone in $S$ is
$(1/2)^{10}$.  So the probability such a player did not beat everyone in
the group is $1-(1/2)^{10}$.  There are $n-10$ players outside of the
group, so
\[
\pr{B_S} = \brac{1-(1/2)^{10}}^{n-10} \,.
\]
\end{solution}

\problempart\label{qkalpha} Let $Q$ be the event that the
tournament digraph is \emph{not} $10$-neutral.  Prove that
\[
\prob{Q} \leq \binom{n}{10} \alpha^{n-10},
\]
where $\alpha \eqdef 1 - (1/2)^{10}$.

\hint Let $S$ range over the size-$10$ subsets of players, so
\[
Q = \lgunion_S B_S \, .
\]
Use Boole's inequality.

\examspace[1.5in]

\begin{solution}

\begin{align*}
\prob{Q} 
 & =\prob{\lgunion_{S} B_S} & \text{(hint)}\\
 & \leq \sum_{S} \prob{B_S} & \text{(Boole's Inequality)}\\
 & = \card{\set{S \suchthat \card{S} = 10}}\cdot \prob{B_S}\\
 & = \binom{n}{10} \alpha^{n-10} & \text{(part~\eqref{prbsn})}
\end{align*}

\end{solution}

\ppart Conclude that if $n$ is large enough, then
$\pr{Q} < 1$.

\hint Show that the limit as $n$ approaches infinity is $0$. Why is this sufficient?

\examspace[2.0in]

\begin{solution}
\begin{align}
\pr{Q}
   & \leq \binom{n}{10}\alpha^{n-10}
       & \text{(by part~\eqref{qkalpha})}\nonumber\\
   & = (1/\alpha)^{10} \binom{n}{10}\alpha^n\,.\label{prob-Q}
\end{align}
But this value approaches $0$ as $n$ approaches infinity. To see why, note that $0 < \alpha < 1$, so $1/\alpha >
1$ and hence $\binom{n}{10} = \Theta(n^{10}) = o((1/\alpha)^n)$ by Corollary~\bref{xbax}.

Because the value of~\eqref{prob-Q} approaches $0$ as $n\to\infty$, it must be less than $1$ when $n$ is large enough.
\end{solution}

\problempart Explain why the previous result implies that there is an $n$-player $10$-neutral tournament (for a large enough $n\in\nngint$).

\examspace[1in]

\begin{solution}
Suppose $n$ is large enough that $\pr{Q} < 1$.  Then $\pr{\bar{Q}}
> 0$, which implies there must be at least one tournament graph
(outcome) in $\bar{Q}$, that is, at least one tournament graph that
is $10$-neutral. In fact, since $\lim_{n\to\infty} \prob{Q} = 0$, it follows that almost all $n$-player tournaments will be
$10$-neutral for large enough $n$.

Note: The number $10$ is not special. The same argument shows that for any fixed $k$, almost all $n$-player tournaments are $k$-neutral if $n$ is large enough (relative to $k$). Some small $(k,n)$ pairs that are known to work are:
\[
(1,3), (2,21), (3,33), (4,46), (5,59), (6,72), (7,85), (8,98)
\]
\end{solution}

\end{problemparts}

\end{problem}
                                 

%%%%%%%%%%%%%%%%%%%%%%%%%%%%%%%%%%%%%%%%%%%%%%%%%%%%%%%%%%%%%%%%%%%%%
% Problem ends here
%%%%%%%%%%%%%%%%%%%%%%%%%%%%%%%%%%%%%%%%%%%%%%%%%%%%%%%%%%%%%%%%%%%%%

\endinput

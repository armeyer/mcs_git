\documentclass[problem]{mcs}

\begin{pcomments}
    \pcomment{TP_RSA_reversed}
    \pcomment{draft by David Shi, rewritten by ARM 10/11/11}
    \pcomment{Was PS_ but too light weight---ARM 3/3/13}
    \pcomment{subsumed by MQ_RSA_reversed}
\end{pcomments}

\pkeywords{
  RSA
  encryption
  signature
}

\begin{problem}
Using the RSA encryption system, \emph{P}ete the \emph{p}ublisher
generates a private key $(d,n)$ and posts a public key $(e,n)$ which
anyone can use to send encrypted messages to Pete.

RSA has the useful property that these same keys can switch roles: if
Pete wants to broadcast an unforgeable ``signed'' message, he can
encrypt his message using his private key as though it was someone's
public key.  That is, from a plain text $m \in [0,n)$, Pete would
  broadcast a ``signed'' version $s \eqdef \rem{m^d}{n}$.

Then anyone can decrypt and read Pete's broadcast message
by using Pete's public key as though it were their own private key.
Readers of Pete's message can be sure the message came from Pete if
they believe that the only way to generate such a message is by using
the private key which Pete alone knows.  (This belief is widely
accepted, but not certain.)
\bparts

\ppart\label{sfromm} Explain exactly what calculation must be
performed on $s$ to recover $m$ using the public key $(e,n)$.

\begin{solution}
\[
m = \rem{s^e}{n}.
\]
\end{solution}

\ppart Explain why the calculation of part~\eqref{sfromm} yields the
plain text $m$.

\begin{solution}
The requirement for RSA decryption, namely that $de \equiv 1
\pmod{(p-1)(q-1)}$, is symmetric in $d$ and $e$, so reversing their
roles allows the private key $d$ to be used to encrypt and the public
key $e$ to decrypt.  See Problem~\bref{CP_RSA_proving_correctness} for
the detailed justification of the decryption method.
\end{solution}

\eparts

\end{problem}

\endinput


\iffalse
We write ``$\equiv$'' for  ``$\equiv p\; mod \; n$:''
\begin{align*}
s^e & \equiv (m^d)^e & \text{($s \equiv m^d$ by def)}\\
    & \equiv  m^{de} & ((m^d)^e = m^{de})\\
    & \equiv m^1 \equiv m
        & \text{by Euler's theorem since $de \equiv 1 \phi(n)$}
\end{align*}    
Since $m \in [0,n)$, it follows that $m = \rem{s^e}{n}$.\fi


\iffalse

\bparts

Consider communication between server Sierra and client Charlie where
Sierra generates the public key $(e,n)$ and the private key $(d,n)$
for RSA.  When Charlie wants to send Sierra a message $m$ secretly,
Charlie can send the encrypted message $\widehat{m}=\rem{m^{e}}{n}$ to
Sierra.  Sierra can reproduce $m$ from $\widehat{m}$ by computing
$m=\rem{\widehat{m}^{d}}{n}$ and no one else can read $\widehat{m}$
because they do not know the private key.  However, since the public
key is known to everyone, a third party opponent Oscar can pretend to
be Charlie and send Sierra a message encrypted by the public key.  In
this problem we will see RSA can also prevent forged messages when the
keys are used in reverse.

\ppart Using the same keys Sierra had generated for RSA, consider when
Sierra wants to send Charlie a message $m$.  Sierra can encrypts $m$
using the private key $(d, n)$ to produce $\widehat{m}=\rem{m^{d}}{n}$
and sends $\widehat{m}$ to Charlie.  Explain how Charlie can reproduce
the original message $m$ with only $\widehat{m}$ and $(e,n)$ and why
this method will return $m$.

\begin{solution}
Charlie can reproduce $m$ by computing $m=\rem{\widehat{m}{e}}{n}$. 
Since the normal RSA scheme decrypts messages using the property
$\rem{\rem{m^{e}}{n}^d}{n} = \rem{m^{ed}}{n} = m$,
with the keys reversed, the message can still be decrypted because that
$\rem{\rem{m^{d}}{n}^e}{n} = \rem{m^{de}}{n} = m$.
\end{solution}

\ppart
%n = p * q = 29 * 37 = 1073 
%phi(n) = 28 * 26 = 1008
%e,d = 605,5 
Given the public key (5, 1073), and the encrypted message 33, 
what is the original message?

\begin{solution}
The original message is $\rem{33^5}{1073} = 937$
\end{solution}

Charlie can be confident that $m$ came from Sierra because the message
is encrpyted with the private key which only Sierra knows.

%\ppart
%Explain in words why $B$ can be confident that this message is not forged?
%
%\begin{solution}
%Because only $A$ knows the private key.
%\end{solution}

\eparts
\fi

\documentclass[problem]{mcs}

\begin{pcomments}
  \pcomment{CP_bogus_unique_prime_factors}
  \pcomment{by ARM 2/18/11}
\end{pcomments}

\pkeywords{
  false_proof
  bogus_proof
  prime
  strong_induction
  weakly_decreasing
  factorization
}

%%%%%%%%%%%%%%%%%%%%%%%%%%%%%%%%%%%%%%%%%%%%%%%%%%%%%%%%%%%%%%%%%%%%%
% Problem starts here
%%%%%%%%%%%%%%%%%%%%%%%%%%%%%%%%%%%%%%%%%%%%%%%%%%%%%%%%%%%%%%%%%%%%%

\begin{problem}
A sequence of numbers is \emph{\idx{weakly decreasing}} when each
number in the sequence is $\ge$ the numbers after it.  (This implies
that a sequence of just one number is weakly decreasing.)

Here's a bogus proof of a very important true fact, every integer greater
than 1 is a \emph{product of a unique weakly decreasing sequence of
  primes}---a pusp, for short.

Explain what's bogus about the proof.

\begin{lemma*}
Every integer greater than 1 is a pusp.
\end{lemma*}

For example, $252 = 7 \cdot 3 \cdot 3 \cdot 2 \cdot 2$, and no other
weakly decreasing sequence of primes will have a product equal to 252.

\begin{bogusproof}
We will prove the lemma by strong induction, letting the induction
hypothesis $P(n)$ be
\[
n \text{ is a pusp}.
\]
So the lemma will follow if we prove that $P(n)$ holds for all $n \geq
2$.

\inductioncase{Base Case} ($n=2$): $P(2)$ is true because $2$ is prime, and
so it is a length one product of primes, and this is obviously the
only sequence of primes whose product can equal 2.

\inductioncase{Inductive step}: Suppose that $n \geq 2$ and that $i$ is a
pusp for every integer $i$ where $2 \leq i < n+1$.  We must show that
$P(n+1)$ holds, namely, that $n+1$ is also a pusp.  We argue by cases:

If $n+1$ is itself prime, then it is the product of a length one
sequence consisting of itself.  This sequence is unique, since by
definition of prime, $n+1$ has no other prime factors.  So $n+1$ is a
pusp, that is $P(n+1)$ holds in this case.

Otherwise, $n + 1$ is not prime, which by definition means $n+1 = km$
for some integers $k,m$ such that $2 \leq k,m < n+1$.  Now by the
strong induction hypothesis, we know that $k$ and $m$ are pusps.  It
follows that by merging the unique prime sequences for $k$
and $m$, in sorted order, we get a unique weakly decreasing sequence
of primes whose product equals $n+1$.  So $n+1$ is a pusp, in this
case as well.

So $P(n+1)$ holds in any case, which completes the proof by strong
induction that $P(n)$ holds for all~$n \ge 2$.

\end{bogusproof}

\begin{solution}
The problem is that even if $n+1=km$ and $k,m$ have unique
factorizations, if $n+1$ is a product of three or more primes, then
$n+1=ij$ for different $(i,j)$ pairs, and it is logically possible
that different $(i,j)$ pairs will produce different weakly decreasing
sequences of primes whose product is $n+1$.  They don't, of course, but
this proof doesn't explain why.
\end{solution}
\end{problem}


\endinput

\documentclass[12pt]{article}
\usepackage{../light}

\newcommand{\mfigure}[3]{\bigskip\centerline{\resizebox{#1}{#2}{\includegraphics{#3}}}\bigskip}

%\hidesolutions
\showsolutions

\newcommand{\naturals}{\mathbb N}
\newcommand{\eqdef}{:=}
\newcommand{\prsub}[2]{\mathop{\textup{Pr}_{#1}}\nolimits\left(#2\right)}

\begin{document}

\recitation{25}{December 10, 2008}


\section{Random walks on graphs}

In lecture yesterday, we saw what happens when you take a random walk
on the roulette wheel.  In today's recitation, we will study a more general
paradigm that allows us to model the typical movement pattern of a
6.042 student right after the final exam.

Let directed graph $G$ have vertices $V$ and edges $E$.
The 6.042 student comes out of the final exam located on a particular node
of the graph, corresponding to the exam room.  What happens next is unpredictable, as the
student is in a total haze.  At each step of the walk, if the 6.042 student 
is at node $u$ at the end of the previous step, he/she picks one of the edges $(u,v)$ 
uniformly at random from the set of all edges directed out of $u$, and
walks to the node $v$.

If $|V|=n$, let the vector $P^{(j)} = (p_1^{(j)},\ldots,p_n^{(j)})$
be such that $p_i^{(j)}$ is the probability of being at node
$i$ after $j$ steps.

\begin{itemize}
\item[a.]
\label{parta}
We will start by looking at a simple graph. If the student starts at node 1 (the top node) in the following graph,
what is $P^{(0)},P^{(1)},P^{(2)}$?  Give a nice expression 
for $P^{(n)}$.  
%What is the stationary distribution?

\mfigure{!}{1.5in}{loop}

\solution[\vspace{2in}]{ $P^{(0)} = (1,0),P^{(1)}=(1/2,1/2),P^{(2)}=(1/4,3/4)$, $P^{(n)} = (1/2^n, 1- 1/2^n)$.}

%\ppart
%Say that the graph is  {\em aperiodic} if the GCD of the
%lengths of all of the cycles in the graph is 1.    
%Is a bipartite graph aperiodic?
%
%\solution{Not unless it has only one node.}
\item[b.]
Given an arbitrary graph, show how to 
write an expression for $p_i^{(j)}$ in terms of the $p_k^{(j-1)}$'s. 

\solution[\vspace{2in}]{We have
$$p_i^{(j)} = \sum_{k \ \mid \ (k,i) \in E} \frac{1}{\textrm{degree}(k)} p_{k}^{(j-1)}.$$
}

\item[c.]
Does your answer to the last part look like any other system of equations you've seen in this course?

\solution[\vspace{1in}]{It should -- these are similar to the equations that
we got from PageRank!}


\item[d.]
Let the {\em limiting distribution} vector $\pi$ be
$$\lim_{k \rightarrow \infty} \frac{\sum_{i = 1}^{k} P^{(i)}}{k}.$$
What is the limiting distribution of the graph from part a?
Would it change if the start distribution 
were $P^{(0)}=(1/2,1/2)$ or $P^{(0)}=(1/3,2/3)$?

\solution[\vspace{2.5in}]{
Say we start with the distribution $(x, 1 - x)$.  The distribution
after $i$ steps is

\[P^{(i)} = (x/2^i, 1 - x/2^i) \]

Plugging this into the formula for the limiting distribution, we
have

\begin{eqnarray*}
\pi & = & \lim_{k \rightarrow \infty} \frac{\sum_{i = 1}^{k} P^{(i)}}{k} \\
& = & \lim_{k \rightarrow \infty} \frac{\sum_{i = 1}^{k} (x/2^i, 1 - x/2^i)}{k} \\
& = & \lim_{k \rightarrow \infty} \left( \frac{\sum_{i = 1}^{k} x / 2^i}{k}, \frac{\sum_{i = 1}^{k} 1 - x / 2^i}{k} \right) \\
& = & \lim_{k \rightarrow \infty} \left( \frac{x - x / 2^k}{k}, \frac{k - (x - x / 2^k))}{k} \right) \\
& = & (0, 1) \\
\end{eqnarray*}
%The limiting distribution  for all of the start
%distributions is (0,1).
}

\newpage
\item[e.]
Let's consider another directed graph. 
If the student starts at node 1 with probability 1/2
and node 2 with probability 1/2,
what is $P^{(0)},P^{(1)},P^{(2)}$ in the following graph?
What is the limiting distribution?

\mfigure{!}{1.5in}{triLoop}

\solution[\vspace{1in}]{$P^{(0)} = (1/2, 1/2, 0)$, $P^{(1)} = (1/3, 1/3, 1/3)$, $P^{(2)} = (1/3, 1/3, 1/3)$, and in general, the limiting distribution is $(1/3, 1/3, 1/3)$.}

\item[f.]
Now we are ready for the real problem.  In order to make it home,
the poor 6.042 student is faced with $n$ doors along a 
long hall way.  Unbeknownst to him, the door that goes outside
to paradise (that is, freedom from 6.042 and more importantly,
vacation!) is at the {\em very end}.  At each step
along the way, he passes by a door which he opens up and
goes through with probability 1/2.  Every time he does this,
he gets teleported back to the 6.042 exam room.
Let's figure out how long it will take the poor guy to escape from
6.042.  What is  $P^{(0)},P^{(1)},P^{(2)}$? 
What is the limiting distribution?

\mfigure{!}{1.5in}{lineMark}

\solution[\vspace{1.5in}]{$P^{(0)}$ just has $p_0^{(0)} = 1$ and all other probabilities $0$, since the student is at the exam room and has just started. $P^{(1)}$ just has $p_1^{(1)} = 1$ since the student made the only move possible, and all other probabilities $0$. $P^{(2)}$ has $p_0^{(2)} = 1/2$, since the student opens the first door with probability $1/2$, and $p_2^{(2)} = 1/2$, since the student passes the first door with probability $1/2$.

Eventually, we reach node $n$, at which point we stay there forever. Thus, the limiting distribution has $p_i = 0$ for $i = 0, 1, \ldots, n-1$, and $p_n = 1$.}

\item[g.]
Show that the expected number of teleportations $T(n)$ you make back to the 
exam room before you escape to the outside world is $2^{n-1}-1$.

\solution{
The probability that you manage to reach the last door to the outside
without getting teleported back to the 6.042 exam room is $\frac{1}{2^{n-1}}$.
Thus, by the mean time to failure formula, $T(n)$,
the expected number of times
you get teleported back to the 6.042 exam room is $2^{n-1}-1$ (on the last try you don't get teleported back, but rather, you succeed.)  
}

\iffalse
\item[h.]
The poor 6.042 student has a friend that
also took the exam, and thinks he did very well on it.
The friend brags that he can use Markov's inequality
to upper bound the probability that the 6.042 student gets
teleported back to the 6.042 exam room more than
$6T(n)$ times before he finally escapes.  What bound
can he get?

\solution{
Using Markov's inequality, the probability is at most $1/6$.
}

\item[i.]
At first the poor 6.042 student wants to push his friend
into the first teleportation black hole that he can find.
But, then the poor 6.042 student thinks hard and realizes
that he can get a much better and much easier upper bound
on the probability that he (the poor 6.042 student) gets
teleported back to the 6.042 exam room more than
$6T(n)$ times before he finally escapes.  What bound
can he get?

\solution{
The probability that the 6.042 student
does not escape by the time he gets teleported 
$s$ teleportations is
$(1-2^{-n})^{s}$. So, the probability that he
does not escape by time $6T(n)$ is at most
$(1-2^{-n})^{6T(n)} =(1-2^{-n})^{62^n} = e^{-6}$. 
}

\fi

\end{itemize}
\newpage
\section{More random walks}
Consider an undirected connected graph $G = (V,E)$. It turns out that such
graphs have a unique limiting distribution, independent of the
initial distribution. For node $i$, let $\textrm{deg}(i)$ be its degree. Let
$m = \sum_i \textrm{deg}(i) = 2|E|$, and $n = |V|$. Consider the vector of probabilities
$$\pi^* = \left (\frac{\textrm{deg}(1)}{m}, \frac{\textrm{deg}(2)}{m}, \ldots, \frac{\textrm{deg}(n)}{m} \right ).$$
We will show that $\pi^*$ is a limiting distribution of $G$.
%
%Use the definition of a limiting distribution together with part b of the previous problem to
%show that $\pi^*$ is indeed a limiting distribution of $G$.  In other words, pick a starting
%distribution and show that $\pi^*$ is the limiting distribution by using the definition.

Note that in general, such a clean description of a limiting distribution does not exist for directed graphs, such as for the web graph that PageRank uses. Intuitively this makes sense, as otherwise one could create a lot of dummy links that point to your web site to increase its degree, and therefore artificially increase its rank. 

\begin{itemize}
\item[a.]
In order to show that $\pi^*$ is a limiting distribution, we need to pick a starting distribution.  Let's choose as our starting distribution $P^{(0)} = \pi^*$. Using the formula from part 1b, prove by induction that $P^{(n)} = P^{(0)}$ for all $n$.

\solution[\vspace{3in}]{
Let $P(n)$ be the proposition that $P^{(n)} = P^{(0)}$.

{\bf Base case: } Clearly, $P^{(0)} = P^{(0)}$.

{\bf Inductive step: } We know from problem 1b that $p_i^{(n + 1)} = \sum_{k \ \mid \ (k,i) \in E} \frac{1}{\textrm{degree}(k)} p_{k}^{(n)}$.  Using this fact, we have

\begin{align*}
p_i^{(n + 1)} & = \sum_{k \ \mid \ \{i,k\} \in E} \frac{1}{\textrm{deg}(k)} \cdot p_k^{(n)} \\
& = \sum_{k \ \mid \ \{i,k\} \in E} \frac{1}{\textrm{deg}(k)} \cdot p_k^{(0)} & \textrm{(Inductive Hypothesis)}\\
& = \sum_{k \ \mid \ \{i,k\} \in E} \frac{1}{\textrm{deg}(k)} \cdot \frac{\textrm{deg}(k)}{m} & \textrm{($\pi^*$ is the starting distribution)}\\
& = \sum_{k \ \mid \ \{i,k\} \in E} \frac{1}{m} & \textrm{(simplifications)}\\
& = \frac{\textrm{deg}(i)}{m} \textrm{(definition of degree)}\\
& = p_i^{(0)} & \textrm{($\pi^*$ is the starting distribution),}\\
\end{align*}
as desired.
}

\item[b.]
Use the definition of a limiting distribution along with the results of part a to show that $\pi^*$ is a limiting distribution of $G$.

\solution[\vspace{3in}]{
Using the definition of a limiting distribution, we have

\begin{align*}
\pi & = \lim_{k \rightarrow \infty} \frac{\sum_{j = 1}^k P^{(j)}}{k} \\
& = \lim_{k \rightarrow \infty} \frac{\sum_{j = 1}^k P^{(0)}}{k} & \textrm{(from part 2a)}\\
& = \lim_{k \rightarrow \infty} \frac{k P^{(0)}}{k} & \textrm{(simplification)}\\
& = P^{(0)} & \textrm{(evaluation of the limit)}\\
& = \pi^* & \textrm{($\pi^*$ is the starting distribution)}\\
\end{align*}

%We need to show that for all nodes $i$,
%$$\pi(i) = \sum_{k \ \mid \ \{i,k\} \in E} \frac{1}{\textrm{deg}(k)}\pi(k).$$
%The following derivation shows this,
%\begin{eqnarray*}
%\pi(i) & = & \sum_{k \ \mid \ \{i,k\} \in E} \frac{1}{\textrm{deg}(k)} \cdot \frac{\textrm{deg}(k)}{m}\\
%& = & \sum_{k \ \mid \ \{i,k\} \in E} \frac{1}{m}\\
%& = & \frac{\textrm{deg}(i)}{m},
%\end{eqnarray*}
%as desired.
}

\end{itemize}

\end{document}

\documentclass[handout]{mcs}

\begin{document}

\inclassproblems{11, Wed.}

%%%%%%%%%%%%%%%%%%%%%%%%%%%%%%%%%%%%%%%%%%%%%%%%%%%%%%%%%%%%%%%%%%%%%
% Problems start here
%%%%%%%%%%%%%%%%%%%%%%%%%%%%%%%%%%%%%%%%%%%%%%%%%%%%%%%%%%%%%%%%%%%%%

\pinput{CP_binom_coeff}
%\pinput{CP_multinomial_theorem}
%\pinput{CP_multinomial_fermat}

\iffalse

\section*{Combinatorial proof}

\textbox{
\centerline{\textbf{Combinatorial proofs of identities}}
Recall the basic plan for a combinatorial proof of an identity $x=y$:
\begin{enumerate}
\item Define a set $S$.
\item Show that $\card{S} = x$ by counting one way.
\item Show that $\card{S} = y$ by counting another way.
\item Conclude that $x = y$.
\end{enumerate}
}
\fi

\pinput{CP_startup}
\pinput{CP_com_proof}
%\pinput{CP_bit_string}


%%%%%%%%%%%%%%%%%%%%%%%%%%%%%%%%%%%%%%%%%%%%%%%%%%%%%%%%%%%%%%%%%%%%%
% Problems end here
%%%%%%%%%%%%%%%%%%%%%%%%%%%%%%%%%%%%%%%%%%%%%%%%%%%%%%%%%%%%%%%%%%%%%
\end{document}

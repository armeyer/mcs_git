\documentclass[problem]{mcs}

\begin{pcomments}
  \pcomment{FP_sequence_gcd}
  \pcomment{zabel, s18}

  \pcomment{This has been superceded by FP_polynomial_modk}
\end{pcomments}

\pkeywords{
  sequence
  gcd
  lcm
  number_theory
  modular
}

%%%%%%%%%%%%%%%%%%%%%%%%%%%%%%%%%%%%%%%%%%%%%%%%%%%%%%%%%%%%%%%%%%%%%
% Problem starts here
%%%%%%%%%%%%%%%%%%%%%%%%%%%%%%%%%%%%%%%%%%%%%%%%%%%%%%%%%%%%%%%%%%%%%

\begin{problem}
Define the sequence $(a_0,a_1,a_2,\ldots)$ by $a_n\eqdef
(n^2-4)(n^2-9)$.  What is the largest integer that divides every
term of this sequence?  Explain.

\begin{solution}
Any integer that divides every term in the sequence must divide $a_0 =
36$ and $a_1 = 24$ and therefore must divide their difference, $12$.
We'll show that $12$ divides every term in the sequence and is thus
the desired answer.

To see that every term is divisible by $3$, observe that
$(n^2-4)(n^2-9)\equiv (m^2-4)(m^2-9)\pmod 3$ whenever $n\equiv m\pmod
3$, so in particular, $a_n\equiv a_{\rem{n}{3}}\pmod 3$, meaning every
$a_n$ is congruent to $a_0 = 36$, $a_1 = 24$, or $a_2 = 0$ modulo
$3$.  These three values are congruent to $0\bmod 3$, so each $a_n$ is
also congruent to $0\bmod 3$.

Similarly, every $a_n$ is congruent to one of $a_0$, $a_1$, $a_2$, or
$a_3 = 0$ modulo $4$ and is thus divisible by $4$.

Because $3 \divides a_n$ and $4\divides a_n$, it follows that
$\lcm(3,4) = 12$ divides $a_n$, as required.
\end{solution}
  
\end{problem}

\endinput

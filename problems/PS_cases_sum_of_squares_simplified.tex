\documentclass[problem]{mcs}

\begin{pcomments}
  \pcomment{PS_cases_sum_of_squares_simplified}
  \pcomment{F16.ps1 and earlier even-year Fall terms}
  \pcomment{edited by ARM and fncv 2/5/17}
\end{pcomments}

\pkeywords{
  cases
  sum
  square
  even
}

%%%%%%%%%%%%%%%%%%%%%%%%%%%%%%%%%%%%%%%%%%%%%%%%%%%%%%%%%%%%%%%%%%%%%
% Problem starts here
%%%%%%%%%%%%%%%%%%%%%%%%%%%%%%%%%%%%%%%%%%%%%%%%%%%%%%%%%%%%%%%%%%%%%
\begin{problem}
The \emph{parity} of an integer is \emph{even} when it is divisible by two;
otherwise its parity is \emph{odd}.

Suppose that 
\[
x^2 + y^2 = z^2,
\]
where $x, y, z$ are nonnegative integers.
\begin{claim}
The integer $z$ is even if and only if $x$ and $y$ are even.
\end{claim}
This Claim can be proved by cases according to the parities of $x$ and
$y$.  Here are four ways to break the proof into cases.

\begin{enumerate}
   \item Case 1: $x,y$ both even,
         Case 2: $x,y$ both odd,
         Case 3: $x$ even, $y$ odd,
         Case 4: $x$ odd, $y$ even.

   \item\label{evoddiff} Case 1: $x,y$ both even,
         Case 2: $x,y$ both odd.
         Case 3: $x,y$ different parities,

   \item Case 1: $x,y$ have the same parity,
         Case 2: $x$ even, $y$ odd,
         Case 3: $x$ odd, $y$ even.

   \item Case 1: $x,y$ have the same parity,
         Case 2: $x,y$ have different parity,

\end{enumerate}

\bparts

\ppart Use one of the four sets of cases above to prove the Claim.

\iffalse
\hint An odd number equals $2m+1$ for some integer $m$, so its square
equals $4(m^2 + m) + 1$.
\fi

\examspace[4in]

\begin{solution}
Use the cases in~\ref{evoddiff}.

Note that a number $n$ is even iff $n^2$ is even.  In addition, the
square of any even member must be a multiple of 4.

\inductioncase{Case 1}: if $x,y$ are different parities, then one of
$x,y$ is even, and the other one is odd.  Since the square of an even
number is even, the square of an odd number is odd, and the sum of an
odd number and a even number is odd, it follows that $x^2+y^2=z^2$ is
odd, so $z$ cannot be even.

\inductioncase{Case 2}: if $x,y$ are both even, then both $x^2$ and
$y^2$ are multiples of 4. So $x^2+y^2=z^2$ is also a multiple of 4,
and hence $z$ is even.

\inductioncase{Case 3}: if $x,y$ are both odd, let $x=2m+1, y=2n+1$
for some nonnegative integers $m,n$. Then,
\[
z^2 = (2m+1)^2 + (2n+1)^2 = 4(m^2+m+n^2+n) + 2.
\]
Since $z^2$ leaves a remainder of 2 on division by 4, it is not a
multiple of 4.  But the square of any even number must be a multiple
of 4, a contradiction.  So this case is impossible and the claim is
vacuously true.

We have showed that in all cases the claim is true.
\end{solution}

\ppart In retrospect, which of the four sets of cases would lead to
the simplest proof that did not involve subcases.

\begin{solution}
The case~\ref{evoddiff} works best: the proof for the same-parity case
requires subcases since the proofs for both even is quite different
from the proof for both odd.  But a proof for the different parity
case need not depend on which is even abd which is odd, so there is no
need to split it into subcases.
\end{solution}

\eparts

\end{problem}

%%%%%%%%%%%%%%%%%%%%%%%%%%%%%%%%%%%%%%%%%%%%%%%%%%%%%%%%%%%%%%%%%%%%%
% Problem ends here
%%%%%%%%%%%%%%%%%%%%%%%%%%%%%%%%%%%%%%%%%%%%%%%%%%%%%%%%%%%%%%%%%%%%%
\endinput

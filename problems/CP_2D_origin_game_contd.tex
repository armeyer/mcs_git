\documentclass[problem]{mcs}

\begin{pcomments}
  \pcomment{from: S09.cp4t}
  \pcomment{this should probably be combined with CP_2D_origin_game}
\end{pcomments}

\pkeywords{
  well-ordering
  WOP
  fun_game
  chains_and_antichains
}

%%%%%%%%%%%%%%%%%%%%%%%%%%%%%%%%%%%%%%%%%%%%%%%%%%%%%%%%%%%%%%%%%%%%%
% Problem starts here
%%%%%%%%%%%%%%%%%%%%%%%%%%%%%%%%%%%%%%%%%%%%%%%%%%%%%%%%%%%%%%%%%%%%%

\begin{problem}
This is an informative problem that no one got to in class, so we're
adding it to pset 5.

\bparts \ppart Even if the players in Problem~\ref{nngame} conspire to
keep the game going as long as possible, it will necessarily come to an
end.  Prove this as follows:

\item[i.] At any point in the game, let $x_m$ be the minimum of the $x$
  coordinates of all of the previous moves, and likewise, $y_m$ be the
  minimum of the $y$ coordinates of all of the previous moves.  Verify
  that after any move, the value of $x_m+y_m$ is the same or a smaller
  nonnegative integer.

\iffalse
\begin{solution}
That's because min's cannot increase as more moves are made, so
  neither can their sum.  
\end{solution}
\fi

\item[ii.] Suppose $a$ is the least number such that a move $(x_m,a)$
  has been made, and likewise $b$ is the least number such that a move $(b,y_m)$
  has been made.  The \emph{bounded moves} are defined to be the
  possible moves in the rectangle with corners at $(x_m,a)$ and
  $(b,y_m)$.  Explain why the only moves that do not decrease
  $x_m+y_m$ must be bounded moves.

\hint Draw a picture of the bounded-move rectangle after a few moves
have been made.

\iffalse
\begin{solution}
Moves North of the rectangle are not allowed because they
  would be coordinatewise larger than $(x_m,a)$, moves East of the
  rectangle are not allowed because they would be coordinatewise
  larger than $(b,y_m)$, and moves Northeast would be bigger than
  both.  Moves South of the rectangle decrease the minimum value of
  $y$, moves West decrease the minimum $x$, and moves Southwest
  decrease both.  That leaves only moves within the rectangle as
  possibly allowed moves that do not decrease $x_m+y_m$.
\end{solution}
\fi

\item[iii.] Define the \term{size}, $\sigma$, of a game position to be
  $(x_m+y_m,k)$ where $k$ is the number of bounded moves.  Explain why
  every move makes $\sigma$ decrease under lexicographic order,
  $\lex<$.\footnote{\term{Lexicographic order}, $\lex<$, on $\naturals^2$
    is defined by the rule:
\[
(x,y) \lex< (x',y') \qiff x < x'\ \QOR\ [x = x'\ \QAND\ y < y'].
\]} 
  Conclude that the game always comes to an end.
\iffalse

\begin{solution}
Week 3 Notes explained that lexicographic order on $\naturals^2$
  is a well-founded total order, and so it has no infinite decreasing
  sequences.  So the sizes of all the positions reached in any game must
  have a minimum value.  Since any further move would decrease this value,
  the position with such a minimum value must be one from which no move is
  possible, namely the final position at the origin.
\end{solution}
\fi

\iffalse

\ppart \emph{(Optional)} Is there a winning strategy for the first player
that guarantees a bound on the number of game moves?

\begin{solution}
No, because neither of the winning strategies described in the
  solution to part~\eqref{winstrat} guarantees a bound, and there are no
  other winning strategies.

To see this, suppose Player 1 starts with any legal move other than
the ones above. Player 2 will be able to adopt one of the two winning
strategies for himself and win for sure:
\begin{itemize}
\item
If Player 1 picks some $(m,n)$ where $m,n\geq{1}$ and
$(m,n)\neq(1,1)$, then Player 2 can pick $(1,1)$ next, and then use
the first strategy to win against Player 1.
\item
If Player 1 picks some $(0,n)$ or $(n,0)$ where $n\geq{3}$, then
Player 2 can respectively pick $(0,2)$ or $(2,0)$ next, and then use
the second strategy to win against Player 1.
\item
If Player 1 picks $(1,0)$ or $(0,1)$ then Player 2 picks the other one
and wins.
\end{itemize}

\end{solution}
\fi

\ppart Conclude that there is no infinite antichain 
 under the coordinatewise partial order on $\naturals^2$.\footnote
{\term{Coordinatewise partial order}, $\coordle$, on $\naturals^2$
is defined by the rule:
\[
(x,y) \coordle (x',y') \qiff [x \leq x'\ \QAND\ y \leq y'].
\]}

\iffalse
\begin{solution}
The elements of an antichain, listed in any order, is an allowed
  sequence of moves.  So an infinite antichain would describe a
  nonterminating game, contradicting the fact that the game must
  terminate.
\end{solution}
\fi

\ppart Show that for each $n>1$, there is an antichain of size $n$ under the
coordinatewise partial order on $\naturals^2$.

\iffalse
\begin{solution}
$(0,n-1),(1,n-2),(2,n-3),\dots,(n-1,0)$ 
\end{solution}
\fi

\ppart Instead of using lexicographic order, a simpler way to prove that
all game plays must end would be to assign to each position a nonnegative
integer-valued ``size'' that decreases with each move.  Explain why this is
impossible.

\iffalse
\begin{solution}
If the size of a game position was $\nu \in \naturals$, and size
  decreased with each move, then from that position, the game would have
  to end after at most $\nu$ moves.  But for any starting position besides
  the origin, play can continue for any finite number of moves, so no
  bound, $\nu$, is possible.
\end{solution}
\fi

\eparts

\end{problem}

%%%%%%%%%%%%%%%%%%%%%%%%%%%%%%%%%%%%%%%%%%%%%%%%%%%%%%%%%%%%%%%%%%%%%
% Problem ends here
%%%%%%%%%%%%%%%%%%%%%%%%%%%%%%%%%%%%%%%%%%%%%%%%%%%%%%%%%%%%%%%%%%%%%

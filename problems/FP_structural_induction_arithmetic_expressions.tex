\documentclass[problem]{mcs}

\begin{pcomments}
  \pcomment{FP_structural_induction_arithmetic_expressions}
  \pcomment{from: S09.final; S08 final}
  \pcomment{adapted by ARM 12/8/09}
\end{pcomments}

\pkeywords{
  structural_induction
  polynomial
  arithmetic
  derivative
}

%%%%%%%%%%%%%%%%%%%%%%%%%%%%%%%%%%%%%%%%%%%%%%%%%%%%%%%%%%%%%%%%%%%%%
% Problem starts here
%%%%%%%%%%%%%%%%%%%%%%%%%%%%%%%%%%%%%%%%%%%%%%%%%%%%%%%%%%%%%%%%%%%%%

\def\ArF{\text{ArF}}

\begin{problem}

\begin{definition*}
  The set $\ArF$ of \term{arithmetic functions} of one argument is
  defined recursively as follows:

\begin{itemize}
\item \textbf{Base cases:}

\begin{enumerate}

\item The identity function, $\ide_\reals$ on real numbers is an $\ArF$.

\item\label{AFc} Every real-valued constant function is in $\ArF$.

\end{enumerate}

\item \textbf{Constructor cases:} If $e,f \in \ArF$, then
\begin{enumerate}
\setcounter{enumi}{2}

\item\label{AF+} $e + f \in \ArF$,

\item\label{AF*} $e \cdot f \in \ArF$.

\end{enumerate}
\end{itemize}
\end{definition*}

Prove by structural induction that $\ArF$ is closed
under taking derivatives.  That is, using the induction hypothesis,
\[
P(h) \eqdef [h^{\prime} \in \ArF],
\]
where $h^{\prime} \eqdef d\,h(x)/dx$, prove that $P(h)$ holds for all $h
\in \ArF$.  Make sure to indicate explicitly each of the base cases and the
constructor cases of the structural induction.

\examspace[4in]
\begin{solution}

\begin{proof}
  \textbf{base cases}: We must show $P(\ide_\reals)$ and
  $P(\text{constant-function})$.  But $(\ide_\reals)^{\prime}$ is the
  constant function 1, and the derivative of a constant function is
  the constant function 0, and these are in $\ArF$ by~\eqref{AFc} in
  the definition of $\ArF$.
  
  \textbf{constructor cases}: Given $f, g \in \ArF$, we may assume by
  structural induction that $P(f)$ and $P(g)$ both hold, and must prove
  $P(h)$ where

\emph{case} $h= f + g$:  In this case,
\[
h^{\prime} = f^{\prime} + g^{\prime},
\]
and since $f^{\prime},g^{\prime} \in \ArF$ by hypothesis, so is their
sum by the constructor rule~\eqref{AF+}.  This proves $P(h)$ in this
case.

\emph{case} $h= f \cdot g$:

The rule for the derivative of a product is:
\begin{equation}\label{fgderiv-arith}
h^{\prime} =  f^{\prime} \cdot g + f \cdot g^{\prime}.
\end{equation}
Since $f, f^{\prime}, g, g^{\prime} \in \ArF$ by hypothesis, so is the
right-hand side of~\eqref{fgderiv-arith} by the constructor rules~\eqref{AF+}
and~\eqref{AF*}.  This proves $P(h)$ in this case.

Thus the induction hypothesis holds in all Constructor cases, which
completes the proof by structural induction.  Hence, $P(h)$ holds for
all $h \in \ArF$.
\end{proof}

\end{solution}

\end{problem}

\endinput

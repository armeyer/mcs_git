\documentclass[problem]{mcs}

\begin{pcomments}
  \pcomment{from: F09.mq3}
  \pcomment{similar to CP_10_and_15_cent_stamps_by_WOP but slightly different and uses induction}
\end{pcomments}

\pkeywords{
  induction
  strong_induction
}

%%%%%%%%%%%%%%%%%%%%%%%%%%%%%%%%%%%%%%%%%%%%%%%%%%%%%%%%%%%%%%%%%%%%%
% Problem starts here
%%%%%%%%%%%%%%%%%%%%%%%%%%%%%%%%%%%%%%%%%%%%%%%%%%%%%%%%%%%%%%%%%%%%%

\begin{problem}

  Find all possible amounts of postage that can be paid exactly
  using 10 and 15 cent stamps. Use induction to prove that
  your answer is correct.
  
  Recall that, in the last miniquiz, we have proven by WOP that 
  every amount of postage payable using 10 and 15 cent stamps is 
  divisible by 5.
  
\bparts
\ppart

  Let $S(n)$ mean that exactly $n$ cents of postage can be paid using 
  only 10 and 15 cent stamps. 
  
  Find the smallest $k$ divisible by 5 such that the following 
  proposition is true.
  \[ 
  \forall n.\ ( 5 \divides n \QAND n \geq k ) \QIMPLIES\ S(n).
  \]
  
\begin{solution}
  
  The only case where $ 5 \divides n $ does not imply $ S(n) $ is
  when $n = 5$. Therefore, $k = 10$.
  
\end{solution}
  
\ppart

  Use induction to prove that your answer to the previous problem 
  part is correct. 
  
  \hint Let $P(m)$ be the proposition that a postage of
  $5m \geq k$ can be paid by 10 or 15 cent stamps.
  
\begin{solution}
  
  \begin{proof}
    The proof is by strong induction on $m$.

    \textbf{Base cases:} $P(2)$ and $P(3)$ are trivially true because they
    may be payable by one 10 cent stamp or one 15 cent stamp, respectively.

    \textbf{Inductive step:} For all $m \geq 3$, we assume that $P(2)$,
    $\dots$, $P(m)$ are true in order to prove that $P(m+1)$ is true.

    By the assumption that $P(m-1)$ is true, we know that the postage value
    $5(m-1)$ can be paid with 10 and 15 cent stamps. By adding one 10 cent
    stamp to that postage, we will be able to pay for a postage of 
    $5(m-1)+10 = 5(m+1)$ cents, showing that $P(m+1)$ is true. It follows
    by strong induction that $P(m)$ holds for all $n \geq 2$.
    
    We have therefore shown that all postage values $ \geq 10 $ that are 
    multiples of $5$ can be paid by 10 and 15 cent stamps.
  \end {proof}
  
\end{solution}

\ppart
  
  What are all the possible nonzero postage amount that can be paid 
  using 10 and 15 cent stamps?
  
\begin{solution}
  
  From the previous part, we know that all multiples of 5 that are
  $ \geq 10 $ can be paid using 10 and 15 cent stamps. In other words, 
  all positive multiples of 5 excluding 5 itself may be paid using 
  10 and 15 cent stamps.
  
\end{solution}

\eparts

\end{problem}

%%%%%%%%%%%%%%%%%%%%%%%%%%%%%%%%%%%%%%%%%%%%%%%%%%%%%%%%%%%%%%%%%%%%%
% Problem ends here
%%%%%%%%%%%%%%%%%%%%%%%%%%%%%%%%%%%%%%%%%%%%%%%%%%%%%%%%%%%%%%%%%%%%%

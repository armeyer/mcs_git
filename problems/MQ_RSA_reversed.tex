\documentclass[problem]{mcs}

\begin{pcomments}
    \pcomment{MQ_RSA_reversed}
    \pcomment{revised ARM 3/15/15, 10/6/15}
    \pcomment{subsumes TP_RSA_reversed}
\end{pcomments}

\pkeywords{
  RSA
  modulo
  encrypt
  decrypt
  exponent
}

\begin{problem}
Suppose Alice and Bob are using the RSA cryptosystem to send secure
messages.  Each of them has a public key visible to everyone and a
private key known only to themselves, and using RSA in the usual way,
they are able to send secret messages to each other over public
channels.

But a concern for Bob is how he knows that a message he gets is
actually from Alice---as opposed to some imposter claiming to be
Alice.  This concern can be met by using RSA to add unforgeable
``signatures'' to messages.  To send a message $m$ to Bob with an
unforgeable signature, Alice uses RSA encryption on her message $m$,
but instead using Bob's public key to encrypt $m$, she uses her own
\emph{private} key to obtain a message $m_1$.  She then sends $m_1$ as
her ``signed'' message to Bob.

\bparts

\ppart Explain how Bob can read the original message $m$ from Alice's
signed message $m_1$.  (Let $(n_A,e_A)$ be Alice's public key and
$d_A$ her private key.  Assume $m \in \Zintvco{0}{n_A}$.)

\examspace[3in]

\begin{solution}
By definition of RSA, the message $m_1$ will be
\[
m_1 \eqdef \rem{m^{d_A}}{n_A},
\]
where $d_A$ is Alice's private key.

RSA encryption is based on the choice of a private key $d$ and a
public key $(e,n)$ which satisfy the condition that $d \cdot e \equiv
1 \pmod{\phi(n)}$.  But this condition is symmetric in $d$ and $e$, so
reversing their roles allows Alice's private key $d_A$ to be used to
``encrypt'' $m$ as the message $m_1$.  Now Bob can apply RSA to $m_1$
using Alice's public key $e_A$ in place of his private key to
reconstruct $m$ from $m_1$:
\[
m = \rem{m_1^{e_A}}{n}
\]
\end{solution}

\ppart Briefly explain why Bob can be confident, assuming RSA is
secure, that $m_1$ came from Alice rather than some imposter.

\examspace[2in]

\begin{solution}
The message $m$ that Bob reconstructs from $m_1$ can only have been
``encrypted'' using Alice's private key $d_A$.  Assuming RSA is
secure, only Alice knows her private key $d_A$, so Bob can conclude
the message came from Alice.
\end{solution}

\ppart Notice that not only Bob, but \emph{anyone} can use Alice's
public key to reconstruct her message $m$ from its signed version
$m_1$.  So how can Alice send a secret signed message to Bob over
public channels?

\begin{solution}
After signing her message with her private key to obtain $m_1$, Alice
can use RSA in the usual way to encrypt $m_1$ using Bob's public key
and send it to Bob.  Now only Bob can read the signed message.
\end{solution}
\eparts

\end{problem}

\endinput


\iffalse
\ppart
%n = p * q = 29 * 37 = 1073 
%phi(n) = 28 * 26 = 1008
%e,d = 605,5 
Given the public key (5, 1073), and the encrypted message 33, 
what is the original message?

\begin{solution}
The original message is $\rem{33^5}{1073} = 937$
\end{solution}
\fi

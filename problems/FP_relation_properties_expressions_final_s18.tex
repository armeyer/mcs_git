\documentclass[problem]{mcs}

\begin{pcomments}
  \pcomment{FP_relation_properties_expressions_final_f13}
  \pcomment{variant of TP_relation_properties_expressions for final
    s18} \pcomment{original by 10/12/13 ARM}
\end{pcomments}

\pkeywords{ irreflexive transitive reflexive antisymmetric asymmetric
  identity_relation }

%%%%%%%%%%%%%%%%%%%%%%%%%%%%%%%%%%%%%%%%%%%%%%%%%%%%%%%%%%%%%%%%%%%%%
% Problem starts here
% %%%%%%%%%%%%%%%%%%%%%%%%%%%%%%%%%%%%%%%%%%%%%%%%%%%%%%%%%%%%%%%%%%%%

\begin{problem}

\bparts \ppart\label{ident-pred-formulas}

Let $R$ be a binary relation on a set $D$.  Each of the following
formulas expresses the fact that $R$ has a familiar relational
property such as reflexivity, asymmetry, transitivity, etc.  For each
of the five predicate formulas below, identify the \textbf{name} of
that property.

% \begin{enumerate}[(1)] \item \label{irreflexform} $\forall d.\,
% \quad \QNOT(d\mrel{R}d)$ \item \label{symmiffform} $\forall c,d.\,
% \quad c\mrel{R}d\ \QIFF\ d\mrel{R}c$ \item \label{asymmform}
% $\forall c, d.\, \quad
% c\mrel{R}d\ \QIMPLIES\ \QNOT(d\mrel{R}c)$ \item \label{transexform}
% $\forall b,d.\, \quad [\exists c.\ (b\mrel{R}c
% \ \QAND\ c\mrel{R}d)]\ \QIMPLIES\ b\mrel{R}d$ \item \label{antisymmform}
% $\forall c \neq d.\, \quad c\mrel{R}d \QIMPLIES
% \QNOT(d\mrel{R}c)$ \end{enumerate}





\begin{enumerate}[(1)]
  \setlength\itemsep{1em}
  
\item \label{symmiffform} $\forall c,d.\, \quad
  c\mrel{R}d\ \QIFF\ d\mrel{R}c$ \hfill \examrule[1in]
  \begin{solution}
    symmetric
  \end{solution}

\item \label{irreflexform} $\forall d.\, \quad \QNOT(d\mrel{R}d)$
  \hfill \examrule[1in]
  \begin{solution}
    irreflexive
  \end{solution}
  
\item \label{asymmform} $\forall c, d.\, \quad
  c\mrel{R}d\ \QIMPLIES\ \QNOT(d\mrel{R}c)$ \hfill \examrule[1in]
  \begin{solution}
    asymmetric
  \end{solution}
  
\item \label{transexform} $\forall b,d.\, \quad [\exists
  c.\ (b\mrel{R}c \ \QAND\ c\mrel{R}d)]\ \QIMPLIES\ b\mrel{R}d$ \hfill
  \examrule[1in]
  \begin{solution}
    transitive
  \end{solution}
  
\item \label{antisymmform} $\forall c \neq d.\, \quad c\mrel{R}d
  \QIMPLIES \QNOT(d\mrel{R}c)$ \hfill \examrule[1in]
  \begin{solution}
    antisymmetric
  \end{solution}
  
\end{enumerate}


% \begin{align} %\item\label{reflexform} \forall d.\quad
% d\mrel{R}d \label{irreflexform} \forall d. &\quad
% \QNOT(d\mrel{R}d)\\ \label{symmiffform} \forall c,d. &\quad
% c\mrel{R}d\ \QIFF\ d\mrel{R}c\\ %\label{symmimpform} \forall
% c,d. &\quad c\mrel{R}d \QIMPLIES d\mrel{R}c \label{asymmform}
% \forall c, d. &\quad
% c\mrel{R}d\ \QIMPLIES\ \QNOT(d\mrel{R}c)\\ %\label{antisymmform}
% \forall c \neq d. &\quad c\mrel{R}d \QIMPLIES \QNOT(d\mrel{R}c)
% %\label{tournamentform} \forall c \neq d. &\quad c\mrel{R}d \QIFF
% \QNOT(d\mrel{R}c) %\label{transform} \forall b,c,d. &\quad
% (b\mrel{R}c
% \ \QAND\ c\mrel{R}d)\ \QIMPLIES\ b\mrel{R}d\\ \label{transexform}
% \forall b,d. &\quad [\exists c.\ (b\mrel{R}c
% \ \QAND\ c\mrel{R}d)]\ \QIMPLIES\ b\mrel{R}d\\ \label{denseform}
% \forall b,d. &\quad b\mrel{R}d\ \QIMPLIES\ [\exists c.\ (b\mrel{R}c
% \ \QAND\ c\mrel{R}d)] \end{align}


\ppart Below are five relational formulas encoding the \textbf{same}
five familiar properties as in part~\eqref{ident-pred-formulas}, in
scrambled order.  Match these five relational formulas to the
predicate formulas from part~\eqref{ident-pred-formulas} by writing
the \textbf{number} that each corresponds to.  Your answers should
simply be the numbers 1--5 in some order.

In these formulas, $\ident{D}$ is the ``identity function'' on $D$,
defined by $\ident{D}(d) \eqdef d\ \text{for}\ d \in D$.

% Next to each of these relational formulas, write the \textbf{letter} that it corresponds to from part~\eqref{ident-pred-formulas}. Your answers should simply be the letters \eqref{} through \eqref{} in some order.


% Predicate
% formulas are numbered (1), (2), \dots, and relational formulas
% (equalities and containments) are labelled with letters (a),(b),\dots.

% Next to each of the relational formulas, write the number of the
% predicate formula equivalent to it.  It is not necessary to name the
% property expressed, but you can get partial credit if you do.  For
% example, part~\eqref{irrpart} gets the label ``(1).''  It expresses
% \emph{irreflexivity}.

\begin{enumerate}[(1)]
  \setlength\itemsep{1em}
  \setcounter{enumi}{5}

\item $R \intersect \ident{D} = \emptyset$
  \hfill \examrule[0.5in]
\begin{solution}
\eqref{irreflexform}: irreflexive
\end{solution}

\item $R \subseteq \inv{R}$
  \hfill \examrule[0.5in]
\begin{solution}
\eqref{symmiffform}: symmetric
\end{solution}

\item $R \compose R \subseteq R$
  \hfill \examrule[0.5in]
\begin{solution}
\eqref{transexform}: transitive
\end{solution}

\item $R \intersect \inv{R} \subseteq \ident{D}$
  \hfill \examrule[0.5in]
\begin{solution}
\eqref{antisymmform}: antisymmetric
\end{solution}

\item $R \intersect \inv{R} = \emptyset$
  \hfill \examrule[0.5in]
\begin{solution}
\eqref{asymmform}: asymmetric
\end{solution}

\end{enumerate}


%% \ppart $R = \inv{R}$ \hfill \examrule[0.5in]
%% \begin{solution}
%% \eqref{symmimpform}, \eqref{symmiffform}: \textbf{symmetric}
%% \end{solution}

%% \ppart $\ident{D} \subseteq R$ \hfill \examrule[0.5in]
%% \begin{solution}
%% \eqref{reflexform}: \textbf{reflexive}
%%\end{solution}


% \ppart $R \subseteq R \compose R$ \hfill \examrule[0.5in]
% \begin{solution}
% \eqref{denseform}: \textbf{dense}
% \end{solution}


%% \ppart $\bar{R} \subseteq \inv{R} $ \hfill \examrule[0.5in]
%% \begin{solution}
%% \eqref{asymmform}: \textbf{asymmetric}
%% \end{solution}

%% \ppart $\bar{R} \intersect \ident{R} = \inv{R} \intersect \ident{R}$ \hfill \examrule[0.5in]
%% \begin{solution}
%% \eqref{tournamentform}: \textbf{tournament graph}
%% \end{solution}


\eparts
\end{problem}
%%%%%%%%%%%%%%%%%%%%%%%%%%%%%%%%%%%%%%%%%%%%%%%%%%%%%%%%%%%%%%%%%%%%%
% Problem ends here
%%%%%%%%%%%%%%%%%%%%%%%%%%%%%%%%%%%%%%%%%%%%%%%%%%%%%%%%%%%%%%%%%%%%%

\endinput


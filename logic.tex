\chapter{First-Order Logic}

\newcommand{\solves}{\text{Solves}}
\newcommand{\probs}{\text{Probs}}
\newcommand{\even}{\text{Evens}}
\newcommand{\primes}{\text{Primes}}


\section{Logical Deductions }

Logical deductions or \emph{inference rules} are used to prove new
propositions using previously proved ones.

A fundamental inference rule is \emph{modus ponens}.  This rule says that
a proof of $P$ together with a proof of $P \implies Q$ is a proof of
$Q$.

Inference rules are sometimes written in a funny notation.  For example,
\emph{modus ponens} is written:
\begin{rul*}
\Rule{P, \quad P \implies Q}{Q}
\end{rul*}

When the statements above the line, called the \emph{antecedents}, are
proved, then we can consider the statement below the line, called the
\emph{conclusion} or \emph{consequent}, to also be proved.

A key requirement of an inference rule is that it must be \emph{sound}: any
assignment of truth values that makes all the antecedents true must also
make the consequent true.  So if we start off with true axioms and apply
sound inference rules, everything we prove will also be true.

There are many other natural, sound inference rules, for example:
\begin{rul*}
\Rule{P \implies Q, \quad Q \implies R}{P \implies R}
\end{rul*}

\begin{rul*}
\Rule{\neg{P} \implies Q, \quad \neg{Q}}{P}
\end{rul*}

\begin{rul*}
\Rule{\neg{P} \implies \neg{Q}}{Q \implies P}
\end{rul*}

On the other hand,
\begin{rul*}
\Rule{\neg{P} \implies \neg{Q}}{P \implies Q}
\end{rul*}
is not sound: if $P$ is assigned $\true$ and $Q$ is assigned $\false$, then
the antecedent is true and the consequent is not.

\begin{notesproblem}
Prove that a propositional inference rule is sound iff the conjunction
(AND) of all its antecedents implies its consequent.
\end{notesproblem}

As with axioms, we will not be too formal about the set of legal inference
rules.  Each step in a proof should be clear and ``logical''; in
particular, you should state what previously proved facts are used to
derive each new conclusion.

\hyperdef{preds}{preds}{\section{Predicates}}

A \term{predicate} is a proposition whose truth depends on the value of
one or more variables.  For example,
%
\begin{center}
``$n$ is a perfect square''
\end{center}
%
is a predicate whose truth depends on the value of $n$.  The predicate
is true for $n = 4$ since four is a perfect square, but false for $n =
5$ since five is not a perfect square.

Like other propositions, predicates are often named with a letter.
Furthermore, a function-like notation is used to denote a predicate
supplied with specific variable values.  For example, we might name
our earlier predicate $P$:
%
\[
P(n) \eqdef \text{``$n$ is a perfect square''}
\]
%
Now $P(4)$ is true, and $P(5)$ is false.

This notation for predicates is confusingly similar to ordinary function
notation.  If $P$ is a predicate, then $P(n)$ is either \textit{true} or
\textit{false}, depending on the value of $n$.  On the other hand, if $p$
is an ordinary function, like $n^2 + 1$, then $p(n)$ is a
\textit{numerical quantity}.  \textbf{Don't confuse these two!}

\subsection{Quantifying a Predicate}

There are a couple of assertions commonly made about a predicate: that it
is \textit{sometimes} true and that it is \textit{always} true.  For
example, the predicate
%
\[
\text{``$x^2 \geq 0$''}
\]
%
is always true when $x$ is a real number.  On the other hand, the
predicate
%
\[
\text{``$5x^2 - 7 = 0$''}
\]
%
is only sometimes true; specifically, when $x = \pm \sqrt{7/5}$.

There are several ways to express the notions of ``always true'' and
``sometimes true'' in English.  The table below gives some general
formats on the left and specific examples using those formats on the
right.  You can expect to see such phrases hundreds of times in
mathematical writing!
%
\begin{center}
\begin{tabular}{ll}
\multicolumn{2}{c}{\textbf{Always True}} \\[1ex]
For all $n$, $P(n)$ is true. & For all $x$, $x^2 \geq 0$. \\
$P(n)$ is true for every $n$. & $x^2 \geq 0$ for every $x$. \\[2ex]
\multicolumn{2}{c}{\textbf{Sometimes True}} \\[1ex]
There exists an $n$ such that $P(n)$ is true. & There exists an $x$ such that $5x^2 - 7 = 0$.\\
$P(n)$ is true for some $n$. & $5x^2 - 7 = 0$ for some $x$.\\
$P(n)$ is true for at least one $n$. & $5x^2-7=0$ for at least one $x$.
\end{tabular}
\end{center}

All these sentences quantify how often the predicate is true.
Specifically, an assertion that a predicate is always true is called a
\term{universal} quantification, and an assertion that a predicate is
sometimes true is an \term{existential} quantification.  Sometimes the
English sentences are unclear with respect to quantification:
%
\begin{center}
  ``If you can solve any problem we come up with, then you get an \emph{A}
  for the course.''
\end{center}
%
The phrase ``you can solve any problem we can come up with'' could
reasonably be interpreted as either a universal or existential
quantification:
%
\begin{quote}
``you can solve \textit{every} problem we come up with,''
\end{quote}
or maybe
\begin{quote}
``you can solve \textit{at least one} problem we come up with.''
\end{quote}
%
In any case, notice that this quantified phrase appears inside a
larger if-then statement.  This is quite normal; quantified statements
are themselves propositions and can be combined with and, or, implies,
etc., just like any other proposition.

\subsection{More Cryptic Notation}

There are symbols to represent universal and existential
quantification, just as there are symbols for ``and'' ($\wedge$),
``implies'' ($\implies$), and so forth.  In particular, to say that a
predicate, $P$, is true for all values of $x$ in some set, $D$, one
writes:
%
\[
\forall x \in D.\; P(x)
\]
%
The symbol $\forall$ is read ``for all'', so this whole expression is
read ``for all $x$ in $D$, $P(x)$ is true''.  To say that a predicate
$P(x)$ is true for at least one value of $x$ in $D$, one writes:
%
\[
\exists x \in D.\; P(x)
\]
%
The backward-E is read ``there exists''.  So this expression would be
read, ``There exists an $x$ in $D$ such that $P(x)$ is true.''  The
symbols $\forall$ and $\exists$ are always followed by a variable
---usually with an indication of the set the variable ranges over ---and
then a predicate, as in the two examples above.

As an example, let $\probs$ be the set of problems we come up with,
$\solves(x)$ be the predicate ``You can solve problem $x$'', and $G$ be
the proposition, ``You get an \emph{A} for the course.''  Then the two
different interpretations of
%
\begin{quote}
``If you can solve any problem we come up with, then you get an \emph{A} for the course.''
\end{quote}
%
can be written as follows:
%
\[
(\forall x \in \probs.\; \solves(x)) \implies G,
\]
or maybe
\[
(\exists x \in \probs.\; \solves(x)) \implies G.
\]

\subsection{Mixing Quantifiers}

Many mathematical statements involve several quantifiers.  For
example, Goldbach's Conjecture states:
%
\begin{center}
``Every even integer greater than 2 is the sum of two primes.''
\end{center}
%
Let's write this more verbosely to make the use of quantification
clearer:
%
\begin{quote}
For every even integer $n$ greater than 2,
there exist primes $p$ and $q$ such that $n = p + q$.
\end{quote}
%
Let $\even$ be the set of even integers greater than 2, and let $\primes$ be the
set of primes.  Then we can write Goldbach's Conjecture in logic
notation as follows:
%
\[
\underbrace{\forall n \in \even}_{\substack
    {\text{for every even} \\
     \text{integer $n > 2$}}}
\
\underbrace{\exists p \in \primes\ \exists q \in \primes.}_{\substack
    {\text{there exist primes} \\
     \text{$p$ and $q$ such that}}}
\ n = p + q.
\]

\subsection{Order of Quantifiers}

Swapping the order of different kinds of quantifiers (existential or
universal) usually changes the meaning of a proposition.  For example, let's return to one of our initial, confusing statements:
\begin{center}
``Every American has a dream.''
\end{center}

This sentence is ambiguous because the order of quantifiers is
unclear.  Let $A$ be the set of Americans, let $D$ be the set of
dreams, and define the predicate $H(a, d)$ to be ``American $a$ has
dream $d$.''.  Now the sentence could mean there is a single dream
that every American shares:
\[
\exists\, d \in D\; \forall a \in A.\; H(a, d)
\]
For example, it might be that every American shares the dream of owning
their own home.

Or it could mean that every American has a personal dream:
\[
\forall a \in A\; \exists\, d \in D.\; H(a, d)
\]
For example, some Americans may dream of a peaceful retirement, while
others dream of continuing practicing their profession as long as they
live, and still others may dream of being so rich they needn't think at
all about work.

Swapping quantifiers in Goldbach's Conjecture creates a patently false
statement that every even number $\geq 2$ is the sum of \emph{the same}
two primes:
\[
\underbrace{\exists\, p \in \primes\ \exists\, q \in \primes}_{\substack
    {\text{there exist primes} \\
     \text{$p$ and $q$ such that}}}
\
\underbrace{\forall n \in \even.}_{\substack
    {\text{for every even} \\
     \text{integer $n > 2$}}}
\ n = p + q.
\]

\subsubsection{Variables over One Domain}
When all the variables in a formula are understood to take values from the
same nonempty set, $D$, it's conventional to omit mention of $D$.  For
example, instead of $\forall x \in D\; \exists y \in D.\; Q(x,y)$ we'd write
$\forall x \exists y.\; Q(x,y)$.  The unnamed nonempty set that $x$ and
$y$ range over is called the \term{domain} of the formula.

It's easy to arrange for all the variables to range over one domain.  For
example, Goldbach's Conjecture could be expressed with all variables
ranging over the domain $\naturals$ as
\[
\forall n.\; n \in \even \implies (\exists\, p \exists\, q.\; p \in \primes \land
q \in \primes \land n = p + q).
\]

\subsection{Negating Quantifiers}

There is a simple relationship between the two kinds of quantifiers.  The
following two sentences mean the same thing:
%
\begin{quote}

It is not the case that everyone likes to snowboard.

There exists someone who does not like to snowboard.

\end{quote}
%
In terms of logic notation, this follows from a general property of
predicate formulas:
%
\[
\neg \forall x.\; P(x)
\hspace{0.1in} \text{is equivalent to} \hspace{0.1in}
\exists x.\; \neg P(x).
\]
%
Similarly, these sentences mean the same thing:
%
\begin{quote}
There does not exist anyone who likes skiing over magma.

Everyone dislikes skiing over magma.
\end{quote}
%
We can express the equivalence in logic notation this way:
%
\begin{equation}\label{nE}
(\neg \exists x.\; P(x))  \iff \forall x.\; \neg P(x).
\end{equation}
%
The general principle is that \textit{moving a ``not'' across a
quantifier changes the kind of quantifier.}

\iffalse Logicians have worked very hard to define strict rules for the
use of logic notation so that ideas can be expressed with absolute rigor.
It's all quite charming and clever.  However, the sad irony is that
applied mathematicans usually use their beloved notation as a crude
shorthand, breaking the rules and abusing the notation willy-nilly ---sort
of like pounding nails with fine china.  \fi

\subsection{Validity}

A propositional formula is called \term{valid} when it evaluates to \true\
no matter what truth values are assigned to the individual propositional
variables.  For example, the propositional version of the Distributive Law
is that $P \conj (Q \disj R)$ is equivalent to $(P \conj Q) \disj (P \conj
R)$.  This is the same as saying that
\[
[P \conj (Q \disj R)] \iff [(P \conj Q) \disj (P \conj R)]
\]
is valid.

The same idea extends to predicate formulas, but to be valid, a
formula now must evaluate to true no matter what values its variables
may take over any unspecified domain, and no matter what
interpretation a predicate variable may be given.  For example, we
already observed that the rule for negating a quantifier is captured
by the valid assertion~\eqref{nE}.

Another useful example of a valid assertion is
\[
\exists x \forall y.\; P(x,y) \implies \forall y \exists x.\; P(x,y).
\]
We could prove this as follows:
\begin{proof}
Let $D$ be the domain for the variables and $P_0$ be some
binary predicate\footnote{That is, a predicate that depends on two variables.}
on $D$.  We need to show that if $\exists x \in D\; \forall y \in D.\;
P_0(x,y)$ holds under this interpretation, then so does $\forall y \in D\;
\exists x \in D.\; P_0(x,y)$.

So suppose $\exists x \in D\; \forall y \in D.\; P_0(x,y)$.  Then some
element $x_0 \in D$ has the property that $P_0(x_0, y)$ is true for all $y
\in D$.  So for every $y \in D$, there is some $x \in D$, namely $x_0$,
such that $P_0(x,y)$ is true.  That is, $\forall y \in D\exists x \in D.\;
P_0(x,y)$ holds; that is, $\forall y\; \exists x.\; P(x,y)$ holds under this
interpretation, as required.
\end{proof}

On the other hand,
\[
\forall y \exists x.\; P(x,y) \implies \exists x \forall y.\; P(x,y).
\]
is \emph{not} valid.  We can prove this simply by describing an
interpretation where the hypothesis, $\forall y \exists x.\; P(x,y)$, is
true but the conclusion, $\exists x \forall y.\; P(x,y)$, is not true.
For example, let the domain be the integers and $P(x,y)$ mean $x > y$.
Then the hypothesis would be true because, given a value, $n$, for $y$ we
could choose the value of $x$ to be $n+1$, for example.  But under this
interpretation the conclusion asserts that there is an integer that is
bigger than all integers, which is certainly false.  An interpetation like
this which falsifies an assertion is called a \emph{counter model} to the
assertion.

\newpage
\section{Glossary of Symbols}
\begin{center}
\begin{tabular}{ll}
symbol &  meaning\\
\hline
$\eqdef$ & is defined to be\\
$\land$ & and\\
$\lor$ & or\\
$\implies$ & implies\\
$\neg$    & not\\
$\neg{P}$ & not $P$\\
$\bar{P}$ & not $P$\\
$\iff$    & iff\\
$\iff$    & equivalent\\
$\oplus$   & xor\\
$\exists$ & exists\\
$\forall$ & for all\\
$\in$   &  is a member of\\

\iffalse

$\subseteq$ & is a subset of\\
$\subset$ & is a proper subset of\\
$\union$  & set union\\
$\intersect$ & set intersection\\
$\bar{A}$ & complement of a set, $A$\\
$\power(A)$ & powerset of a set, $A$\\
$\emptyset$ & the empty set, $\set{}$\\
$\naturals$ & nonnegative integers \\
$\integers$ & integers\\
$\integers^+$ & positive integers\\
$\integers^-$ & negative integers\\
$\rationals$ & rational numbers\\
$\reals$ & real numbers\\
$\complexes$ & complex numbers\\
$\emptystring$ & the empty string/list
\fi

\end{tabular}
\end{center}

\endinput


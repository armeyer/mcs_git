\documentclass[twoside,12pt]{article}
\usepackage{light}

\begin{document}

\generic{Top 10 Proof Techniques NOT Allowed in 6.042}{September 11, 2014}

\begin{description}
\item[{\bf 10.}]{\bf Proof by throwing in the kitchen sink:} The author writes down
every theorem or result known to mankind and then adds a few more just
for good measure.  When questioned later, the author correctly
observes that the proof contains all the key facts needed to actually
prove the result.  Very popular strategy on 6.042 exams.  Known to
result in extra credit with sufficient whining.
\item[{\bf 9.}]{\bf Proof by example:} The author gives only the case n = 2 and suggests
that it contains most of the ideas of the general proof.
\item[{\bf 8.}]{\bf Proof by vigorous handwaving:} A faculty favorite.  Works well in
any classroom or seminar setting.
\item[{\bf 7.}]{\bf Proof by cumbersome notation:} Best done with access to at least
four alphabets and special symbols.  Helps to speak several foreign
languages.
\item[{\bf 6.}]{\bf Proof by exhaustion:} An issue or two of a journal devoted to your proof is useful.  Works well in combination with proof by throwing in
the kitchen sink and proof by cumbersome notation.
\item[{\bf 5.}]{\bf Proof by omission:} \\
    ``The reader may easily supply the details.'' \\
    ``The other 253 cases are analogous.''\\
    ``...''
\item[{\bf 4.}]{\bf Proof by picture:} A more convincing form of proof by example.
Combines well with proof by omission.
\item[{\bf 3.}]{\bf Proof by vehement assertion:} It is useful to have some kind of authority in relation to the audience.
\item[{\bf 2.}]{\bf Proof by appeal to intuition:} Cloud-shaped drawings frequently help here.  Can be seen on 6.042 exams when there was not time to
include a complete proof by throwing in the kitchen sink.
\item[{\bf 1.}]{\bf Proof by reference to eminent authority:}\\
``I saw Fermat in the elevator and he said he had a proof . . .''
\end{description}

\newpage

Here are some other common proof techniques that can be very useful,
but which are not recommended for this class.

\begin{itemize}

\item{\bf Proof by intimidation:} Can involve phrases such
as: ``Any moron knows that...'' or ``You know the Zorac Theorem of
Hyperbolic Manifold Theory, right?''  Sometimes seen in 6.042
tutorials.

\item{\bf Proof by intimidation (alternate form):} Consists
of a single word: ``Trivial.''  Often used by faculty who don't know
the proof.

\item {\bf Proof by reference to inaccessible literature:}
The author cites a simple corollary of a theorem to be found in a
privately circulated memoir of the Slovenian Philological Society,
1883.  It helps if the issue has not been translated.\item
{\bf Proof by semantic shift:} Some standard but inconvenient
definitions are changed for the statement of the result.

\item {\bf Proof by cosmology:} The negation of the
proposition is unimaginable or meaningless.  Popular for proofs of the
existence of God.

\item {\bf Proof by obfuscation:} A long plotless sequence of
true and/or meaningless syntactically related statements.

\item {\bf Proof by wishful citation:} The author cites the
negation, converse, or generalization of a theorem from the literature
to support his claims.

\item {\bf Proof by funding:} How could three different government
agencies be wrong?

\item {\bf Proof by personal communication:}\\ ``$x^n$ + $y^n$ $\neq$
$z^n$ for $n>2$'' [Fermat, personal communication].

\item {\bf Proof by importance:} A large body of useful consequences
all follow from the proposition in question.

\item {\bf Proof by accumulated evidence:} Long and diligent search
has not revealed a counterexample.

\item {\bf Proof by mutual reference:} In reference A, Theorem 5 is
said to follow from Theorem 3 in reference B, which is shown from
Corollary 6.2 in reference C, which is an easy consequence of Theorem
5 in reference A.

\item {\bf Proof by ghost reference:} Nothing even remotely resembling
the cited theorem appears in the reference given.

\item {\bf Proof by forward reference:} Reference is usually to a
forthcoming paper of the author, which is often not as forthcoming as
the first.

\item {\bf Proof by metaproof:} A method is given to construct the
desired proof. The correctness of the method is proved by any of the
above techniques.

\end{itemize}
\end{document}

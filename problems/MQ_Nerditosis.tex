
\documentclass[problem]{mcs}

\begin{pcomments}
  \pcomment{MQ_Nerditosis}
\end{pcomments}

\pkeywords{
  conditional_probability
  tree_diagram
}

%%%%%%%%%%%%%%%%%%%%%%%%%%%%%%%%%%%%%%%%%%%%%%%%%%%%%%%%%%%%%%%%%%%%%
% Problem starts here
%%%%%%%%%%%%%%%%%%%%%%%%%%%%%%%%%%%%%%%%%%%%%%%%%%%%%%%%%%%%%%%%%%%%%

% F09, S09, S07

\begin{problem}
There is a rare and deadly disease called Nerditosis which afflicts
about 10 MIT students in 100.  One symptom is a compulsion to refer to
everything---fields of study, classes, buildings, etc.---using
numbers.  It's horrible.  As victims enter their final, downward
spiral, they're awarded a degree from MIT. Year ago, Professor Meyer
came up with a test to diagnose this tragic disease, but it's not
perfect.

The facts of the test:
\begin{itemize}

\item If you have Nerditosis, there's a 10\% chance the test will say you do not.

\item If you don't have it, there's a 30\% chance the test will say you do.

\end{itemize}
A random MIT student is tested for the disease. If the test is
positive, then what is the probability that the person has the
disease?

Hint: Use the four-step method and a tree diagram.

\begin{solution}
\TBA{tree diagram}

An alternative approach uses Bayes Theorem~\bref{bayesrule}.  Let $A$
be the event that the person has the disease.  Let $B$ be the event
that the test is positive.  We know
\[
\pr{A} = 1/10, \qquad \prcond{\bar{B}}{A} = 1/10, \qquad \prcond{B}{\bar{A}} = 3/10\,.
\]

So by Total Probability~\bref{total_prob_Ebar},
\begin{align*}
\pr{B} & = \prcond{B}{A} \prob{A} +  \prcond{B}{\bar{A}}\prob{\bar{A}}\\
       & = (1- \prcond{\bar{B}}{A}) \prob{A} +  \prcond{B}{\bar{A}}(1-\prob{A})\\
       & = 9/10 \cdot 1/10 + 3/10 \cdot 9/10\\
       & = 36/100.
\end{align*}

Then by Bayes Rule
\[
\prcond{A}{B} = \frac{\prcond{B}{A}\cdot \pr{A}}{\pr{B}} = \frac{9/10 \cdot 1/10}{36/100} = \frac{1}{4}.
\]

\end{solution}
\end{problem}

%%%%%%%%%%%%%%%%%%%%%%%%%%%%%%%%%%%%%%%%%%%%%%%%%%%%%%%%%%%%%%%%%%%%%
% Problem ends here
%%%%%%%%%%%%%%%%%%%%%%%%%%%%%%%%%%%%%%%%%%%%%%%%%%%%%%%%%%%%%%%%%%%%%

\endinput

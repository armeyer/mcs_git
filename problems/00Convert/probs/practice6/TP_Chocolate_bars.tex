\documentclass[problem]{mcs}

\begin{pcomments}
    \pcomment{TP_Chocolate_bars}
    \pcomment{Converted from prob3.scm
              by scmtotex and dmj
              on Sun 13 Jun 2010 02:58:06 PM EDT}
 \pcomment{revised by drewe 27 July 2011}
\end{pcomments}

\begin{problem}

%% type: multi-part
%% title: Chocolate bars

We are given a chocolate bar with $m \times n$ squares of chocolate,
and our task is to divide it into $mn$ individual squares.  We are
only allowed to split a chocolate bar using a vertical or a horizontal
cut.

For example, suppose that the chocolate bar is $3 \times 2$.  A
horizontal cut between the first and second rows of squares splits it
into two bars, a $1 \times 2$ bar and a $2 \times 2$ bar.  One cut of
the $1 \times 2$ splits it into individual squares, and three more
cuts splits the $2 \times 2$ bar into squares.  So a total of 5 cuts
splits the whole $3 \times 2$ bar into squares.

At each step of the division process, let $s$ be the number of splits
already performed, and $p$ the number of pieces of chocolate obtained.

\bparts

\ppart
%% type: short-answer
%% title: 

Which of the following predicates are \emph{preserved invariants} for
this process?

\begin{enumerate}
  
\item $s = p - 1$
  
\item $s \neq p$
  
\item $s = mn - p$

\end{enumerate}

\begin{solution}

1, 
2.

At every step, each of $s$ and $p$ increases by 1.  So, each side of
the equation in~(1) and inequation in~(2) increases by the same
amount.  Therefore predicates (1) and~(2) are preserved.  In contrast,
the left side of~(3) increases while the right side decreases.
\end{solution}


\ppart
%% type: short-answer
%% title: 

Which of the following derived variables are \emph{strictly
decreasing}?

\begin{enumerate}
       
\item $mn - p$
       
\item $s$
       
\item $p - s$
             
\end{enumerate}

\begin{solution}

1.

At every step, each of $s$ and $p$ increases by~1. So, $mn-p$ is
strictly decreasing, $s$ is strictly increasing, and $p-s$ is
constant.
\end{solution}

\ppart
%% type: short-answer
%% title: 

What is the number of pieces $p$ of chocolate at the end of the division
process?

\begin{solution}

$p=mn$.

At the end of the process every square is separated, since otherwise,
more cuts could be performed.  So $p=mn$.
\end{solution}

\ppart
%% type: short-answer
%% title: 

What is the number of splits performed to reach the end of the
division ?

\begin{solution}

$s=mn-1$.

By Part~1, $s=p-1$ is an invariant. Moreover, it is true at the
beginning (because then $s=0$ and $p=1$.  Therefore, by the Invariant
Theorem, the predicate is true throughout the process. In particular,
it is true at the end. By Part~3, we know that at the end
$p=mn$. Substituting in the invariant, we get $s=mn-1$.
\end{solution}

\eparts

\end{problem}

\endinput

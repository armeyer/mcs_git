\documentclass[problem]{mcs}

\begin{pcomments}
  \pcomment{from: F06.ps8}
\end{pcomments}

\pkeywords{
   %I don't know the format for these keywords, how do I look them up?
	recursion
        linear recurrence
        plug_and_chug
}

%%%%%%%%%%%%%%%%%%%%%%%%%%%%%%%%%%%%%%%%%%%%%%%%%%%%%%%%%%%%%%%%%%%%%
% Problem starts here
%%%%%%%%%%%%%%%%%%%%%%%%%%%%%%%%%%%%%%%%%%%%%%%%%%%%%%%%%%%%%%%%%%%%%

\begin{problem} %from Leighton's PS8
Less well-known than the Towers of Hanoi--- but no less fascinating---
are Towers of Sheboygan, WI.  As in Hanoi, the puzzle in Sheboygan
involves 3 posts and $n$ disks of different sizes.  Initially, all the
disks are on post \#1:

\begin{center}
\unitlength=0.6pt
\begin{picture}(600,190)(0,-30)
% \put(0,-30){\dashbox(600,190){}} % bounding box

\put(99,0){\dashbox(2,140){}}
\put(99,140){\framebox(2,20){}}
\put(30,0){\framebox(140,20){}}
\put(40,20){\framebox(120,20){}}
\put(50,40){\framebox(100,20){}}
\put(60,60){\framebox(80,20){}}
\put(70,80){\framebox(60,20){}}
\put(80,100){\framebox(40,20){}}
\put(90,120){\framebox(20,20){}}
\put(299,0){\framebox(2,160){}}
\put(499,0){\framebox(2,160){}}
\put(0,-5){\framebox(600,5){}}
\put(100,-20){\makebox(0,0){Post \#1}}
\put(300,-20){\makebox(0,0){Post \#2}}
\put(500,-20){\makebox(0,0){Post \#3}}
\end{picture}
\end{center}

The objective is to transfer all $n$ disks to post \#2 via a sequence
of moves.  A move consists of removing the top disk from one post and
dropping it onto another post with the restriction that a larger disk
can never lie above a smaller disk.  Furthermore, a local ordinance
requires that \textit{a disk can be moved only from post \#1 to post
\#2, from \#2 to \#3, or from \#3 to \#1.}  Thus, for example, moving
a disk directly from post \#1 to post \#3 is not permitted.
\end{problem}

\bparts
\ppart Briefly describe a solution to the Towers of Sheboygan puzzle.

\begin{solution} 
Use a recursive procedure: to move an initial stack of $n$
blocks to the next post, move the top stack of $n-1$ disks to the furthest
post by moving it to the next post two times, then move the big, $n$th
disk to the next post, and finally move the top stack another two times to
land on top of the big disk.

This procedure leads to a simple linear recurrence, and gets full credit
as an answer.  But it turns out not to be the most efficient procedure for
moving the stack.

Namely, a better (indeed optimal, but we won't prove this) procedure can
be defined in terms of two mutually recursive procedures, procedure
$P_1(n)$ for moving a stack of $n$ disks 1 pole forward, and $P_2(n)$
for moving a stack of $n$ disks 2 poles forward.  It's obvious how to do
this for $n=1$.  For $n>1$, define:

$P_1(n)$: Apply $P_2(n-1)$ to move the top $n-1$ disks two poles forward
to the third pole.  Then move the remaining big disk once to land on
the second pole.  Then apply $P_2(n-1)$ again to move the stack of $n-1$
disks two poles forward from the third pole to land on top of the big
disk.

$P_2(n)$: Apply $P_2(n-1)$ to move the top $n-1$ disks two poles forward
to land on the third pole.  Then move the remaining big disk to the second
pole.  Then apply $P_1(n-1)$ to move the stack of $n-1$ disks one pole
forward to land on the first pole.  Now move the big disk 1 pole forward
again to land on the third pole.  Finally, apply $P_2(n-1)$ again to move
the stack of $n-1$ disks two poles forward to land on the big disk.
\end{solution} 

\ppart Let $S_n$ be the number of moves needed to solve the $n$-disk
problem.  Express $S_n$ with a recurrence equation and sufficient base
cases.

\begin{solution} 
For the first procedure, we have
\begin{align}
S_1 & = 1,\notag\\
S_n & = 2S_{n-1} + 1 + 2S_{n-1} = 4S_{n-1} + 1 & \text{for }n > 1\label{4S}.
\end{align}

For the second procedure, in addition to the number, $S_n$, of steps to
move a stack of $n$ disks one pole forward a stack of $n$, let 
$T_n$ be the number of steps to move a stack of $n$ disks two poles
forward.  From the definitions of procedures $P_1$ and $P_2$ we have
\begin{align}
S_1 & = 1,\notag\\
T_1 & = 2,\notag\\
S_n & = T_{n-1} + 1 + T_{n-1} & \text{for } n > 1,\label{ST}\\
T_n & = T_{n-1} + 1 + S_{n-1} + 1 + T_{n-1} & \text{for } n > 1.\label{TT}
\end{align}
From these equations we first calculate that $T_2=7$.  Then,
using~\eqref{ST} to substitute for $S_{n-1}$ in~\eqref{TT}, we conclude
that for $n > 2$,
\begin{equation}\label{Trec}
T_n = 2T_{n-1} + 2 + (2T_{n-2} +1) = 2T_{n-1} + 2T_{n-2} + 3.
\end{equation}
\end{solution} 

\ppart Find a closed-form expression for $S_n$ by solving the
recurrence.

\begin{solution} 
For recurrence~\eqref{4S}, Plug \& Chug works nicely:
\begin{align*}
S_{n} & = 4S_{n-1} + 1\\
      & = 4(4S_{n-2}+1) + 1\\
      & = 4^2 S_{n-2} + 4 + 1\\
      & = 4^2 (4S_{n-3}+1) + 4 + 1\\
      & = 4^3 S_{n-3}+ 4^2 + 4 + 1 = \dots\\
      & = 4^{n-1} S_1 + 4^{n-2} + \dots + 4^2 + 4 + 1\\
      & = 4^{n-1} + (4^{n-1}-1)/3\\
      & = \frac{4^n - 1}{3}
\end{align*}

For recurrence~\eqref{Trec}, we apply the general approach for
inhomogeneous linear recurrences: the characteristic polynomial is $x^2 -
2x - 2 $ with roots $1 \pm \sqrt{3}$, so the general solution to the
homogenous part of~\eqref{Trec} is
\[
A(1 + \sqrt{3})^n + B(1 - \sqrt{3})^n.
\]

Since the inhomogeneous term of~\eqref{Trec} is constant, we guess that a
particular solution will be some constant, $c$.  This requires that $c =
2c + 2c + 3$, namely, $c=-1$.  So the most general solution
to~\eqref{Trec} is of the form
\begin{equation}\label{AB}
A(1 + \sqrt{3})^n + B(1 - \sqrt{3})^n - 1.
\end{equation}
Now we use the boundary conditions $T_1 =2, T_2 = 7$ and~\eqref{AB} to
obtain linear equations in $A$ and $B$:
\begin{align*}
A(1 + \sqrt{3}) + B(1 - \sqrt{3}) - 1 & = 2\\
A(1 + \sqrt{3})^2 + B(1 - \sqrt{3})^2 - 1 & = 7.
\end{align*}


Solving these equations, we find $A = (3+2\sqrt{3})/6$ and $B =
(3-2\sqrt{3})/6$, so
\[
T_n = \frac{3+2\sqrt{3}}{6} (1 + \sqrt{3})^n + \frac{3-2\sqrt{3}}{6} (1 - \sqrt{
3})^n
      -1.
\]
Now using~\eqref{ST}, we conclude that
\[
S_n = \frac{3+2\sqrt{3}}{3}(1 + \sqrt{3})^{n-1} + \frac{3-2\sqrt{3}}{3}(1 - \sqr
t{3})^{n-1} - 1.
\]
In particular, we conclude that $S_n = \Theta((1+ \sqrt{3})^n) = o(4^n)$,
so the second procedure for moving a stack of $n$ disks is vastly more
efficient thah the first one.
\end{solution}

\eparts



\documentclass[problem]{mcs}

\begin{pcomments}
  \pcomment{PS_counting_graphs}
  \pcomment{subsumes FP_counting_graphs_f13}
  \pcomment{more suitable as CP--ARM 11/7/13}
  \pcomment{from: F08 PS9 -> S09 PS9; S09 CP7R}
  \pcomment{part(c) rewritten 4/8/11 by ARM}
  \pcomment{edited 11/3/11 by ARM }
\end{pcomments}

\pkeywords{
  counting
  graph
  asymmetric
  linear
  permutation
  simple_graph
}

%%%%%%%%%%%%%%%%%%%%%%%%%%%%%%%%%%%%%%%%%%%%%%%%%%%%%%%%%%%%%%%%%%%%%
% Problem starts here
%%%%%%%%%%%%%%%%%%%%%%%%%%%%%%%%%%%%%%%%%%%%%%%%%%%%%%%%%%%%%%%%%%%%%

\begin{problem}
This problem is about binary relations on the set of integers in the
interval $\Zintv{1}{n}$ and digraphs whose vertex set is
$\Zintv{1}{n}$.

\bparts

\ppart How many digraphs are there?

\exambox{1.0in}{0.6in}{0in}

\begin{solution}
\[
2^{n^2}
\]

There are $n^2$ potential edges, each of which may or
may not appear in a given graph.

\iffalse
Therefore, the number of graphs is:
\[
2^{n^2}
\]
\fi

\end{solution}

\ppart How many simple graphs are there?

\exambox{1.0in}{0.6in}{0in}

\begin{solution}
\[
2^{\binom{n}{2}}
\]

There are $\binom{n}{2}$ potential edges, each of which may or
may not appear in a given graph.

\iffalse
Therefore, the number of graphs is:
\[
2^{\binom{n}{2}}
\]
\fi

\end{solution}

\ppart How many asymmetric binary relations are there?

\exambox{1.0in}{0.6in}{0in}

\begin{solution}
\[
3^{\binom{n}{2}}
\]

There are no self-loops in an asymmetric relation and 
for each of the $\binom{n}{2}$ sets of two elements $a \neq b$,
either
\begin{enumerate}
\item $a \mrel{R} b$, or
\item $b \mrel{R} a$, or
\item neither,
\end{enumerate}
but not both.  
\iffalse
Therefore, the number of asymmetric binary relations is
\[
3^{\binom{n}{2}}
\]\fi

\end{solution}

\ppart How many linear strict partial orders are there?

\exambox{1.0in}{0.6in}{0in}

\begin{solution}
$n!$.

Since the partial order is linear, there is a unique listing of
the elements in decreasing partial order.  This listing defines a
bijection between the linear strict partial orders and the
permutations of $\Zintv{1}{n}$.
\end{solution}

\eparts
\end{problem}

%%%%%%%%%%%%%%%%%%%%%%%%%%%%%%%%%%%%%%%%%%%%%%%%%%%%%%%%%%%%%%%%%%%%%
% Problem ends here
%%%%%%%%%%%%%%%%%%%%%%%%%%%%%%%%%%%%%%%%%%%%%%%%%%%%%%%%%%%%%%%%%%%%%

\endinput

\documentclass[problem]{mcs}

\begin{pcomments}
  \pcomment{TP_Stable_Marriage_Invariants_F17}
  \pcomment{Converted from
  ./00Convert/probs/practice6/stable-marriage-invariants.scm by
  scmtotex and drewe on Thu 28 Jul 2011 01:05:16 PM EDT}
  \pcomment{similar to TP_mating_ritual_invariant}
\end{pcomments}

\begin{problem}
We are interested in preserved invariants of the Mating
Ritual\inbook{~\bref{mating_ritual_sec}} for finding stable marriages
when there are an equal number of boys and girls.  Let Angelina and
Jen be two of the girls, and Keith and Tom be two of the boys.

Which of the following predicates are preserved invariants of the
Mating Ritual no matter what the preferences are among the boys and
girls?

\iffalse (Remember that a predicate that is always false is an
invariant---check the definition of invariant to see why.)\fi

\bparts

\iffalse
\ppart Angelina is crossed off Tom's list and she has a suitor that
she prefers to Tom.

\begin{solution}
Invariant: This is the basic preserved invariant used to verify the Ritual.
\end{solution}
\fi

\ppart Tom is serenading Jen.

\begin{solution}
Not invariant: If Tom serenades Jen and gets rejected by her,
he will stop serenading her.
\end{solution}

\ppart Tom is not serenading Jen.

\begin{solution}
Not invariant: Tom might serenade Angelina, get rejected by
her, and then serenade Jen next.
\end{solution}

\ppart Tom's list of girls to serenade is empty.

\begin{solution}
Invariant: With an equal number of boys and girls, Tom's list will
never be empty, since the Ritual guarantees he will be married in the
end.  So this predicate is a preserved invariant because it is always
false.

In fact, it would still be a preserved invariant even if there were
more boys than girls.  That's because no girl ever gets added to a
boy's list, so if his list becomes empty, it will stay empty.
\end{solution}

\iffalse
\ppart
 All the boys have the same number of girls left uncrossed in their lists.

\begin{solution}
 Not invariant: Suppose all the boys like Angelina best.  The she will
 reject all but her favorite, say it's Tom, on the first day.  Now, on
 the second day, Tom's list will be one longer than all the other
 boys' lists.
\end{solution}
\fi

% Zoran: commenting this part out for Fall15 2nd midterm
%\ppart Jen is crossed off Keith's list.
%
%\begin{solution}
%Invariant: Again, once Keith crosses a girl off his list, he will
%never un-cross her.
%\end{solution}

\ppart Jen is crossed off Keith's list and Keith prefers Jen to anyone
he is serenading.

\begin{solution}
Invariant: Keith crosses off girls in order of preference, so if Jen
is crossed off, Keith likes her better than anybody that's left.
\end{solution}

\ppart Jen is the only girl on Keith's list.

\begin{solution}
Invariant: No girls get added to a list, so if Jen is the only girl on
Keith's list, she must be the girl he marries at the end.

Note that in the case of more boys than girls, this need not by an
invariant.
\end{solution}

\eparts

\end{problem}

\endinput

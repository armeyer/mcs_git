\documentclass[problem]{mcs}

\begin{pcomments}
  \pcomment{MQ_fkbij}
  \pcomment{ARM 1/27/12}
  \pcomment{done more simply in number_theory}
\end{pcomments}

\pkeywords{
  modular_arithmetic
  relatively_prime
  inverse
  bijection
}

\begin{problem}
For any set $P$ of numbers in $[0,n)$ and any $m \in [0,n)$, define
\[
mP \eqdef \set{m \cdot_n p \suchthat p \in P}.
\]
Prove that if $m$ is relatively prime to $n$, then the function $f_m:
P \to mP$ defined by the rule
\[
f_m(p) \eqdef m \cdot_n p,
\]
is a bijection.

\begin{solution}

\begin{proof}
The function $f_m$ is total and surjective ($[= 1\ \text{out}, \geq
  1\ \text{in}]$) by definition.  It is also an injection, because
\[
f_m(p_1) = f_m(p_2)
\]
is equivalent to
\[
m \cdot_n p_1 = m \cdot_n p_2, 
\]
and since $m$ is relatively prime to $n$, we can cancel $m$ to conclude
that $p_1 = p_2$.

So $f_k$ is a bijection.
\end{proof}

\end{solution}

\end{problem}

\endinput

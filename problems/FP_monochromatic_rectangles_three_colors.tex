\documentclass[problem]{mcs}

\begin{pcomments}
  \pcomment{FP_monochromatic_rectangle_three_colors.tex}
  \pcomment{modified from PS_monochromatic_rectangle for f17 midterm4}
  \pcomment{extended in PS_monochromatic_rectangle_4}
  \pcomment{Author: Justin Venezuela (jven@mit.edu)}
  \pcomment{Generalized from USAMTS Problem 4/1/18 (usamts.org)}
  \pcomment{edited 11/3/11 by ARM}
  \pcomment{last part suggested by CH, added by ARM, 4/26/14}
\end{pcomments}

\pkeywords{
  counting
  pigeonhole
}

%%%%%%%%%%%%%%%%%%%%%%%%%%%%%%%%%%%%%%%%%%%%%%%%%%%%%%%%%%%%%%%%%%%%%
% Problem starts here
%%%%%%%%%%%%%%%%%%%%%%%%%%%%%%%%%%%%%%%%%%%%%%%%%%%%%%%%%%%%%%%%%%%%%

\begin{problem}

Let $R$ be a rectangular cheesboard with $82$ rows and $4$ columns. Each square of $R$ may be colored either red, white or blue.

\bparts
\ppart How many different colorings of $R$ are possible?

\begin{center}
\exambox{1in}{0.5in}{0.0in}
\end{center}

\begin{solution}
  There are $82\cdot 4$ squares in $R$ and each square can independently be colored in three different ways. By the Product Rule, the total number of colorings of $R$ is $3^{82\cdot 4}$.
\end{solution}

\examspace[0.5in]

\ppart
Explain why in every coloring,  at least two of
the 82 rows in $R$ must have identical color patterns.

\examspace[1.5in]

\begin{solution}
There are only $3^4=81$ ways to color the squares in a row of length 4,
and there are 82 rows, so by the Pigeonhole Principle, some pair of rows
must have identical color patterns.
\end{solution}

\ppart\label{ppart:rect} Prove that any coloring of $R$ contains four squares with the same color that
form the corners of a rectangle.

\examspace[1.5in]

\begin{solution}
Each row is length 4 and there are only 3 colors, so some color must
occur (at least) twice.  So the two rows with identical color patterns have the same
color occurring (at least) twice in corresponding positions, and these four
points form the corners of a rectangle.
\end{solution}

\eparts
\end{problem}

%%%%%%%%%%%%%%%%%%%%%%%%%%%%%%%%%%%%%%%%%%%%%%%%%%%%%%%%%%%%%%%%%%%%%
% Problem ends here
%%%%%%%%%%%%%%%%%%%%%%%%%%%%%%%%%%%%%%%%%%%%%%%%%%%%%%%%%%%%%%%%%%%%%

\endinput

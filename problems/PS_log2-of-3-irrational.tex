\documentclass[problem]{mcs}

\begin{pcomments}
  \pcomment{new by ARM from Jean Gallier email 8/21/09}
  \pcomment{Fix hard reference to Week 1 notes.}
\end{pcomments}

\pkeywords{
  irrational
  power 
  contradiction
}

%%%%%%%%%%%%%%%%%%%%%%%%%%%%%%%%%%%%%%%%%%%%%%%%%%%%%%%%%%%%%%%%%%%%%
% Problem starts here
%%%%%%%%%%%%%%%%%%%%%%%%%%%%%%%%%%%%%%%%%%%%%%%%%%%%%%%%%%%%%%%%%%%%%

\begin{problem}
  A recent class problem %~\ref{CP_irrational_raised_to_an_irrational},
  proved that the were irrational numbers $a,b$ such that $a^b$ was
  rational.  Unfortunately, that proof was \emph{nonconstructive}: it
  didn't reveal a specific pair, $a,b$, with this property.  But in fact,
  it's easy to do this: let $a \eqdef \sqrt{2}$ and $b \eqdef 2\log_2 3$.
  
  We know $\sqrt{2}$ is irrational, and obviously $a^b =3$.  Finish the
  proof that this $a,b$ pair works, by showing that $2\log_2 3$ is
  irrational.

\begin{solution}
\begin{proof}
  Suppose to the contrary that $2\log_2 3$ was rational.  Then $\log_2 3$
  must also be rational, say $\log_2 3 =m/n$ for some positive integers
  $m$ and $n$.  So $m = n\log_2 3$.  Now raising 2 to each side of this
  equation gives
\begin{equation}\label{2m2n3}
2^m = 2^{n \log_2 3} =  2^n\cdot 3.
\end{equation}
But this is impossible, since right hand side of~\eqref{2m2n3} is
divisible by 3 and the left hand side is not.

So $2\log_2 3$ must be irrational.
\end{proof}
\end{solution}
\end{problem}

%%%%%%%%%%%%%%%%%%%%%%%%%%%%%%%%%%%%%%%%%%%%%%%%%%%%%%%%%%%%%%%%%%%%%
% Problem ends here
%%%%%%%%%%%%%%%%%%%%%%%%%%%%%%%%%%%%%%%%%%%%%%%%%%%%%%%%%%%%%%%%%%%%%

\documentclass[problem]{mcs}

\begin{pcomments}
  \pcomment{MQ_m_envelopes_induction}
  \pcomment{Adapted by Chinmay Hegde 2/20/14 from question on stackexchange.}
  \pcomment{closely related to TP_m_envelopes_WOP}
\end{pcomments}

\pkeywords{
  induction
  envelopes
  dollars
  binary representations of integers
  base-two representations
}

%%%%%%%%%%%%%%%%%%%%%%%%%%%%%%%%%%%%%%%%%%%%%%%%%%%%%%%%%%%%%%%%%%%%%
% Problem starts here
%%%%%%%%%%%%%%%%%%%%%%%%%%%%%%%%%%%%%%%%%%%%%%%%%%%%%%%%%%%%%%%%%%%%%

\begin{problem}

You are given $n$ envelopes, numbered $0, 1, \ldots, n-1$.  Envelope 0
contains $2^0 =1$ dollar, Envelope 1 contains $2^1 =2 $ dollars,
\dots, and Envelope $n-1$ contains $2^{n-1}$ dollars.  Let
$P(n)$ be the assertion that:
\begin{quote}
For all nonnegative integers $k < 2^n$, there is a subset of the
$n$ envelopes whose contents total to exactly $k$ dollars.
\end{quote}
Prove by induction that $P(n)$ holds for all integers $n \geq 1$.

%\hint Use induction on the number of envelopes $m$.

\begin{solution}

\inductioncase{Base case} ($n = 1$): The only possible values of $k$
are 0 and 1.  The sole envelope contains the needed \$1 and the empty
subset---that is, not using any envelopes---gives \$0.

\inductioncase{Inductive step} Assume that $P(n)$ is true for some $n
\geq 1$.  We need to show that, given $n+1$ envelopes, there is a
subset that sums to exactly $k$ dollars for $0 \leq k < 2^{n+1}$.

There are two cases to consider:
\begin{enumerate}
\item $k < 2^n .$  By the Induction Hypothesis, we can get exactly
  $k$ dollars using a subset of the first $k$ envelopes.

\item $2^n \leq k < 2^{n+1} .$ Then, $k = j + 2^n$, where $0 \leq j <
  2^n$.  Now by Induction Hypothesis, we can get exactly $j$ dollars
  using a subset of the first $n$ envelopes.  Then this subset along
  with the $(n+1)$st envelope sums to $j + 2^n = k$.
 \end{enumerate}

In either case, we are able to get exactly $k$ dollars
using a subset of the envelopes.  Therefore, $P(n+1)$ is true.  By
induction, we conclude that $P(n)$ is true for all $n \geq 1$.

\end{solution}

\end{problem}

%%%%%%%%%%%%%%%%%%%%%%%%%%%%%%%%%%%%%%%%%%%%%%%%%%%%%%%%%%%%%%%%%%%%%
% Problem ends here
%%%%%%%%%%%%%%%%%%%%%%%%%%%%%%%%%%%%%%%%%%%%%%%%%%%%%%%%%%%%%%%%%%%%%

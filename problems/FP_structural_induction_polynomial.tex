\documentclass[problem]{mcs}

\begin{pcomments}
  \pcomment{FP_structural_induction_polynomial}
  \pcomment{overlaps FP_structural_ind_polynomials}
  \pcomment{first part of CP_polynomials_produce_multiples}
\end{pcomments}

\pkeywords{
  structural_induction
  polynomial
  multiple
  congruence 
}

%%%%%%%%%%%%%%%%%%%%%%%%%%%%%%%%%%%%%%%%%%%%%%%%%%%%%%%%%%%%%%%%%%%%%
% Problem starts here
%%%%%%%%%%%%%%%%%%%%%%%%%%%%%%%%%%%%%%%%%%%%%%%%%%%%%%%%%%%%%%%%%%%%%

\begin{problem}

\begin{definition*}
The set $P$ of single variable integer polynomials can be defined
recursively:

\inductioncase{Base cases}:
  \begin{itemize}

   \item the identity function, $\ident{\integers}(x) \eqdef x$ is in $P$.

   \item for any integer $m$ the constant function, $c_m(x) \eqdef m$ is in $P$.
  \end{itemize}

\inductioncase{Constructor cases}.  If $r,s \in P$, then $r+s$
  and $r \cdot s \in P$.

\end{definition*}

\medskip
Prove by structural induction that for all $q \in P$,
\[
j \equiv k \pmod{n}\quad \QIMPLIES\quad q(j) \equiv q(k) \pmod n,
\]
for all integers $j,k,n$ where $n>1$.

Be sure to clearly state and label your Induction Hypothesis, Base
case(s), and Constructor step.

%\examspace[4.0in]

\begin{solution}
The proof is by structural induction on the definition of $P$.  The
hypothesis $H(q)$ is:
\begin{align*}
H(q) & \eqdef\  [j\equiv k \pmod{n}\quad \QIMPLIES\quad q(j) \equiv q(k) \pmod{n},\\
     & \qquad \text{for all } j,k,n \in \integers,\text{ where } n>1].
\end{align*}

\inductioncase{Base cases}:

\inductioncase{Case}: ($q = \ident{\integers}$).

$H(q)$ holds because if $j\equiv k \pmod{n}$, then
\begin{align*}
q(j) & \eqdef \ident{\integers}(j)\\
    & = j\\
    & \equiv k \pmod{n}\\
    & = \ident{\integers}(k)\\
    & = q(k),
\end{align*}
so $q(j) \equiv q(k) \pmod{n}$, as required.

\inductioncase{Case}: ($q = c_m$).
$H(c_m)$ holds because $c_m(j) = c_m(k)$, and therefore certainly
$c_m(j) \equiv c_m(k) \pmod{n}$.

\inductioncase{Constructor cases}:

We may assume by structural induction that $H(r)$ and $H(s)$ both hold.

\inductioncase{Case}: ($q = r+s$).  To show $H(q)$, suppose $j \equiv k \pmod n$.  Since
  $H(r)$ holds, we have that $r(j) \equiv r(k) \pmod n$.  Likewise, $s(j)
  \equiv s(k) \pmod n$.  So
\[
r(j) + s(s) \equiv r(j) + s(k) \pmod n,
\]
by additivity of congruences
Lemma~\bref{mod_congruence_lem}~(\bref{mod_congruence_lem+}), that is,
$q(j) \equiv q(k) \pmod n$, as required.

\inductioncase{Case}: ($q = r \cdot s$).  The proof in this case is
the same as the previous with ``$\cdot$'' replacing ``$+$.''

\end{solution}

\begin{staffnotes}
rubric from Stephan Boyer for Fall '13 final exam:

* Induction hypothesis: 4 points for correct induction hypothesis, -2
for inappropriate use of quantifiers (e.g., for all q in P)

* Base cases: 4 points for both correct, -2 points for listing both
but incorrect/incomplete explanations

* Constructor cases: 6 points total, -3 points for writing a
"backwards" proof, incorrect use of structural induction,
handwaviness, incorrect use of number theory
\end{staffnotes}

\end{problem}


%%%%%%%%%%%%%%%%%%%%%%%%%%%%%%%%%%%%%%%%%%%%%%%%%%%%%%%%%%%%%%%%%%%%%
% Problem ends here
%%%%%%%%%%%%%%%%%%%%%%%%%%%%%%%%%%%%%%%%%%%%%%%%%%%%%%%%%%%%%%%%%%%%%

\endinput


\documentclass[handout]{mcs}

\begin{document}

\inclassproblems{15, Wed.}

%%%%%%%%%%%%%%%%%%%%%%%%%%%%%%%%%%%%%%%%%%%%%%%%%%%%%%%%%%%%%%%%%%%%%
% Problems start here
%%%%%%%%%%%%%%%%%%%%%%%%%%%%%%%%%%%%%%%%%%%%%%%%%%%%%%%%%%%%%%%%%%%%%

\pinput{CP_RSA_between_tables}
\pinput{CP_findphi}
\pinput{CP_RSA_proving_correctness}

%\pinput{MQ_factoring_cracks_RSA}
%\pinput{MQ_RSA_collision_probability_200}

\instatements{\examspace

\textbox{
\begin{center}
\textboxheader{\large The RSA Cryptosystem}
\end{center}

A \textbf{Receiver} who wants to be able to receive secret numerical
messages creates a \emph{private key}, which they keep secret, and a
\emph{public key}, which they make publicly available.  Anyone with the
public key can then be a \textbf{Sender} who can publicly send secret
messages to the \textbf{Receiver}---even if they have never
communicated or shared any information besides the public key.

Here is how they do it:

\begin{description}

\item[Beforehand] The \textbf{Receiver} creates a public key and a private key
as follows.

\begin{enumerate}

\item Generate two distinct primes, $p$ and $q$.  These are used to
  generate the private key, and they must be kept hidden.  (In current
  practice, $p$ and $q$ are chosen to be hundreds of digits long.)

\item Let $n \eqdef pq$.

\item Select an integer $e \in [1,n)$ such that $\gcd(e, (p-1)(q-1)) = 1$.

  The \emph{public key} is the pair $(e, n)$.  This should be
  distributed widely.

\item Compute $d \in [1,n)$ such that $de \equiv 1 \pmod{(p-1)(q-1)}$.
  This can be done using the Pulverizer.

  The \emph{private key} is the pair $(d, n)$.  This should be kept
  hidden!

\end{enumerate}

\item[Encoding]

To transmit a message $m \in [0,n)$ to \textbf{Receiver}, a \textbf{Sender} uses the
  public key to encrypt $m$ into a numerical message
\[
\widehat{m} \eqdef \rem{m^e}{n}.
\]
The \textbf{Sender} can then publicly transmit $\widehat{m}$ to the
\textbf{Receiver}.

\item[Decoding] The \textbf{Receiver} decrypts message $\widehat{m}$
  back to message $m$ using the private key:
\[
m = \rem{\widehat{m}^d}{n}.
\]

\end{description}
}}
%%%%%%%%%%%%%%%%%%%%%%%%%%%%%%%%%%%%%%%%%%%%%%%%%%%%%%%%%%%%%%%%%%%%%
% Problems end here
%%%%%%%%%%%%%%%%%%%%%%%%%%%%%%%%%%%%%%%%%%%%%%%%%%%%%%%%%%%%%%%%%%%%%
\end{document}

\documentclass[problem]{mcs}

\begin{pcomments}
  \pcomment{TP_bogus_discrimination_contradiction}
  \pcomment{renamed from PS_}
  \pcomment{revised ARM 11/10/15}
\end{pcomments}

\pkeywords{
  conditional_probability
  paradox
  Simpson
}

%%%%%%%%%%%%%%%%%%%%%%%%%%%%%%%%%%%%%%%%%%%%%%%%%%%%%%%%%%%%%%%%%%%%%
% Problem starts here
%%%%%%%%%%%%%%%%%%%%%%%%%%%%%%%%%%%%%%%%%%%%%%%%%%%%%%%%%%%%%%%%%%%%%

\begin{problem}
\inhandout{Define the following events as in Section~\bref{discrimination_subsec}:
%
\begin{itemize}
\item $A \eqdef$ a random applicant is admitted,
\item $F_{EE} \eqdef$ the applicant is a woman for the EE department,
\item $F_{CS} \eqdef$ the applicant is a woman for the CS department,
\item $M_{EE} \eqdef$ the applicant is a man for the EE department,
\item $M_{CS} \eqdef$ the applicant is a man for the CS department,
\end{itemize}
where all applicants are assumed to be either men or women, and no
applicants is in both departments.}

\inbook{Define the events $A, F_{EE}, F_{CS}, M_{EE}$, and $M_{CS}$ as
  in Section~\bref{discrimination_subsec}.}

In these terms, the plaintiff in a discrimination suit against a
university makes the argument that in both departments, the
probability that a female is admitted is less than the probability for
a male.  That is,
\begin{align}
\prcond{A}{F_{EE}} & < \prcond{A}{M_{EE}} \quad\text{and}\label{AFEE<}\\
\prcond{A}{F_{CS}} & < \prcond{A}{M_{CS}}.\label{AFCS<}
\end{align}

The university's defence attorneys retort that \emph{overall}, a female
applicant is \emph{more} likely to be admitted than a male,
namely, that
\begin{equation}\label{AFEEunion}
    \prcond{A}{F_{EE} \union F_{CS}} > \prcond{A}{M_{EE} \union M_{CS}}.
\end{equation}

The judge then interrupts the trial and calls the plaintiff and
defence attorneys to a conference in his office to resolve what he
thinks are contradictory statements of facts about the admission
data.  The judge points out that:
\begin{align*}
\lefteqn{\prcond{A}{F_{EE} \cup F_{CS}}}\\
    & = \prcond{A}{F_{EE}} + \prcond{A}{F_{CS}}
       & \text{(because $F_{EE}$ and $F_{CS}$ are disjoint)}\\
    & <	\prcond{A}{M_{EE}} + \prcond{A}{M_{CS}}
       & \text{(by~\eqref{AFEE<} and \eqref{AFCS<})} \\
    & =\prcond{A}{M_{EE} \cup M_{CS}}
       & \text{(because $M_{EE}$ and $M_{CS}$ are disjoint)}
\end{align*}
so
\[
\prcond{A}{F_{EE} \cup F_{CS}} < \prcond{A}{M_{EE} \cup M_{CS}},
\]
which directly contradicts the university's position~\eqref{AFEEunion}!

Of course the judge is mistaken; an example where the plaintiff and
defence assertions are all true appears in \inhandout{the
  text}\inbook{Section~\bref{discrimination_subsec}}.  What is the
mistake in the judge's proof?

\begin{solution}
The two equalities asserted by the judge are based on:
\begin{falseclm*}
If $B$ and $C$ are disjoint events, then for any event $A$,
\begin{equation}\label{backwardssum}
\prcond{A}{B \union C} = \prcond{A}{B} + \prcond{A}{C}.
\end{equation}
\end{falseclm*}
But~\eqref{backwardssum} is false in general.  For example, let $B$
and $C$ be any disjoint events that both have nonzero probability, and
let $A \eqdef B \union C$.  Then
\[
\prcond{A}{B \union C} = 1 \neq 1+1 = \prcond{A}{B} + \prcond{A}{C}.
\]
In this particular case, the identity
\[
\prcond{A}{F_{EE} \cup F_{CS}} = \prcond{A}{F_{EE}} + \prcond{A}{F_{CS}}
\]
would fail if more than half of all female applicants were admitted in
each department, since then the right-hand side of this equation would
be greater than~1.
\end{solution}

\end{problem}

%%%%%%%%%%%%%%%%%%%%%%%%%%%%%%%%%%%%%%%%%%%%%%%%%%%%%%%%%%%%%%%%%%%%%
% Problem ends here
%%%%%%%%%%%%%%%%%%%%%%%%%%%%%%%%%%%%%%%%%%%%%%%%%%%%%%%%%%%%%%%%%%%%%

\endinput

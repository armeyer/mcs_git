\documentclass[problem]{mcs}

\begin{pcomments}
    \pcomment{TP_average_up_down}
    \pcomment{BY ARM from active learning site Prob & Stat 8/27/11}
\end{pcomments}

\pkeywords{
  average
}

%%%%%%%%%%%%%%%%%%%%%%%%%%%%%%%%%%%%%%%%%%%%%%%%%%%%%%%%%%%%%%%%%%%%%
% Problem starts here
%%%%%%%%%%%%%%%%%%%%%%%%%%%%%%%%%%%%%%%%%%%%%%%%%%%%%%%%%%%%%%%%%%%%%

\begin{problem}
A news article reporting on the departure of a school official from
California to Alabama dryly commented that this move would raise the
average IQ in both states.  Explain.

\begin{solution}
If a below(above)-average number is removed from a set of numbers, the average
of the remaining numbers rises(falls).  So the official's IQ must have
been below the California average and above the Alabama average.
\end{solution}

\end{problem}

\endinput

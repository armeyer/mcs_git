\documentclass[problem]{mcs}

\newcommand{\glb}{\operatorname{glb}}

\begin{pcomments}
  \pcomment{PS_Markov_with_bounded_RVs}
  \pcomment{by Kazerani 4/30/11 w/ARM minor edits}
\end{pcomments}

\pkeywords{
  Markov_bound
  deviation
  expectation
}

%%%%%%%%%%%%%%%%%%%%%%%%%%%%%%%%%%%%%%%%%%%%%%%%%%%%%%%%%%%%%%%%%%%%%
% Problem starts here
%%%%%%%%%%%%%%%%%%%%%%%%%%%%%%%%%%%%%%%%%%%%%%%%%%%%%%%%%%%%%%%%%%%%%

\begin{problem}
If $R$ is a nonnegative random variable, then Markov's Theorem gives
an upper bound on $\pr{R \geq x}$ for any real number $x >
\expect{R}$.  If $b$ is a lower bound on $R$, then Markov's
Theorem can also be applied to $R-b$ to obtain a possibly different
bound on $\pr{R \geq x}$.

\iffalse
\footnote{The less demanding constraint $x > 0$ would still
  allow Markov's Theorem to be used, but the resulting bound would be
  useless if $x\leq\expect{R}$.}, direct application of Markov's
Theorem gives
\begin{equation}\label{MarkovBound}
\pr{R\geq x}\leq\frac{\expect{R}}{x}
\end{equation}
\begin{problemparts}
\fi

\bparts \ppart\label{boundpart} Show that if $b>0$, applying Markov's
Theorem to $R-b$ gives a smaller upper bound on $\pr{R \geq x}$ than
simply applying Markov's Theorem directly to $R$.

\begin{solution}
Define
\[
T\eqdef R-b.
\]
Then $T$ is a nonnegative random variable and Markov's Theorem can
therefore be applied to $T$ to give
\[
\prob{T \geq x - b} \leq \frac{\expect{T}}{x - b} = \frac{\expect{R} - b}{x - b}.
\]
But the event $[T \geq x - b]$ is the same as $[R \geq x]$, so
\[
\prob{R \geq x} \leq \frac{\expect{R} - b}{x - b}.
\]

So we want to show that
\[
\frac{\expect{R} - b}{x - b} < \frac{\expect{R}}{x}.
\]
Since $x$, $b$ and $x-b$ are all positive,
\begin{align}
\frac{\expect{R} - b}{x - b}
    & < \frac{\expect{R}}{x}           &\QIFF\label{R-bx-bexRx}\\
x \expect{R} - bx
    & < x\expect{R} - b\expect{R}      &\QIFF\notag\\
-bx & < -b\expect{R}                   &\QIFF\notag\\
  x & > \expect{R}.\label{x>exR}
\end{align}
But~\eqref{x>exR} is given, which shows that \eqref{R-bx-bexRx} holds,
as required.
\end{solution}

\ppart What value of $b\geq 0$ in part~\eqref{boundpart} gives the
best bound?

\begin{solution}
With $b\geq 0$, $R-b$ is nonnegative iff $b \leq
\glb(\range{R})$.\footnote{``glb'' = ``greatest lower bound.''}  So
for any such $b$, applying Markov's Theorem to $R-b$ gives
\[
\prob{R \geq x} \leq \frac{\expect{R} - b}{x - b}.
\]
It is easy to check, by simple algebra or taking derivatives wrt to
$b$,\footnote{Differentiating this upper bound with respect to $b$ gives
\[
\frac{d}{db}\paren{\frac{\expect{R} - b}{x - b}} = \frac{\expect{R}-x}{\paren{x-b}^2}.
\]
Since $x>\expect{R}$ and $x\neq b$, this derivative is negative---and
so the bound as a function of $b$ is strictly decreasing---for all $0
\leq b \leq \glb(\range{R})$.

Alternatively, to prove that the bound is strictly decreasing in $b$,
suppose $0 < b_1,b_2 \leq \glb(\range{R})$.  Since $x-b_1>0$,
$x-b_2>0$, and $x>\expect{R}$,
\begin{align*}
\frac{\expect{R} - b_1}{x - b_1}
   & < \frac{\expect{R} - b_2}{x - b_2}              &\QIFF\\
x\expect{R} - b_1x - b_2\expect{R} + b_1b_2
   & < x\expect{R} - b_1\expect{R} - b_2x + b_1b_2   &\QIFF\\
(b_1-b_2)\expect{R}
   & < (b_1-b_2)x                                    &\QIFF\\
b_1-b_2
   & > 0                                             &\QIFF\\
b_1& > b_2.
\end{align*}
} that the right-hand bound is strictly decreasing in $b$, so the
upper bound is least when $b=\glb(\range{R})$.
\end{solution} 

\end{problemparts}

\end{problem} 

%%%%%%%%%%%%%%%%%%%%%%%%%%%%%%%%%%%%%%%%%%%%%%%%%%%%%%%%%%%%%%%%%%%%% 
% Problem ends here 
%%%%%%%%%%%%%%%%%%%%%%%%%%%%%%%%%%%%%%%%%%%%%%%%%%%%%%%%%%%%%%%%%%%%%

\endinput

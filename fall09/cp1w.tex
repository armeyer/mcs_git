\documentclass[handout]{mcs}

\begin{document}

\inclassproblems{1, Wed.}

%%%%%%%%%%%%%%%%%%%%%%%%%%%%%%%%%%%%%%%%%%%%%%%%%%%%%%%%%%%%%%%%%%%%%
% Problems start here
%%%%%%%%%%%%%%%%%%%%%%%%%%%%%%%%%%%%%%%%%%%%%%%%%%%%%%%%%%%%%%%%%%%%%

\begin{problem}

%?? ARM ASKS 9/4/09 WHAT IS THIS:

%See theorem~\ref{C0301:sum-to-n}.

Identify exactly where the bugs are in each of the following bogus
proofs.\footnote{From Stueben, Michael and Diane Sandford. \emph{Twenty
Years Before the Blackboard}, Mathematical Association of America, \copyright 1998.}

\bparts

\problempart \textbf{Bogus Claim}: $1/8 > 1/4.$
\begin{bogusproof}
\begin{align*}
    3 &> 2 \\
    3 \log_{10} (1/2) &> 2 \log_{10}(1/2) \\
    \log_{10} (1/2)^3 &> \log_{10} (1/2)^2 \\
    (1/2)^3 &> (1/2)^2,
\end{align*}
and the claim now follows by the rules for multiplying fractions.
\end{bogusproof}

\begin{solution}
$\log x < 0$, for $0<x<1$, so since both sides of the inequality
``$3 > 2$'' are being multiplied by the negative quantity
$\log_{10}(1/2)$, the ``$>$'' in the second line should have been
``$<$.''
\end{solution}

\ppart \emph{Bogus proof}: $1 \mbox{\textcent} = \$0.01 = (\$0.1)^2 = (10\mbox{\textcent})^2 =
100\mbox{\textcent} = \$1.\qed$

\begin{solution}
$\$0.01 = \$(0.1)^2 \neq (\$0.1)^2$ because the units $\$^2$ and
$\$$ don't match (just as in physics the difference between $sec^2$ and
$sec$ indicates the difference between acceleration and velocity).
Similarly, $(10\mbox{\textcent})^2 \neq 100$\textcent.
\end{solution}

%FALSE PROOF ARITHMETIC, from Spring94 revised ARM 9/3/01

\ppart \textbf{Bogus Claim}: If $a$ and $b$ are two equal real numbers,
then $a=0$.
\begin{bogusproof}
\begin{eqnarray*}
a&=&b \\
a^2&=&ab \\
a^2-b^2&=&ab-b^2 \\
(a-b)(a+b)&=&(a-b)b\\ % \label{cancel}\\
a+b&=&b\\ % \label{bug}\\
a&=&0.
\end{eqnarray*}
\end{bogusproof}

\begin{solution}
The bug is at the fifth line:
\iffalse~(\ref{bug})\fi
one cannot cancel $(a-b)$ from
both sides of the equation on the fourth line
\iffalse~(\ref{cancel})\fi
because $a-b = 0$.
\end{solution}

\eparts
\end{problem}

\instatements{\newpage}

\begin{problem} It's a fact that the Arithmetic Mean is at least as
large the Geometric Mean, namely,
\[
\frac{a + b}{2} \geq \sqrt{a b}
\]
for all nonnegative real numbers $a$ and $b$.  But there's something
objectionable about the following proof of this fact.  What's the
objection, and how would you fix it?

\begin{bogusproof}
\begin{align*}
\frac{a + b}{2} & \stackrel{?}{\geq} \sqrt{a b}, & \text{ so} \\
a + b  & \stackrel{?}{\geq} 2 \sqrt{a b}, & \text{ so} \\
a^2 + 2 a b + b^2  & \stackrel{?}{\geq} 4 a b, & \text{ so}\\
a^2 - 2 a b + b^2  & \stackrel{?}{\geq} 0, & \text{ so}\\
(a - b)^2  & \geq 0 & \text{ which we know is true.}
\end{align*}

The last statement is true because $a - b$ is a real number, and the
square of a real number is never negative.  This proves the claim.
\end{bogusproof}
\end{problem}

\begin{solution}
In this argument, we started with what we wanted to prove and
then reasoned until we reached a statement that is surely true.  The
little question marks presumably are supposed to indicate that we're not
quite certain that the inequalities are valid until we get down to the
last step.  At that step, the inequality checks out, \emph{but that
doesn't prove the claim}.  All we have proved is that \textbf{if} $(a + b)/2
\geq \sqrt{a b}$, \textbf{then} $(a - b)^2 \geq 0$, which is not very
interesting, since we already knew that the square of any nonnegative
number is nonnegative.

To be fair, this bogus proof is pretty good: if it was written in reverse
order -- or if ``is implied by'' was simply inserted after each line -- it
would actually prove the Arithmetic-Geometric Mean Inequality:

\begin{proof}

\begin{align*}
\frac{a + b}{2} & \geq \sqrt{a b} & \text{ is implied by}\\
a + b  & \geq 2 \sqrt{a b},  & \text{ which is implied by}\\
a^2 + 2 a b + b^2  & \geq 4 a b, & \text{ which is implied by}\\
a^2 - 2 a b + b^2  & \geq 0, & \text{ which is implied by}\\
(a - b)^2  & \geq 0.
\end{align*}

The last statement is true because $a - b$ is a real number, and the
square of a real number is never negative.  This proves the claim.
\end{proof}

But the problem with the bogus proof as written is that it reasons
backward, beginning with the proposition in question and reasoning to a
true conclusion.  This kind of backward reasoning can easily ``prove''
false statements.  Here's an example:

\textbf{Bogus Claim}: $0 = 1$.
\begin{bogusproof}
\begin{align*}
0 & \stackrel{?}{=} 1, & \text{ so} \\
1 & \stackrel{?}{=} 0, & \text{ so} \\
0+1 & \stackrel{?}{=} 1+0, & \text{ so} \\
1 & = 1 & \text{ which is trivially true,}
\end{align*}
which proves $0=1$.
\end{bogusproof}

We can also come up with very easy ``proofs'' of true theorems, for
example, here's an easy ``proof'' of the Arithmetic-Geometric Mean
Inequality:

\begin{bogusproof}
\begin{align*}
\frac{a + b}{2} & \stackrel{?}{\geq} \sqrt{a b}, & \text{ so}\\
0 \cdot \frac{a + b}{2} & \stackrel{?}{\geq} 0 \cdot \sqrt{a b}, & \text{ so}\\
0 & \geq 0 & \text{ which is trivially true.}\quad \qedhere
\end{align*}
\end{bogusproof}

So watch out for backward proofs!
\end{solution}

\begin{problem} Albert announces that he plans a surprise 6.042
quiz next week.  His students wonder if the quiz could be next Friday.
The students realize that it obviously cannot, because if it hadn't been
given before Friday, everyone would know that there was only Friday left
on which to give it, so it wouldn't be a surprise any more.

So the students ask whether Albert could give the surprise quiz Thursday?
They observe that if the quiz wasn't given \emph{before} Thursday, it
would have to be given \emph{on} the Thursday, since they already know it
can't be given on Friday.  But having figured that out, it wouldn't be a
surprise if the quiz was on Thursday either.  Similarly, the students
reason that the quiz can't be on Wednesday, Tuesday, or Monday.  Namely,
it's impossible for Albert to give a surprise quiz next week.  All the
students now relax, having concluded that Albert must have been bluffing.

And since no one expects the quiz, that's why, when Albert gives it on
Tuesday next week, it really is a surprise!

What do you think is wrong with the students' reasoning?

\begin{solution}
The basic problem is that ``surprise'' is not a mathematical
concept, nor is there any generally accepted way to give it a mathematical
definition.  The ``proof'' above assumes some plausible axioms about
surprise, without defining it.  The paradox is that these axioms are
inconsistent.  But that's no surprise \texttt{:-)}, since ---mathematically
speaking ---we don't know what we're talking about.

Mathematicians and philosophers have had a lot more to say about what might
be wrong with the students' reasoning, (see Chow, Timothy Y.
\href{http://courses.csail.mit.edu/6.042/fall09/surprise-paradox.pdf}
{\emph{The surprise examination or unexpected hanging paradox}}, American
Mathematical Monthly (January 1998), pp.41--51.)
\end{solution}

\end{problem}


%%%%%%%%%%%%%%%%%%%%%%%%%%%%%%%%%%%%%%%%%%%%%%%%%%%%%%%%%%%%%%%%%%%%%
% Problems end here
%%%%%%%%%%%%%%%%%%%%%%%%%%%%%%%%%%%%%%%%%%%%%%%%%%%%%%%%%%%%%%%%%%%%%
\end{document}

\documentclass[problem]{mcs}

\begin{pcomments}
  \pcomment{asg_M2_Euler_congruence}
  \pcomment{variation of TP_Eulers_Theorem}
  \pcomment{F15.mid2}
  \pcomment{Converted from euler-theorem.scm by scmtotex and dmj
    on Sat 12 Jun 2010 09:09:34 PM EDT}
  \pcomment{subsumed by MQ_modular_arithmetic}
\end{pcomments}

\begin{problem}

\bparts

\ppart
What is the value of $\phi(800)$, where $\phi$ is Euler's function?

\begin{solution}
$800 = 2^{5} \cdot 5^{2}$, so $\phi(800)=(2^{5} - 2^{4})(5^{2} - 5^{1}) = 16 \cdot 20 = 320$.
\end{solution}

\examspace[1.0in]

\ppart
What is the remainder of $21^{962}$ divided by 800?

\begin{solution}
\begin{align*}
21^{962} &= 21^{(320 \cdot 3) +2} \\
           & = (21^{320})^{3}  \cdot  21^{2} \\
           & \equiv  1^{3}  \cdot  441 \bmod 800\\
           & \equiv  441 \bmod 800.
\end{align*}
\end{solution}

\examspace[2.0in]

\ppart Find the inverse of 21 modulo 800 in the interval
$\Zintvcc{1}{799}$?

\begin{solution}
We first use the Pulverizer to find $s,t$ such that $\gcd(21, 800) = s\cdot 21 + t\cdot 800$:
\[
\begin{array}{ccccrcl}
x & \quad & y & \quad & \rem{x}{y} & = & x - q \cdot y \\ \hline
800 && 21 && 2  & = &   800 - 38\cdot 21 \\
21 && 2 && 1 & = &   21 - 10\cdot 2 \\
&&&&            & = &   21 - 10\cdot (800 - 38\cdot 21)\\
&&&&            & = &  381\cdot 21 - 10\cdot 800\\
\end{array}
\]

\[
1 = 381 \cdot 21 - 10 \cdot 800.
\]
This implies that $s = 381$ is an inverse of 21 modulo 800.
\end{solution}

\examspace[4.0in]

\iffalse
\ppart Find the inverse of 22 modulo 175 in the interval
$\Zintvcc{1}{174}$?

\begin{solution}
Note that $8 \cdot 22 - 175 = 1$. Therefore, $8$ is an inverse of $22$ modulo $175$.

\begin{staffnotes}
Change to need Pulverizer.
\end{staffnotes}
\end{solution}
\fi

\iffalse
%\examspace[2.0in]

\ppart
What is the remainder of $22^{11999}$ divided by 175?

\begin{solution}

\begin{align*}
22^{11999} & \equiv 22^{11999} \cdot 22 \cdot 8 \bmod 175 \\
           & \equiv 22^{12000} \cdot 8 \bmod 175 \\
           &= 22^{(120 \cdot 100)} \cdot 8 \bmod 175 \\
           & = (22^{120})^{100} \cdot 8 \bmod 175 \\
           & \equiv  1^{100} \cdot 8 \bmod 175\\
           & \equiv  8 \bmod 175.
\end{align*}
\end{solution}

\examspace[2.0in]
\fi

\eparts

\end{problem}

\endinput

%ARM Aug 2010 version

\section{The Alternating Bit Protocol}

\begin{center}
{\large \textcolor{red}{DRAFT}}
\end{center}

We've already demonstrated the use of state machines as abstract
models of step-by-step processes.  In this chapter we examine two
further applications of state machines, first as models of concurrent
computational processes, and second as the basis for reasoning about
programs.

\begin{editingnotes}
Lynch Notes S07.H11-sm
\end{editingnotes}

The Alternating Bit Protocol is a well-known two-process communication
protocol that achieves reliable FIFO communication over unreliable
channels that operate concurrently.  The unreliable channels may lose
or duplicate messages, but are assumed not to reorder them.  We'll use
the Invariant Method to verify the Protocol.

The Protocol allows a \textbf{Sender} process to send a sequence of
messages from a message alphabet $M$ to a \textbf{Receiver}
process. It works as follows.

\textbf{Sender} repeatedly sends the rightmost message in its
\textbf{outgoing-queue} of messages, tagged with a \textbf{tagbit}
that is initially $1$.  When \textbf{Receiver} receives this tagged
message, it sets its \textbf{ackbit} to be the message tag $1$, and
adds the message to the left-hand end of its \textbf{received-msgs}
list.  Then as an acknowledgement, \textbf{Receiver} sends back
\textbf{ackbit} 1 repeatedly.  When \textbf{Sender} gets this
acknowledgement bit, it deletes the rightmost outgoing message in its
queue, sets its \textbf{tagbit} to 0, and begins sending the new
rightmost outgoing message, tagged with \textbf{tagbit}.

\textbf{Receiver}, having already accepted the message tagged with
\textbf{ackbit} $1$, ignores subsequent messages with tag $1$, and
waits until it sees the first message with tag $0$; it adds this
message to the left-hand side of its \textbf{received-msgs} list, sets
\textbf{ackbit} to 0 and acknowledges repeatedly with with
\textbf{ackbit} $0$.  \textbf{Sender} now waits till it gets
acknowledgement bit $0$, then goes on to send the next outgoing
message with tag $1$.  In this way, it alternates use of the tags $1$
and $0$ for successive messages.

We claim that this causes \textbf{Sender} to receive \emph{suffix}
original \textbf{outgoing-msgs} queue.  That is, at any stage in the
process when the the \textbf{outgoing-msgs}

(The fact that \textbf{Sender} actually outputs the entire outgoing
queuee is a \emph{liveness} claim---liveness properties are a
generalization of termination properties.  We'll ignore this issue for
now.)

We formalize the description above as a state whose states consist of:
\begin{tabbing}
XXX\=XXX\= \kill \> $\text{\textbf{outgoing-msgs}}$, a finite sequence of $M$,
whose initial value is called \textbf{all-msgs}\\ \> $\text{\textbf{tagbit}} \in
\set{0, 1}$, initially $1$ \\ \\ \> $\text{\textbf{received-msgs}}$, a finite
sequence of $M$, initially empty\\ \> $\text{\textbf{ackbit}}( \in \set{0, 1}$,
initially $0$ \\ \\ \> $\text{\textbf{msg-channel}}$, a finite sequence of $M
\times \set{0, 1}$, initially empty, \\ \> $\text{\textbf{ack-channel}}$, a
finite sequence of $\set{0, 1}$, initially empty
\end{tabbing}

The transitions are:
\begin{enumerate}
\item[\textbf{SEND:}]

\begin{enumerate}
\item
\prcef{$\text{\textbf{send-msg}}(m,b)$}
{% Pre: 
   $m = \text{\textbf{rightend}}(\text{\textbf{outgoing-msgs}}) \QAND
   b = \text{\textbf{tagbit}}$}
{% Eff: 
   add $(m,b)$ to the left-hand end of $\text{\textbf{msg-channel}}$,
   any number $\geq 0$ of times}

\item
\prcef{$\text{\textbf{send-ack}}(b)$}
{% Pre: 
   $b = \text{\textbf{ackbit}}$}
{% Eff: 
   add $b$ to the right-hand end of $\text{\textbf{ack-channel}}$,
   any number $\geq 0$ of times}
\end{enumerate}

\item[\textbf{RECEIVE:}]

\begin{enumerate}
\item
\prcef{$\text{\textbf{receive-msg}}(m,b)$}
 {% Pre: 
   $(m,b) = \text{\textbf{rightend}}(\text{\textbf{msg-channel}})$}
{% Eff: 
   remove $\text{\textbf{rightend}}$ of $\text{\textbf{msg-channel}}$;\\
   if $b \neq \text{\textbf{ackbit}}$, then
   [add $m$ to the left-hand end of $\text{\textbf{receive-msgs}}$;
   $\text{\textbf{ackbit}} := b$.]}

\item \prcef{$\text{\textbf{receive-ack}}(b)$}
{% Pre: 
   $b = \text{\textbf{leftend}}(\text{\textbf{ack-channel}}()$}
{% Eff: 
   remove $\text{\textbf{leftend}}$ of $\text{\textbf{ack-channel}}$.\\
   if $b = \text{\textbf{tagbit}}$, then
   [remove $\text{\textbf{rightend}}$ of $\text{\textbf{outgoing-msgs}}$ (if nonempty);
   $\text{\textbf{tagbit}} := \overline{\text{\textbf{tagbit}}}$]}
\end{enumerate}

\end{enumerate}

Our goal is to show that when $\text{\textbf{tagbit}} \neq
\text{\textbf{ackbit}}$, then\\
\begin{equation}\label{all-msgs-preserved}
\text{\textbf{outgoing-queue}}\cdot \text{\textbf{received-msgs}} =
\text{\textbf{all-msgs}}.
\end{equation}

This requires three auxiliary invariants.
For the first of these, we need a definition.

Let $\text{\textbf{tag-sequence}}$ be the sequence consisting of bits in
$\text{\textbf{ack-channel}}$, in right-to-left order, 
followed by $\text{\textbf{tagbit}}$, 
followed by the tag components of the elements of
$\text{\textbf{msg-channel}}$, in left-to-right order, 
followed by $\text{\textbf{ackbit}}$. 

{\bf Property 2:}
$\text{\textbf{tag-sequence}}$ consists of one of the following:
\begin{enumerate}
\item
All $0$'s.
\item
All $1$'s. 
\item
A positive number of $0$'s followed by a positive number of $1$'s.
\item
A positive number of $1$'s followed by a positive number of $0$'s.
\end{enumerate}
What is being ruled out by these four cases is the situation where
the sequence contains more than one switch of tag value. 

The fact that Property 2 is an invariant can be proved easily by
induction.  We also need: 

{\bf Property 3:}
If $(m,\text{\textbf{tag}})$ is in $\text{\textbf{msg-channel}}$ then $m =
\text{\textbf{rightend}}(\text{\textbf{outgoing-queue}})$. \\

\begin{proof}
(That Property 3 is an invariant)

By induction, using Property 2. 

Base: Obvious, since no message is in the channel initially. 

Inductive step: It is easy to see that the property is preserved by
$\text{\textbf{send}}_{m,b}$, which adds new messages to
$\text{\textbf{channel}}_{1,2}$.  The only other case that could cause
a problem is $\text{\textbf{receive}}(b)_{2,1}$, which could cause
$\text{\textbf{tag}}_1$ to change when there is another message
already in $\text{\textbf{channel}}_{1,2}$ with the same tag.  But
this can't happen, by Property 2 applied before the step---since the
incoming tag $g$ must be equal to $\text{\textbf{tag}}_1$ in this
case, all the tags in $\text{\textbf{tag-sequence}}$ must be the same.
\end{proof}

Finally, we need that the following counterpart
to~\eqref{all-msgs-preserved}: when $\text{\textbf{tagbit}} =
\text{\textbf{ackbit}}$, then
\begin{equation}\label{all-msgs-preserved2}
\text{\textbf{lefttail}(\textbf{outgoing-queue})}\cdot \text{\textbf{received-msgs}} =
\text{\textbf{all-msgs}},
\end{equation}
where $\text{\textbf{lefttail}}(\textbf{outgoing-queue})$ all but the
rightmost message, if any, in \textbf{outgoing-queue}.

Property 4, part 2, easily implies the goal Property 1.
It also implies that $\text{\textbf{work-buf}}_2$ is always nonempty when 
$\text{\textbf{receive}}(b)_{2,1}$ occurs with equal tags; therefore, the
parenthetical check in the code always works out to be true.

\begin{proof}
(That Property 4 is an invariant)

By induction. 
Base: In an initial state, the tags are unequal, 
$\text{\textbf{work-buf}}_1 = \text{\textbf{buf}}_1$ and $\text{\textbf{buf}}_2$ is empty.  This
suffices to show part 1.  part 2 is vacuous.

Inductive step:
When a $\text{\textbf{send}}$ occurs, the tags and buffers are unchanged, so the
truth of the invariants must be preserved. 
It remains to consider $\text{\textbf{receive}}$ events.

$\text{\textbf{receive}}(m,b)_{1,2}$:

If $b = \text{\textbf{tag}}_2$, nothing happens, so the invariants are
preserved.  So suppose that $b \neq \text{\textbf{tag}}_2$.  Then
Property 2 implies that $b = \text{\textbf{tag}}_1$, and then Property
3 implies that $m$ is the first message on
$\text{\textbf{work-buf}}_1$.  The effect of the transition is to
change $\text{\textbf{tag}}_2$ to make it equal to
$\text{\textbf{tag}}_1$, and to replicate the first element of
$\text{\textbf{work-buf}}22_1$ at the end of $\text{\textbf{buf}}_2$.

The inductive hypothesis implies that, before the step,
$\text{\textbf{buf}}_2 \cdot \text{\textbf{work-buf}}_1 =
\text{\textbf{buf}}_1$.  The changes caused by the step imply that,
after the step, $\text{\textbf{tag}}_1 = \text{\textbf{tag}}_2$,
$\text{\textbf{work-buf}}_1$ and $\text{\textbf{buf}}_2$ are nonempty,
$\text{\textbf{head}}(\text{\textbf{work-buf}}_1) =
\text{\textbf{last}}(\text{\textbf{buf}}_2)$, and
$\text{\textbf{buf}}_2 \cdot
\text{\textbf{tail}}(\text{\textbf{work-buf}}_1) =
\text{\textbf{buf}}_1$.  This is as needed.

$\text{\textbf{receive}}(b)_{2,1}$:

The argument is similar to the one for
$\text{\textbf{receive}}(m,b)_{1,2}$.  If $b \neq
\text{\textbf{tag}}_1$, nothing happens so the invariants are
preserved.  So suppose that $b = \text{\textbf{tag}}_1$.  Then
Property 2 implies that $b = \text{\textbf{tag}}_2$, and the step
changes $\text{\textbf{tag}}_1$ to make it unequal to
$\text{\textbf{tag}}44_2$.  The step also removes the first element of
$\text{\textbf{work-buf}}_1$.  The inductive hypothesis implies that,
before the step, $\text{\textbf{work-buf}}_1$ and
$\text{\textbf{buf}}_2$ are nonempty,
$\text{\textbf{head}}(\text{\textbf{work-buf}}_1) =
\text{\textbf{last}}(\text{\textbf{buf}}_2)$, and
$\text{\textbf{buf}}_2 \cdot
\text{\textbf{tail}}(\text{\textbf{work-buf}}_1) =
\text{\textbf{buf}}_1$.  The changes caused by the step imply that,
after the step, $\text{\textbf{tag}}_1 \neq \text{\textbf{tag}}_2$ and
$\text{\textbf{buf}}_2 \cdot \text{\textbf{work-buf}}_1 =
\text{\textbf{buf}}_1$.  This is as needed.
\end{proof}

\section{Reasoning About \textbf{While} Programs}\label{while_chap}

Real programs and programming languages are often huge and complicated,
making them hard to model and even harder to reason about.  Still, making
programs ``reasonable'' is a crucial aspect of software engineering.  In
this section we'll illustrate what it means to have a clean mathematical
model of a simple programming language and reasoning principles that go
with it---if only real programming languages allowed for such simple,
accurate modeling.

\subsection{\textbf{While} Programs}

The programs we'll study are called ``\term{\while\ programs}.''  We
can define them as a recursive data type:
\begin{definition}\label{whiledef} \mbox{}

\textbf{base cases:}
\begin{itemize}

\item $x \assigned e$ is a \while\ program, called an \term{assignment
    statement}, where $x$ is a variable and $e$ is an expression.

\item \term{$\halt$} is a \while\ program.

\end{itemize}

\textbf{constructor cases:}
If $C$ and $D$ are \while\ programs, and $T$ is a test, then the following are also
\while\ programs:
\begin{itemize}

\item $\seqcomm{C}{D}$---called the \term{sequencing} of $C$ and $D$,

\item $\condcomm{T}{C}{D}$---called a \term{conditional} with
  \term{test} $T$ and \term{branches} $C$ and $D$,

\item $\wloop{T}{C}$---called a \term{while loop} with \term{test},
  $T$, and \term{body} $C$.

\end{itemize}
\end{definition}

For simplicity we'll stick to \while\ programs operating on integers.  So by
expressions we'll mean any of the familiar integer valued
expressions involving integer constants and operations such as addition,
multiplication, exponentiation, quotient or remainder.  As \term{tests},
we'll allow propositional formulas built from basic formulas of the form
$e \leq f$ where $e$ and $f$ are expressions.  For example, here is the
Euclidean algorithm for $\gcd(a,b)$ expressed as a \while\ program.
\begin{tabbing}
XXX\=XXX\=XXX\=XXX\kill
$\mathtt{x} \assigned a$\textbf{;} \\
$\mathtt{y} \assigned b$\textbf{;} \\
\while\ $\mathtt{y} \neq 0$ \docomm\\
   \> $\mathtt{t} \assigned \mathtt{y}$\textbf{;}\\
   \> $\mathtt{y} \assigned \rem{\mathtt{x}}{\mathtt{y}}$\textbf{;}\\
   \> $\mathtt{x} \assigned \mathtt{t}$
\odcomm\\
\end{tabbing}

\subsection{The \textbf{While}  Program State Machine}

A \while\ program acts as a pure command: it is run solely for its
side effects on stored data and it doesn't return a value.  The data
consists of integers stored as the values of variables, namely
environments:
\begin{definition}
An \term{environment} is a total function from variables to integers.
Let $\Env$ be set of all environments.
\end{definition}
So if $\rho$ is an environment and $x$ is a variable, then $\rho(x)$ is an
integer.  More generally, the environment determines the integer value of
each expression $e$ and the truth value of each test $T$.  We can think
of an expression, $e$ as defining a function $\mean{e}:\Env \to
\integers$, and refer to this function $\mean{e}$ as the \term{meaning}
of $e$; likewise for tests.

It's standard in programming language theory to write $\mean{e}\rho$
as shorthand for $\mean{e}(\rho)$, that is, applying
the \emph{meaning}, $\mean{e}$, of $e$ to $\rho$.  For example, if
$\rho(\mathtt{x}) =4$, and $\rho(\mathtt{y}) =-2$, then
\begin{equation}\label{x2y311}
\mean{\mathtt{x^2+y -3}}\rho = \rho(\mathtt{x})^2 + \rho(y) -3 = 11.
\end{equation}

\iffalse
(A variable is a special case of an expression, so $\mean{x}\rho \eqdef \rho(x)$.)
\fi

Executing a program causes a succession of changes to the
environment\footnote{More sophisticated programming models distinguish
the environment from a \term{store} which is affected by commands, but
this distinction is unnecessary for our purposes.}  which may continue
until the program halts.  Actually the only command which immediately
alters the environment is an assignment command.  Namely, the effect
of the command
\[
x \assigned e
\]
on an environment is that the value assigned to the variable $x$ is
changed to the value of $e$ in the original environment.  We can say
this precisely and concisely using the following notation:
$\patch{f}{a}{b}$ is a function that is the same as the function $f$
except that when applied to element $a$ its value is $b$.  Namely,
\begin{definition}
If $f:A \to B$ is a function and $a,b$ are arbitrary elements, define
  \[
\patch{f}{a}{b}
\]
to be the function $g$ such that
\begin{equation*}
g(u) = \begin{cases}
         b & \text{ if } u = a.\\
         f(u) & \text{otherwise.}
       \end{cases}
\end{equation*}
\end{definition}

Now we can specify the step-by-step execution of a \while\ program as a state
machine, where the states of the machine consist of a \while\ program paired with an
environment.  The transitions of this state machine are defined
recursively on the definition of \while\ programs.

\begin{definition}

The transitions $\step{C}{\rho}{D}{\rho'}$ of the \while\ program state machine are
defined as follows:

\textbf{base cases:}

\[
\step{x \assigned e}{\rho}{\halt}{\patch{\rho}{x}{\mean{e}\rho}}
\]

\textbf{constructor cases:}
If $C$ and $D$ are \while\ programs, and $T$ is a test, then:

\begin{itemize}

\item if $\step{C}{\rho}{C'}{\rho'}$, then
\[
\step{\seqcomm{C}{D}}{\rho}{C';D}{\rho'}.
\]
Also,
\[
\step{\seqcomm{\halt}{D}}{\rho}{D}{\rho}.
\]

\item if $\mean{T}\rho = \true$, then
\[
\step{\condcomm{T}{C}{D}}{\rho}{C}{\rho},
\]
or if $\mean{T}\rho = \false$, then
\[
\step{\condcomm{T}{C}{D}}{\rho}{D}{\rho}.
\]

\item if $\mean{T}\rho = \true$, then
\[
\step{\wloop{T}{C}}{\rho}{\seqcomm{C}{\wloop{T}{C}}}{\rho}
\]
or if $\mean{T}\rho = \false$, then
\[
\step{\wloop{T}{C}}{\rho}{\halt}{\rho}.
\]
\end{itemize}
\end{definition}

Now \while\ programs are probably going to be the simplest kind of programs you will
ever see, but being condescending about them would be a mistake.  It turns
that \emph{every function on nonnegative integers that can be computed by
  any program} on any machine whatsoever can also be computed by \while\ programs
(maybe more slowly).  We can't take the time to explain how such a
sweeping claim can be justified, but you can find out by taking a course
in computability theory such as 6.045 or 6.840.

\subsection{Denotational Semantics}

The net effect of starting a \while\ program in some environment is reflected in the
final environment when the program halts.  So we can think of a \while\ program $C$
as defining a function $\mean{C}:\Env \to \Env$ from initial
environments to environments at halting.  The function $\mean{C}$ is
called the \term{meaning} of $C$.

$\mean{C}$ of a \while\ program, $C$ to be a partial function from $\Env$ to $\Env$
mapping an initial environment to the final halting environment.

We'll need one bit of notation first.  For any function $f:S \to S$, let
$f^{(n)}$ be the composition of $f$ with itself $n$ times where $n \in
\nngint$.   Namely,
\begin{align*}
f^{(0)} & \eqdef \ident{S}\\
f^{(n+1)} & \eqdef f \compose f^{(n)},
\end{align*}
where ``$\compose$'' denotes functional composition.

The recursive definition of the meaning of a program follows the
definition of the \while\ program recursive data type.
\begin{definition}

\textbf{base cases:}
\begin{itemize}

\item $\mean{x \assigned e}$ is the function from $\Env$ to $\Env$ defined
  by the rule:
\[
\mean{x \assigned e}\rho \eqdef \patch{\rho}{x}{\mean{e}\rho}.
\]

\item
\[
\mean{\halt} \eqdef \ident{\Env}
\]
where $\ident{\Env}$ is the identity function on $\Env$.  In other words,
$\mean{\halt}\rho \eqdef \rho$.

\end{itemize}

\textbf{constructor cases:}
If $C$ and $D$ are \while\ programs, and $T$ is a test, then:
\begin{itemize}

\item
\[
\mean{\seqcomm{C}{D}} \eqdef \mean{D} \compose \mean{C}
\]
That is,
\[
\mean{\seqcomm{C}{D}}\rho \eqdef \mean{D}(\mean{C}\rho).
\]

\item
\[
\mean{\condcomm{T}{C}{D}}\rho
\eqdef
\begin{cases}
\mean{C}\rho & \text{if } \mean{T}\rho = \true\\
\mean{D}\rho & \text{if } \mean{T}\rho = \false.
\end{cases}
\]

\item
\[
\mean{\wloop{T}{C}}\rho \eqdef \mean{C}^{(n)}\rho
\]
where $n$ is the least nonnegative integer such that
$\mean{T}(\mean{C}^{(n)}\rho) = \false$.  (If there is no such $n$, then
$\mean{\wloop{T}{C}}\rho$ is undefined.)
\end{itemize}

\end{definition}

We can use the denotational semantics of \while\ programs to reason about
\while\ programs using structural induction on programs, and this is often
much simpler than reasoning about them using induction on the number of
steps in an execution.  This is OK as long as the denotational semantics
accurately captures the state machine behavior.  In particular, using the
notation $\movesto^*$ for the transitive closure of the transition
relation:
\begin{theorem}\label{sound-semantics}
\[
\haltswith{C}{\rho}{\rho'} \qiff \mean{C}\rho = \rho'
\]
\end{theorem}
Theorem~\ref{sound-semantics} can be proved easily by induction; it
appear in~Problem\ref{PS_while_program_semantic_soundness_proof}.

\begin{problems}
\homeworkproblems
\pinput{PS_while_program_semantic_soundness_proof}
\end{problems}

\subsection{Logic of Programs}

A typical program specification describes the kind of inputs and
environments the program should handle, and then describes what should
result from an execution.  The specification of the inputs or initial
environment is called the \term{precondition} for program execution, and
the prescription of what the result of execution should be is called the
\emph{postcondition}.  So if $P$ is a logical formula expressing the precondition
for a program $C$ and likewise $Q$ expresses the postcondition, the
specification requires that
\begin{quote}
If $P$ holds when $C$ is started, then $Q$ will hold if and when $C$ halts.
\end{quote}
We'll express this requirement as a formula
\[
\after{P}{C}{Q}
\]
called a \term{partial correctness assertion}.

For example, if $E$ is the \while\ program above for the Euclidean
algorithm, then the partial correctness of $E$ can be expressed as
\begin{equation}\label{eucpc}
\after{\paren{a, b \in \nngint\ \QAND\  \mathtt{a} \neq 0}}{E}
                   {\paren{\mathtt{x} = \gcd(a,b)}}.
\end{equation}

More precisely, notice that just as the value of an expression in an
environment is an integer, the value of a logical formula in an
environment is a truth value.  For example, if $\rho(\mathtt{x}) =4$,
and $\rho(\mathtt{y}) =-2$, then by~\eqref{x2y311},
$\mean{\mathtt{x^2+y -3}}\rho = 11$, so
\begin{align*}
\mean{\exists \mathtt{z}.\, \mathtt{z} > 4 \QAND \mathtt{x}^2+\mathtt{y} -3=\mathtt{z}}\rho & = \true,\\
\mean{\exists \mathtt{z}.\, \mathtt{z} > 13 \QAND \mathtt{x}^2+\mathtt{y} -3=\mathtt{z}}\rho & = \false.
\end{align*}

\begin{definition}\label{def_afterPCQ}
For logical formulas $P$ and $Q$, and \while\ program $C$ the partial correctness assertion
\[
\after{P}{C}{Q}
\]
is true proving that for all environments $rho$ if $\mean{P}\rho$ is
true, and $\haltswith{C}{\rho}{\rho'}$ for some $\rho'$, then
$\mean{Q}\rho'$ is true.
\end{definition}

In the 1970's, 
\begin{editingnotes}\TBA{check Hoare ref}\end{editingnotes}
Prof.\ Tony Hoare (now Sir Anthony) at
Univ. Dublin formulated a set of inference rules for proving partial
correctness formulas.  These rules are known as \term{Hoare Logic}.

The first rule captures the fact that strengthening the preconditions
and weakening the postconditions makes a partial correctness
specification easier to satisfy:

\Rule{P \ \QIMPLIES\  R,
     \quad \after{R}{C}{S},
     \quad S \ \QIMPLIES\  Q}{\after{P}{C}{Q}}

The rest of the logical rules follow the recursive definition of \while\ programs.
There are axioms for the base case commands:
\[\begin{array}{c}
  \after{P(x)}{x \assigned e}{P(e)}\\
  \after{P}{\halt}{P},
\end{array}\]
and proof rules for the constructor cases:

\begin{itemize}

\item
\Rule{\after{P}{C}{Q} \ \QAND\  \after{Q}{D}{R}}{\after{P}{\seqcomm{C}{D}}{R}}

\item
\Rule{\after{P\ \QAND\ T}{C}{Q}}
        {\after{P\ \QAND\ T}{\condcomm{T}{C}{D}}{Q\ \QAND\ T}}

\item
\Rule{\after{P\ \QAND\ T}{C}{P}}
     {\after{P}{\wloop{T}{C}}{P\ \QAND\ \QNOT(T)}}
\end{itemize}

\begin{example}
\TBA{Formal correctness proof of~\eqref{eucpc} for the Euclidean algorithm.}
\end{example}

\begin{editingnotes}
\TBA{Brief discussion of ``relative completeness''.}
\end{editingnotes}

\endinput

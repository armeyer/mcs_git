\documentclass[problem]{mcs}

\begin{pcomments}
  \pcomment{TP_which_are_partial_orders}
  \pcomment{from: S02.cp3f}
  \pcomment{S10: added by estick}
  \pcomment{5/19/18 needs update re weak/strict partial orders -- ARM}
\end{pcomments}

\pkeywords{
  partial orders
  transitive
  antisymmetric
  reflexive
  total_order
  linear
}

%%%%%%%%%%%%%%%%%%%%%%%%%%%%%%%%%%%%%%%%%%%%%%%%%%%%%%%%%%%%%%%%%%%%%
% Problem starts here
%%%%%%%%%%%%%%%%%%%%%%%%%%%%%%%%%%%%%%%%%%%%%%%%%%%%%%%%%%%%%%%%%%%%%

\begin{problem}

In each of the following examples, $R$ is a binary relation on a set
$A$.  Indicate which examples are weak partial orders or strong partial orders, and when they are, indicate whether they are linear. \textbf{Explain} your answers.

\begin{problemparts}

\problempart $A=\set{a,b,c},\  R=\set{(a,a),(b,a),(b,b),(b,c),(c,c)}.$

\begin{solution}
WPO, not Linear. $R$ is reflexive, transitive, and antisymmetric.  Therefore, $R$ is
a weak partial order on $A$.  However, it is not a linear order on $A$ since
neither the pair $(a,c)$ nor $(c,a)$ is in $R$.
\end{solution}

\examspace[1.5in]
\problempart $A=$ the set of all English words,
\[
R = \set{(x,y) \in A \times A \suchthat x \text{ comes before $y$ alphabetically}}.
\]

\begin{solution}
  SPO, Linear.  $R$ is transitive and asymmetric, so it is a strong partial order. For any pair of distinct words, one of them comes before the other alphabetically, so $R$ is linear.
\end{solution}

\examspace[1.5in]

\problempart $A=\reals,\  R= \set{(x,y) \in \reals \times \reals \suchthat
  \abs{x} < \abs{y} \QOR x=y}$.

\begin{solution}
WPO, not Linear. $R$ is reflexive, transitive, and antisymmetric.  Therefore, $R$ is a
  weak partial order on $A$.  However, $R$ is not a linear order on $A$.
  For example, if $x = 5, y = -5$, then none of the following holds:
  $x = y, \abs{x} < \abs{y}, \abs{y} < \abs{x}$.
\end{solution}

\examspace[1.5in]

\problempart $A=\reals,\ R=\set{(x,y) \in \reals \times \reals \suchthat
  \abs{x} \leq \abs{y}}$.

\begin{solution}
  Neither. $R$ is neither asymmetric nor antisymmetric, so it is neither a WPO nor a SPO. As a counterexample, $\abs{-3} \leq \abs{3}, \abs{3} \leq \abs{-3}$,
  but $3 \neq -3$.
\end{solution}

\end{problemparts}

\end{problem}


%%%%%%%%%%%%%%%%%%%%%%%%%%%%%%%%%%%%%%%%%%%%%%%%%%%%%%%%%%%%%%%%%%%%%
% Problem ends here
%%%%%%%%%%%%%%%%%%%%%%%%%%%%%%%%%%%%%%%%%%%%%%%%%%%%%%%%%%%%%%%%%%%%%

\endinput

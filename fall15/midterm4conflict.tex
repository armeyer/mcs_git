\documentclass[quiz]{mcs}

\renewcommand{\exampreamble}{   % !! renew \exampreamble
%    \textbf{Indicate your}\ \teaminfo

\begin{center}
{\large   \textbf{Circle your} \qquad \teaminfo}
\end{center}

  \begin{itemize}

  \item
   This exam is \textbf{closed book} except for a 2-sided cribsheet.
   Total time is 60 minutes. 

  \item
   Write your solutions in the space provided.  If you need more
   space,  \textbf{write on the back} of the sheet containing the problem.

%   Please keep your entire answer to a problem on that problem's page.
   
   \item In answering the following questions, you may use without
     proof any of the results from class or text.
     \iffalse (unless explicitly instructed otherwise).\fi


\iffalse
  \item
   GOOD LUCK!
\fi

  \end{itemize}}

\begin{document}

\conflictmidterm{November 24}

%%%%%%%%%%%%%%%%%%%%%%%%%%%%%%%%%%%%%%%%%%%%%%%%%%%%%%%%%%%%%%%%%%%%%
% Problems start here
%%%%%%%%%%%%%%%%%%%%%%%%%%%%%%%%%%%%%%%%%%%%%%%%%%%%%%%%%%%%%%%%%%%%

%\examspace
\begin{staffnotes}
\begin{center}
{\Large DRAFT}
\end{center}
\end{staffnotes}

\begin{staffnotes}
\begin{center}
{\large Counting}
\end{center}
\end{staffnotes}

\pinput[points = 15, title= \textbf{Counting}]{FP_counting_various_f15_ver1}
\examspace
\begin{staffnotes}
FP\_counting\_various\_f15\\
A compilation of counting problems. Includes parts of (or similar to)
PS\_alphabet. 
\end{staffnotes}

%\pinput[points = 15, title= \textbf{Arranging letters}]{PS_alphabet}
\begin{staffnotes}
PS\_alphabet\\
Zoran: Good problem. (a) and (b) are enough.
I came up with slightly different ones along the same lines and will
write and sub later.\\
 (c) is hard.\\
 (d) uses a ``mapping-to-sum-equation'' technique that
is currently already used in another problem.
\end{staffnotes}

%\pinput[points = 15, title= \textbf{Counting}]{FP_counting_grades_assignments}
\begin{staffnotes}
FP\_counting\_grades\_assignments\\
Zoran: Tests if students can map a problem to a known problem.
Would be fine, but uses the same ``mapping-to-sum-equation'' technique
already used in another problem.
\end{staffnotes}

%\pinput[points = 15, title= \textbf{Counting solutions}]{MQ_more_counting_practice}
\begin{staffnotes}
MQ\_more\_counting\_practice\\
Zoran: Good, but uses a twist to ``mapping-to-sum-equation'' technique
already used in another problem.
\end{staffnotes}

%\pinput[points = 15, title= \textbf{Robot Paths}]{MQ_counting_robot_paths}
\begin{staffnotes}
MQ\_counting\_robot\_paths\\
Zoran: Fine, but simple. Could be added as a part of a counting problem with various parts
if there is space.
\end{staffnotes}

%\pinput[points = 15, title= \textbf{Counting}]{CP_nonadjacent_books} % was in CP

%\pinput[points = 15, title= \textbf{Size of powersets}]{CP_induction_numberofsubsets} 
%out of place for mid4 elementary induction & sets, not much counting.

%\pinput[points = 15, title= \textbf{Books on bookshelf}]{CP_nonadjacent_books_counting_sequel} % was in CP

%\pinput[points = 15, title= \textbf{Binomial Coefficients}]{CP_binom_coeff} % was in CP


\begin{staffnotes}
\begin{center}
{\large Pigeonhole, Inclusion-Exclusion, Combinatorial.}
\end{center}
\end{staffnotes}

\pinput[points = 8, title= \textbf{Pigeonhole Principle}]{PS_pigeonhole-diff_2}
\examspace
\begin{staffnotes}
PS\_pigeonhole-power\_of\_2\\
Zoran: Fine problem (with hint).
\end{staffnotes}

%\pinput[points = 8, title= \textbf{Pigeonhole Principle}]{CP_pigeonhole_777000}
\begin{staffnotes}
CP\_pigeonhole\_777000\\
Zoran: I like this problem and would choose either this or CP\_factorial\_sum.
\end{staffnotes}

\pinput[points = 15, title= \textbf{Counting, Inclusion-Exclusion}]{FP_counting_4_digit_numbers_with_sum_of_digits_22_ver1}
\examspace
\begin{staffnotes}
FP\_counting\_4\_digit\_numbers\_with\_sum\_of\_digits\_22\\
Zoran: Combines a bijection to solutions of a ``sum equation'' with inclusion-exclusion.
I suggest this one out of the two I proposed.
\end{staffnotes}

%\pinput[points = 3, title= \textbf{Quotient is power of 3}]{PS_pigeonhole-power_of_3}
\begin{staffnotes}
PS\_pigeonhole-power\_of\_3\\
Zoran: A bit confusing.
\end{staffnotes}

%\pinput[points = 3, title= \textbf{k appears in kth, in place}]{CP_factorial_sum}
\begin{staffnotes}
CP\_factorial\_sum\\
Zoran: This is a nice combinatorial-proof problem. Choosing it instead of CP\_pigeonhole\_777000
would be fine.
\end{staffnotes}

%\pinput[points = 3, title= \textbf{Student lineup}]{FP_count_lineup}
\begin{staffnotes}
FP\_count\_lineup\\
Zoran: A bit boring.
\end{staffnotes}

%\pinput[points = 3, title= \textbf{Smallest hand for flush}]{MQ_pigeonhole_cards_flush}
%shallow: TP_ level at best


\begin{staffnotes}
\begin{center}
{\large Generating Functions}
\end{center}
\end{staffnotes}

%\examspace
%\pinput[points = 7, title= \textbf{Generating Functions}]{MQ_linear_recur_even}
\begin{staffnotes}
MQ\_linear\_recur\_even\\
Zoran: A relatively simple GF problem, but good for test.
\end{staffnotes}

\pinput[points = 10, title= \textbf{Generating Functions}]{FP_linear_recur_squares_f15_ver1}
\examspace
\begin{staffnotes}
FP\_linear\_recur\_simplified\\
Zoran: The recurrence is the same as in one of the ps problems, although it is given directly
rather than using a textual description.
Would be fine if numbers are changed. Could be used instead of MQ\_linear\_recur\_even
if a somewhat harder GF problem is needed (depends on other problems selected).
\end{staffnotes}

%\pinput[points = 15, title= \textbf{Generating Bookkeeper}]{CP_nth_derivative_of_A}
% tedious, mainly induction

%\pinput[points = 6, title= \textbf{Boat trip}]{FP_boat_trip} % was in CP


\begin{staffnotes}
\begin{center}
{\large Probability}
\end{center}
\end{staffnotes}

%\examspace
%\pinput[points = 15, title= \textbf{Probability}]{MQ_a_baseball_series}
%\pinput[points = 15, title= \textbf{Probability}]{MQ_a_baseball_series_revised}
\begin{staffnotes}
MQ\_a\_baseball\_series\_revised\\
Zoran: I like this as a basic probability question with not a long text. Gives an option to ask various
subquestions (conditional prob.\ as well) -- can make variations for conflicts.
\end{staffnotes}

% BACKUP PROBLEMS

%\pinput{MQ_voldemort_returns}
\pinput[points = 12, title= \textbf{Probability}]{MQ_voldemort_returns_again_ver1}
%\pinput[points = 15, title= \textbf{Probability}]{MQ_voldemort_really_returns}
\begin{staffnotes}
MQ\_voldemort\_really\_returns\\
Zoran: Good problem, but a bit long text. Similar type as MQ-a-baseball-series-revised, but with
conditional prob.
\end{staffnotes}

%\pinput{MQ_Nerditosis}
\begin{staffnotes}
MQ\_Nerditosis\\
Zoran: Good problem, but MQ-a-baseball-series-revised allows for more creative questions.
\end{staffnotes}

%\pinput{FP_probability}
\begin{staffnotes}
FP\_probability\\
Zoran: Good problem, but mostly a counting problem -- good to show the connection to prob.
We can add some parts if there is space.
\end{staffnotes}

%\pinput{FP_probable_isomorphism}
\begin{staffnotes}
FP\_probable\_isomorphism\\
Zoran: Very nice problem, but basically a counting problem.
Also, it refers to graph isomorphisms and, even though one may figure out what is asked,
some may be unprepared. Great for Finals!
\end{staffnotes}

%\pinput{PS_conditional_space}
\begin{staffnotes}
PS\_conditional\_space\\
Zoran: It shows an important properly of prob.\ distributions. The proof itself
is not so interesting, but ok.
\end{staffnotes}

%\pinput{TP_bogus_discrimination_contradiction_shorter}
\begin{staffnotes}
TP\_bogus\_discrimination\_contradiction\_shorter\\
Revise from male-female admission rate to comparison of
two drugs each treating two diseases.\\
Zoran: Simpson's paradox is a topic for Monday's class
-- the day before the exam, which is not counted for midterm 4.
\end{staffnotes}

%\pinput{TP_Fun_with_coins}  NOT IN REPO

%\pinput{FP_ran3p_graph}   expectation not on mid4

%\pinput{PS_conditional_aces}  %too long for midterm


%%%%%%%%%%%%%%%%%%%%%%%%%%%%%%%%%%%%%%%%%%%%%%%%%%%%%%%%%%%%%%%%%%%%%
% Problems end here
%%%%%%%%%%%%%%%%%%%%%%%%%%%%%%%%%%%%%%%%%%%%%%%%%%%%%%%%%%%%%%%%%%%%%
\end{document}

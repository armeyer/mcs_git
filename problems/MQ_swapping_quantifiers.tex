\documentclass[problem]{mcs}

\begin{pcomments}
  \pcomment{MQ_swapping_quantifiers}
  \pcomment{from Drew & ARM 3/16/12}
\end{pcomments}

\pkeywords{
  quantifiers
  counter_model
  invalid
  implication
  alternating_quantifiers
  valid
}

%%%%%%%%%%%%%%%%%%%%%%%%%%%%%%%%%%%%%%%%%%%%%%%%%%%%%%%%%%%%%%%%%%%%%
% Problem starts here
%%%%%%%%%%%%%%%%%%%%%%%%%%%%%%%%%%%%%%%%%%%%%%%%%%%%%%%%%%%%%%%%%%%%%

\begin{problem}

One of the implications below is not valid\dots which one? \examboxplain{0.2in}{0in}{0.2in}
\begin{align}
\forall x\, \exists y.\, P(x, y) & \QIMPLIES \exists y\, \forall x.\, P(x, y),\label{AEPneq}\\
\exists y\, \forall x.\, P(x, y) & \QIMPLIES \forall x\, \exists y.\, P(x, y).
\end{align}

\examspace[0.2in]

Provide a counter-model for this implication with domain of discourse
\[
D \eqdef \set{3, 5}.
\]
You do not need to explain your answer.

\begin{solution}
The first implication~\eqref{AEPneq} is not valid.

A simple counter-model is when the predicate $P(x, y)$ means $x \neq
y$.

\iffalse with $D \eqdef \set{\True, \False}$, equivalent ways to
define $P(x,y)$ are $\QNOT(x \QIFF y)$ and $x \QXOR y$.
\fi

The antecedent of the implication is true: for every $x \in D$, there
exists a $y \in D$ which is $\neq x$, namely, let $y$ be the other element in $D$. %$\QNOT(x)$.

The consequent is not true since there is no $x$ that differs from
everything, because $x$ does not differ from itself.
\end{solution}

\end{problem}

%%%%%%%%%%%%%%%%%%%%%%%%%%%%%%%%%%%%%%%%%%%%%%%%%%%%%%%%%%%%%%%%%%%%%
% Problem ends here
%%%%%%%%%%%%%%%%%%%%%%%%%%%%%%%%%%%%%%%%%%%%%%%%%%%%%%%%%%%%%%%%%%%%%

\endinput

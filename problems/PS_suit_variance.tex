\documentclass[problem]{mcs}

\begin{pcomments}
  \pcomment{PS_suit_variance}
  \pcomment{S01.practice-last-pset: was commented out}
  \pcomment{added, edited ARM 5/20/17}
  \pcomment{PART(C) NEEDS SOLN, DO NOT USE UNTIL CHECKED, ARM 11/19/17}
\end{pcomments}

\pkeywords{
  expectation
  variance
}

%%%%%%%%%%%%%%%%%%%%%%%%%%%%%%%%%%%%%%%%%%%%%%%%%%%%%%%%%%%%%%%%%%%%%
% Problem starts here
%%%%%%%%%%%%%%%%%%%%%%%%%%%%%%%%%%%%%%%%%%%%%%%%%%%%%%%%%%%%%%%%%%%%%

\begin{problem}
Suppose you are playing the card game ``Hearts'' with three of your
friends.  Hearts uses a standard deck of fifty-two cards with thirteen
cards of each suit: spades, hearts, diamonds, clubs.  At the start,
thirteen of the cards are randomly dealt to each player.  Let $H$ be
the number of hearts in your hand.

\bparts

\ppart\label{numHinexp}
Express $\expect{H}$ as a fraction.

\exambox{0.6in}{0.6in}{-0.1in}

\examspace[0.7in]

\begin{solution}
Let $H_i$ be the indicator variable that equals one if your $i$th card
is a heart and equals zero otherwise.  So $H = \sum_{i=1}^{13} H_i$.
Because $H_i$ is an indicator variable, its expectation is the
probability that the $i$th card is a heart, namely, 1/4.  So
\[
\expect{H} = \sum_1^{13} 1/4 = 13/4.
\]
\end{solution}

\ppart\label{numHinvar} Write a closed form numerical formula for
$\variance{H}$.  Your formula may contain binomial coefficients; you
do not need to evaluate it.

\exambox{3.0in}{0.6in}{-0.1in}

\examspace[1.5in]

\begin{solution}
\[
\Variance{H}
  = \paren{13 \cdot \frac{1}{4} + 156 \cdot \frac{3}{51}} - \paren{\frac{13}{4}}^2
%  \approx 1.863970588.
\]

To compute variance, we use the fact that
\[
\Variance{H} = \Expect{H^2} - \Expectsq{H}.
\]
Now we need only find $\Expect{H^2}$:
\begin{align}
\Expect{H^2}
    & \eqdef \Expect{\paren{\sum_{i=1}^{13} H_i}^2} \notag\\
    & = \Expect{\sum_{i=1}^{13} H_i^2 + \sum_{i\neq j} H_iH_j} \notag\\
    & = \sum_{i=1}^{13} \Expect{H_i^2} + \sum_{i\neq j} \Expect{H_iH_j}.\tag{E2Eij}
\end{align}
Now we note that $\expect{H_i^2} = \expect{H_i}$ since $H_i$ is an
indicator variable, so $\expect{H_i^2} = 1/4$.

When $i \neq j$, the product $H_iH_j = 1$ iff the $i$th and $j$th
cards are both hearts.  The probability that both cards are hearts is
$(1/4)(12/51)$, and thus
\[
\expect{H_i H_j} = (1/4)(12/51) = 3/51.
\]
In the summation~(E2Eij), there are 13 $\expect{H_i^2}$ terms,
and $2\binom{13}{2} = 156$ terms of the form $\expect{H_iH_j}$.  Thus,
\[
\Expect{H^2} = 13\cdot\frac{1}{4} + 156\cdot \frac{3}{51},
\]
and so
\[
\Variance{H}
  = \paren{13 \cdot \frac{1}{4} + 156 \cdot \frac{3}{51}} - \paren{\frac{13}{4}}^2
%  \approx 1.863970588.
\]

\end{solution}

\eparts

\bigskip

Let $N$ be the number of suits that appear in your hand.

%---so the range of $N$ is the integer interval $\Zintv{0}{4}$

\bparts

\ppart Write a simple formula for $\expect{N}$.

\exambox{1.2in}{0.5in}{-0.1in}

\examspace[0.4in]

\begin{solution}
\TBA{soln}

\iffalse
Let $H_h$ be the indicator variable which is 1 if there is a heart in
the hand and 0 otherwise. Define $H_s, H_d, H_c$ similarly for the
three remaining suits.  Then
\[
N = H_h + H_s + H_d + H_c.
\]

BUGGY:   \textcolor{red}{Note that $H_h$ is the same as $H$, so}
\[
\expect{N} \eqdef \expect{H_s + H_h + H_d + H_c} = 4 \expect{H} %\approx 3.94884.
\]
\fi

\end{solution}

\examspace[0.4in]

\ppart Write a numerical formula for
\inhandout{$\prob{N=2}$}\inbook{$\variance{N}$}.  Your formula may
contain binomial coefficients; you do not need to evaluate it.

\exambox{4.0in}{0.6in}{-0.1in}

\examspace[3.0in]

\begin{solution}
\inbook{\[
\variance{N} \eqdef \sum_{i=1}^4 \prob{N=i}(i - \expect{N})^2.
\]
We need only calculate each of the probabilities.  First, there are
four possible hands of all one suit, so
\[
\pr{N=1} = \frac{4}{\binom{52}{13}}   %= 6.299 \times 10^{-12},
\]

Next, t}\inhandout{T}here are $\binom{4}{2}$ ways to choose two suits and there are
$\binom{26}{13}$ ways to choose 13 cards from the 26 possible cards of
those two suits.  Since two of those $\binom{26}{13}$ hands actually
contain only one suit, we have $\binom{26}{13}-2$ that contain exactly
those two suits.  Therefore,
\[
\pr{N=2} = \frac{\binom{4}{2}\paren{\binom{26}{13} - 2}}{\binom{52}{13}}
%             = 9.827 \times 10^{-5}.
\] 

\inbook{Similarly, there are $\binom{4}{3}$ ways to choose the three suits.
There are $\binom{39}{13}$ ways to choose 13 cards from the 39 cards
of those three suits.  But that also includes hands with two suits and
one suit.  There are $\binom{3}{2}$ ways to choose two suits from the
three suits and $\binom{26}{13}$ to choose 13 cards from the 26
cards of those two suits.  So the number of hands with at most two
suits is $\binom{3}{2}\binom{26}{13}$.  Finally, there are three hands
containing exactly one of the three suits.  Now by inclusion-exclusion, there are
\[
\binom{4}{3}\paren{\binom{39}{13} - \binom{3}{2}\binom{26}{13} + 3}
\]
hands containing exactly three suits.  Therefore,
\[
\pr{N=3} = \frac{\binom{4}{3}\paren{\binom{39}{13} -
\binom{3}{2}\binom{26}{13} + 3}}{\binom{52}{13}} %= 0.05097.
\]

Finally, 
\[
\pr{N=4} = 1 - (\pr{N=1} +\pr{N=2} + \pr{N=3}) %\approx 0.9489.
\]

This now allows us to calculate the numerical value of the variance,
which turns out to be approximately 0.04874.}
\end{solution}

\eparts


\end{problem}

\endinput

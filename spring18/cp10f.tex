\documentclass[handout]{mcs}

\begin{document}

\inclassproblems{10, Fri.}

%%%%%%%%%%%%%%%%%%%%%%%%%%%%%%%%%%%%%%%%%%%%%%%%%%%%%%%%%%%%%%%%%%%%%
% Problems start here
%%%%%%%%%%%%%%%%%%%%%%%%%%%%%%%%%%%%%%%%%%%%%%%%%%%%%%%%%%%%%%%%%%%%%

\begin{staffnotes}
Chapter~\bref{Turing_sec}.\ \emph{Turing}
through~\bref{sec:inverse}.\ \emph{Cancelling $\pmod{n}$}
\end{staffnotes}

%\pinput{CP_fast_exponentiation}
%\pinput{CP_calculating_inverses_fermat} %PS_calculating_inverses (b)

\pinput{FP_inverse17mod29}

%\pinput{CP_7777}

\begin{staffnotes}
  For Problem 2, briefly review the rules for simplifying sums and products in modular arithmetic: can always replace a number being added or multiplied with anything it is equivalent to. But \emph{not} exponents and \emph{not} denominators (sometimes).
\end{staffnotes}
\begin{staffnotes}
  Reinforce the relationship between mod and rem. Mod is an \emph{(equivalence) relation} (two numbers are equivalent or not), whereas $\rem{a}{n}$ is a \emph{function} (it returns one number in the range $[0,n)$.) Don't use mod as a function: $a\bmod n$ on its own doesn't mean anything.
\end{staffnotes}

\pinput{CP_remainder_computation_practice}

\pinput{CP_multiples_of_9_and_11}


%\pinput{CP_divisible_by_24}

\pinput{MQ_congruent_mod_product}  %lightweight intro to Chinese remainder

\pinput{CP_chinese_remainder} % Rearranged things since I'm guessing we want students to get to this one?

%\textbf{Supplemental problems:}



%\pinput{PS_check_factor_by_digits}


%\pinput{CP_polynomials_produce_multiples} 

%\pinput{CP_pirate_treasure}

%%%%%%%%%%%%%%%%%%%%%%%%%%%%%%%%%55
% Problems end here
%%%%%%%%%%%%%%%%%%%%%%%%%%%%%%%%%%%%%%%%%%%%%%%%%%%%%%%%%%%%%%%%%%%%%

\end{document}

\endinput


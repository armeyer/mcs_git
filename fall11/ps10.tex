
\documentclass[handout]{mcs}

%sums and products
%asymptotics

\begin{document}

\renewcommand{\reading}{ \emph{For this pset}:
  Chapter~\bref{bijection_counting_sec}{--\bref{sec:counting_sequences}
    Counting with
    Bijections},~\bref{generalized_product_sec}{--\bref{combinations_sec}{
      Generalized Product and Division
      Rules}},~\bref{bookkeeper_sec}{--\bref{pigeon_hole_sec}{
      Counting Repetitions and Pigeon Hole Principle}}.

For Friday lecture:
Chapter~\bref{cardmagic_sec}{--\bref{inc-ex_sec}{ Card Magic and Inclusion-Exclusion}}.
}

\problemset{10}

%%%%%%%%%%%%%%%%%%%%%%%%%%%%%%%%%%%%%%%%%%%%%%%%%%%%%%%%%%%%%%%%%%%%%
% Problems start here
%%%%%%%%%%%%%%%%%%%%%%%%%%%%%%%%%%%%%%%%%%%%%%%%%%%%%%%%%%%%%%%%%%%%%

%\pinput{PS_5_card_poker}   %like the colored dice below
%\pinput{PS_bijective_FLT}  %nice, but needs work
%\pinput{PS_more_numbered_trees}  %too long
\pinput{PS_counting_colored_dice}

\pinput{PS_monochromatic_rectangle}  %nice, pigeonhole
%\pinput{PS_alphabet}     %routine
%\pinput{PS_counting_robot_paths}  %too easy
%\pinput{PS_counting_graphs}  %good, but enough from the colored dice

\pinput{CP_multinomial_fermat}
%\pinput{PS_counting_problems}  %uninspired



%%%%%%%%%%%%%%%%%%%%%%%%%%%%%%%%%%%%%%%%%%%%%%%%%%%%%%%%%%%%%%%%%%%%%
% Problems end here
%%%%%%%%%%%%%%%%%%%%%%%%%%%%%%%%%%%%%%%%%%%%%%%%%%%%%%%%%%%%%%%%%%%%%

\end{document}


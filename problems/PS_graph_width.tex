\documentclass[problem]{mcs}

\begin{pcomments}
  \pcomment{PS_graph_width}
  \pcomment{motivated by PS_graph_colorable}
  \pcomment{ARM 5/19/14}
\end{pcomments}

\pkeywords{
 graph theory
 colorable
 width
 handshaking
 average_degree
}

\begin{problem}
A sequence of vertices of a graph has \index{graph!width}\emph{width}
$w$ iff each vertex is adjacent to at most $w$ vertices that precede
it in the sequence.  A simple graph $G$ has width $w$ if
there is a width-$w$ sequence of all its vertices.

\bparts

\ppart\label{widthmindeg} Explain why the width of a graph must
be at least the minimum degree of its vertices.

\begin{solution}
However the vertices are lined up in sequence, the last vertex in the
list will be preceded by all the vertices adjacent to it, so the width
will be at least the degree of the last vertex.
\end{solution}

\ppart Prove that if a finite graph has width $w$, then there is a
width-$w$ sequence of all its vertices that ends with a minimum degree
vertex.

\begin{solution}
Removing a vertex from a width-$w$ sequence leaves a width-$w$
sequence, and adding a new vertex of degree $d$ to the end of a
width-$w$ sequence yields a width-$\max(w,d)$ sequence.

So if a sequence of all the vertices of a graph has width $w$, then,
since $w$ is at least the minimum degree, moving a minimum degree
vertex to the end of the list will yield another width-$w$ sequence of
all the vertices.
\end{solution}

\ppart Describe a simple algorithm to find the minimum width of a graph.

\begin{solution}
A recursive ``greedy'' algorithm is easy: remove any min degree vertex
$v$, create a min width sequence for the remaining subgraph and put
$v$ at the end of the sequence.

The same idea is the basis for a nonrecursive algorithm: start with
the whole graph and an empty sequence of vertices.  At each staqe,
find a minimum degree vertex $v$ in the current graph and append $v$
to the end of the current sequence of vertices.  Then update the
current graph by deleting $v$.  The last stage is when the current
graph has one vertex.  The final result will be the reversal of the
current sequence.
\end{solution}

\eparts

\end{problem}
\endinput

\documentclass[problem]{mcs}

\begin{pcomments}
  \pcomment{CP_conditional_even_throws2}
  \pcomment{complements PS_conditional_expectation_even_throws}
  \pcomment{ARM 5/27/18}
\end{pcomments}

\pkeywords{
  conditional_probability
  dice
  independence
  total_probability
  geometric_series
}

%%%%%%%%%%%%%%%%%%%%%%%%%%%%%%%%%%%%%%%%%%%%%%%%%%%%%%%%%%%%%%%%%%%%%
% Problem starts here
%%%%%%%%%%%%%%%%%%%%%%%%%%%%%%%%%%%%%%%%%%%%%%%%%%%%%%%%%%%%%%%%%%%%%

\begin{problem}
A fair six-sided die is repeatedly thrown until a six appears.

A natural sample space modelling this situation is the set of finite
strings of integers from one to six that end at the first occurrence
of a six.  That is, $\sspace \eqdef \strings{\Zintv{1}{5}}6$.

For example, 256 is the outcome corresponding to successively throwing
a two, a five and a six.  The length-one string $6 \in \sspace$ is the
outcome corresponding to six appearing on the first throw.

\bparts \ppart\label{throwprob} What should $\pr{256}$ be defined to
be? \dots $\pr{6}$?  What about the probability of an arbitrary
outcome $s \in \sspace$?

\begin{solution}
\[
\pr{s} \eqdef (1/6)^{\lnth{s}},
\]
so
\begin{align*}
\pr{256} & = (1/6)^3 = 1/216,\\
\pr{6} & = 1/6.
\end{align*}
\end{solution}

\ppart Verify that $\sspace$ with the probabilities assigned in
part~\eqref{throwprob} defines a probability space.  What does this
imply about the possibility of never throwing a six?

\begin{solution}
To confirm the specified probabilities define a probability space, we
need to verify that the probabilities sum to one:

\begin{align*}
\sum_{s \in \sspace} \pr{s}
  & \eqdef \sum_{s \in \strings{\Zintv{1}{5}}6} (1/6)^{\lnth{s}}\\
  & = \frac16 \sum_{r \in \strings{\Zintv{1}{5}}} (1/6)^{\lnth{r}}\\
  & = \frac16 \sum_{n=0}^\infty \sum_{r \in \Zintv{1}{5}^n} (1/6)^n
             = \frac16 \sum_{n=0}^\infty 5^n (1/6)^n\\
  &  = \frac16 \sum_{n=0}^\infty (5/6)^n
             \qquad = \frac16 \cdot \frac{1}{1-(5/6)} \quad = 1.
\end{align*}

So if we wanted to include an additional sample point $\omega$
corresponding to possibly an infinite number of throws without a six,
the only choice would be $\pr{\omega} = 0$.

In a discrete probability space, the probability of an event is the
same as the probability of the nonzero-probability outcomes in the
event.  It follows that we can safely restrict our probability spaces
to having only nonzero outcomes.  So we omit~$\omega$.
\end{solution}

\eparts
\medskip

For any string $r \in \strings{\Zintv{1}{5}}$, let $F_r$ be the event
that values of the initial throws are are the successive elements of
$r$.  Let $V$ be the event that all the dice throws are e\emph{V}en.  That
is, $V$ is the event $\strings{\set{2,4}}6$ that all throws are twos
and fours until the first six.

\bparts

\ppart Suppose $t$ is a string of twos and fours, that is, $t \in
\strings{\set{2,4}}$.  Explain why
\begin{equation}\label{VFtV}
\prcond{V}{F_t} = \pr{V}.
\end{equation}

\begin{solution}
If the first throws are a sequence $t$ of twos and fours, then we
are in the same situation as at the start, so the probability that
all the remaining throws will be even is the same as the probability
of all evens in the first place.

More formally, for any event $A \subseteq \sspace$ and die values $r
\in \strings{\Zintv{1}{5}}$, let
 \[
rA \eqdef \set{rs \suchthat s \in A},
\]
So $F_r \eqdef r\sspace$.  Independence of die throws implies that
\[
  \pr{rA} =\pr{F_r}\pr{A}.
\]
Now for $t \in \strings{\set{2,4}}$,
\[
F_t \intersect V = tV,
\]
so
\[
\prcond{V}{F_t}
     \eqdef \frac{\pr{F_t \intersect V}}{\pr{F_t}}
     = \frac{\pr{tV}}{\pr{F_t}} = \frac{\pr{F_t}\pr{V}}{\pr{F_t}} = \pr{V}.
\]
\end{solution}

\ppart\label{partFtVFt} Explain why equation~\eqref{VFtV} implies that for
$t\in \strings{\set{2,4}}$,
\begin{equation}\insolutions{\label{FtVFt}}\instatements{\notag}
\prcond{F_t}{V} = \pr{F_t}.
\end{equation}
Conclude that
\begin{equation}\label{6V23}
\prcond{6}{V} = \frac23,
\end{equation}

\begin{solution}
Equation~\eqref{VFtV} means that $V$ is \emph{independent} of $F_t$,
so conversely $F_t$ is \emph{independent} of $V$, namely,~\eqref{FtVFt}.

But given $V$, the first throw must be two, four or six, so
\[
1 =  \prcond{F_2}{V} + \prcond{F_4}{V} + \prcond{6}{V},
\]
and therefore,
\[
\prcond{6}{V} = 1 - \pr{F_2} - \pr{F_4} = 1 - \frac16  - \frac16 = \frac23.
\]
\end{solution}

\ppart Given that all throws are even, the only possible first throws
are two, four and six.  Since the die is fair, \textcolor{red}{these
  are all equally likely, so the probability $\prcond{6}{V}$ that the
  first throw is a six must be 1/3}, contradicting
equation~\eqref{6V23}!  Explain.\footnote{If you're thrown by this,
  you are not alone.  There are several
  \href{http://www.untrammeledmind.com/2017/12/counterintuitive-dice-probability-how-many-rolls-expected-to-get-a-6-given-only-even-outcomes}{websites}
  devoted to explanations of this seductive problem.  In fact, when it
  came up at the MIT Theory of Computation faculty lunch in April
  2018, several attendees confidently defended this mistaken
  reasoning.}

\begin{solution}
We're only interested in the first throw, so it's tempting to think
that the probability of a first-throw event conditioned on all throws
being even would be the same as just conditioning on the first throw
being even; that is a mistake.  In particular, two, four, and six are
equally likely to be the first value thrown given that the
\emph{first} throw is even, but they are \emph{not} equally likely
given that \emph{all} throws are even.  In fact, the solution to
part~\eqref{partFtVFt} shows that for $t\in \strings{\set{2,4}}$,
\begin{align*}
\prcond{F_{t2}}{V} & = \pr{F_{t2}} = \pr{F_t}\cdot \pr{F_2}\\
    & \quad = \pr{F_t}\cdot \frac16,\\
\prcond{F_{t4}}{V} & = \pr{F_{t4}} = \pr{F_t}\cdot \pr{F_4}\\
    & \quad = \pr{F_t}\cdot \frac16,\\
\prcond{\text{outcome}\ t6}{V} & \eqdef \frac{\pr{\set{t6}\intersect V}}{\pr{V}} = \frac{\pr{t6}}{\pr{V}} = \frac{\pr{F_t}\cdot \pr{6}}{\pr{V}}\\
    & =  \pr{F_t} \cdot \frac{\pr{6}}{\pr{V}} = \pr{F_t} \cdot \prcond{6}{V}\\
    & \quad = \pr{F_t} \cdot \frac23.
\end{align*}
This means that conditioned on all even throws, the resulting
probability space is the same as the one that comes from independently
throwing a biased three-sided die until a six appears, with
probabilities of throwing two, four, and six respectively equal to
1/6, 1/6 and 2/3.
\end{solution}

\ppart Conclude immediately from~\eqref{6V23} that
\begin{equation}\label{pV14}
\pr{V} = \frac14.
\end{equation}

\begin{solution}
\[
\prcond{6}{V} \eqdef \frac{\pr{6 \intersect V}}{\pr{V}},
\]
so
\[
\pr{V} = \frac{\pr{6 \intersect V}}{\prcond{6}{V}} = \frac{\pr{6}}{2/3} = \frac{1/6}{2/3} = \frac14.
\]

Less immediately, we could have calculated $\pr{V}$ as a sum over
outcomes:
\begin{align*}
\pr{V} & \eqdef \sum_{n=0}^\infty \sum_{s \in \set{2,4}^n6} \pr{s}\\
       &  = \sum_{n=0}^\infty \sum_{s \in \set{2,4}^n6} (1/6)^{n+1}
             = \frac{1}{6}\sum_{n=0}^\infty 2^n (1/6)^n\\
       &  = \frac{1}{6}\sum_{n=0}^\infty (1/3)^n
             = \frac{1}{6} \cdot \frac{1}{1 - (1/3)} \quad = \frac{1}{4}.
\end{align*}
\end{solution}

\eparts
\end{problem}

%%%%%%%%%%%%%%%%%%%%%%%%%%%%%%%%%%%%%%%%%%%%%%%%%%%%%%%%%%%%%%%%%%%%%
% Problem ends here
%%%%%%%%%%%%%%%%%%%%%%%%%%%%%%%%%%%%%%%%%%%%%%%%%%%%%%%%%%%%%%%%%%%%%

\endinput


\iffalse
\[\begin{array}{lcll}
\lefteqn{\sum_{s \in \sspace}}\\
 & \eqdef & \sum_{s \in \strings{\Zintv{1}{5}}6} (1/6)^{\lnth{s}}
            & = (1/6) \sum_{r \in \strings{\Zintv{1}{5}}} (1/6)^{\lnth{r}}\\
\\
 & =      & (1/6) \sum_{n=0}^\infty \sum_{r \in \Zintv{1}{5}^n} (1/6)^n
            & = (1/6) \sum_{n=0}^\infty 5^n (1/6)^n\\
\\
 & =      & (1/6) \sum_{n=0}^\infty (5/6)^n
            & = (1/6)\frac{1}{1-(5/6)} = 1.
\end{array}\]\fi

\documentclass[problem]{mcs}

\begin{pcomments}
  \pcomment{FP_monochromatic_rectangles}
  \pcomment{Simpler version of PS_monochromatic rectangle along with xtra inc-exc part}
  \pcomment{CH, Spring '14, edited ARM 4/24/14}
  \pcomment{ARM 11/21/17; corrected soln to pigeonhole part;
    suggestion for revising the problem below}
\end{pcomments}

\pkeywords{
  counting
  pigeonhole
  inclusion-exclusion
}

%%%%%%%%%%%%%%%%%%%%%%%%%%%%%%%%%%%%%%%%%%%%%%%%%%%%%%%%%%%%%%%%%%%%%
% Problem starts here
%%%%%%%%%%%%%%%%%%%%%%%%%%%%%%%%%%%%%%%%%%%%%%%%%%%%%%%%%%%%%%%%%%%%%

\begin{problem}
Let $R$ be a rectangular chessboard with 3 rows and 9 columns.  Each
square of $R$ may be colored either black or white.

\bparts

\ppart How many different colorings of $R$ are possible?

\begin{solution}
There are 27 squares in $R$ and each square can be colored either
black or white.  Therefore, by the Product Rule, the total number of
colorings of $R$ is $2^{27}$. 
\end{solution}

\begin{center}
\exambox{0.5in}{0.5in}{0.0in}
\end{center}

\examspace[0.5in]

\ppart Explain why every coloring will include two columns with the
same coloring.

\examspace[1.5in]

\begin{solution} 
There are $2^3 = 8$ ways to color each column of $R$, while there are
$9$ columns.  Therefore, by the Pigeonhole Principle, some pair of columns
must be colored the same. 
\end{solution}

\ppart Prove that every coloring includes four squares that lie at
the corners of a rectangle and have the same color.

\examspace[1.5in]

\begin{solution}
There must be two columns with the same coloring.  These columns have
3 rows and therefore some color must be repeated.  The first two
squares of this repeated color in each of the columns form the corners
of the desired rectangle.
\end{solution}

% \begin{staffnotes}
% ARM 11/21/17

% I've moved the Pigeonhole argument to beginning of problem above.  The
% following parts are independent of the Pigeonhole parts; the reader
% should be warned of this---even better, split them into separate
% problems.

% Suggested revision for Inclusion-Exclusion part below: Define $M_I$ as
% given in soln, and ask for

% ppart   size of $M_1$ 

% \begin{center}
% \exambox{0.5in}{0.5in}{0.0in}
% \end{center}

% \examspace[0.5in]


% ppart   size of $M_{1,2}$

% \begin{center}
% \exambox{0.5in}{0.5in}{0.0in}
% \end{center}

% \examspace[0.5in]

% then the following last part.
% \end{staffnotes}

% \ppart Write a simple arithmetic formula for the number of possible
% colorings of $R$ in which at least one row is \emph{monochromatic},
% that is, all squares are the same color.

% \begin{solution}
% For $I \subsetedq \Zintv{1}{3}$, let $M_I$ be the set of colorings
% where rows number $i$ is monochromatic for each $i \in I$.  For
% example, $M_{1,3}$ is the set of colorings whose first and third rows
% are monochromatic; the second row of a coloring in $M_{1,3}$ may or
% may not be monochromatic.  So $M_{\emptyset}$ is the set of all
% $2^{27}$ possible colorings.

% Now $M_1 \union M_2 \union M_3$ is the set of colorings with at least
% one monochromatic row.  By Inclusion-Exclusion
% \[
% \card{M_1 \union M_2 \union M_3} =
%          \card{M_1} + \card{M_2} + \card{M_3}
%        - \card{M_{12}} - \card{M_{13}} - \card{M_{23}}
%        + \card{M_{123}}.
% \]

% But $\card{M_i} = 2 \cdot 2^{18}$ since there are two ways to color
% row $i$ and $2^{18}$ ways to color the remaining squares.  Likewise, 
% $\card{M_{ij}} = 2 \cdot 2 \cdot 2^{9} = 2^{11}$, and $\card{M_{123}} = 2 \cdot
%   2 \cdot 2 = 8$.  So
% \[
% \card{M_1 \union M_2 \union M_3} = 3 \cdot  2^{19} - 3 \cdot  2^{11} + 2^3.
% \]
% \end{solution}

\eparts
\end{problem}

%%%%%%%%%%%%%%%%%%%%%%%%%%%%%%%%%%%%%%%%%%%%%%%%%%%%%%%%%%%%%%%%%%%%%
% Problem ends here
%%%%%%%%%%%%%%%%%%%%%%%%%%%%%%%%%%%%%%%%%%%%%%%%%%%%%%%%%%%%%%%%%%%%%

\endinput

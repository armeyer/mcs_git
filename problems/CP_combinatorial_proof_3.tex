\documentclass[problem]{mcs}

\begin{pcomments}
  \pcomment{CP_combinatorial_proof_3}
  \pcomment{suggested by Darij Grinberg 5/14/18}
\end{pcomments}

\pkeywords{
  combinatorial_proof
  binomial_coefficient
}

%%%%%%%%%%%%%%%%%%%%%%%%%%%%%%%%%%%%%%%%%%%%%%%%%%%%%%%%%%%%%%%%%%%%%
% Problem starts here
%%%%%%%%%%%%%%%%%%%%%%%%%%%%%%%%%%%%%%%%%%%%%%%%%%%%%%%%%%%%%%%%%%%%%

\begin{problem}
Give a combinatorial proof for this identity:
\[
\sum_{r=0}^n \binom{n}{r} \binom{m}{k-r} = \binom{n+m}{k}
\]

\begin{staffnotes}
\hint Theorem~\bref{th:comb-ex}
\end{staffnotes}

\begin{solution}
\begin{proof}
The proof is essentially the same as for Theorem~\bref{th:comb-ex}:

Let $S$ be all $k$-card hands that can be dealt from a deck containing
$n$ different red cards and $m$ different black cards.  So we know
\[
\card{S} = \binom{n+m}{k}.
\]

From another perspective, the number of $k$-card hands with exactly $r$ red
cards is
\[
\binom{n}{r} \binom{m}{k - r}
\]
because there are $\binom{n}{r}$ ways to choose the $r$ red cards and
$\binom{m}{k - r}$ ways to choose the remaining $k - r$ black cards.  Since the
number of red cards can be anywhere from 0 to $n$, the total number of
$n$-card hands is:
\[
    \card{S} = \sum_{r=0}^n \binom{n}{r} \binom{m}{k-r}.
\]
Equating these two expressions for $\card{S}$ proves the theorem.
\end{proof}
\end{solution}

\end{problem}

\endinput

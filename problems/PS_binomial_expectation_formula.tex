\documentclass[problem]{mcs}

\begin{pcomments}
  \pcomment{PS_binomial_expectation_formula}
  \pcomment{ARM 8/6/12 from yanked footnote}
\end{pcomments}

\pkeywords{
  expectation
  binomial_distribution
  differentiate
}

%%%%%%%%%%%%%%%%%%%%%%%%%%%%%%%%%%%%%%%%%%%%%%%%%%%%%%%%%%%%%%%%%%%%%
% Problem starts here
%%%%%%%%%%%%%%%%%%%%%%%%%%%%%%%%%%%%%%%%%%%%%%%%%%%%%%%%%%%%%%%%%%%%%

\begin{problem}
Applying linearity of expectation to the binomial distribution
$f_{n,p}$ immediately yielded the identity~\bref{binomial-expectsum}:
\begin{equation}\label{fnpsum}
  \expect{f_{n,p}} \eqdef \sum_{k = 0}^n k \binom{n}{k} p^k (1 -  p)^{n - k} = pn.
\end{equation}
Though it might seem daunting to prove this equation without appeal to
linearity, it is, after all, pretty similar to the binomial identity,
and this connection leads to an immediate alternative algebraic
derivation.

\bparts

\ppart Starting with the binomial identity for $(x+y)^n$, prove that
\begin{equation}\label{xnx+y}
xn(x + y)^{n-1} = \sum_{k = 0}^n k\binom{n}{k} x^{k} y^{n - k}.
\end{equation}

\begin{solution}
Differentiating both sides of the binomial identity
\[
(x + y)^n = \sum_{k = 0}^n \binom{n}{k} x^k y^{n - k},
\]
wrt to $x$ yields
\[
n(x + y)^{n-1} = \sum_{k = 0}^n k\binom{n}{k} x^{k-1} y^{n - k}.
\]
The multiplying both sides by $x$ gives~\eqref{xnx+y}.
\end{solution}

\ppart Now conclude equation~\eqref{fnpsum}.

\begin{solution}
Plugging $p$ for $x$ and $1-p$ for $y$ in~\eqref{xnx+y} yields~\eqref{fnpsum}.
\end{solution}

\eparts

\end{problem}

\endinput

\documentclass[problem]{mcs}

\begin{pcomments}
  \pcomment{MQ_divide_product_induction}
  \pcomment{tweak of TP_divide_product_induction}
  \pcomment{minor revision of part(b) of PS_induction_mod_proof}
  \pcomment{by ARM 10/8/13, edited 10/10/13}
\end{pcomments}

\pkeywords{
  divides
  prime
  induction
}

%%%%%%%%%%%%%%%%%%%%%%%%%%%%%%%%%%%%%%%%%%%%%%%%%%%%%%%%%%%%%%%%%%%%%
% Problem starts here
%%%%%%%%%%%%%%%%%%%%%%%%%%%%%%%%%%%%%%%%%%%%%%%%%%%%%%%%%%%%%%%%%%%%%

\begin{problem}
This problem asks for the ``routine induction'' that proves
\inbook{Lemma~\bref{lem:prime-divides-ind}---}the key lemma used to
prove unique factorization of integers:
\begin{lemma*}
If a prime divides a finite product of integers, then it divides one
of the integer factors.
\end{lemma*}
In proving it, you may assume the special case given in
Lemma~\bref{lem:prime-divides}\inhandout{ of the text}, namely, if a
prime divides a product of \emph{two} integers, then it divides one of
them.  You may \emph{not} assume unique factorization of integers.

\bparts

\ppart Reformulate the Lemma as a statement of the form $\forall n \in
\nngint.\, P(n)$ where $P(n)$ is suitable as an induction hypothesis
for a proof by induction.

\begin{center}
$\mathbf{\mathit{P(n)} \eqdef}$ \examboxplain{5.0in}{0.6in}{-0.3in}
 \end{center}

\begin{solution}
Let $P(n)$ be [if a prime divides a product of $n > 0$ integers, then it
  divides one of the them].

In more detail,
\[
P(n) \eqdef\ \forall a_1,a_2,\dots,a_n \in \integers.\,
  \forall p \in \text{primes}.\ p \divides a_1 a_2 \cdots
  a_n\ \QIMPLIES \exists i.\ p \divides a_i.
\]
\end{solution}

\ppart State and prove the Base Case(s) of your induction proof.
\iffalse (You may assume any of the previously proved facts.)\fi

\examspace[1.5in]

\begin{solution}
We proceed by ordinary induction on $n$.

\inductioncase{Base case}: ($n = 1$).  The case $n=1$ reduces to the
tautology ``$p \divides a_1 \QIMPLIES p \divides a_1$.
\end{solution}

\ppart Prove the induction step, carefully indicating exactly where
you use the fact that the divisor is prime.

\begin{solution}
\inductioncase{Inductive step}: Now we assume $P(n)$ holds for some $n
\geq 1$ and prove $P(n + 1)$.

Let $a \eqdef a_1 a_2 \cdots a_n a_{n+1}$ and suppose that $p \divides
a$.  We need to prove that $p \divides a_i$ for some $i \in
\Zintv{1}{n+1}$

Let $b \eqdef a_1 \cdots a_{n}$, so $a = ba_{n+1}$.  Now since $p
\divides ba_{n+1}$ and \textbf{$p$ is prime}, we know from
\inhandout{the text} Lemma~\bref{lem:prime-divides} that $p
\divides b$ or $p \divides a_{n+1}$.  If $p \divides a_{n+1}$, then
$P(n+1)$ follows immediately by letting $i = n+1$.  If $p \divides b$,
then the induction hypothesis $P(n)$ implies that $p \divides a_i$ for
some $i \in \Zintv{1}{n}$, which also implies $P(n+1)$.  So in either case,
$P(n+1)$ holds, which completes the inductive step.

By induction, the claim holds for all $n \geq 1$.
\end{solution}
  
\eparts
\end{problem}

%%%%%%%%%%%%%%%%%%%%%%%%%%%%%%%%%%%%%%%%%%%%%%%%%%%%%%%%%%%%%%%%%%%%%
% Problem ends here
%%%%%%%%%%%%%%%%%%%%%%%%%%%%%%%%%%%%%%%%%%%%%%%%%%%%%%%%%%%%%%%%%%%%%

\endinput

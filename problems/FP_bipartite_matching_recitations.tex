\documentclass[problem]{mcs}

\begin{pcomments}
  \pcomment{FP_bipartite_matching_recitations}
  \pcomment{from S090.MQmar19}
  \pcomment{slightly revised ARM 3/22/14}
\end{pcomments}

\pkeywords{
  graphs
  bipartite
  matching
  Halls_theorem
}

%%%%%%%%%%%%%%%%%%%%%%%%%%%%%%%%%%%%%%%%%%%%%%%%%%%%%%%%%%%%%%%%%%%%%
% Problem starts here
%%%%%%%%%%%%%%%%%%%%%%%%%%%%%%%%%%%%%%%%%%%%%%%%%%%%%%%%%%%%%%%%%%%%%

\begin{problem}

  Overworked and over-caffeinated, the Teaching Assistant's (TA's)
  decide to oust the lecturer and teach their own recitations.  They
  will run a recitation session at 4 different times in the same room.
  There are exactly 20 chairs to which a student can be assigned in
  each recitation.  Each student has provided the TA's with a list of
  the recitation sessions her schedule allows and each student's
  schedule conflicts with at most two sessions.  The TA's must assign
  each student to a chair during recitation at a time she can attend,
  if such an assignment is possible.

  \bparts

  \ppart Describe how to model this situation as a matching problem.
  Be sure to specify what the vertices/edges should be and briefly
  describe how a matching would determine seat assignments for each
  student in a recitation that does not conflict with his schedule.
  (This is a \emph{modeling problem}; we aren't looking for a description
  of an algorithm to solve the problem.)
%\begin{staffnotes}
%4pts
%\end{staffnotes}

 \begin{solution}
  There will be one vertex for each student, and 20 vertices for each
  recitation time slot (one for each chair).  There is an edge between
  a student and all chair vertices for a particular recitation time
  slot if that time slot does not conflict with her schedule.  A
  matching for the students assigns a student to a chair in a
  recitation that he can attend and assigns at most 20 students to
  any recitation.

  It is possible to assign the students to recitations iff a matching
  exists.
\end{solution}

\examspace[3in]

  \ppart Suppose there are 41 students.  Given the information provided
  above, is a matching guaranteed?  Briefly explain.

\begin{solution}
Not enough information.

It is easy to see that there could be a matching, if, for example, the
first 20 students could attend the first session, the second 20 could
attend the second session, and the 41st student could attend the third
recitation.

On the other hand, it is easy to see that there could be a bottleneck
and hence no matching.  For example, it is possible that all 41
students can only attend the first two sessions, in which case the
number of students exceeds the number of chairs to which they can be
assigned.
\end{solution}

\eparts

\end{problem}


%%%%%%%%%%%%%%%%%%%%%%%%%%%%%%%%%%%%%%%%%%%%%%%%%%%%%%%%%%%%%%%%%%%%%
% Problem ends here
%%%%%%%%%%%%%%%%%%%%%%%%%%%%%%%%%%%%%%%%%%%%%%%%%%%%%%%%%%%%%%%%%%%%%

\endinput

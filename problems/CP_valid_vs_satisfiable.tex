\documentclass[problem]{mcs}

\begin{pcomments}
  \pcomment{from: S09.cp2m}
  \pcomment{commented out in S09}
\end{pcomments}

\pkeywords{
  logic
  environment
  valid
  satisfiable
}

%%%%%%%%%%%%%%%%%%%%%%%%%%%%%%%%%%%%%%%%%%%%%%%%%%%%%%%%%%%%%%%%%%%%%
% Problem starts here
%%%%%%%%%%%%%%%%%%%%%%%%%%%%%%%%%%%%%%%%%%%%%%%%%%%%%%%%%%%%%%%%%%%%%

\begin{problem}
\bparts

\ppart A propositional formula is \term{valid} iff it is equivalent to
\true.  Verify by truth table that
\[
(P\ \QIMP\ Q)\ \QOR\ (Q \ \QIMP\ P)
\]
is valid.

\solution{
\[
\begin{array}{cc|c|c|c}
P & Q & P\ \QIMP\ Q & Q\ \QIMP\ P  & (P\ \QIMP\ Q)\ \QOR\ (Q \ \QIMP\ P)\\ \hline
\true & \true & \true & \true & \true\\
\true & \false & \false & \true & \true\\
\false & \true & \true & \false & \true\\
\false & \false & \true & \true & \true
\end{array}
\]
}

\ppart Let $P$ and $Q$ be propositional formulas.  Describe a single
propositional formula, $R$, involving $P$ and $Q$ such that $R$ is valid
iff $P$ and $Q$ are equivalent.

\solution{\[R \eqdef (P \ \QIFF\ Q)\] or
\[R \eqdef (P \ \QIMP\ Q)\ \QAND\ (Q \ \QIMP\ P)\]
or
\[R \eqdef (P\ \QAND\  Q)\ \QOR\ [\text{NOT($P$)}\ \QAND
\text{NOT($Q$)}]\]
are all possible solutions.
}

\ppart\label{sat} A propositional formula is \emph{satisfiable} iff there
is an assignment of truth values to its variables ---an \emph{environment}
---which makes it true.  Explain why
\begin{quote}
$P$ is valid iff NOT($P$) is \emph{not} satisfiable.
\end{quote}

\solution{
To prove the iff, we prove first, that the left hand statement implies the
right hand one, and second, vice-versa.

\textbf{(left-to-right case)}: If $P$ is valid, then NOT($P$) is \emph{not}
satisfiable.

\begin{proof}
Now $P$ is true in an environment iff NOT($P$) is false in that
environment.  Since $P$ is valid, it is true in every environment, which
means that NOT($P$) is false in every environment.  So no environment makes
NOT($P$) true, which means that NOT($P$) is \emph{not} satisfiable.

\end{proof}

\textbf{(right-to-left case)}: If NOT($P$) is \emph{not}
satisfiable, the $P$ is valid.

\begin{proof}
  Since NOT($P$) is not satisfiable, every truth assignment makes it
  false.  This implies that every truth assignment makes $P$ true, that
  is, $P$ is valid.

\end{proof}}

\ppart A set of propositional formulas $P_1,\dots,P_k$ is
\emph{consistent} iff there is an environment in which they are all
true.  Write a formula, $S$, so that the set $P_1,\dots,P_k$ is \emph{not}
consistent iff $S$ is valid.

\solution{Note that the set $P_1,\dots,P_k$ is consistent iff $(P_1 \text{
AND } P_2\ \QAND\ \dots\ \QAND\ P_k)$ is satisfiable.  So by
part~\eqref{sat}
\[
S \eqdef \text{ NOT }(P_1\ \QAND\ P_2\ \QAND\ \dots\ \QAND\ P_k)
\]
is the desired formula. In more concise notation this would written
\[
\QNOT\paren{\Land_{i=1}^k P_i}
\]
or, using DeMorgan's Law:
\[
\Lor_{i=1}^k \bar{P_i}.
\]
}

\eparts
\end{problem}

%%%%%%%%%%%%%%%%%%%%%%%%%%%%%%%%%%%%%%%%%%%%%%%%%%%%%%%%%%%%%%%%%%%%%
% Problem ends here
%%%%%%%%%%%%%%%%%%%%%%%%%%%%%%%%%%%%%%%%%%%%%%%%%%%%%%%%%%%%%%%%%%%%%

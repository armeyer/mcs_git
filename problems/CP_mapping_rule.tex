%CP_mapping_rule

\documentclass[problem]{mcs}

\begin{pcomments}
  \pcomment{written by ARM 9/20/09}
  \pcomment{problems need Appendix below or NOTES access}
\end{pcomments}

\pkeywords{
 mapping rule
 images
 relations
 functions
 injections
 surjections
}

%%%%%%%%%%%%%%%%%%%%%%%%%%%%%%%%%%%%%%%%%%%%%%%%%%%%%%%%%%%%%%%%%%%%%
% Problem starts here
%%%%%%%%%%%%%%%%%%%%%%%%%%%%%%%%%%%%%%%%%%%%%%%%%%%%%%%%%%%%%%%%%%%%%

\begin{problem}
  Let $R: A \to B$ be a binary relation.  Use an arrow counting argument
  to prove the following generalization of the Mapping Rule:
\begin{lemma*}
  If $R$ is a function, and $X \subseteq A$, then
\[
\card{X} \geq \card{XR}.
\]
\end{lemma*}

\begin{solution}
\begin{proof}
The proof is virtually a repeat of the proof in the Appendix for the first
Mapping Rule.

Since $R$ is a function, the number of arrows whose starting point is an
element of $X$ is at most the number of elements in $X$,  That is,
\[
\card{X} \geq \#\text{arrows from $X$}.
\]
Also, each element of $XR$ is, by definition, the endpoint of at least one
arrow starting from $X$, so there must be at least as many arrows starting
from $X$ as the number of elements of $XR$.  That is,
\[
\#\text{arrows from $X$} \geq \card{XR}.
\]
Combining these inequalities immediately implies that $\card{X} \geq \card{XR}$.
\end{proof}

An alternative proof appeals to the original Mapping Rule:

\begin{proof}
  Consider the relation $R'$ whose domain is $X$, whose codomain is $XR$,
  and whose arrows are just the arrows of $R$ that start from $X$. (These
  arrows necessarily end in $XR$ by definition of $XR$.)  Since $R$ is a
  function, $R'$ will be one too, and by definition of $XR$, the relation
  $R'$ is a surjection.  Hence the first Mapping Rule implies that
  $\card{X} \geq \card{XR}$.
\end{proof}

\end{solution}

\end{problem}

\iffalse

\Section{Appendix}

Let $R: A \to B$ be a binary relation.

For any subset $X \subseteq A$
\[
XR \eqdef \set{b \in B \suchthat \exist x \in X.\, x\mrel{R}b}
\]
In other words, $XR$ is the set of endpoints of arrows that start in $X$.

\begin{lemma}[Mapping Rule] \mbox{}
\begin{enumerate}

\item If $R$ is a surjective function, then
\[
\card{A} \geq \card{B}.
\]

\begin{proof}
  Since $R$ is a function, every arrow in the diagram for $R$ comes from
  exactly one element of $A$, so the number of arrows is at most the
  number of elements in $A$.  That is, since $R$ is a function,
\[
\card{A} \geq \#\text{arrows}.
\]
Similarly, since $R$ is surjective, every element of $B$ has at least one
arrow into it, so there must be at least as many arrows as the number of
elements of $B$.  That is, since $R$ is surjective,
\[
\#\text{arrows} \geq \card{B}.
\]
Combining these inequalities immediately implies that $\card{A} \geq \card{B}.$

\end{proof}

\item If $R$ is total and injective, then $\card{A} \leq \card{B}$.

\item If $R$ is a bijection, then $\card{A} = \card{B}$.

\end{enumerate}

\end{lemma}
\fi

\endinput


\documentclass[problem]{mcs}

\begin{pcomments}
  \pcomment{PS_little_oh_properties_arm}
  \pcomment{subsumes PS_little_oh_properties because of update to
    Lemma logxxe in asymptotics chapter}
  \pcomment{ARM 4/25/18}
\end{pcomments}

\pkeywords{
  little_oh
  asymptotics
  partial_order
  calculus
}

%%%%%%%%%%%%%%%%%%%%%%%%%%%%%%%%%%%%%%%%%%%%%%%%%%%%%%%%%%%%%%%%%%%%%
% Problem starts here
%%%%%%%%%%%%%%%%%%%%%%%%%%%%%%%%%%%%%%%%%%%%%%%%%%%%%%%%%%%%%%%%%%%%%

\begin{problem}
\bparts

\ppart
Prove that the relation $R$ on positive functions such that $f\mrel{R}g$ iff $g = o(f)$
is a strict partial order.

\begin{solution}
We must show that $R$ is transitive and irreflexive.  To show that $R$
is irreflexive, for any positive function $f$ we have
\[
\lim_{x\to \infty} f(x)/f(x) = \lim_{x\to\infty} 1 = 1,
\]
so in particular this limit cannot equal $0$.  This means $f = o(f)$
does not hold for any $f$.

For transitivity, assume $f \mrel{R} g$ and $g\mrel{R} h$, that is, $h =
o(g)$ and $g = o(f)$; we must prove that $h = o(f)$.  From the
definition of little-o,
  \begin{equation*}
    \lim_{x\to\infty} \frac{g(x)}{f(x)} = \lim_{x\to\infty} \frac{h(x)}{g(x)} = 0.
  \end{equation*}
We may now compute that
  \begin{align}
\lefteqn{\lim_{x\to\infty} \frac{h(x)}{f(x)}}\notag\\
    & = \lim_{x\to\infty} \frac{h(x)}{g(x)}\cdot\frac{g(x)}{f(x)} \notag\\
    & = \paren{\lim_{x\to\infty} \frac{h(x)}{g(x)}}
        \cdot \lim_{x\to\infty} \frac{g(x)}{f(x)}\tag{*}\\ %\label{limit-product-split}
    & = 0 \cdot 0 = 0.\notag
  \end{align}
Splitting the limit into the product of two limits in~(*) %line~\eqref{limit-product-split}
is justified because the latter two limits are both finite.  This
shows that $h = o(f)$, that is, $f\mrel{R} h$, as desired.
\end{solution}

\ppart If $g$ is a positive function, prove than $f \sim g$ iff $f = g
+ h$ for some function $h = o(g)$.

\begin{solution}
The statement $f\sim g$ means
\[
    \lim_{x\to\infty} \frac{f(x)}{g(x)} = 1.
\]
By subtracting $1$ from each side, this is equivalent to
\[
    \lim_{x\to\infty} \frac{f(x)-g(x)}{g(x)} = 0,
\]
which in turn is precisely the definition of $f-g = o(g)$.  So $f\sim
g$ is true \emph{if and only if} the function $h \eqdef f-g$ is
$o(g)$, as desired.

\emph{Note}: In this proof, we were careful to indicate that every
logical step was an \emph{equivalence} (as in $\QIFF$), not just an
implication.  If instead we said something like ``this implies''
instead of ``this is equivalent to,'' we would have ended up proving
$f\sim g \QIMPLIES f-g = o(g)$, which is only half of the required
claim.
\end{solution}

\eparts

\end{problem}

%%%%%%%%%%%%%%%%%%%%%%%%%%%%%%%%%%%%%%%%%%%%%%%%%%%%%%%%%%%%%%%%%%%%%
% Problem ends here
%%%%%%%%%%%%%%%%%%%%%%%%%%%%%%%%%%%%%%%%%%%%%%%%%%%%%%%%%%%%%%%%%%%%%


\endinput

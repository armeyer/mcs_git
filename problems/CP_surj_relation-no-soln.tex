%CP_surj_relation-n-soln

\documentclass[problem]{mcs}

\begin{pcomments}
  \pcomment{from: excerpted from CP_relational_properties_table
            do not use with that problem}
  \pcomment{this version has solns copmmented out for reuse as PS}
\end{pcomments}

\pkeywords{
  relations
  functions
  injections
  surjections
}

%%%%%%%%%%%%%%%%%%%%%%%%%%%%%%%%%%%%%%%%%%%%%%%%%%%%%%%%%%%%%%%%%%%%%
% Problem starts here
%%%%%%%%%%%%%%%%%%%%%%%%%%%%%%%%%%%%%%%%%%%%%%%%%%%%%%%%%%%%%%%%%%%%%

\newcommand{\surj}{\mrelt{surj}}
\newcommand{\inj}{\mrelt{inj}}

\begin{problem}
The following problem has been carried over to Pset 2; the solution will
appear there.

Define a \term{surjection relation}, $\surj$\footnote{This is the same as the
  ``\emph{as big as}'' relation defined in the Notes, but we're giving it
  a less suggestive name to avoid assumptions about properties it may have
  because of it name.}, on sets by the rule
\begin{definition*}
  $A \surj B$ iff there is a surjective function from $A$ to $B$.
\end{definition*}
Define the \term{injection relation}, $\inj$, on sets by the rule
\begin{definition*}
  $A \inj B$ iff there is a total injective relation from $A$ to $B$.
\end{definition*}

\bparts

\ppart\label{surjinj} Explain why $A \surj B \qiff B \inj A$.

\iffalse

\begin{solution}
\begin{proof}

(right to left): By definition of $\inj$, there is a total injective
  relation, $R:B \to A$.  By problem~\ref{TP_inverse_relation_table}, this
  implies that $\inv{R}$ is a surjective function from $A$ to $B$.

(left to right): By definition of $\surj$, there is a surjective function,
$F:A \to B$.
By problem~\ref{TP_inverse_relation_table}, this
 implies that $\inv{F}$ is a total injective relation from $A$ to $B$.
\end{proof}
\end{solution}
\fi

\ppart\label{surjsurj} Prove that if $A \surj B$ and $B \surj C$, then $A \surj C$.
\iffalse

\begin{solution}
By definition of $\surj$, there area surjective functions,
$F:A \to B$ and $G:B \to C$.

Let $H \eqdef G \compose F$ be the function equal to the composition of
$G$ and $F$, that is
\[
H(a) \eqdef G(F(a)).
\]
We show that $H$ is surjective, which will complete the proof.  So suppose
$c \in C$.  Then since $G$ is a surjection, $c = G(b)$ for some $b \in B$.
Likewise, $b = F(a)$ for some $a \in A$.  Hence $c = G(F(a)) = H(a)$,
proving that $c$ is in the range of $H$, as required.
\end{solution}
\fi

\ppart Conclude from~\eqref{surjinj} and~\eqref{surjsurj} that if $A \inj
B$ and $B \inj C$, then $A \inj C$.
\iffalse

\begin{solution}
From~\eqref{surjinj} and~\eqref{surjsurj} we have that if $C \inj B$ and
$B \inj A$, then $C \inj A$, so just switch the names $A$ and $C$.
\end{solution}
\fi

\eparts
\end{problem}


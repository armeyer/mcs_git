%final S15

\documentclass[quiz]{mcs}

%\renewcommand{\examspace}{}

\renewcommand{\exampreamble}{   % !! renew \exampreamble

\begin{center}
{\large   \textbf{Circle your}\qquad   \teaminfo}
\end{center}

  \begin{itemize}

  \item
   This exam is \textbf{closed book} except for two 2-sided cribsheets.
   Total time is 180 minutes.

  \item

   Write your solutions in the space provided.  If you need more
   space, \textbf{write on the back} of the sheet containing the
   problem.

%   Please keep your entire answer to a problem on that problem's page.   
   \item In answering the following questions, you may use without
     proof any of the results from class or text.

   \item Incorrect short answers are eligible for part credit when
     there is an explanation.

\iffalse
  \item
   GOOD LUCK!
\fi

  \end{itemize}}

\begin{document}

\final

%%%%%%%%%%%%%%%%%%%%%%%%%%%%%%%%%%%%%%%%%%%%%%%%%%%%%%%%%%%%%%%%%%%%%
% Problems start here
%%%%%%%%%%%%%%%%%%%%%%%%%%%%%%%%%%%%%%%%%%%%%%%%%%%%%%%%%%%%%%%%%%%%

%\pinput[points = 6, title= \textbf{Set Formulas}]{FP_basic_set_formulas}


\pinput[points = 6, title= \textbf{Probable Satisfiability)}]{FP_satisfy_implies_probability_alt}

%\pinput[points = 10, title= \textbf{Induction, ADAM}]{FP_prove_strong_induction}
    %will confuse students on an exam

%\examspace
%\pinput[points = 10, title= \textbf{induction, TASHA}]{CP_3_and_7_cent_stamps_by_induction}

%\examspace
%\pinput[points = 8, title= \textbf{Induction, Recurrence}]{FP_recurrence_strong_induction}
   %, Congruence, ARM

\examspace
\pinput[points = 10, title= \textbf{Induction, Trees}]{FP_graph_width_oneS15b}
      %, MISHA
     % Material covered elsewhere, & proofs a bit tricky to grade, probably. -AC

%\examspace
%\pinput[points = 5, title= \textbf{State Machines}]{MQ_red_blue_machine}
  %, TASHA/ARM
% Proof part seems tricky to grade because of unclear target level of rigor. -AC

%\examspace
%\pinput[points = 10, title= \textbf{state machine, TASHA}]{FP_divide_using_3}

\examspace
\pinput[points = 8, title= \textbf{Number Theory}]{FP_number_short}
   %ADAM

%\examspace
%\pinput[points = 6, title= \textbf{Number Theory, Probability}]{TP_relative_prime_probability}
   %ARM NEEDS CHECKING & SOLN
% Save on generating full solution? :-) -AC

%\examspace
%\pinput[points = 10, title= \textbf{digraph, TASHA}]{FP_digraph_triangle_inequality}
   %boring, picky

\examspace
\pinput[points = 8, title= \textbf{Scheduling \& DAGs}]{FP_chains_scheduling}
    % , TASHA/ARM

%\examspace
%\pinput[points = 10, title= \textbf{poset schedule, TASHA}]{CP_conquering_the_galaxy}
   %too long for final

\examspace
\pinput[points = 8, title= \textbf{Simple Graphs}]{FP_degree_sequencesS15}
   %ADAM
   % Probably don't want to use this one _and_ next one. -AC

%\examspace
%\pinput[points = 8, title= \textbf{Graph Coloring}]{FP_chromatic_union}
  % ADAM GOOD PROB, use for conflict
% Or use for main exam?

%\examspace
%\pinput[points = 10, title= \textbf{simple graph, BEN}]{CP_Handshaking_Lemma}
    %anything from class can be assumed

%\examspace
%\pinput[points = 10, title= \textbf{3-Coloring}]{FP_3color_XOR}
  %BEN/ARM;  Hard to grade? -AC


%\examspace
%\pinput[points = 10, title= \textbf{Induction, Graphs}]{FP_cycles_components_induction}
   %, ARM

%\examspace
%\pinput[points = 8, title= \textbf{Stable Marriage (tweak invars)}]{FP_Stable_Marriage_Invariants}
    %ADAM; TP_Stable_Marriage_Invariants, TP_mating_ritual_invariant, MISHA/ARM

%\examspace
%\pinput[points = 10, title= \textbf{Asymptotics}]{FP_Oh_not_Theta}
    %ADAM
% Too intricate, especially for 2nd part? -AC

%\examspace
%\pinput[points = 10, title= \textbf{geometric sum}]{CP_neat_trick_for_geometric_sum}
    % MISHA, maybe cut intro?
% A probability problem below now covers summations. -AC

%\examspace
%\pinput[points = 10, title= \textbf{harmonic nums, ANNIE (revise)}]{PS_bug_on_rug_harmonic_number}

%\examspace
%\pinput[points = 10, title= \textbf{Graphs, Counting}]{PS_3_friends}
  %, ANNIE/ARM  USED S14.

\examspace
\pinput[points = 6, title= \textbf{Big Oh}]{TP_asymptotics_define_functions}

\examspace
\pinput[points = 10, title= \textbf{Counting}]{FP_counting_poker_high_cardsS15}
  %, ADAM

%\examspace
%\pinput[points = 10, title= \textbf{Pigeonholes, Inclusion-Exclusion}]{FP_monochromatic_rectangles}
  %, ADAM

\examspace
\pinput[points = 10, title= \textbf{Conditional Probability}]{FP_monty_hall_variantS15}
   %, ADAM

%\examspace
%\pinput[points = 10, title= \textbf{Conditional Probability}]{MQ_conditional_prob_inequality}
  %, ARM
% Last question more interesting? -AC

%\examspace
%\pinput[points = 10, title= \textbf{conditional prob, MISHA-BEN}]{PS_conditional_aces}
   %too hard for exam

%\examspace
%\pinput[points = 10, title= \textbf{conditional prob, MISHA-BEN}]{CP_conditional_prob_says_so_bug}
   %too easy--everyone in class got it 

%\examspace
%\pinput[points = 10, title= \textbf{Conditional probability}]{FP_conditional_beaver_fever}
 %, MISHA-BEN

%\examspace
%\pinput[points = 10, title= \textbf{Expectation}]{FP_consecutive_coin_flips}
  %, TASHA/ARM
% Overlaps too much with Monty Hall? -AC

\examspace
\pinput[points = 6, title= \textbf{Expectation}]{FP_random_graphsS15}
   %ADAM, ARM

%\examspace
%\pinput[points = 10, title= \textbf{Expectation)}]{FP_class_expectation}
% Formulas too intricate? -AC

\examspace
\pinput[points = 10, title= \textbf{Variance, Sums)}]{FP_variance_dice_sum}
  %, ADAM
% Also covers summations! -AC

%\examspace
%\pinput[points = 10, title= \textbf{Markov's Bound}]{FP_hot_cows_markov}
  %, TASHA/ARM

%\examspace
%\pinput[points = 10, title= \textbf{Chebyshev's Bound}]{FP_hot_cows_chebyshevS15}
  %ARM
% Solution needs revising if we're going to use this one. -AC

\examspace
\pinput[points = 12, title= \textbf{Markov \& Chebyshev}]
  {FP_gambling_manS15}
  %, TASHA
% Previous question more interesting? -AC

%\examspace
%\pinput[points = 10, title= \textbf{Sampling}]{FP_sampling_concepts}
  %, ADAM
% Too much reading? -AC

%\examspace
%\pinput[points = 10, title= \textbf{Random Walks}]{FP_uniform_stationary_distribution}
  %, ADAM

\examspace
\pinput[points = 6, title= \textbf{Random Walks}]{FP_random_walk_examples}
    %ARM

%%%%%%%%%%%%%%%%%%%%%%%%%%%%%%%%%%%%%%%%%%%%%%%%%%%%%%%%%%%%%%%%%%%%%
% Problems end here
%%%%%%%%%%%%%%%%%%%%%%%%%%%%%%%%%%%%%%%%%%%%%%%%%%%%%%%%%%%%%%%%%%%%%
\end{document}

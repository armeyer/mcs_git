\documentclass[problem]{mcs}

\begin{pcomments}
  \pcomment{from: S09.ps7}
\end{pcomments}

\pkeywords{
  number_theory
  linear_combinations
  gcd
  state_machines
  fun_game
}

%%%%%%%%%%%%%%%%%%%%%%%%%%%%%%%%%%%%%%%%%%%%%%%%%%%%%%%%%%%%%%%%%%%%%
% Problem starts here
%%%%%%%%%%%%%%%%%%%%%%%%%%%%%%%%%%%%%%%%%%%%%%%%%%%%%%%%%%%%%%%%%%%%%

\begin{problem}
Here is a {\em very, very fun} game.  We start with two distinct,
positive integers written on a blackboard.  Call them $a$ and $b$.
You and I now take turns.  (I'll let you decide who goes first.)  On
each player's turn, he or she must write a new positive integer on the
board that is the difference of two numbers that are already there.
If a player can not play, then he or she loses.

For example, suppose that 12 and 15 are on the board initially.  Your
first play must be 3, which is $15 - 12$.  Then I might play 9, which
is $12 - 3$.  Then you might play $6$, which is $15 - 9$.  Then I can
not play, so I lose.

\bparts

\ppart Show that every number on the board at the end of the game is a
multiple of $\gcd(a, b)$.

\begin{solution}

Thinking of the game as a state machine, we observe that the property that
$\gcd(a,b)$ divides all the numbers on the board is an invariant.
This follows because the next state (board) is the same as the previous
state, except for an additional number which is the difference of two
numbers already there.  Assuming these two numbers are divisible by
$\gcd(a,b)$, we know that their difference will be as well, which proves
that the next state satisfies the invariant.

\end{solution}

\ppart Show that every positive multiple of $\gcd(a, b)$ up to $\max(a,
b)$ is on the board at the end of the game.

\begin{solution}
 Assume without loss of generality that $a > b$.  Let $s$ be the
smallest number on the board at the end of the game.  So $a = q s + r$
where $0 \leq r < s$ by the division algorithm.  Then $a - s$ must be on
the board and thus so must $a - 2s$, $a - 3s$, \ldots, $a - (q-1)s$.
However, $r = a - q s$ cannot be on the board, since $r < s$ and $s$ is
defined to be the smallest number there.  The only explanation is that $r
= 0$, which implies that $s \mid a$.  By the same argument, $s \mid b$.
Therefore, $s$ is a common divisor of $a$ and $b$.  Since $s$ is a
multiple of the greatest common divisor of $a$ and $b$ by the preceding
problem part, $s$ must actually be the greatest common divisor.  We
already argued that $a$, $a - s$, $a - 2s$, \ldots, $a - (q-1)s$ must be
on the board, and these are all the positive multiples of $\gcd(a, b)$ up
to $\max(a, b)$.

\end{solution}

\ppart Describe a strategy that lets you win this game every time.

\begin{solution}
Assume without loss of generality that $a = \max(a,b)$.  By the
previous parts, the numbers that appear on the final board are precisely
all the multiples $\leq a$ of $gcd(a,b)$.  Thus, for each game, we know
\emph{exactly} how many values will be placed on the board before the game
ends.  So if an odd number of values will appear on the final board (which
happens precisely when $a$ is an even multiple of $\gcd(a,b)$), then choose
to go first.  
\end{solution}

\eparts

\end{problem}

%%%%%%%%%%%%%%%%%%%%%%%%%%%%%%%%%%%%%%%%%%%%%%%%%%%%%%%%%%%%%%%%%%%%%
% Problem ends here
%%%%%%%%%%%%%%%%%%%%%%%%%%%%%%%%%%%%%%%%%%%%%%%%%%%%%%%%%%%%%%%%%%%%%

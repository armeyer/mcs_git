\documentclass[problem]{mcs}

\begin{pcomments}
  \pcomment{TP_divisibility_and_congruence_conflict}
  \pcomment{adapted by Emanuele from TP_Divisibility_and_Congruence}
  \pcomment{S16.mid3conflict}
\end{pcomments}

\pkeywords{
  divides
  multiple
  modulo
  congruence
}

\begin{problem}
Assume that
\[
a \equiv  b \pmod n,
\]
where $n > 1$ and $a$ and $b$ are integers.
      
\begin{staffnotes}
5 points for each part.
\end{staffnotes}

\bparts

\ppart 
For what positive integer value(s) of $k$ does $2a \equiv 2b \pmod {kn}$ hold?
\examspace[1.0in]

\begin{solution}
Only for $k=1,2$ are the only values that work for all $a,b,n$.
\end{solution}

\ppart For what positive integer value(s) of $k$ does $a^k \equiv b^k
\pmod n$ hold?  For what values are the two statements equivalent
(remember that for the two statements to be equivalent both directions
of implications must hold)?
\examspace[1.0in]

\begin{solution}
The statement holds for all positive integer $k$, but the two
statements are equivalent only in the trivial case where $k=1$.
\end{solution}

\ppart  Is this equivalent to saying that $\gcd(a,n) = \gcd(b,n)$?
Either show the equivalence or give a counterexample.
\examspace[3.0in]

\begin{solution}
\textbf{Not equivalent.}

There is a one-way implication: if $a \equiv b \pmod n$, then the
gcd's are equal, but the converse fails.  For example,
\[
\gcd(3,11) = \gcd(7,11) = 1 \quad \text{but} \quad 3 \not\equiv 7 \pmod {11}.
\]

\end{solution}

\eparts

\end{problem}

\endinput

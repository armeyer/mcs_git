%PS_brents_theorem

\documentclass[problem]{mcs}

\begin{pcomments}
  \pcomment{from: F06 pset5Back.tex, maybe same as F04 pset5}
\end{pcomments}

\pkeywords{
  partial_orders
  chains
  antichain
}

%%%%%%%%%%%%%%%%%%%%%%%%%%%%%%%%%%%%%%%%%%%%%%%%%%%%%%%%%%%%%%%%%%%%%
% Problem starts here
%%%%%%%%%%%%%%%%%%%%%%%%%%%%%%%%%%%%%%%%%%%%%%%%%%%%%%%%%%%%%%%%%%%%%

\begin{problem}
We consider DAG's where each vertex represents a task to be completed.  If
there is a path from one vertex, $v$, to another vertex, $w$, then the $v$
task must be completed before the $w$ task.  Assuming all tasks take unit
time to complete, we showed in the Notes that the minimum time schedule to
complete all the tasks is the size (number of vertices), $t$, of the
longest path (chain) in the DAG.

Formally, a \emph{schedule} for a DAG is a partition of the vertices.
Each block of the partition is supposed to correspond to a set of tasks
that are to be performed simultaneously.  The \emph{number of processors
required} by a schedule is the maximum number of tasks that are scheduled
to be performed simultaneously.
\bparts

\ppart Describe purely in terms of graph, partition, and partial order
properties (no informal descriptions in terms of ``jobs,'' ``parallel
processing,'' etc.):

\begin{itemize}
\item exactly the properties a vertex partition of a DAG must satisfy in
order to represent a possible schedule for the vertex tasks,
\item the total time required to complete a schedule,
\item the number of processors required by a schedule.

\end{itemize}

\begin{solution}

\begin{itemize}

\item A schedule for a DAG, $G$, is a partition of the edges of $G$ into a
sequence of blocks, $B_1,B_2,\dots, B_k$ such that if $a \in B_i$, $b\in
B_j$, and $a < b$ (that is, there is a path of positive length from vertex
$a$ to vertex $b$), then $i < j$.  Another way to say this is that the
blocks are anti-chains, and the sequence consisting of the elements in
$B_1$ in any order, followed by the elements of $B_2$ in any order,
through the elements of $B_k$, is a topological sort of the partial order
defined by $G$.

\item The total time required to complete a schedule is the number, $k$,
of blocks it has.

\item The number of processors required by a schedule is the size of the
largest block.
\end{itemize}

\end{solution}

\ppart Give a small example of a DAG with more than one minimum time
schedule.

\begin{solution}

$V= \set{1,2,3}, E= \set{\diredge{1}{2}}$.  There are two minimum time
schedules: $\set{\set{1,3}\set{2}}$ and $\set{\set{1}\set{2,3}}$.

\end{solution}

\ppart Explain why any schedule that requires only $p$ processors to
complete $n$ tasks must take time at least $\ceil{n/p}$.

\begin{solution}
 If there are $k < \ceil{n/p}$, then the integer $k$ is less
than $n/p$.  
\end{solution}

\ppart\label{timeD} Let $D_{n,t}$ be the DAG with $n$ vertices that
consists of a directed path of $t-1$ vertices ending with edges from the
final, $(t-1)$st, vertex on the path directly to each of the remaining
$n-(t-1)$ vertices, as in the following figure:

\mfigure{!}{2.5in}{pset5-hasse}

What is the minimum time schedule for $D_{n,t}$?  Explain why it is
unique.  How many processors does it require?

\begin{solution}
There's no choice but to schedule each of the $t-1$ vertices on
the path one at a time in order.  A minimum time schedule then does all
the remaining $n-(t-1)$ vertices at the $t$th time interval.  The number
of processors required is therefore $n-t+1$.  
\end{solution}

\ppart Describe a minimum time $p$-processor schedule for $D_{n,t}$.
Write a simple formula for this minimum time, $M(n,t,p)$.

\begin{solution}
As in part~\eqref{timeD}, there's no choice but to schedule each
of the $t-1$ vertices on the path one at a time in order.  A minimum time
schedule then does all the remaining $n-(t-1)$ vertices $p$ at a time, for
a total time of
\begin{equation}\label{tnp}
M(n,t,p) \eqdef (t-1) + \ceil{\frac{n-(t-1)}{p}}.
\end{equation}

\end{solution}

\ppart Show that \emph{every} DAG with $n$ vertices and maximum chain
size, $t$, has a $p$-processor schedule that runs in time $M(n,t,p)$.

\hint Induction -- you decide on what variable.  You may find it
helpful to use the fact that if $a\geq b\geq 0$, then
\begin{equation}\label{cab}
\ceil{a-b} \leq 1 + \ceil{a} - \ceil{b}
\end{equation}
for all real numbers $a,b$.

\begin{solution}

\begin{proof}
%There has to be a simpler proof than this mess of calculations with
%fractions --ARM 10/3/09

Induction on $t$.  Induction hypothesis:
\begin{quote}
$P(t) \eqdef \forall \text{ DAGs } G, \forall n,p \in \naturals^+$, if $G$
has $n$ vertices and maximum chain size $t$, then there is a $p$-processor
schedule for $G$ that takes time $M(n,t,p)$.
\end{quote}

\textbf{Base case $t=1$:} In this case there are $n$ vertices and no edges
between them.  So any partition of the vertices into $\ceil{n/p}$ blocks
of size at most $p$ will be a $p$-processor schedule taking time
$\ceil{n/p} = 0 + \ceil{(n-0)/p} = M(n,1,p)$.

\textbf{Inductive step:}
Assume $P(t)$ and conclude $P(t+1)$ where $t \geq 1$.

Let $G$ be any DAG with $n$ vertices and maximum chain size $t+1$.
Suppose $k$ vertices are endpoints of maximum-size chains in $G$.  Note
that no edge can leave any of these endpoint vertices, for otherwise there
would be a chain of length one more than the maximum chain size.  Let $H$
be the subgraph of $G$ obtained by removing these $k$ vertices.

Now $H$ is a DAG with $n-k$ vertices and maximum chain size $t$, so by
Induction Hypothesis, there is a $p$-processor schedule for $H$ taking
time $M(n-k,t,p)$.

This $p$-processor schedule for $H$ can be extended to one for $G$ by
adding $\ceil{k/p}$ disjoint blocks of the endpoints, all of size $\leq p$.
So the time for this schedule for $G$ is
\begin{align}
\lefteqn{M(n-k,t,p) + \ceil{\frac{k}{p}}}\notag\\
  & = (t-1) + \ceil{\frac{n-k-(t-1)}{p}} + \ceil{\frac{k}{p}} & \text{(def of $M$)}\notag\\
  & = (t-1) + \ceil{\frac{n-t}{p} - \frac{k-1}{p}} + \ceil{\frac{k}{p}}
\label{t1}
\end{align}

We complete the proof by showing that the expression~\eqref{t1} is $\leq
M(n, t+1, p)$.  To do this, we consider two cases:

\begin{itemize}

\item \textbf{Case 1} ($k-1$ is not a multiple of $p$):
We have
\begin{equation}\label{k1}
\ceil{\frac{k-1}{p}} = \ceil{\frac{k}{p}},
\end{equation}
so
\begin{align*}
\text{\eqref{t1}}
   & \leq (t-1) + \left(1+ \ceil{\frac{n-t}{p}} -
   \ceil{\frac{k-1}{p}}\right) + \ceil{\frac{k}{p}} & \text{(by~\eqref{cab})}\\
   & = (t-1) + \left(1+ \ceil{\frac{n-t}{p}} -
      \ceil{\frac{k}{p}}\right) + \ceil{\frac{k}{p}} & \text{(by~\eqref{k1})}\\
   & = t + \ceil{\frac{n-t}{p}}\\
   & = M(n, t+1, p). & \text{(def of $M$)}
\end{align*}

\item \textbf{Case 2} ($k-1$ is a multiple of $p$): 
Now we have
\begin{equation}
\ceil{\frac{k}{p}}=1+\frac{k-1}{p},\label{k+}
\end{equation}
so
\begin{align*}
\text{\eqref{t1}}
  & = (t-1) + \left(\ceil{\frac{n-t}{p}} - \frac{k-1}{p}\right) +
       \ceil{\frac{k}{p}} & \text{(since $(k-1)/p \in \integers$)}\\
  & = (t-1) + \ceil{\frac{n-t}{p}} - \frac{k-1}{p} +
  \left(1+\frac{k-1}{p}\right) & \text{(by~\eqref{k+})}\\
   & = t + \ceil{\frac{n-t}{p}}\\
   & = M(n, t+1, p). & \text{(def of $M$)}
\end{align*}

\end{itemize}
\end{proof}

\end{solution}

\eparts

\end{problem}

%%%%%%%%%%%%%%%%%%%%%%%%%%%%%%%%%%%%%%%%%%%%%%%%%%%%%%%%%%%%%%%%%%%%%
% Problem ends here
%%%%%%%%%%%%%%%%%%%%%%%%%%%%%%%%%%%%%%%%%%%%%%%%%%%%%%%%%%%%%%%%%%%%%

\endinput



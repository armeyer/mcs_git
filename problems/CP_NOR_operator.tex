\documentclass[problem]{mcs}

\begin{pcomments}
    \pcomment{CP_NOR_operator}
    \pcomment{variation of TP_only_and_or_not}
    \pcomment{ARM 2/7/13}
\end{pcomments}

\pkeywords{
  propositional formula
  logical_connectives
  disjunctive_form
}

\begin{problem}
The propositional connective \QNOR\ is defined by the rule
\[
P \QNOR Q\ \eqdef\ (\QNOT(P) \QAND \QNOT(Q)).
\]
Explain why every propositional formula---possibly involving any of
the usual operators such as $\QIMPLIES$, $\QXOR$, \dots---is equivalent
to one whose only connective is $\QNOR$.

\begin{staffnotes}
\hint Show how to express \QAND\ and \QNOT\ solely in terms or \QNOR.

With this hint, the problem becomes TP\_ rather than CP\_ level.

Also refer to Problem~\bref{TP_only_and_or_not} for detailed version
that does not depend on disjuctive form.
\end{staffnotes}

\begin{solution}
Section~\bref{normal_form_sec} explains why every formula is
equivalent to one in \idx{disjunctive form}---which only involves
$\QAND$, $\QOR$ and $\QNOT$.  \idx{DeMorgan's Law}~(\bref{DeMQAND})
shows how to express $\QAND$ in terms of $\QOR$ and $\QNOT$, so all
that's needed is a way to express $\QOR$ and $\QNOT$ in terms of
$\QNOR$:
\begin{align*}
\QNOT(P) & \text{ is equivalent to } (P \QNOR P)\\
P \QOR Q & \text{ is equivalent to } \QNOT(P \QNOR Q).
\end{align*}
\end{solution}

\end{problem}

\endinput


\documentclass[problem]{mcs}

\begin{pcomments}
  \pcomment{CP_irrational_raised_to_an_irrational}
  \pcomment{from: S09.cp1r}
%  \pcomment{}
%  \pcomment{}
\end{pcomments}

\pkeywords{
  square_root_of_2
  irrational
  rational
  proof_by_cases
}

%%%%%%%%%%%%%%%%%%%%%%%%%%%%%%%%%%%%%%%%%%%%%%%%%%%%%%%%%%%%%%%%%%%%%
% Problem starts here
%%%%%%%%%%%%%%%%%%%%%%%%%%%%%%%%%%%%%%%%%%%%%%%%%%%%%%%%%%%%%%%%%%%%%

\begin{problem}
If we raise an irrational number to an irrational power, can the result be
rational?  Show that it can by considering $\sqrt{2}^{\sqrt{2}}$ and
arguing by cases.

\begin{solution}
We want to find irrational numbers $a,b$ such that $a^b$ is rational.
We argue by cases.

Case 1: [$\sqrt{2}^{\sqrt{2}}$ is rational].  Let $a = b = \sqrt{2}$.  Now $a$
and $b$ are irrational, since $\sqrt{2}$ is irrational as we know.  Also,
$a^b$ is rational by case hypothesis.  So we have found the required $a$
and $b$ in this case.

Case 2: [$\sqrt{2}^{\sqrt{2}}$ is irrational].  Let $a =\sqrt{2}^{\sqrt{2}}$
and $b = \sqrt{2}$.  Then $a$ is irrational by case hypothesis, we know
$b$ is irrational, and
\[
a^b = \paren{\sqrt{2}^{\sqrt{2}}}^{\sqrt{2}} =
      \sqrt{2}^{\sqrt{2} \cdot \sqrt{2}} = \sqrt{2}^2 = 2,
\]
which is rational.  So we have found the required $a$
and $b$ in this case also.

So in any case, there will be irrational $a,b$ such that $a^b$ is
rational.  Note that we have no clue about which case is
true,\footnote{A number which is a root of some polynomial with integer
  coefficients is called an \term{algebraic number}.  A number that is
  not algebraic is called \term{transcendental}.

  The number $\sqrt{2}^{\sqrt{2}}$, known as the Gelfond-Schneider
  constant, was proven to be transcendental by Rodion Kuzmin in
  1930. In 1934, Aleksandr Gelfond~\cite{Gelfond34} proved the more
  general Gelfond-Schneider theorem, which implies that if $a$ and $b$
  are algebraic real numbers with $a \neq 0,1$ and $b$ irrational,
  then $a^b$ is a transcendental number.} but that didn't matter.
\end{solution}

\end{problem}

%%%%%%%%%%%%%%%%%%%%%%%%%%%%%%%%%%%%%%%%%%%%%%%%%%%%%%%%%%%%%%%%%%%%%
% Problem ends here
%%%%%%%%%%%%%%%%%%%%%%%%%%%%%%%%%%%%%%%%%%%%%%%%%%%%%%%%%%%%%%%%%%%%%

\endinput

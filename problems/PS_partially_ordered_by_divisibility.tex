\documentclass[problem]{mcs}

\begin{pcomments}
  \pcomment{from: F03.ps2}
\end{pcomments}

\pkeywords{
  partial_ordered_by_divisibility
  unique_minimal_and_maximal_elements
  infinite_chain
  infinite_antichain
}

%%%%%%%%%%%%%%%%%%%%%%%%%%%%%%%%%%%%%%%%%%%%%%%%%%%%%%%%%%%%%%%%%%%%%
% Problem starts here
%%%%%%%%%%%%%%%%%%%%%%%%%%%%%%%%%%%%%%%%%%%%%%%%%%%%%%%%%%%%%%%%%%%%%

\begin{problem}
Consider the natural numbers partially ordered by divisibility.

\begin{problemparts}

\problempart 
Show that this partial order has a unique minimal element.

\begin{solution}
1 is minimal as there is no other natural number that divides 1.
It is unique because all other numbers are divisible by 1 and therefore
are not minimal.
\end{solution}


\problempart 
Show that this partial order has a unique maximal element.

\begin{solution}
0 is maximal: all natural numbers divide zero.  It is the only
maximal element, because for every positive natural number, $n$, we have
that $n$ is strictly ``smaller'' than $2n$ under divisibility.
\end{solution}


\problempart  
Prove that this partial order has an infinite chain.

\begin{solution}
1 2 4 8 16 \dots is a chain with infinite length.
\end{solution}

\problempart
Prove that this partial order has an infinite antichain. \hint The
primes.

\begin{solution}
The set of prime numbers is infinite.  Since no prime divides
another, any two primes are incomparable.  So the set of prime numbers is
an antichain.
\end{solution}

\end{problemparts}

\end{problem}


%%%%%%%%%%%%%%%%%%%%%%%%%%%%%%%%%%%%%%%%%%%%%%%%%%%%%%%%%%%%%%%%%%%%%
% Problem ends here
%%%%%%%%%%%%%%%%%%%%%%%%%%%%%%%%%%%%%%%%%%%%%%%%%%%%%%%%%%%%%%%%%%%%%
\documentclass[problem]{mcs}

\begin{pcomments}
  \pcomment{PS_erase_aexps}
  \pcomment{by  ARM 2/24/11}
\end{pcomments}

\pkeywords{
  arithmetic
  aexp
  matched
  brackets
  erase
}

\newcommand{\erasefunc}[1]{\mopt{erase}(#1)}

%%%%%%%%%%%%%%%%%%%%%%%%%%%%%%%%%%%%%%%%%%%%%%%%%%%%%%%%%%%%%%%%%%%%%
% Problem starts here
%%%%%%%%%%%%%%%%%%%%%%%%%%%%%%%%%%%%%%%%%%%%%%%%%%%%%%%%%%%%%%%%%%%%%

\begin{problem}
\bparts

\ppart\label{PS_erase_def} Give a recursive definition of a function
$\erasefunc{e}$ that erases all the symbols in $e \in \aexp$ but the
brackets.  For example
\[
\erasefunc{\lefbrk \lefbrk \mtt{3} \prodsym \lefbrk x \prodsym x\rhtbrk\rhtbrk \sumsym \lefbrk \lefbrk \mtt{2} \prodsym x\rhtbrk \sumsym \mtt{1}\rhtbrk\rhtbrk} =
\lefbrk \lefbrk \lefbrk \rhtbrk\rhtbrk \lefbrk \lefbrk \mtt{2} \prodsym x\rhtbrk \sumsym \mtt{1}\rhtbrk\rhtbrk.
\]

\begin{solution}
TBA
\end{solution}

\ppart
Prove that $\erasefunc{e} \in \RM$ for all $e \in \aexp$.

\ppart Give an example of a small string $s \in \RM$ such that
$\lefbrk s \rhtbrk \neq \erasefunc{e}$ for any $e \in \aexp$.

\begin{solution}
$s \eqdef \lefbrk \rhtbrk \lefbrk \rhtbrk \lefbrk \rhtbrk$ corresponds
to erasure of a a three argument operator, but operators in \aexp's
take $\le 2$ arguments.
\end{solution}

\eparts
\end{problem}

%%%%%%%%%%%%%%%%%%%%%%%%%%%%%%%%%%%%%%%%%%%%%%%%%%%%%%%%%%%%%%%%%%%%%
% Problem ends here
%%%%%%%%%%%%%%%%%%%%%%%%%%%%%%%%%%%%%%%%%%%%%%%%%%%%%%%%%%%%%%%%%%%%%

\endinput

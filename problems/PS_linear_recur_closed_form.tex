%PS_binomial_problem

\documentclass[problem]{mcs}

\begin{pcomments}
  \pcomment{PS_linear_recur_closed_form}
  \pcomment{from: F06.ps8 modified to use generating function}
  \pcomment{soln needs to be revised to style of PS_Fibonacci_and_bunnies}
\end{pcomments}

\pkeywords{
  generating_functions
  linear_recurrence
  partial_fractions
}

%%%%%%%%%%%%%%%%%%%%%%%%%%%%%%%%%%%%%%%%%%%%%%%%%%%%%%%%%%%%%%%%%%%%%
% Problem starts here
%%%%%%%%%%%%%%%%%%%%%%%%%%%%%%%%%%%%%%%%%%%%%%%%%%%%%%%%%%%%%%%%%%%%%

\begin{problem}

  Let $x_0 \eqdef 0, x_1 \eqdef 1$ and for $n \geq 2$, let $x_n$ be
  defined by the linear recurrence:

  % \bparts

%   \ppart $x_{n} = 12 x_{n-2} - 16 x_{n-3} \quad
%     (x_0 = 1, x_1 = 2, x_2 = 3)$
% 
%   Hint: 2 is a root.
% 
% \begin{solution}
%   The characteristic equation is $r^3 - 12 r +  16 = 0$.
%   Solving a cubic equation can be messy process, but in this case the
%   roots are easy to find:
%   \begin{align*}
%   r_1 & =  2 \\
%   r_2 & =  2 \\
%   r_3 & = -4 
%   \end{align*}
% 
%   Therefore a general form for a solution is
%   \[
%   x_n  = A 2^n + B n 2^n + C (-4)^n.
%   \]
% 
%   Substituting the initial conditions into this general form gives a
%   system of linear equations.
%   \begin{align*}
%   1 & =  A + C \\
%   2 & =  2 A + 2 B -4 C \\
%   3 & =  4 A + 8 B + 16 C
%   \end{align*}
% 
%   The solution to this linear system is $A = 37/36$, $B = -1/12$ and
%   $C = -1/36$.  The complete solution to the recurrence is therefore
%   \[
%   x_n  =  \frac{37}{36}2^{n} - \frac{1}{12}n 2^{n} - \frac{1}{36}(-4)^{n}.
%   \]
% \end{solution}

  % \ppart
  \[
  x_{n} = 3x_{n-1} - 2 x_{n-2} + n.  % \quad (x_0 = 0, x_1 = 1)
  \]
  Find a closed form expression for $x_n$.  

\iffalse
  \hint The solution should be of the form 
  \[
  x_n = A \cdot 2^n + B \cdot n^2 + C \cdot n + D.
  \]
\fi

\begin{solution}
  
  We begin by finding the generating function for the sequence as defined
  by the recurrence:
\[
X(z) \eqdef x_0+x_1z+x_2z^2+\cdots+x_nz^n+\cdots.
\]
\iffalse

  \[
  \ang{0,\ 1,\ 3 f_1 - 2 f_0 + 2,\ 3 f_2 - 2 f_1 + 3,\ 3 f_2 - 2 f_1 + 4,\ \dots}
  \]
\fi
  
We can do this by breaking the sequence into a sum of three sequences:
\[
X(z) = 3zX(z) -2z^2X(z) + \frac{z}{(1-z)^2}
\]  
\iffalse
  \[
  \begin{array}{ccccccccccl}
    & \langle & 0, & 3 f_0, & 3 f_1, & 3 f_2, & \dots & \rangle
      & \corresp & 3 x F(x) \\
    & \langle & 0, & 0, & -2 f_0, & -2 f_1, & \dots & \rangle
      & \corresp & -2 x^2 F(x) \\
  + & \langle & 0, & 1, & 2, & 3, & \dots & \rangle
      & \corresp & \frac{x}{(1-x)^2} \\ \hline
    & \langle & 0, & 3 f_0 + 1, & 3 f_1 - 2 f_0 + 2, & 3 f_2 - 2 f_1 + 3, & \dots & \rangle
      & \corresp & 3 x F(x) - 2 x^2 F(x) + \frac{x}{(1-x)^2} \\
  \end{array}
  \]
 

  As we can see, the sum of the three sequences is equal to the sequence defined by the 
  recurrence, so we may solve for $X(z)$ by solving the following:
  \[
  X(z) = 3 x X(z) - 2 x^2 X(z) + \frac{x}{(1-x)^2}
  \]
\fi

  So
  \[
  X(z) \cdot (1 - 3z + 2z^2) = \frac{z}{(1-z)^2}
  \]
  Since $(1 - 3z + 2z^2) = (1 - 2z)(1 - z)$, we get:
  \[
  X(z) = \frac{z}{(1-z)^3 (1-2z)}
  \]
  
  Now, to find the closed form for the $n$th coefficient of this generating function, 
  let's expand $X(z)$ into partial fractions:
  \[
  \frac{z}{(1-z)^3  (1-2z)} = \frac{A}{(1-z)^3} + \frac{B}{(1-z)^2} + \frac{C}{1-z} + \frac{D}{1-2z}
  \]
  
  To find the constants $A,B,C,D$, we can multiply both sides by the
  denominator $(1-z)^3 (1-2z)$, so
\[
  z = A(1-2z) + B(1-z)(1-2z) + C(1-z)^2(1-2z) + D(1-z)^3
\]
  Now letting $z=1$, yields $A = -1$, and letting $z=1/2$ yields $1/2
  = D/2^3$, that is, $D=4$.  Then letting $z=0$ then yields $B+C=-3$,
  and letting $z=2$ yields $B-C=1$, so $B=-1$ and $C=-2$.

\iffalse
  We can solve for the other terms via a similar process:
  \begin{gather*}
  A = -1 \\
  B = -1 \\
  C = -2 \\
  D = 4
  \end{gather*}
\fi
  
  So
  \[
  X(z) = \frac{z}{(1-z)^3 (1-2z)} = - \frac{1}{(1-z)^3} - \frac{1}{(1-z)^2} - \frac{2}{1-z} + \frac{4}{1-2z}.
  \]
  
  We know the coefficient of $z^n$ in the power series for each term in this
  partial fraction expansion:
  \begin{align*}
  - \frac{1}{(1-z)^3} & = - (1 + z +  \cdots + \binom{n+2}{2}z^n +\cdots) \\
  - \frac{1}{(1-z)^2} & = - (1 + z +  \cdots + \binom{n+1}{1}z^n +\cdots) \\
  - \frac{2}{1-z} & = -2 (1 + z +  \cdots + z^n +\cdots) \\
  \frac{4}{1-2z} & = 4 (1 + z +  \cdots + 2^n z^n +\cdots)
  \end{align*}
  
\iffalse
  (Note that $\frac{1}{(1-x)^3}$ is $\frac{1}{2}$ the derivative of
  $\frac{1}{(1-x)^2}$, so its coefficient for $x^n$ is $\frac{1}{2}$
  the coefficient of $\frac{1}{(1-x)^2}$ left shifted by 1 and
  multiplied by $(n+1)$.)
\fi
  
  Summing up the coefficients of $z^n$ in each of these power series gives:
  \begin{align*}
  x_{n} & = - \frac{(n+1)(n+2)}{2} - (n+1) - 2 + 4 \cdot 2^n \\
        & = - \paren{\frac{n^2}{2} + \frac{3n}{2} +1} - n - 1 - 2 + 4 \cdot 2^n \\
        & =  2^{n+2} - \frac{n^2}{2} - \frac{5n}{2} - 4.
  \end{align*}
  
\end{solution}

\end{problem}

%%%%%%%%%%%%%%%%%%%%%%%%%%%%%%%%%%%%%%%%%%%%%%%%%%%%%%%%%%%%%%%%%%%%%
% Problem ends here
%%%%%%%%%%%%%%%%%%%%%%%%%%%%%%%%%%%%%%%%%%%%%%%%%%%%%%%%%%%%%%%%%%%%%

\endinput

\documentclass[twoside,12pt]{article}
\newcommand{\tab}{\hspace*{2em}}
\usepackage{light}
\usepackage{subfigure}
\usepackage{graphicx}
\usepackage{amsmath}
\usepackage{verbatim}

\usepackage{amsfonts}

\newcommand{\lr}{l_{right}}
\renewcommand{\ll}{l_{left}}

\newcommand{\hint}[1]{({\it Hint: #1})}
\newcommand{\card}[1]{\left|#1\right|}
\newcommand{\union}{\cup}
\newcommand{\lgunion}{\bigcup}
\newcommand{\intersect}{\cap}
\newcommand{\lgintersect}{\bigcap}
\newcommand{\cross}{\times}
\newcommand{\rubric}[3][Any correct proof.]
  {
  	\begin{center}
	\fbox{\begin{minipage}{35em}
	\textbf{Rubric}
	\par
	[#3pts] #1
		\begin{center}
		\textbf{or}
		\end{center}
	#2
	\end{minipage}}
	\end{center}
  }

\hidesolutions
\showsolutions

\newlength{\strutheight}
\newcommand{\prob}[1]{\mathop{\textup{Pr}} \nolimits \left( #1 \right)}
\newcommand{\prsub}[2]{\mathop{\textup{Pr}_{#1}}\nolimits\left(#2\right)}
\newcommand{\prcond}[2]{%
  \ifinner \settoheight{\strutheight}{$#1 #2$}
  \else    \settoheight{\strutheight}{$\displaystyle#1 #2$} \fi %
  \mathop{\textup{Pr}}\nolimits\left(
    #1\,\left|\protect\rule{0cm}{\strutheight}\right.\,#2
  \right)}
\newcommand{\cE}{\mathcal{E}}
\renewcommand{\setminus}{-}
\renewcommand{\complement}[1]{\overline{#1}}

\providecommand{\abs}[1]{\lvert#1\rvert}

\begin{document}
\problemset{2}{September 13, 2016}{Monday, September 19}


%%%%%%%%%%%%%%%%%%%%%%%%%%%%%%%%%%%%%%%%%%%%%%%%%%%%
\noindent \textbf{Reading Assignment:}   Sections 2.5-2.7, 3.0-3.4, \& 3.5 (optional)
\\

\begin{problem}{10}
\bparts
\ppart{5} Prove that the number of subsets of $\{ 1, 2, \ldots n \}$ for positive integers $n$ is $2^n$.
\rubric{
[1pt] Base case


[2pts] Correct inductive hypothesis


[2pts] Inductive step


[$-$1pt] Math errors
}{5}
\solution{
\textbf{Base Case}: There are only 2 subsets of $S = \{ 1 \}$, namely $S$ itself and the empty set.  

\textbf{Inductive Hypothesis}:  Assume that there are $2^{k-1}$ subsets for $S = \{ 1, 2, \ldots k-1 \}$.  

\textbf{Inductive Step}: Consider $S = \{1, 2, \ldots k \}$.  We know that there are $2^{k-1}$ subsets on $\{1, 2, \ldots k-1 \}$ from the inductive hypothesis, which is just the subsets of $S$ not including element $k$.  However, we can just add the element $k$ to any of these subsets to create all subsets including the element $k$.  Therefore the total number of subsets is just $2^{k-1} + 2^{k-1} = 2^k$.

As the statement holds for $n = k$, by induction we have that the statement holds for all positive integers $n$. 
}
\rubric{
[1pt] Base case


[2pts] Correct inductive hypothesis


[2pts] Inductive step


[$-$1pt] Math errors
}{5}
\ppart{5} Prove that for $n \in \mathbb{N}$, $2^{3n} - 1$ is divisible by $7$ using induction.
\solution{
\textbf{Base Case}: $2^0 - 1 = 0$, which is divisible by 7.

\textbf{Inductive Hypothesis}: Assume $2^{3k-1} - 1$ is divisible by $7$ for some $k > 0 \in \mathbb{N}$.

\textbf{Inductive Step}: Consider $s = 2^{3k} - 1$.
\begin{equation}
\begin{split}
s &= 2^{3k} - 1 \\
&= 8 \cdot 2^{3k-1} - 1 \\
&= (7 + 1) \cdot 2^{3k-1} - 1 \\
&= 7 \cdot 2^{3k-1} + (2^{3k-1} - 1) \\
\end{split}
\end{equation}

Clearly $7 \cdot 2^{3k-1}$ is divisible by 7 since $k > 0$, and by our inductive hypothesis $2^{3k-1} - 1$ is divisible by 7.

Hence the statement is true for $k$, and so the statement is true for all $n \in \mathbb{N}$.  
}
\eparts

\end{problem}


\begin{problem}{15}
\bparts
\ppart{5} Consider three integers x, y, z written down on a piece of paper.  Any of the integers may be replaced by the sum of the other two plus $1$.  This operation is repeated a number of times until the final result is $11031, 19871, 16343$.  Is it possible that the initial integers were $2, 4, 6$? Prove it. (\textit{Hint}: Try out a few rounds of operations starting with $2, 4, 6$ and see if a pattern emerges) 
\rubric{
[1pt] No

[1pt] Proof by cases

[1pt] One operation with three evens ends up with two even and one odd

[2pt] Two even and one odd can only lead to two even and one odd (1 point for adding up one even and one odd and 1 point for adding up the two evens)
}{5}

\solution{
No.  We prove this by considering the parity of the three integers.  Suppose that we start with three even integers.  Then after one operation, we will end up with two even integers and an odd integer.  From this point, if we choose to do the operation with one odd and one even integer, then we will end up with two even and one odd.  If we choose to do the operation with two even integers, then we again end up with two even and one odd.  Thus, if we start with three even integers, we will always have two even and one odd integer after one operation.  Therefore, we have no way of making three odd integers from an initial configuration of three even integers.
}

\ppart{10} Prove that there is no way to cover a 6 x 6 board with rectangles of size 1 x 4 such that each square is covered by exactly one rectangle.  As a hint, one way to prove that it is not possible involves reasoning about the following assignment of letters $A, B, C, D$ to each square of the board.  
 \[
\begin{array}{|c|c|c|c|c|c|}
\hline
A & B & C & D & A & B\\ \hline
B & C & D & A & B & C\\ \hline
C & D & A & B & C & D\\ \hline
D & A & B & C & D & A\\ \hline
A & B & C & D & A & B\\ \hline
B & C & D & A & B & C\\ \hline
\end{array}
\]
\rubric{[5pts] Realizing that there are 9 $A's$, 10 $B's$, 9 $C's$, and 8 $D's$ 

\par
[5pts] Realizing that in possible configurations, there needs to be an equal number of $A's$, $B's$, $C's$, and $D's$ 
}{10}

\solution{
Observing the tiling, we see that each 1 x 4 rectangular tile must cover exactly one of each of $A, B, C, D$ based on the tiling.  However, there are 9 $A's$, 10 $B's$, 9 $C's$, and 8 $D's$ in our tiling, meaning that there is no way to tile a 6 x 6 board with such rectangles.
}
\eparts
\end{problem}

\begin{problem}{20}
The following problem is fairly tough until you hear a certain
one-word clue. The solution is elegant but is slightly tricky, so don't hesitate to ask for hints!

During 6.042, the students are sitting in
an $n\times n$ grid. A sudden outbreak of beaver flu (a rare variant of bird flu that lasts forever; symptoms include year\
ning for problem sets and craving for ice cream study sessions) causes some students to get infected. Here is
an example where $n = 6$ and infected students are marked $\times$.

\[
\begin{array}{|c|c|c|c|c|c|}
\hline
\times& & & &\times & \\ \hline
 &\times& & & & \\ \hline
& &\times&\times& & \\ \hline
& & & & & \\ \hline
& &\times& & & \\ \hline
& & &\times& &\times \\ \hline
\end{array}
\]

\noindent Now the infection begins to spread every minute (in discrete time-steps). Two students are considered \textit{adjacent} if they
share an edge (i.e., front, back, left or right, but NOT diagonal); thus, each student is adjacent to 2, 3 or 4 others.  A
student is infected in the next time step if either

\begin{itemize}
\item the student was previously infected (since beaver flu lasts forever), or
\item the student is adjacent to \textit{at least two} already-infected students.
\end{itemize}

In the example, the infection spreads as shown below.
%
\[
\begin{array}{|c|c|c|c|c|c|}
\hline
\times& & & &\times& \\ \hline
 &\times& & & & \\ \hline
& &\times&\times& & \\ \hline
& & & & & \\ \hline
& &\times& & & \\ \hline
& & &\times& &\times \\ \hline
\end{array}
\Rightarrow
\begin{array}{|c|c|c|c|c|c|}
\hline
\times&\times& & &\times& \\ \hline
\times&\times&\times& & & \\ \hline
&\times&\times&\times& & \\ \hline
& &\times& & & \\ \hline
& &\times&\times& & \\ \hline
& &\times&\times&\times&\times \\ \hline
\end{array}
\Rightarrow
\begin{array}{|c|c|c|c|c|c|}
\hline
\times&\times&\times& &\times& \\ \hline
\times&\times&\times&\times& & \\ \hline
\times&\times&\times&\times& & \\ \hline
&\times&\times&\times& & \\ \hline
& &\times&\times&\times& \\ \hline
& &\times&\times&\times&\times \\ \hline
\end{array}
\]
%
In this example, over the next few time-steps, all the students in class will become infected.

\begin{theorem*}
If fewer than $n$ students in class are initially infected, the whole class will never be completely infected.
\end{theorem*}

Prove this theorem.

\textit{Hint:} To understand how a system such as the above ``evolves" over time, it is usually a good strategy to (1) identify an appropriate  property of the system at the initial stage, and (2) prove, by induction on the number of time-steps, that the property is preserved at every time-step. So look for a property (of the set of infected students) that remains invariant as time proceeds.

If you are stuck, ask your recitation instructor for the one-word clue and even more hints!
\rubric{
[2pts] Make a mention of perimeter


[2pts] Proof by induction


[4pts] Base case


[6pts] Perimeter of infected set of students never gets larger


[6pts] Inductive step, adding a newly infected student removes at least two edges and adds at most two edges to the perimeter.



}{20}
\solution{
\begin{proof}
Define the {\em perimeter} of an infected set of students to be the number of
edges with infection on exactly one side.  Let $I$ denote the
perimeter of the initially-infected set of students.

Now we use induction on the number of time steps to prove that the
perimeter of the infected region never increases.  Let $P(k)$ be the
proposition that after $k$ time steps, the perimeter of the infected
region is at most $I$.

{\bf Base case:} $P(0)$ is true by definition; the
perimeter of the infected region is at most $I$ after 0 time steps,
because $I$ is defined to be the perimeter of the initially-infected
region.

{\bf Inductive step:} Now we must show that $P(k)$ implies
$P(k+1)$ for all $k \geq 0$.  So assume that $P(k)$ is true, where $k \geq
0$; that is, the perimeter of the infected region is at most $I$ after $k$
steps.  The perimeter can only change at step $k + 1$ because some squares
are newly infected.  By the rules above, each newly-infected square is
adjacent to at least two previously-infected squares.  Thus, for each
newly-infected square, at least two edges are removed from the perimenter
of the infected region, and at most two edges are added to the perimeter.
Therefore, the perimeter of the infected region can not increase and is at
most $I$ after $k + 1$ steps as well.  This proves that $P(k)$ implies
$P(k+1)$ for all $k \geq 0$.

By the principle of induction, $P(k)$ is true for all $k \geq 0$.

If an $n \times n$ grid is completely infected, then the perimeter of
the infected region is $4n$.  Thus, the whole grid can become infected
only if the perimeter is initially at least $4n$.  Since each square
has perimeter 4, at least $n$ squares must be infected initially for the whole grid to be infected.
\end{proof}

The above proof shows that if initially $k$ students  are infected, then the perimeter of the infected region will never e\
xceed $4k$. The largest number of students that can be contained within a region with perimeter $\leq 4k$ is equal to $k^2\
$, therefore, if $k$ students in class are initially infected, then at most $k^2$ students will eventually be infected. This feels intuitively true after having done the previous proof. However, to give a formal proof requires some case analysi\
s (try it!).
}

\end{problem}

%%%%%%%%%%%%%%%%%%%%%%%%%%%%%%%%%


\begin{problem}{10}
Can raising an irrational number $a$ to an irrational power $b$ result in a
rational number? Provide a proof that it can by considering $\sqrt{5}^{\sqrt{2}}$
and using proof by cases.
\rubric{

[2pts] Proof by cases.


[4pts] Case where $\sqrt{5}^{\sqrt{2}}$ is rational.


[4pts] Case where $\sqrt{5}^{\sqrt{2}}$ is irrational.

[$-$1pt] Math errors
}{10}
\solution{
This proof is by cases. We will consider $\sqrt{5}^{\sqrt{2}}$, and there are
two cases.

\begin{itemize}

\item $\sqrt{5}^{\sqrt{2}}$ is rational. In this case, since $\sqrt{2}$ is
irrational and $\sqrt{5}$ is irrational, we have found an $a$ and $b$ such that
both are irrational and $a^b$ is rational.

\item $\sqrt{5}^{\sqrt{2}}$ is irrational. In this case, notice that
$(\sqrt{5}^{\sqrt{2}})^{\sqrt{2}} = \sqrt{5}^{2} = 5$. Since $\sqrt{5}^{\sqrt{2}}$
and $\sqrt{2}$ are both irrational and $5$ is rational, we have found an
$a$ and $b$.

\end{itemize}

In all cases, we have found some irrational $a$ such that when raised to an
irrational power $b$, $a^b$ is rational.
}
\end{problem}

\begin{problem}{15}
For any nonempty set $C$, let $f(C)$ be the square of the product of the elements in $C$.  For example, if $C = \{ 1, 4, 5 \}$, then $f(C) = (1 \cdot 4 \cdot 5)^2 = 400$.  Show that the sum of $f(S)$ for all nonempty subsets of $\{1, 2, \ldots, n\}$ containing no consecutive elements is $ (n+1)! - 1$.  For example, if we consider $\{1, 2, 3\}$, then we have $f(\{1, 3\}) + f(\{1\}) + f(\{2\}) + f(\{3\}) = 23 = 4! - 1$.  (Hint: Use strong induction)
\end{problem}
\rubric{

[1pts] Proof by strong induction

[2pts] Base Case

[2pts] Inductive Hypothesis

[2pts] $f(S')$ for nonempty subsets of $\{1, 2,\ldots, k-1\}$ (which is without $k$) is $k! - 1$

[3pts] $k$ can be added to all subsets $S''$ if $\{1, 2,\ldots, k-2\}$ (because no consecutive elements)

[3pts] adding {k} to the final sum

[2pts] correct math resulting in $(k+1)! - 1$
}{15}
\solution{
We will prove this using strong induction.  

Base Case: We have that for $n = 1$, there is only one nonempty subset $S = \{1\}$.  Thus, $f(S) = 1 = (1+1)! -1$.

Strong Inductive Hypothesis: Suppose that the statement holds for all $n = 1, 2, \ldots, k-1$.  

Inductive Step: Suppose we consider the sum of $f(S)$ for nonempty subsets $S$ of $\{1, 2, \ldots , k\} $.  Then if we consider all subsets without element $k$, we have the sum of $f(S')$ for nonempty subsets $S'$ of $\{1, 2, \ldots k-1 \}$, which by our inductive hypothesis is just $k! - 1$.  

Now if our nonempty subsets $S$ of $\{1, 2, \ldots, k\}$ include the element $k$, then they cannot include element $k-1$.  Thus we consider the sum of $f(S'')$ for nonempty subsets $S''$ of $\{1, 2, \ldots, k-2 \}$, since we can safely add back $k$ into any of these subsets.  Then by our inductive hypothesis, this gives us another $k^2((k-1)! - 1)$.  However, we must finally consider the subset $\{k\}$, which contributes $k^2$ to our sum of $f(S)$.  

Adding over all the cases, we have that the sum of $f(S)$ for all nonempty subsets of $\{1, 2, \ldots, k\}$ such that $S$ contains no consecutive elements is
\begin{equation*}
\begin{split}
k! - 1 + k^2((k-1)! - 1) + k^2 &= k! (1 + k) -1 - k^2 + k^2 \\
&= (k+1)! - 1 \\
\end{split}
\end{equation*}
Thus as the statement holds for $n = k$, the statement holds for all $n = 1, 2, \ldots$. 
}

%%%%%%%%%%%%%%%%%%%%%%%%%%%%%%%%%%%%%%%%
\begin{problem}{15}
  A group of $n \ge 1$ people can be divided into teams, each
  containing either 5 or 6 people.  What is the largest $n$ for which the group cannot be divided into such teams?  Use induction to prove that your answer is correct.
\rubric{
[1pt] Proof by strong induction


[5pts] 1 point for each base case


[5pts] Inductive step


[4pts] $n = 19$
}{15}
\solution{
We begin by observing that the following numbers of people can be divided
into teams with 5 or 6 people per team:
\begin{align*}
5 & = 5 \\
6 & = 6 \\
10 &= 5 + 5 \\
11 &= 5 + 6 \\
12 &= 6 + 6 \\
15 &= 5 + 5 + 5 \\
16 &= 5 + 5 + 6 \\
17 &= 5 + 6 + 6 \\
18 &= 6 + 6 + 6 \\
20 &= 5 + 5 + 5 + 5 \\
21 &= 5 + 5 + 5 + 6 \\
22 &= 5 + 5 + 6 + 6 \\
23 &= 5 + 6 + 6 + 6 \\
24 &= 6 + 6 + 6 + 6 \\
\end{align*}
and these are the only numbers less than $25$ that can be divided into such
teams.  Now we claim that every group of $n \ge 20$ people can be divided
into teams, each containing either 5 or 6 people.

\begin{proof}
  The proof is by strong induction on $n$.  Let $P(n)$ be the proposition
  that a group of $n \ge 20$ people can be divided into teams, with each
  containing either 5 or 6 people.

  \textbf{Base cases:} As shown above $P(20)$, $P(21)$, $P(22)$, and
  $P(23)$, and $P(24)$ are true.

\textbf{Inductive step:} For all $n \ge 25$, we assume that $P(20)$,
$P(21)$, $\dots$, $P(n)$ are true in order to prove that $P(n+1)$ is true.

Since $n+1 = (n-4) + 5$, a team of 5 people can be removed from the set of
$n+1$ people, leaving $n-4 \ge 25$ people.  By induction hypothesis, the
$n-4$ people can be further divided into disjoint teams with 5 or 6
people.  Since this divides the $n+1$ people into teams with 5 or 6, we
have shown that $P(n+1)$ is true.  It follows by strong induction that
$P(n)$ holds for all $n \ge 25$.

So the largest possible $n$ for which the group cannot be divided into teams of size 5 or 6 is for $n = 19$.
\end {proof}
}
\end{problem}

%%%%%%%%%%%%%%%%%%%%%%%%%%%%%%%%%%%%%%%%%%

\begin{problem}{15}\textit{The Well Ordering Principle (WOP)} states that ``every \textit{nonempty} set of 
\textit{nonnegative} integers has a \textit{smallest} element." (See Section 3.1 of the text \textit{Mathematics 
for Computer Science}.)  It captures a special property about nonnegative integers and can be extremely 
useful in proofs.  

Prove using the Well Ordering Principle that for all nonnegative integers, $n$: 
\begin{equation}\label{sum-to-n}
\sum_{i=0}^{n} i^3 = \left(\frac{n(n+1)}{2}\right)^2.
\end{equation}
 (\textit{Hint:} Begin by pretending that the equation is false and consider the 
nonempty set of nonnegative integers for which the equation does not hold. Try to show that this set has no least element, which would contradict the Well Ordering Principle)
\rubric[Does not use WOP]{
[2pts] Use WOP


[2pts] $C$ has a minimum $c$ that is the smallest counterexample (and that is nonnegative)


[2pts] True for $n=0$, so $c > 0$


[3pts] $c - 1$ is nonnegative, and equation holds true


[3pts] Add $c^3$ to both sides to show it does hold for $c$


[2pts] Correct math
}{0}
\solution{
\begin{proof}
The proof uses WOP. Assume that equation \eqref{sum-to-n} is false. Then, some nonnegative integers serve as counterexamples to it.
Let's collect these counterexamples in a set:
$C ::= \{ n \in \mathbb{N} | \sum_{i=0}^{n} i^3 \neq \left(\frac{n(n+1)}{2}\right)^2 \}$.

By our assumption that \eqref{sum-to-n} admits counterexamples, $C$ is a nonempty set
of nonnegative integers. So, by the Well Ordering Principle, $C$ has a minimum
element, call it $c$. That is, $c$ is the smallest counterexample to \eqref{sum-to-n}.

Since $c$ is the smallest counterexample, we know that equation \eqref{sum-to-n} is false for $n = c$, but it is true for all nonnegative integers $n < c$.  However, equation \eqref{sum-to-n} is true for $n = 0$ since $\sum_{i=0}^{0} i^3 = 0 = \left(\frac{0(0+1)}{2}\right)^2$. Hence, $c > 0$. This means $c - 1$ is a nonnegative integer, and since it is less than $c$, equation \eqref{sum-to-n} is true for $c - 1$. That is,

\begin{equation}\label{sum-to-c-1}
\sum_{i=0}^{c-1} i^3 = \left(\frac{(c-1)c}{2}\right)^2.
\end{equation}

But then, adding $c^3$ to both sides of equation \eqref{sum-to-c-1} gives us
\[
\sum_{i=0}^{c} i^3
\]
on the left hand side. And the right hand side now equals
\begin{align*}
\left(\frac{(c-1)c}{2}\right)^2 + c^3 &= \frac{(c-1)^2c^2+4c^3}{2^2}\\
 &= \frac{c^2\left((c-1)^2+4c \right)}{2^2}\\
&= \frac{c^2\left(c^2-2c+1+4c \right)}{2^2}\\
&= \frac{c^2\left(c^2+2c+1 \right)}{2^2}\\
&= \frac{c^2(c+1)^2}{2^2}\\
&= \left(\frac{c(c+1)}{2}\right)^2.
\end{align*}

That is,
\[
\sum_{i=0}^{c} i^3 = \left(\frac{c(c+1)}{2}\right)^2,
\]
which means that equation \ref{sum-to-n} does hold for c, after all! This is a contradiction, and we are done.

\end{proof}
}

\end{problem}



\end{document}

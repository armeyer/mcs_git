\documentclass[12pt]{article}
\usepackage{light}
\renewcommand{\divides}{\mathop{\mid}}


\begin{document}

\generic{Some useful facts about divisibility and modulo arithmetic}{18 September 2012}

\section*{Divisibility}
\renewcommand{\labelenumi}{D\theenumi.}
\begin{enumerate}
%\item If $a \divides b$, then $a \divides bc$ for all $c$.

\item\label{lem:divtrans} If $a \divides b$ and $b \divides c$, then $a \divides c$.

\item\label{lem:divsbtc} If $a \divides b$ and $a \divides c$, then $a \divides sb + tc$
  for all $s$ and $t$.

\item\label{lem:divcancel} For all $c \neq 0$, $a \divides b$ if and only if $ca \divides
  cb$.
\end{enumerate}

\section*{Greatest common divisor}
\renewcommand{\labelenumi}{G\theenumi.}
\begin{enumerate}
%\item Every common divisor of $a$ and $b$ divides $\gcd(a, b)$.
\item\label{gcd2} $\gcd(k a, k b) = k \cdot \gcd(a, b)$ for all $k > 0$.
\item\label{gcd3} If $\gcd(a, b) = 1$ and $\gcd(a, c) = 1$, then $\gcd(a, bc) = 1$.
\item\label{gcd4} If $a \divides b c$ and $\gcd(a, b) = 1$, then $a \divides c$.
\item If $m \divides a$ and $m \divides b$, then $m \divides \gcd(a,b)$.
%\item\label{gcd5} $\gcd(a, b) = \gcd(b, \rem{a}{b})$. already in lem:gcdrem

\end{enumerate}

\section*{Modulo arithmetic}
\renewcommand{\labelenumi}{M\theenumi.}
\begin{enumerate}
\item $a \equiv a \pmod{n}$
\item $a \equiv b \pmod{n}$ implies $b \equiv a \pmod{n}$
\item $a \equiv b \pmod{n}$ and $b \equiv c \pmod{n}$ implies $a \equiv c \pmod{n}$
\item $a \equiv b \pmod{n}$ implies $a + c \equiv b + c \pmod{n}$
\item $a \equiv b \pmod{n}$ implies $a c \equiv b c \pmod{n}$
\item $a \equiv b \pmod{n}$ and $c \equiv d \pmod{n}$ imply $a + c
\equiv b + d \pmod{n}$
\item $a \equiv b \pmod{n}$ and $c \equiv d \pmod{n}$ imply $a c
\equiv b d \pmod{n}$
\end{enumerate}

\textbf{Warning:} it is \emph{not} the case that $ak \equiv bk \pmod{n}$ implies $a \equiv b \pmod{n}$ in general. It \emph{is} true however if $\gcd(n,k) = 1$; in particular, if $n$ is prime and $k$ is not a multiple of $n$.

\end{document}


\documentclass[problem]{mcs}

\begin{pcomments}
  \pcomment{FP_logic_relations}
  \pcomment{from: F08.final}
  \pcomment{BUGGY, edited ARM 2/20/17; still needs correction}
\end{pcomments}

\pkeywords{
  partial_order
  reflexive
  symmetric
  antisymmetric
  transitive
}

%%%%%%%%%%%%%%%%%%%%%%%%%%%%%%%%%%%%%%%%%%%%%%%%%%%%%%%%%%%%%%%%%%%%%
% Problem starts here
%%%%%%%%%%%%%%%%%%%%%%%%%%%%%%%%%%%%%%%%%%%%%%%%%%%%%%%%%%%%%%%%%%%%%

\begin{problem}
We define a relation $R$ on pairs of the Boolean values $\true$ and
$\false$ by

\begin{staffnotes}
In second \QIMPLIES\ there were \QOR s, making it identical to the
first \QIMPLIES, so changed those two \QOR s to \QAND s.
\end{staffnotes}

\[
(p_1,p_2)\mrel{R}(q_1,q_2) \qiff \left[
\begin{array}{ll}
(p_1\QOR\ p_2) \QIMPLIES (q_1\QOR\ q_2) & \mbox{ and } \\
(p_1\QAND\ p_2) \QIMPLIES (q_1 \QAND\ q_2)     
\end{array} \right].
\]

For example, since $(\false \QOR\ \true) \QIMPLIES\ (\true \QOR\ \false)$
and $(\false \QAND\ \true) \QIMPLIES\ (\true \QAND\ \false)$,
\[
(\false,\true)\mrel{R}(\true,\false).
\]

\bparts

\ppart Give another of Boolean values $p_1,p_2,q_1,q_2$ for which
$(p_1,p_2)\mrel{R}(q_1,q_2)$.

\begin{solution}
For example, $(\false,\false)\mrel{R}(\true,\true)$.
\end{solution}
\examspace[1cm]

\ppart Give an example of Boolean values $p_1,p_2,q_1,q_2$ for which
$(p_1,p_2)\mrel{R}(q_1,q_2)$ does not hold.

\begin{solution}
For example, $(\true,\true)\mrel{R}(\false,\false)$.
\end{solution}
\examspace[1cm]
\eparts

\bigskip
For each of the following properties, briefly explain why $R$ has the
the property or give a counterexample to show it does not have the
property.

\bparts

\ppart Reflexive.

\begin{solution}
Yes, $(p_1\QOR\ p_2) \QIMPLIES\ (p_1\QOR\ p_2)$ and $(p_1\QAND\ p_2)
\QIMPLIES\ (p_1\QAND\ p_2)$.
\end{solution}
\examspace[3cm]

\ppart Symmetric.

\begin{solution}
No, $(\true, \false)\mrel{R}(\true, \true)$ but not $(\true,
\true)\mrel{R}(\true, \false)$.
\end{solution}
\examspace[3cm]

\ppart Antisymmetric.

\begin{solution}
No, $(\true, \false)\mrel{R}(\false,\true)$,
$(\false,\true)\mrel{R}(\true,\false)$, and $(\true,\false)\neq
(\false,\true)$.
\end{solution}
\examspace[3cm]

\begin{staffnotes}
OMITTED  in F08 presumably because proof is wrong.

\ppart Transitive.

\begin{solution}
Yes, this follows directly from the transitivity of $QIMPLIES$.  If
\[
[(p_1\QOR\ p_2) \QIMPLIES\ (q_1\QOR\ q_2)]
   \QAND 
[(q_1\QAND\ q_2) \QIMPLIES\ (z_1\QAND\ z_2),
\]
then
\[
(p_1\QOR\ p_2) \QIMPLIES\ (z_1\QOR\ z_2).
\]

If
\[
[(p_1\QAND\ p_2) \QIMPLIES\ (q_1\QAND\ q_2)]
    \QAND 
[(q_1\QAND\ q_2) \QIMPLIES\ (z_1\QAND\ z_2)],
\]
then
\[
(p_1\QAND\ p_2) \QIMPLIES\ (z_1\QAND\ z_2).
\]
So,
\[
[(p_1,p_2)\mrel{R}(q_1,q_2)]
\QAND
[(q_1,q_2)\mrel{R}(z_1,z_2)]
\]
implies
\[
(p_1,p_2)\mrel{R}(z_1,z_2).
\]
\end{solution}
\end{staffnotes}

\eparts

\end{problem}

\endinput

\documentclass[problem]{mcs}

\begin{pcomments}
  \pcomment{from: S09.ps6}
  \pcomment{commented out in S09}
  \pcomment{solution missing}
  \pcomment{contains href to the lecture notes}
\end{pcomments}

\pkeywords{
}

%%%%%%%%%%%%%%%%%%%%%%%%%%%%%%%%%%%%%%%%%%%%%%%%%%%%%%%%%%%%%%%%%%%%%
% Problem starts here
%%%%%%%%%%%%%%%%%%%%%%%%%%%%%%%%%%%%%%%%%%%%%%%%%%%%%%%%%%%%%%%%%%%%%

\begin{problem}
  \href{http://courses.csail.mit.edu/6.042/spring09/ln6.pdf#switch.edges}
  {Lemma 7.8} in Notes 6 established a key property of planar embeddings:
  two edges that could be successively added to an embedding can be added
  in either order.  Carefully prove this property for the case that the
  two edges are both added by the split-a-face constructor.

\solution{TBA}

\end{problem}

%%%%%%%%%%%%%%%%%%%%%%%%%%%%%%%%%%%%%%%%%%%%%%%%%%%%%%%%%%%%%%%%%%%%%
% Problem ends here
%%%%%%%%%%%%%%%%%%%%%%%%%%%%%%%%%%%%%%%%%%%%%%%%%%%%%%%%%%%%%%%%%%%%%

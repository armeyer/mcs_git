\documentclass[problem]{mcs}

\begin{pcomments}
  \pcomment{PS_nim_strategy}
  \pcomment{F14.ps2}
  \pcomment{edited ARM 9/24/14}
\end{pcomments}

\pkeywords{
  Nim
  xor
  binary
  strategy
  state_machines
  invariant
  binary
  representation
  game
}

%%%%%%%%%%%%%%%%%%%%%%%%%%%%%%%%%%%%%%%%%%%%%%%%%%%%%%%%%%%%%%%%%%%%%
% Problem starts here
%%%%%%%%%%%%%%%%%%%%%%%%%%%%%%%%%%%%%%%%%%%%%%%%%%%%%%%%%%%%%%%%%%%%%

\begin{problem}
Nim is a two-person game that starts with some piles of stones.  A
player's move consists of removing one or more stones from a single
pile.  \inbook{The players alternate making moves, and whoever takes
  the last stone wins.}\inhandout{Player-1 and player-2 alternate
  making moves, and whoever takes the last stone wins.

So if there is only one pile, then the first player to move wins by
taking the whole pile.  On the hand, if the game starts with just two
piles, each with the same number of stones, then the player who moves
second can guarantee a win simply by mimicking the first player.  This
means, for example, that if the first player removes three stones from
one pile, then the second player removes three stones from the other
pile.  At this point, it's worth thinking for a moment about
\textbf{why the mimicking strategy guarantees a win} for the second
player.}

It turns out there is a winning strategy for one of the players that
is easy to carry out but is not so obvious.

\iffalse If you are positioned to be the winning player, it's going to
be a while before your opponent figures out why you keep winning.\fi

To explain the winning strategy, we need to think of a number in two
ways: as a nonnegative integer and as the bit string equal to the
binary representation of the number---possibly with leading zeroes.

For example, the $\QXOR$ of \emph{numbers} $r,s,...$ is defined in
terms of their binary representations: combine the corresponding bits
of the binary representations of $r, s,...$ using $\QXOR$, and then
interpret the resulting bit-string as a number.  For example,
\[
2 \QXOR 7 \QXOR 9 = 12
\]
because, taking $\QXOR$'s down the columns, we have
\[\begin{array}{ccccl}
 0 & 0 & 1 & 0 & \text{(binary rep of 2)}\\
 0 & 1 & 1 & 1 & \text{(binary rep of 7)}\\
 1 & 0 & 0 & 1 & \text{(binary rep of 9)}\\
\hline
 1 & 1 & 0 & 0 & \text{(binary rep of 12)}
\end{array}\]

This is the same as doing binary addition of the numbers, but throwing
away the carries (see Problem~\bref{CP_binary_adder_logic}).

The $\QXOR$ of the numbers of stones in the piles is called their
\emph{Nim sum}.  In this problem we will verify that if the Nim sum is
not zero on a player's turn, then the player has a winning strategy.
For example, if the game starts with five piles of equal size, then
the first player has a winning strategy, but if the game starts with
four equal-size piles, then the second player can force a win.

\bparts

\ppart\label{zS-nzS} Prove that if the Nim sum of the piles is zero,
then any one move will leave a nonzero Nim sum.

\begin{solution}
When a player removes stones from a pile, the binary representation of
the number of stones in the pile changes.  Since the other piles stay
the same, the bits in the Nim sum change at the positions where bits
changed in the binary representation.  Since all the bits in the Nim
sum were initially equal to zero, the changed bits must have become
ones.  That makes the Nim sum nonzero.
\end{solution}

\ppart\label{if-r>sxt} Prove that if there is a pile with more stones
than the Nim sum of all the other piles, then there is a move that
makes the Nim sum equal to zero.

\begin{solution}
  If there is a size $r$ greater than the the Nim sum $n$ of the
  other piles, then remove stones from this pile until it is of size
  $n$.  Now the Nim sum of all the piles becomes $n \QXOR n = 0$.
\end{solution}

\ppart\label{nz-r>sxt} Prove that if the Nim sum is not zero, then one
of the piles is bigger than the Nim sum of the all the other piles.

\hint Notice that the largest pile may not be the one that is bigger
than the Nim sum of the others; three piles of sizes 2,2,1 is an
example.

\begin{solution}
 Suppose the Nim sum of all the piles is $s$, and the high order digit
 of the Nim sum, that is, the leftmost 1 in the binary representation
 of $s$, occurs at the $n$th position.  Since this bit of the Nim sum
 equals 1, there must be at least one pile of $r$ stones such that the
 $n$th bit of $r$ is also 1.  This is the pile to choose.

 Notice that, since $r \QXOR r = 0$, we have $r \QXOR (r \QXOR s) = s$.
 This means that $r \QXOR s$ is the Nim sum of all the piles besides
 the pile of $r$.  Now $r \QXOR s$ has $n$th bit 0, since both $r$ and
 $s$ have $n$th bit 1.  Also, since each bit of $s$ above the $n$th is
 zero, each bit of $r$ above the $n$th must equal the corresponding
 bit of $r \QXOR s$.

 So $r$ and $r \QXOR s$ are the same above the $n$th bit, and $r$ is
 bigger at the $n$th bit.  This implies that $r > r \QXOR s$.  That
 is, $r$ is bigger than the Nim sum of all the other piles, as
 claimed.
\end{solution}

\ppart Conclude that if the game begins with a nonzero Nim sum, then
the first player has a winning strategy.

\hint Describe a preserved invariant that the first player can
maintain.

\begin{solution}
By part~\eqref{zS-nzS}, whenever it is the second player's turn to
move, and the Nim sum is zero, the second player will leave a Nim sum
that is not zero.

By parts~\eqref{if-r>sxt} and~\eqref{nz-r>sxt}, whenever it is the
first player's turn to move, and the Nim sum is not zero, the first
player can leave the Nim sum equal to zero.

If the first player always moves to set a nonzero Nim sum to zero,
then the Nim sum being nonzero on the first player's turn and zero on
the second player's turn is a preserved invariant of the game.

Since the total number of stones decreases at every move, the game
must eventually end with no stones are left.  But the Nim sum of no
stones is zero, so by the preserved invariant, it must happen on the
second player's turn.  That is, the second player must lose.
\end{solution}

\ppart (Extra credit) Nim is sometimes played with winners and losers
reversed, that is, the person who takes the last stone \emph{loses}.
This is called the \emph{mis\`ere} version of the game.  Use ideas
from the winning strategy above for regular play to find one for
\emph{mis\`ere} play.

\begin{solution}
\TBA{DRAFT} Follow the same strategy until just 2 piles remain, or
there are only piles of size 1, and adapt for \emph{mis\`ere}.
\end{solution}

\eparts
\end{problem}

\endinput

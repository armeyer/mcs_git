\documentclass[problem]{mcs}

\begin{pcomments}
  \pcomment{PS_transitive_closure_proof}
  \pcomment{added by ARM 10/16/11}
\end{pcomments}

\pkeywords{
  transitivity
  walk_relation
  merge
}

%%%%%%%%%%%%%%%%%%%%%%%%%%%%%%%%%%%%%%%%%%%%%%%%%%%%%%%%%%%%%%%%%%%%%
% Problem starts here
%%%%%%%%%%%%%%%%%%%%%%%%%%%%%%%%%%%%%%%%%%%%%%%%%%%%%%%%%%%%%%%%%%%%%

\begin{problem}
Prove that if $R$ is a transitive binary relation on a set $A$ then
$R= R^+$.

\begin{solution}
Let $R^{n+}$ be the length-$(\leq n)$ positive walk relation of $R$,
that is $a \mrel{R^{n+}} b$ iff there is a positive length path of
length $\leq n$ from $a$ to $b$ in the graph of $R$.  So by
definition, $a R^+ b$ iff $\exists n in \integers^+.\, a \mrel{R^{n+}}
b$.

So we need only show that if $R$ is transitive, then
\begin{equation}\label{RnsubeqR}
R^{n+} = R
\end{equation}
for all $n$.  The proof is by induction on $n$ with
induction hypothesis~\eqref{RnsubeqR}.

\inductioncase{base case} ($n =1$):  $R^{1+} = R$ by definition,
so~\eqref{RnsubeqR} is immediate in this case.

\inductioncase{induction step}: There is a path in the graph of $R$ of
length at most $n+1$ from vertex $a$ to vertex $b$ iff there is a path
between them that is of length at most $n$, or a path that is the merge of a
path of length at most $n$ with a path of length 1.  That is, for
all $a,b \in A$,
\begin{align*}
a \mrel{R^{(n+1)+}} b
   & \qiff a\mrel{ R^n }b \QOR\ \exists c \in A.\, a \mrel{R^{(n+}} c \QAND c \mrel{R^1} b\\
   & \qiff  a \mrel{ R }b \QOR\ \exists c \in A.\, a \mrel{R} c \QAND c \mrel{R} b
         & \text{(by induction hyp.)}\\
   & \qiff  a \mrel{ R }b
         & \text{(by transitivity)}.
\end{align*}
This proves that $R^{(n+1)+} = R$, which completes the induction.
\end{solution}

\end{problem}

%%%%%%%%%%%%%%%%%%%%%%%%%%%%%%%%%%%%%%%%%%%%%%%%%%%%%%%%%%%%%%%%%%%%%
% Problem ends here
%%%%%%%%%%%%%%%%%%%%%%%%%%%%%%%%%%%%%%%%%%%%%%%%%%%%%%%%%%%%%%%%%%%%%

\endinput

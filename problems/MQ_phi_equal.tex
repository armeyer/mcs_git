\documentclass[problem]{mcs}

\begin{pcomments}
  \pcomment{MQ_phi_equal}
  \pcomment{ARM 4/30/18}
\end{pcomments}

\pkeywords{
  Euler
  phi
  number_theory
  congruence
}

%%%%%%%%%%%%%%%%%%%%%%%%%%%%%%%%%%%%%%%%%%%%%%%%%%%%%%%%%%%%%%%%%%%%%
% Problem starts here
%%%%%%%%%%%%%%%%%%%%%%%%%%%%%%%%%%%%%%%%%%%%%%%%%%%%%%%%%%%%%%%%%%%%%

\begin{problem}
Prove that for all integers $n \geq 2$,
\[
\phi(n) = \phi(2n) \QIFF n \text{ is odd}.
\]

\begin{solution}
If $m$ is odd, then for $k>0$,
\[
\phi(2^k m) = \phi(2^k)\phi(m) = 2^{k-1}\phi(m)
\]
by Theorem~\bref{th:phi}.  So if $n$ is odd, then
\[
\phi(2 n) = \phi(2^1 n) = 2^{1-1}\phi(n) = \phi(n).
\]

Conversely, if $n$ is not odd, then $\phi(n) \neq \phi(2n)$.  This
follows because $n = 2^k m$ for some odd number $m \geq 1$ and $k>0$,
so
\begin{align*}
\phi(n)
 & = \phi(2^k)\phi(m)\\
 & = 2^{k-1}\phi(m)\\
 & \neq 2^k\phi(m) & \text{(since $\phi(2^{k-1}) > 0$)}\\
 & = \phi(2^{k+1})\phi(m)\\
 & = \phi(2^{k+1}m)\\
 & = \phi(2n).
\end{align*}
\end{solution}
\end{problem}

\endinput

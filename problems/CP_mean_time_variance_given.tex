\documentclass[problem]{mcs}

\begin{pcomments}
  \pcomment{CP_mean_time_variance_given}
  \pcomment{based on CP_mean_time_variance_gen_func by ARM 5/1/11}
  \pcomment{was called TP_mean_time_variance_given}
  \pcomment{from: F07.rec14h, S07.cp14w}
\end{pcomments}

\pkeywords{
  random_variable
  expectation
  indicator_variable
  variance
  additivity
}

%%%%%%%%%%%%%%%%%%%%%%%%%%%%%%%%%%%%%%%%%%%%%%%%%%%%%%%%%%%%%%%%%%%%%
% Problem starts here
%%%%%%%%%%%%%%%%%%%%%%%%%%%%%%%%%%%%%%%%%%%%%%%%%%%%%%%%%%%%%%%%%%%%%

\begin{problem}
A computer program crashes at the end of each hour of use with
probability $1/p$, if it has not crashed already.  Let $H$ be the
number of hours until the first crash.

%% , we know 
%% \begin{align*}
%% \expect{H}   & = \frac{1}{p},\\% & \text{(Equation~(\bref{exp_time_to_fail}))}\\
%% \variance{H} & = \frac{q}{p^2},% & \text{(Equation~(\bref{var_time_to_fail}))}
%% \end{align*}
%% where $q \eqdef 1-p$.

\bparts

\ppart\label{Cheb} What is the \idx{Chebyshev bound} on
\[
\pr{\abs{H - (1/p)}> x/p}
\]
where $x > 0$?

\begin{solution}
\[
\frac{1-p}{x^2}.
\]

We know 
\begin{align*}
\expect{H}   & = \frac{1}{p} & \text{(by~(\bref{exp_time_to_fail}))}\\
\variance{H} & = \frac{1-p}{p^2}, & \text{(by~(\bref{var_time_to_fail}))}.
\end{align*}

So by Chebyshev's Theorem~\bref{chebthm}, the bound on the above probability is
\[
\frac{\variance{H}}{\paren{x/p}^2} = \frac{1-p}{p^2\paren{x/p}^2} = 
    \frac{1-p}{x^2}.
\]
\end{solution}

\ppart\label{Hap} Conclude from part~\eqref{Cheb} that for $a \geq 2$,
\[
\pr{H > a/p} \leq \frac{1-p}{(a-1)^2}
\]

\hint Check that $\abs{H - (1/p)} > (a-1)/p$ iff  $H > a/p$.

\begin{solution}
Note that if $H \leq 1/p$, then $\abs{H - (1/p)} = (1/p) - H$,
and since $H > 0$, we have $ (1/p) - H < 1/p \leq (a-1)/p$.  It follows
that $\abs{H - (1/p)} > (a-1)/p$ iff $H - (1/p)> (a-1)/p$ iff $H > a/p$.
So
\begin{align*}
\pr{H > a/p} & = \pr{\abs{H - (1/p)} > (a-1)/p}\\
             & \leq (1-p)/(a-1)^2 & \text{(by part~\eqref{Cheb})}.
\end{align*}
\end{solution}

\ppart What actually is
\[
\pr{H > a/p}?
\]

Conclude that for any fixed $p>0$, the probability that $H > a/p$ is an
asymptotically smaller function of $a$ than the Chebyshev bound of 
part~\eqref{Hap}.

\begin{solution}
\begin{align*}
\pr{H > a/p} & = \sum_{k >a/p} (1-p)^{k-1}p\\
  & = p(1-p)^{\floor{a/p}} \sum_0^{\infty} (1-p)^j\\
  & = p(1-p)^{\floor{a/p}}\cdot\frac{1}{1-(1-p)}\\
  & = (1-p)^{\floor{a/p}}.
\end{align*}
So
\begin{align*}
\pr{H > a/p}
   & \leq (1-p)^{(a-p)/p}\\
   & = \paren{(1-p)^{1/p}}^{a-p}\\
   & < \paren{\paren{e^{-p}}^{1/p}}^{a-p}\\
   & = e^{p-a}.
\end{align*}

But as shown in part~\eqref{Hap}, Chebyshev gives
\[
\pr{H > a/p} \leq (1-p)/(a-1)^2 = \Theta(1/a^2).
\]
Since
\[
e^{p-a} = \frac{p}{e^a} = o(1/a^2),
\]
we conclude that the actual probability is much smaller than we got
from the Chebyshev Bound.
\end{solution}

\eparts
\end{problem}

\endinput

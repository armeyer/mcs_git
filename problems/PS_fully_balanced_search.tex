\documentclass[problem]{mcs}
\begin{pcomments}
  \pcomment{PS_fully_balanced_search}
  \pcomment{ARM 10/3/17}
\end{pcomments}

\pkeywords{
  binary_trees
  search_tree
  balanced
  recursive_data
  structural_induction
  }

%%%%%%%%%%%%%%%%%%%%%%%%%%%%%%%%%%%%%%%%%%%%%%%%%%%%%%%%%%%%%%%%%%%%%
% Problem starts here
% %%%%%%%%%%%%%%%%%%%%%%%%%%%%%%%%%%%%%%%%%%%%%%%%%%%%%%%%%%%%%%%%%%%%

\begin{problem}
Prove that there is a search tree $T_V$ such that
\[
\text{nums} \eqdef \set{\nlbl{(S)} \suchthat S \in \subbrn{T_V}} = V
\]
and
\[
\dpth{T_V} \leq \log_2 \card{V}.
\]

\hint Give a definition of $T_V$ by induction on the $\card{V}$.

\begin{solution}
If $\card{V}=1$, define $T_V$ to be a leaf with the label $v \in V$.

Otherwise, let $v_m$ be the median value in $V$.  This means that if 
\begin{align*}
V_< & \eqdef \set{v \in V \suchthat v < v_m}\\
V_>     & \eqdef \set{v \in V \suchthat v > v_m},
\end{align*}
then $\card{V_<} = \card{V_>}$.
]
Then by induction hypothesis, there are search trees $T_{V_<}$ and $T_{V_>}$


\end{solution}


\end{problem}

%%%%%%%%%%%%%%%%%%%%%%%%%%%%%%%%%%%%%%%%%%%%%%%%%%%%%%%%%%%%%%%%%%%%%
% Problem ends here
%%%%%%%%%%%%%%%%%%%%%%%%%%%%%%%%%%%%%%%%%%%%%%%%%%%%%%%%%%%%%%%%%%%%%

\endinput

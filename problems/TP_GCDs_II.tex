\documentclass[problem]{mcs}

\begin{pcomments}
    \pcomment{TP_GCDs_II}
    \pcomment{Converted from gcd-2.scm by scmtotex and dmj
              on Sat 12 Jun 2010 09:14:23 PM EDT}
    \pcomment{edited ARM 3/13/12}
\end{pcomments}

\begin{problem}

%% type: short-answer
%% title: GCD's II

Let
\begin{align*}
    x & \eqdef 17^{88} \cdot 31^{5} \cdot 37^{2} \cdot 59^{1000}  \\
    y & \eqdef 19^{(9^{22})} \cdot 37^{12} \cdot 53^{3678} \cdot 59^{29}.
\end{align*}

%Please give the prime factorization of your answer as a set of
%(\emph{prime exponent}) pairs.  For example, to write  
%the number $60 = 2^{2} \cdot 3 \cdot 5$, you can write
%\begin{equation*}
%(3 1) (2 2) (5 1)
%\end{equation*}

\bparts
\ppart\label{gcd2:pp1}
What is $\gcd(x,y)$?

\begin{solution}
$37^{2}\cdot59^{29}$.

To get the GCD of two numbers: iterate over all primes that appear in
both factorizations; raise each of them to the \emph{smallest} of its
two exponents; then multiply the resulting powers.

\end{solution}

\ppart What is $\lcm(x,y)$?
 (``$\lcm$'' is \emph{least common multiple}.)

\begin{solution}
$17^{88} \cdot 31^{5} \cdot 37^{12} \cdot 59^{1000}\cdot19^{(9^{22})}\cdot 53^{3678}$.

Replace \emph{smallest} with \emph{greatest} in the procedure for part \ref{gcd2:pp1}.
\end{solution}
\eparts
\end{problem}

\endinput

\documentclass[problem]{mcs}

\begin{pcomments}
  \pcomment{PS_shortest_undirected_closed_walk}
  \pcomment{subsumed by text Theorem th:treeprops}
  \pcomment{first part is MQ_closed_walk_to_cycle}
  \pcomment{S15.mid3}
\end{pcomments}

\pkeywords{
  walk
  path
  cycle
  closed_walk
  simple_graph
  undirected
}

%%%%%%%%%%%%%%%%%%%%%%%%%%%%%%%%%%%%%%%%%%%%%%%%%%%%%%%%%%%%%%%%%%%%%
% Problem starts here
%%%%%%%%%%%%%%%%%%%%%%%%%%%%%%%%%%%%%%%%%%%%%%%%%%%%%%%%%%%%%%%%%%%%%

\begin{problem}

\bparts

\ppart Give an example of a simple graph that has two vertices $u \neq
v$ and two distinct paths between $u$ and $v$, but neither $u$ nor $v$
is on a cycle.

\inhandout{\hint There is an example with five vertices.}

\examspace[1in]

\begin{solution}

\begin{editingnotes}
INSERT:

\begin{figure}
\graphic{Fig_walkpath}
\caption{$u, v$ connected by Distinct paths but neither is on a cycle.}
\label{fig:pathcycle}
\end{figure}

As in Figure~\ref{fig:pathcycle}, define
\end{editingnotes}

Take a triangle and attach $u$ by an edge to one corner and $v$ by an
edge to another corner.  Formally, define
\begin{align*}
V & \eqdef \set{u,v,a,b,c},\\
E & \eqdef \set{\edge{u}{a}, \edge{a}{b}, \edge{b}{c}, \edge{c}{a}, \edge{c}{v} }.
\end{align*}
Two paths from $u$ to $v$ are
\[
u \edge{u}{a} a \edge{a}{c} c \edge{c}{v} v
\]
and
\[
u \edge{u}{a} a \edge{a}{b} b \edge{b}{c} c \edge{c}{v} v.
\]
\end{solution}

\ppart Prove that if there are different paths between two vertices
in a simple graph, then the graph has a cycle.

\begin{staffnotes}
The proof is direct from Theorem~\bref{th:treeprops}.  Tell students
\textbf{not} to look up that proof.
\end{staffnotes}

\begin{solution}

\begin{proof}
  Call two vertices a \emph{different-path-pair} (dpp) if there are
  distinct paths between them.  Let $u \neq v$ be a dpp such that some
  path $\walkv{p}$ between $u$ and $v$ has minimum length among all
  paths between dpp's.  Since $u$ and $v$ are a dpp, there must be
  another path between them.

  Assume for the sake of contradiction that the other path shares with
  $\walkv{p}$ a vertex $w$ different from the endpoints $u$ and $v$.
  Then there either $u$ and $w$, or $w$ and $v$, must be a dpp.  But
  the shortest path between $u$ and $w$ as well as the shortest path
  between $w$ and $v$ are each shorter than $\walkv{p}$, contradicting
  minimality.  So the two paths between $u$ and $v$ do not share any
  vertices other than their endpoints.  This means they form a cycle.
\end{proof}

Note that the proof above is a rephrasing of the proof that a simple
graph is a tree iff there is a unique path between any two vertices,
Theorem~\bref{th:treeprops}.\bref{treeprops:uniquepath}.  In fact,
appeal to this Theorem yields an immediate proof of this part: if
$u,v$ are a dpp, then by the Theorem, the connected component
containing $u,v$ is not a tree, which by definition means it contains
a cycle.

\end{solution}

\eparts

\end{problem}

%%%%%%%%%%%%%%%%%%%%%%%%%%%%%%%%%%%%%%%%%%%%%%%%%%%%%%%%%%%%%%%%%%%%%
% Problem ends here
%%%%%%%%%%%%%%%%%%%%%%%%%%%%%%%%%%%%%%%%%%%%%%%%%%%%%%%%%%%%%%%%%%%%%

\endinput

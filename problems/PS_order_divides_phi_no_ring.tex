\documentclass[problem]{mcs}

\begin{pcomments}
  \pcomment{PS_order_divides_phi_no_ring}
  \pcomment{version of PS_order_divides_phi avoiding $\Zmod{n}$ notation, for s18 mid4.}
\end{pcomments}

\pkeywords{
  Euler_function
  Eulers_theorem
  number_theory
  congruence
  phi
}

%%%%%%%%%%%%%%%%%%%%%%%%%%%%%%%%%%%%%%%%%%%%%%%%%%%%%%%%%%%%%%%%%%%%%
% Problem starts here
%%%%%%%%%%%%%%%%%%%%%%%%%%%%%%%%%%%%%%%%%%%%%%%%%%%%%%%%%%%%%%%%%%%%%

\begin{problem}

\begin{definition*}
Define the \emph{order of $k$ modulo $n$}, written as $\ordmod{k}{n}$, to be the
smallest positive power of $k$ congruent to $1$ modulo $n$, that is,
\[
\ordmod{k}{n} \eqdef \min \set{m > 0 \suchthat\ \ k^m \equiv 1 \pmod n}.
\]
If $k^m$ is \emph{never} congruent to $1 \bmod n$ for any positive
integer $m$, then $\ordmod{k}{n} \eqdef \infty$.
\end{definition*}

\bparts

\ppart For integers $k$ and $n$, show that if $\ordmod{k}{n}$ is
finite then $k$ and $n$ are relatively prime.

\examspace[1.5in]

\begin{solution}
If $\ordmod{k}{n} = m$ then $k^m\equiv 1\pmod n$, that is, $k^m = 1 + q n$
for some integer $q$.  Then $\gcd(k,n)$ must divide the linear
combination $k^{m-1}\cdot k  - q\cdot n = 1$, meaning $\gcd(k,n) = 1$,
as desired.

Alternatively, $k^{m-1}$ is a multiplicative inverse of $k$ modulo
$n$, so $k$ must be relatively prime to $n$ by
Theorem~\bref{thm:mod_inverses}.
\end{solution}

\ppart Show conversely that if $k$ and $n$ are relatively prime then
$\ordmod{k}{n}$ is finite.

\examspace[1.5in]

\begin{solution}
Because $\gcd(k,n) = 1$, by Euler's theorem, we have
$k^{\phi(n)}\equiv 1\pmod n$.  So $\ordmod{k}{n}$ is at most $\phi(n)$
and is therefore finite.

Alternatively, by the Pigeonhole Principle, some pair of entries in
the list
\[
k^0,\ k^1,\ k^2,\ \dots,\ k^{n}
\]
must be congruent modulo $n$, because there are only $n$ possible
remainders and the list has $n+1$ entries.  In other words, $k^i
\equiv k^{i+m}\pmod n$ for some $i,m \in \Zintv{1}{n}$.  But $k$ is
cancellable modulo $n$ because $\gcd(k,n) = 1$, so $k^m \equiv 1\pmod
n$.  This means $\ordmod{k}{n} \le m$.
\end{solution}

\ppart Prove that if $k$ and $n$ are relatively prime, then
$\ordmod{k}{n}$ divides $\phi(n)$.

\hint Let $m = \ordmod{k}{n}$ and  divide $\phi(n)$ by $m$.  So
\[
\phi(n) = q\cdot m  + r \text{ where } 0 \leq r < m.
\]

\iffalse
\hint Let $m = \ordmod{k}{n}$, and consider the quotient and remainder
of $\phi(n)$ when divided by $m$.  In other words, write $\phi(n) =
q\cdot m + r$ where $r=\rem{\phi(n)}{m}$.
\fi

\examspace[2.5in]

\begin{solution}
\begin{proof}
\iffalse %Use if hint is omitted
Let $m = \ordmod{k}{n}$, and write $\phi(n) = q\cdot m + r$ for
integers $q$ (the quotient) and $r$ (the remainder) where $0 \le r <
m$.\fi

We have
\begin{align*}
1 &\equiv k^{\phi(n)} & \text{(Euler)}\\
  & = k^{qm + r} & \\
  & = (k^m)^q\cdot k^r & \\
  &\equiv 1^q \cdot k^r & \text{(Def of $m$)} \\
  &\equiv k^r.
\end{align*}
But $r < m$ and $m$ is the smallest \emph{positive} power of $k$
congruent to $1 \bmod n$, so $r$ must be $0$.  This means $\phi(n) =
qm$, so $m$ divides $\phi(n)$.
\end{proof}

\end{solution}

\eparts
\end{problem}


%%%%%%%%%%%%%%%%%%%%%%%%%%%%%%%%%%%%%%%%%%%%%%%%%%%%%%%%%%%%%%%%%%%%%
% Problem ends here
%%%%%%%%%%%%%%%%%%%%%%%%%%%%%%%%%%%%%%%%%%%%%%%%%%%%%%%%%%%%%%%%%%%%%

\endinput


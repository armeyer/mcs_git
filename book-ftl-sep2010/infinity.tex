\chapter{Infinite Sets}\label{cardinality_chap}

So you might be wondering how much is there to say about an infinite
set other than, well, it has an infinite number of elements.  Of
course, an infinite set does have an infinite number of elements, but
it turns out that not all infinite sets have the same size---some are
bigger than others!  And, understanding infinity is not as easy as you
might think.  Some of the toughest questions in mathematics involve
infinite sets.

Why should you care?  Indeed, isn't computer science only about finite
sets?  Not exactly.  For example, we deal with the set of natural
numbers~$\naturals$ all the time and it is an infinite set.  In fact,
that is why we have induction: to reason about predicates
over~$\naturals$.  Infinite sets are also important in
Part~\ref{part:probability} of the text when we talk about random
variables over potentially infinite sample spaces.

So sit back and open your mind for a few moments while we take a very
brief look at \term{infinity}.

\section{Injections, Surjections, and Bijections}

We know from Theorem~\ref{thm:PA} that if there is an injection or
surjection between two finite sets, then we can say something about
the relative sizes of the two sets.  The same is true for infinite
sets.  In fact, relations are the primary tool for determining the
relative size of infinite sets.

\begin{definition}
Given any two sets $A$ and~$B$, we say that
\begin{center}
\begin{tabular}{lp{3in}}
$A \surj B$     & iff there is a surjection from $A$ to~$B$, \\

$A \inj B$      & iff there is an injection from $A$ to~$B$, \\

$A \bij B$      & iff there is a bijection between $A$ and~$B$, and \\

$A \strict B$   & iff there is a surjection from $A$ to~$B$ but
  there is \emph{no} bijection from $B$ to~$A$.
\end{tabular}
\end{center}
\end{definition}

Restating Theorem~\ref{thm:PA} with this new terminology, we have:
\begin{theorem}\label{thm:PA2}
For any pair of \emph{finite} sets $A$ and~$B$,
\begin{align*}
    \card{A} \ge \card{B} & \qiff A \surj B, \\
    \card{A} \le \card{B} & \qiff A \inj B, \\
    \card{A} = \card{B} & \qiff A \bij B, \\
    \card{A} > \card{B} & \qiff A \strict B.
\end{align*}
\end{theorem}

Theorem~\ref{thm:PA2} suggests a way to generalize size comparisons
to infinite sets; namely, we can think of the relation~$\surj$ as an
``at least as big'' relation between sets, even if they are infinite.
Similarly, the relation~$\bij$ can be regarded as a ``same size''
relation between (possibly infinite) sets, and $\strict$~can be
thought of as a ``strictly bigger'' relation between sets.

Note that we haven't, and won't, define what the size of an infinite
set is.  The definition of infinite ``sizes'' is cumbersome and
technical, and we can get by just fine without it.  All we need are
the ``as big as'' and ``same size'' relations, $\surj$ and~$\bij$,
between sets.

But there's something else to watch out for.  We've referred
to~$\surj$ as an ``as big as'' relation and $\bij$ as a ``same size''
relation on sets.  Most of the ``as big as'' and ``same
size'' properties of $\surj$ and~$\bij$ on finite sets do carry over
to infinite sets, but \emph{some important ones don't}---as we're
about to show.  So you have to be careful: don't assume that $\surj$
has any particular ``as big as'' property on \emph{infinite} sets
until it's been proved.

Let's begin with some familiar properties of the ``as big as'' and
``same size'' relations on finite sets that do carry over exactly to
infinite sets:

\begin{theorem}\label{thm:infinite1}
For any sets, $A$, $B$, and~$C$,
\begin{enumerate}

\item
$\text{$A \surj B$ and $B \surj C$} \QIMPLIES A \surj C$.

\item
$\text{$A \bij B$ and $B \bij C$} \QIMPLIES A \bij C$.

\item
$A \bij B  \QIMPLIES B \bij A$.
\end{enumerate}
\end{theorem}

Parts 1 and~2 of Theorem~\ref{thm:infinite1} follow immediately from
the fact that compositions of surjections are surjections, and
likewise for bijections.  Part~3 follows from the fact that the
inverse of a bijection is a bijection.  We'll leave a proof of these
facts to the problems.

Another familiar property of finite sets carries over to infinite
sets, but this time it's not so obvious:

\begin{theorem}[Schr\"oder-Bernstein]\label{schroder-bernstein}
For any pair of sets $A$ and~$B$, if $A \surj B$ and $B \surj A$, then
$A \bij B$.
\end{theorem}

The Schr\"oder-Bernstein Theorem says that if $A$ is at least as big
as~$B$ and, conversely, $B$ is at least as big as~$A$, then $A$ is the
same size as~$B$.  Phrased this way, you might be tempted to take this
theorem for granted, but that would be a mistake.  For infinite sets
$A$ and~$B$, the Schr\"oder-Bernstein Theorem is actually pretty
technical.  Just because there is a surjective function $f: A \to
B$---which need not be a bijection---and a surjective function $g: B
\to A$---which also need not be a bijection---it's not at all clear
that there must be a bijection $h: A \to B$.  The challenge is to
construct~$h$ from parts of both $f$ and~$g$.  We'll leave the actual
construction to the problems.

\subsection{Infinity Is Different}

A basic property of finite sets that does \emph{not} carry over to
infinite sets is that adding something new makes a set bigger.  That
is, if $A$ is a finite set and $b \notin A$, then $\card{A \union
  \set{b}} = \card{A} + 1$, and so $A$ and $A \union \set{b}$ are not
the same size.  But if $A$ is infinite, then these two sets \emph{are}
the same size!

\begin{theorem}\label{thm:infinite2}
Let $A$ be a set and $b \notin A$.  Then $A$ is infinite iff $A \bij A
\union \set{b}$.
\end{theorem}

\begin{proof}
Since $A$ is \emph{not} the same size as $A \union \set{b}$ when $A$
is finite, we only have to show that $A \union \set{b}$ \emph{is} the
same size as~$A$ when $A$~is infinite.

That is, we have to find a bijection between $A \union \set{b}$
and~$A$ when $A$~is infinite.  Since $A$ is infinite, it certainly has
at least one element; call it~$a_0$.  Since $A$~is infinite, it has at
least two elements, and one of them must not be equal to~$a_0$; call
this new element~$a_1$.  Since $A$ is infinite, it has at least three
elements, one of which must not equal $a_0$ or~$a_1$; call this new
element~$a_2$.  Continuing in this way, we conclude that there is an
infinite sequence $a_0$, $a_1$, $a_2$, \dots, $a_n$, \dots, of
different elements of~$A$.  Now it's easy to define a bijection $f: A
\union \set{b} \to A$:
\begin{align*}
    f(b)    &\eqdef a_0, \\
    f(a_n)  &\eqdef a_{n + 1}   && \text{for $n \in \naturals$}, \\
    f(a)    &\eqdef a           && \text{for $a \in A - \set{b, a_0,
        a_1, \dots}$}. \qedhere
\end{align*}
\end{proof}

\section{Countable Sets}\label{sec:countable}

\subsection{Definitions}

A set~$C$ is \term{countable} iff its elements can be listed in order,
that is, the distinct elements in~$C$ are precisely
\begin{equation*}
    c_0, c_1, \dots, c_n, \dots.
\end{equation*}
This means that if we defined a function~$f$ on the nonnegative
integers by the rule that $f(i) \eqdef c_i$, then $f$~would be a
bijection from $\naturals$ to~$C$.  More formally,

\begin{definition}\label{def:countable}
A set~$C$ is \term{countably infinite} iff $\naturals \bij C$.  A set
is \term{countable} iff it is finite or countably infinite.
\end{definition}

Discrete mathematics is often defined as the mathematics of countable
sets and so it is probably worth spending a little time understanding
what it means to be countable and why countable sets are so special.
For example, a small modification
of the proof of Theorem~\ref{thm:infinite2} shows that countably
infinite sets are the ``smallest'' infinite sets; namely, if $A$ is
any infinite set, then $A \surj \naturals$.

\subsection{Unions}

Since adding one new element to an infinite set doesn't change its
size, it's obvious that neither will adding any \emph{finite} number
of elements.  It's a common mistake to think that this proves that you
can throw in countably infinitely many new elements---just because
it's ok to do something any finite number of times doesn't make it ok
to do it an infinite number of times.

For example, suppose that you have two countably infinite sets $A =
\set{a_0, a_1, a_2, \dots}$ and $B = \set{b_0, b_1, b_2, \dots}$.  You
might try to show that $A \union B$ is countable by making the
following ``list'' for~$A \union B$:
\begin{equation}\label{eqn:13D1}
    a_0, a_1, a_2, \dots, b_0, b_1, b_2, \dots
\end{equation}
But this is not a valid argument because Equation~\ref{eqn:13D1} is
not a list.  The key property required for listing the elements
in a countable set is that for any element in the set, you can
determine its finite index in the list.  For example, $a_i$~shows up
in position~$i$ in Equation~\ref{eqn:13D1}, but there is no index in
the supposed ``list'' for any of the~$b_i$.  Hence,
Equation~\ref{eqn:13D1} is not a valid list for the purposes  of
showing that $A \union B$ is countable when $A$ is infinite.
Equation~\ref{eqn:13D1} is only useful when $A$~is finite.

It turns out you really can add a countably infinite number of new
elements to a countable set and still wind up with just a countably
infinite set, but another argument is needed to prove this.

\begin{theorem}\label{thm:countable_union}
If $A$ and~$B$ are countable sets, then so is~$A \union B$.
\end{theorem}

\begin{proof}

Suppose the list of distinct elements of~$A$ is $a_0$, $a_1$, \dots,
and the list of~$B$ is $b_0$, $b_1$, \dots.  Then a valid way to list
all the elements of~$A \union B$ is
\begin{equation}\label{eqn:countable_union}
    a_0, b_0, a_1, b_1, \dots, a_n, b_n, \dots.
\end{equation}
Of course this list will contain duplicates if $A$ and~$B$ have
elements in common, but then deleting all but the first occurrence of
each element in Equation~\ref{eqn:countable_union} leaves a list of
all the distinct elements of $A$ and~$B$.
\end{proof}

Note that the list in Equation~\ref{eqn:countable_union} does not have
the same defect as the purported ``list'' in Equation~\ref{eqn:13D1},
since every item in~$A \union B$ has a finite index in the list
created in Theorem~\ref{thm:countable_union}.

\subsection{Cross Products}

Somewhat surprisingly, cross products of countable sets are also
countable.  At first, you might be tempted to think that ``infinity
times infinity'' (whatever that means) somehow results in a larger
infinity, but this is not the case.

\begin{theorem}\label{thm:countable_products}\label{thm:13D5}
The cross product of two countable sets is countable.
\end{theorem}

\begin{proof}
Let $A$ and~$B$ be any pair of countable sets.  To show that $C = A
\cross B$ is also countable, we need to find a listing of the elements
\begin{equation*}
    \set{\, (a, b) \mid a \in A, b \in B \,}.
\end{equation*}
There are many such listings.  One is shown in Figure~\ref{fig:13D6}
for the case when $A$ and~$B$ are both infinite sets.  In this
listing, $(a_i, b_j)$ is the $k$th element in the list for~$C$ where
\begin{align*}
    & \text{$a_i$ is the $i$th element in~$A$,} \\
    & \text{$b_j$ is the $j$th element in~$B$, and} \\
    & k = \max(i,j)^2 + i + \max(i - j, 0).
\end{align*}
The task of finding a listing when one or both of $A$ and~$B$ are
finite is left to the problems at the end of the chapter.

\begin{figure}\redrawntrue

\begin{equation*}
\begin{array}{c|ccccc}
\multicolumn{1}{c}{} & b_0    & b_1       & b_2       & b_3    & \dots \\
\cline{2-6}
a_0    & c_0    & c_1       & c_4       & c_9    \\
a_1    & c_3    & c_2       & c_5       & c_{10} \\
a_2    & c_8    & c_7       & c_6       & c_{11} \\
a_3    & c_{15} & c_{14}    & c_{13}    & c_{12} \\
\vdots &        &           &           &       & \ddots
\end{array}
\end{equation*}

\caption{A listing of the elements of $C = A \cross B$ where $A =
  \set{a_0, a_1, a_2, \dots}$ and $B = \set{b_0, b_1, b_2, \dots}$ are
  countably infinite sets.  For example, $c_5 = (a_1, b_2)$.}

\label{fig:13D6}

\end{figure}

\end{proof}

\subsection{$\rationals$ Is Countable}

Theorem~\ref{thm:countable_products} also has a surprising Corollary;
namely that the set of rational numbers is countable.

\begin{corollary}\label{cor:countable_rationals}
The set of rational numbers~$\rationals$ is countable.
\end{corollary}

\begin{proof}
Since $\integers \cross \integers$ is countable by
Theorem~\ref{thm:countable_products}, it suffices to find a
surjection~$f$ from $\integers \cross \integers$ to~$\rationals$.
This is easy to to since
\begin{equation*}
f(a, b) = \begin{cases}
            a/b & \text{if $b \ne 0$} \\
              0 & \text{if $b = 0$}
          \end{cases}
\end{equation*}
is one such surjection.
\end{proof}

At this point, you may be thinking that every set is countable.  That
is \emph{not} the case.  In fact, as we will shortly see, there are
many infinite sets that are uncountable, including the set of real
numbers~$\reals$.

\section{Power Sets Are Strictly Bigger}

It turns out that the ideas behind Russell's Paradox, which caused so
much trouble for the early efforts to formulate Set Theory, also lead
to a correct and astonishing fact discovered by Georg Cantor in the
late nineteenth century: infinite sets are \emph{not all the same
  size}.

\begin{theorem}\label{thm:power_sets}
For any set~$A$, the power set~$\power(A)$ is strictly bigger
than~$A$.
\end{theorem}

\begin{proof}

First of all, $\power(A)$ is as big as~$A$: for example, the partial
function $f : \power(A) \to A$ where $f(\set{a}) \eqdef a$ for $a \in
A$ is a surjection.

To show that $\power(A)$ is strictly bigger than~$A$, we have to show
that if $g$~is a function from~$A$ to~$\power(A)$, then $g$~is not a
surjection.  So, mimicking Russell's Paradox, define
\begin{equation*}
    A_g \eqdef \set{\, a \in A \mid a \notin g(a) \,}.
\end{equation*}
$A_g$ is a well-defined subset of~$A$, which means it is a member
of~$\power(A)$.  But $A_g$ can't be in the range of~$g$, because if it
were, we would have
\begin{equation*}
    A_g = g(a_0)
\end{equation*}
for some $a_0 \in A$. So by definition of~$A_g$,
\begin{equation*}
    a \in g(a_0) \qiff a \in A_g \qiff a \notin g(a)
\end{equation*}
for all $a \in A$.  Now letting $a = a_0$ yields the contradiction
\begin{equation*}
    a_0 \in g(a_0) \qiff a_0 \notin g(a_0).
\end{equation*}
So $g$~is not a surjection, because there is an element in the power
set of~$A$, namely the set~$A_g$, that is not in the range of~$g$.
\end{proof}

\subsection{$\reals$ Is Uncountable}

To prove that the set of real numbers is uncountable, we will 
show that there is a surjection from~$\reals$ to~$\power(\naturals)$
and then apply Theorem~\ref{thm:power_sets} to~$\power(\naturals)$.

\begin{lemma}\label{lem:13D7}
$\reals \surj \power(\naturals)$.
\end{lemma}

\begin{proof}
Let $A \subset \naturals$ be any subset of the natural numbers.  Since
$\naturals$ is countable, this means that $A$ is countable and thus
that $A = \set{a_0, a_1, a_2, \dots}$.  For each $i \ge 0$, define
$\bin(a_i)$ to be the binary representation of~$a_i$.  Let $x_A$ be
the real number using only digits 0, 1, 2 as follows:
\begin{equation}\label{eqn:13D9}
    x_A \eqdef 0.2 \bin(a_0) 2 \bin(a_1) 2 \bin(a_2) 2 \dots
\end{equation}
We can then define a surjection $f : \reals \to \power(\naturals)$ as
follows:
\begin{equation*}
f(x) = \begin{cases}
        A & \text{if $x = x_A$ for some $A \in \naturals$}, \\
        0 & \text{otherwise}.
       \end{cases}
\end{equation*}
Hence $\reals \surj \power(\naturals)$.
\end{proof}

\begin{corollary}\label{cor:reals_uncountable}
$\reals$ is uncountable.
\end{corollary}

\begin{proof}
By contradiction.  Assume $\reals$ is countable.  Then $\naturals
\surj \reals$.  By Lemma~\ref{lem:13D7}, $\reals \surj
\power(\naturals)$.  Hence $\naturals \surj \power(\naturals)$.  This
contradicts Theorem~\ref{thm:power_sets} for the case when $A =
\naturals$.
\end{proof}

So the set of rational numbers and the set of natural numbers have the
same size, but the set of real numbers is strictly larger.  In fact,
$\reals \bij \power(N)$, but we won't prove that here.

Is there anything bigger?

\subsection{Even Larger Infinities}

There are lots of different sizes of infinite sets.  For example,
starting with the infinite set~$\naturals$ of nonnegative integers, we
can build the infinite sequence of sets
\begin{equation*}
    \naturals, \; \power(\naturals), \; \power(\power(\naturals)), \;
    \power(\power(\power(\naturals))), \dots
\end{equation*}
By Theorem~\ref{thm:power_sets}, each of these sets is strictly bigger
than all the preceding ones.  But that's not all, the union of all the
sets in the sequence is strictly bigger than each set in the sequence.
In this way, you can keep going, building still bigger infinities.

\subsection{The Continuum Hypothesis}

Georg Cantor was the mathematician who first developed the theory of
infinite sizes (because he thought he needed it in his study of
Fourier series).  Cantor raised the question whether there is a set
whose size is strictly between the ``smallest'' infinite set,
$\naturals$, and~$\power(\naturals)$. He guessed not:
\begin{cont_hypo*}
There is no set~$A$ such that $\power(\naturals)$ is strictly bigger
than~$A$ and $A$~is strictly bigger than~$\naturals$.
\end{cont_hypo*}

The Continuum Hypothesis remains an open problem a century later.  Its
difficulty arises from one of the deepest results in modern Set
Theory---discovered in part by G\"odel in the 1930s and Paul Cohen in
the 1960s---namely, the ZFC axioms are not sufficient to settle the
Continuum Hypothesis: there are two collections of sets, each obeying
the laws of ZFC, and in one collection, the Continuum Hypothesis is
true, and in the other, it is false.  So settling the Continuum
Hypothesis requires a new understanding of what sets should be to
arrive at persuasive new axioms that extend ZFC and are strong enough
to determine the truth of the Continuum Hypothesis one way or the
other.

\section{Infinities in Computer Science}

If the romance of different size infinities and continuum hypotheses
doesn't appeal to you, not knowing about them is not going to lower
your professional abilities as a computer scientist.  These abstract
issues about infinite sets rarely come up in mainstream mathematics,
and they don't come up at all in computer science, where the focus is
generally on countable, and often just finite, sets.  In practice,
only logicians and set theorists have to worry about collections that
are too big to be sets.  In fact, at the end of the 19th century, even
the general mathematical community doubted the relevance of what they
called ``Cantor's paradise'' of unfamiliar sets of arbitrary infinite
size.

That said, it is worth noting that the proof of
Theorem~\ref{thm:power_sets} gives the simplest form of what is known
as a ``diagonal argument.''  Diagonal arguments are used to prove many
fundamental results about the limitations of computation, such as the
undecidability of the Halting Problem for programs and the inherent,
unavoidable inefficiency (exponential time or worse) of procedures for
other computational problems.  So computer scientists do need to study
diagonal arguments in order to understand the logical limits of
computation.  Ad a well-educated computer scientist will be
comfortable dealing with countable sets, finite as well as infinite.

\problemsection

\endinput

\documentclass[problem]{mcs}

\begin{pcomments}
  \pcomment{FP_asymptotics_short_answer}
  \pcomment{zabel, Jerry Li, and Hongyu Yang for s18 mid4}
  \pcomment{loosely based on PS_asymptotics_and_stirlings and CP_asymptotic_equality_properties} 
\end{pcomments}

\pkeywords{
  asymptotics
  stirling
  little_oh
  Theta
  big_Oh
}

%%%%%%%%%%%%%%%%%%%%%%%%%%%%%%%%%%%%%%%%%%%%%%%%%%%%%%%%%%%%%%%%%%%%%
% Problem starts here
%%%%%%%%%%%%%%%%%%%%%%%%%%%%%%%%%%%%%%%%%%%%%%%%%%%%%%%%%%%%%%%%%%%%%
\begin{problem}
Include brief \textbf{explanations} with your answers to each of
following questions.
  
\bparts

\ppart Let $h(x) = (\log_2 x)^{3}\cdot(x+2)^{3}$.  Is $h(x) = O(x^3)$?
Is $h(x) = O(x^{3.1})$?

\begin{solution}
\textbf{No}, $h(x)$ is \textbf{not} $O(x^3)$ because
\[
\lim_{x\to\infty} \frac{h(x)}{x^3}
= \lim_{x\to\infty} (\log_2 x)^3 \cdot\frac{(x+2)^3}{x^3} = \infty
\]
because the first term tends toward $\infty$ and the second term tends
toward $1$.  On the other hand,

\textbf{Yes}, $h(x)$ \textbf{is} $O(x^{3.1})$, in fact is
$o(x^{3.1})$, because
\[
\lim_{x\to\infty} \frac{h(x)}{x^{3.1}}
= \lim_{x\to\infty} \frac{(\log_2 x)^3}{x^{0.1}} \cdot\lim_{x\to\infty}\frac{(x+2)^3}{x^3}
= 0 \cdot 1 = 0.
\]
\end{solution}

\examspace[1.5in]

\ppart Is it true that $x \log_2 x \sim x \ln x$?  Is it true that
$x\log_2 x = \Theta(x\ln x)$?

\begin{solution}
Theta $\Theta$: \textbf{Yes}.  Asymptotic equality $\sim$: \textbf{No}.

We have $x\log_2 x = (x\ln x)/(\ln 2)$, so the functions $x\log_2 x$
and $x\ln x$ have a constant ratio of $\ln 2 \ne 1$.  This means
$x\log_2 x = \Theta(x\ln x)$ is true but that the functions are not
asymptotically equivalent because their ratio in the limit is not
equal to one.
\end{solution}
  
  % \ppart Consider the sum
  % \begin{equation*}
  %   S(n) \eqdef \sum_{k=1}^n k^2\cdot 2^k.
  % \end{equation*}
  % Is $S(n) = O(2.01^n)$? Is $S(n) = O(2^n)$?  
  % \hint You don't need to evaluate the sum.

\examspace[1.5in]

ppart If $f,g:\nngint^+\to\nngint^+$ and $f\sim g$, must $f^2$ and
$g^2$ be asymptotically equal?

\begin{solution}
\textbf{Yes:}
\[
\lim_{n\to\infty} \frac{f(n)^2}{g(n)^2}
= \lim_{n\to\infty} \frac{f(n)}{g(n)} \cdot \frac{f(n)}{g(n)}
= \lim_{n\to\infty} \frac{f(n)}{g(n)} \cdot \lim_{n\to\infty} \frac{f(n)}{g(n)}
= 1 \cdot 1 = 1.
\]
\end{solution}

\examspace[1.5in]

\ppart If $f,g:\nngint^+\to\nngint^+$ and $f\sim g$, must $2^f$ and
$2^g$ be asymptotically equal?  \hint No.

\begin{solution}
\textbf{No}.  A simple counterexample is
\begin{align*}
  f(n) & \eqdef n+1\\ g(n) & \eqdef n.
\end{align*}
Then $f \sim g$ since $\lim (n+1)/n = 1$, but $2^f= 2^{n+1} = 2\cdot 2^n =
2\cdot 2^g$ so
\[
\lim \frac{2^f}{2^g} = 2 \neq 1.
\]

There are lots of other examples, such as
\[
f(n) \eqdef n,\ g(n) \eqdef n + \log n.
\]
Then
\[
\frac{2^{f(n)}}{2^{g(n)}} = \frac{2^n}{2^{n + \log n}} = \frac{2^n}{2^n2^{\log n}} = \frac{1}{n},
\]
so
\[
\lim_{n\to\infty} \frac{2^{f(n)}}{2^{g(n)}} = \lim_{n\to\infty} \frac{1}{n} = 0 \neq 1
\]
which means in fact that, $2^f = o(2^g)$.
\end{solution}
 
\eparts

\end{problem}

%%%%%%%%%%%%%%%%%%%%%%%%%%%%%%%%%%%%%%%%%%%%%%%%%%%%%%%%%%%%%%%%%%%%%
% Problem ends here
%%%%%%%%%%%%%%%%%%%%%%%%%%%%%%%%%%%%%%%%%%%%%%%%%%%%%%%%%%%%%%%%%%%%%

\endinput

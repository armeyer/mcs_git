\documentclass[problem]{mcs}

\begin{pcomments}
  \pcomment{CP_shortest_directed_closed_walk}
  \pcomment{subsumes part (a) of CP_directed_walks_and_cycles}
  \pcomment{by ARM 3/11/11}
\end{pcomments}

\pkeywords{
  walk
  path
  cycle
  closed_walk
  digraph
}

%%%%%%%%%%%%%%%%%%%%%%%%%%%%%%%%%%%%%%%%%%%%%%%%%%%%%%%%%%%%%%%%%%%%%
% Problem starts here
%%%%%%%%%%%%%%%%%%%%%%%%%%%%%%%%%%%%%%%%%%%%%%%%%%%%%%%%%%%%%%%%%%%%%

\begin{problem}

\bparts

\ppart Give an example of a digraph that has a closed walk including
two vertices but has no cycle including those vertices.

\begin{staffnotes}
\hint There is an example with 3 vertices.
\end{staffnotes}

\begin{solution}

Let the vertices be $a,b,c$ and edges be $(a,b), (b,a), (b,c), (c,b)$.
Now $a$ and $c$ are on the closed walk $a,b,c,b,a$, but every closed
walk from $a$ to $c$ must go through $b$ at least twice, and so will
not be a cycle.

\iffalse

%%v on even length closed walk but no even length cycle.


\begin{figure}
\graphic{Fig_walkpath}
\caption{A closed walk but no cycle through $u$ and $v$.}
\label{fig:walkpath}
\end{figure}

As in Figure~\ref{fig:walkpath}, let
\begin{align*}
V & \eqdef \set{u,v,w,x},\\
E & \eqdef \set{\diredge{u}{w}, \diredge{w}{x}, \diredge{x}{u}, \diredge{v}{w}, \diredge{x}{v}}.
\end{align*}
There is a path
\[
u \diredge{u}{w} w \diredge{w}{x} x \diredge{x}{v}
\]
from~$u$ to $v$, and a path
\[
v \diredge{v}{w} w \diredge{w}{x} x \diredge{x}{u} u
\]
from~$v$ to~$u$, but it is easy to see that there is no \emph{cycle}
from~$u$ to~$u$ \emph{that contains~$v$}.  The reason is that the sole
edge out of $u$ goes to $w$, and the sole edge out of~$v$ likewise
goes to $w$, so any walk from~$u$ to~$u$ that goes through~$v$ must go
through $w$ at least twice and therefore won't be a cycle.
\fi


\end{solution}

\ppart Prove Lemma~\bref{shortestclosedwalk_lem}:
\begin{lemma*}
The shortest positive length closed walk through a vertex is a cycle.
\end{lemma*}

\begin{solution}
\begin{proof}
  Suppose $\walkv{w}$ is a minimum positive length walk from~$u$
  to~$u$.  We claim $\walkv{w}$ is a cycle.

  To prove the claim, suppose to the contrary that $\walkv{w}$ is not
  a cycle.  One way the walk could fail to be a cycle is because some
  vertex besides $u$ occurs twice on the walk:

\textbf{case} (some vertex $x \neq u$ occurs twice in $\walkv{w}$):
Then
\[
\walkv{w} = \catv{\catv{\walkv{e}}{x}{\walkv{f}}}{x}{\walkv{g}}
\]
for some positive length walks $\walkv{e}, \walkv{f}, \walkv{g}$.  But
then ``deleting'' $\walkv{f}$ yields a strictly shorter walk, namely
\[
\catv{\walkv{e}}{x}{\walkv{g}}
\]
is a shorter walk from~$u$ to~$u$, again contradicting the minimality
of $\walkv{w}$.

The other possibility is that no vertex besides $u$ appears two or
more times, but $u$ appears more that two times:

\textbf{case} ($u$ occurs more than two times in $\walkv{w}$): This
means that
 \[
\walkv{w}= \catv{\walkv{e}}{u}{\walkv{f}}
\]
where both $\walkv{e}$ and $\walkv{f}$ have positive length.  Then
$\walkv{e}$ is a shorter positive length walk from $u$ to $u$,
contradicting the minimality of $\walkv{w}$.
\end{proof}

\end{solution}

\eparts

\end{problem}

%%%%%%%%%%%%%%%%%%%%%%%%%%%%%%%%%%%%%%%%%%%%%%%%%%%%%%%%%%%%%%%%%%%%%
% Problem ends here
%%%%%%%%%%%%%%%%%%%%%%%%%%%%%%%%%%%%%%%%%%%%%%%%%%%%%%%%%%%%%%%%%%%%%

\endinput

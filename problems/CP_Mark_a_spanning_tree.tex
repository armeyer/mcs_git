\documentclass[problem]{mcs}

\begin{pcomments}
  \pcomment{from: S09.cp6t}
  \pcomment{commented out in S09 so check/revise before using}
\end{pcomments}

\pkeywords{
  trees
  spanning_trees
  state_machines
  graphs
}

%%%%%%%%%%%%%%%%%%%%%%%%%%%%%%%%%%%%%%%%%%%%%%%%%%%%%%%%%%%%%%%%%%%%%
% Problem starts here
%%%%%%%%%%%%%%%%%%%%%%%%%%%%%%%%%%%%%%%%%%%%%%%%%%%%%%%%%%%%%%%%%%%%%

\begin{problem}
Procedure $Mark$ starts with a connected, simple graph with all edges
unmarked and then marks some edges.  At any point in the procedure a path
that traverses only marked edges is called a \emph{fully marked} path, and
an edge that has no fully marked path between its endpoints is called
\emph{eligible}.

Procedure $Mark$ simply keeps marking eligible edges, and terminates when
there are none.

Prove that $Mark$ terminates, and that when it does, the set of marked
edges forms a spanning tree of the original graph.

\begin{solution}
As a state machine, the start state of $Mark$ is some given
connected graph, $G$.  The rest of the states are copies of $G$ with some
edges marked.

$Mark$ terminates because the number of unmarked edges decreases by one at
each transition, so this number is a strictly decreasing nonnegative
integer-valued variable, which we know implies termination.  (A common
mistake in arguing termination of $Mark$ was to instead say that the number
of eligible edges was strictly decreasing, without any additional
reasoning.  This is true, but can't be taken for granted: you have to
explain why removing an eligible edge does not result in new edges becoming
eligible.)

To prove partial correctness, we show if $Mark$ terminates, the marked
edges of the final state form a spanning tree of $G$.  So we must show
that the marked edges form an acyclic connected graph with the same set of
vertices as $G$.

To do this we verify the invariant:
\begin{quote}
The marked edges form an acyclic graph.  (*)
\end{quote}

To verify~(*) is an invariant, consider a step $H\to H'$, where $H$
satisfies~(*).  This means that $H$ has no fully marked cycles, and $H'$
is the same as $H$ with an edge, $e$, that was one eligible in $H$, now
marked in $H'$.  But in $H'$, the only fully marked path between the
endpoints of $e$ must be the edge $e$ itself, by definition of
``eligible.''  So $e$ is not on a fully marked simple cycle in $H'$.
Since $H$ and $H'$ are otherwise the same, there is no fully marked simple
cycle elsewhere in $H'$.  That is, $H'$ satisfies~(*).

Since the start state $G$ has no marked edges, it satisfies~(*) trivially.
Hence, by the Invariant Principle, any final state of $Mark$
satisfies~(*).

We also claim that in any final state, there is a fully marked path
between any two vertices.  To prove this assume to the contrary that there
were two vertices, $u$ and $v$, with no fully marked path between them.
Since there is a path in $G$ between $u$ and $v$, there must be a path
between $u$ and $v$ traversing the smallest number of unmarked edges, and
this path must contain at least one unmarked edge, $e$.  Now if there was
a fully marked path between the endpoints of $e$, we could replace $e$ by
this path to obtain a path between $u$ and $v$ with fewer unmarked edges.
So there can't be a fully marked path between the endpoints of $e$, which
means that $e$ is eligible, contradicting the fact the state was final.

So in any final state, the marked edges determine an acyclic, connected
graph on the vertices of $G$.  That is, the marked edges determine a
spanning tree of $G$.

\end{solution}

\end{problem}

%%%%%%%%%%%%%%%%%%%%%%%%%%%%%%%%%%%%%%%%%%%%%%%%%%%%%%%%%%%%%%%%%%%%%
% Problem ends here
%%%%%%%%%%%%%%%%%%%%%%%%%%%%%%%%%%%%%%%%%%%%%%%%%%%%%%%%%%%%%%%%%%%%%

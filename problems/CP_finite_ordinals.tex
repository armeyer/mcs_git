\documentclass[problem]{mcs}

\begin{pcomments}
  \pcomment{CP_finite_ordinals}
  \pcomment{ARM 2/17/13}
  \pcomment{revise to use WOP instead of Foundation}
\end{pcomments}

\pkeywords{
  sets
  set_theory
  subset
  power_set
  union
  ordinal
  member
  contains
}

\newcommand{\nextset}[1]{\text{next}(#1)}

\begin{problem}
For any set $x$, define $\nextset{x}$ to be the set consisting of all
the elements of $x$, along with $x$ itself:
\[
\nextset{x}  \eqdef x \union \set{x}
\]
Now we can define a sequence of sets $\nu_0, \nu_1, \nu_2, \dots$
called the \term{finite ordinals} with a simple recursive recipe:
\begin{align*}
\nu_0    & \eqdef \emptyset,\\
\nu_{n+1} & \eqdef \nextset{\nu_n}.
\end{align*}
So we have,
\begin{align*}
\nu_1 & \eqdef \set{\emptyset}\\ %= \set{\nu_0}
\nu_2 & \eqdef \set{\emptyset, \set{\emptyset}}\\  %= \set{\nu_0, \nu_1}
\nu_3 & \eqdef \set{\emptyset, \set{\emptyset}, \set{\emptyset, \set{\emptyset}}} %= \set{\nu_0, \nu_1, \nu_2}
\end{align*}

The finite ordinals are kind of weird, but have some engaging
properties, and more important, they turn out to play a significant
role in set theory.

\bparts
\ppart Prove that 
\begin{equation}\label{nun+1}
\nu_{n+1} = \set{\nu_0, \nu_1,\dots, \nu_n}.
\end{equation}

\begin{solution}

\begin{proof}
Proof by contradiction using WOP.

Suppose equation~\eqref{nun+1} fails for some nonnegative integer,
$n$.  The by WOP, there is a least integer $m$ for which it fails.

Now~\eqref{nun+1} holds for $n=0$ since
\[
\nu_{0+1}
  = \nu_1
  \eqdef \nextset{\nu_0}
  \eqdef \nu_0 \union \set{\nu_0}
  = \emptyset \union \set{\nu_0}
  = \set{\nu_0}.
\]
So $m \geq 1$.

Since $m$ is minimal and $m-1 \geq 0$, equation~\eqref{nun+1} must
hold for $m-1$, namely
\begin{equation}\label{numnum11}
\nu_m = \nu_{(m-1)+1} = \set{\nu_0, \nu_1,\dots, \nu_{m-1}}
\end{equation}

But then
\begin{align*}
\nu_{m+1}
  & \eqdef \nextset{\nu_m}\\
  & \eqdef \nu_m \union \set{\nu_m}\\
  & = \set{\nu_0, \nu_1,\dots, \nu_{m-1}} \union \set{\nu_m}
       & \text{(by \eqref{numnum11})}\\
  & = \set{\nu_0, \nu_1,\dots, \nu_{m-1}, \nu_m}.
\end{align*}
So in fact $m$ also satisfies~\eqref{nun+1}, a contradiction.
Hence, equation~\eqref{nun+1} must hold for all $n\in \nngint$.
\end{proof}

\end{solution}


\iffalse
\eparts

Now equation~\eqref{nun+1} implies that
\begin{align*}
\nu_m \in \nu_n & \QIFF\ m < n\\o
\nu_m \subseteq \nu_n & \QIFF\ m \leq n.
\end{align*}

\bparts
\fi

\ppart Conclude that $\card{\nu_n}=n$.

\hint A set cannot be a member of itself.\footnote{By the Foundation Axiom,
  Section~\bref{ZFC_sec}.}

\begin{editingnotes}
Prove by WOP; no need for Foundation.
\end{editingnotes}

\begin{staffnotes}
You can't claim this just from equation~\eqref{nun+1}: you need to
be sure that all the elements are different.
\end{staffnotes}

\begin{solution}
Clearly, $\card{\nu_0} =0$ by definition, and for $n>0$, 
\[
\nu_{n} = \set{\nu_0, \nu_1,\dots, \nu_{n-1}}
\]
by~\eqref{nun+1}.  Therefore $\card{\nu_n} = \card{\set{\nu_0,
    \nu_1,\dots, \nu_{n-1}}} = n$ providing all the $\nu_i$'s
\emph{are different}.  But this follows immediately
from~\eqref{nun+1}, because if $i<j$, then $\nu_i \in \nu_j$, and so
$\nu_i$ must not equal $\nu_j$ or it would be a member of itself.
\end{solution}

\ppart Conclude that if $\mu,\nu, \rho$ are finite ordinals and $\mu
\in \nu \in \rho$, then $\mu \in \rho$.  Likewise, if $\mu,\nu$ are
different finite ordinals, then $\nu \in \mu \QOR\ \mu \in \nu$.

\begin{solution}
Now from equation~\eqref{nun+1} we have the $\nu_m \in \nu_n$ iff $m <
n$.  So $\mu \in \nu \in \rho$ is equivalent to
\[
\mu = \nu_m, \nu = \nu_n, \rho = \nu_r
\]
for some integers $m<n<r$.  But then $m<r$ so $\mu = \nu_m \in \nu_r =
\rho$.  That is, $\mu \in \rho$ as required.

Likewise, if $\mu$ and $\nu$ are distinct ordinals, then $\mu = \nu_m$
and $\nu = \nu_n$ for some nonnegative integers $m \neq n$.  But then
either $m < n$ or $n < m$, in which case $\mu \in \nu$ or $\nu \in
\mu$, respectively.
\end{solution}

\eparts
\end{problem}

\endinput

\documentclass[problem]{mcs}

\begin{pcomments}
  \pcomment{FP_counting_poker_high_cards}
  \pcomment{S15.final}
  \pcomment{edited ARM 12/13/15}
\end{pcomments}

\pkeywords{
  counting
  rank
  suit
  poker
  inclusion_exclusion
}

%%%%%%%%%%%%%%%%%%%%%%%%%%%%%%%%%%%%%%%%%%%%%%%%%%%%%%%%%%%%%%%%%%%%%
% Problem starts here
%%%%%%%%%%%%%%%%%%%%%%%%%%%%%%%%%%%%%%%%%%%%%%%%%%%%%%%%%%%%%%%%%%%%%

\begin{problem}
In a standard 52-card deck (13 ranks and 4 suits),
a hand is a 5-card subset of the set of 52 cards.
Express the answer to each part as a formula using
factorial, binomial, or multinomial notation.

\bparts

\ppart
Let $H$ be the set of all hands.
What is $\card{H}$?\hfill\examrule[1.0in]
\begin{solution}
\[
\card{H} = \binom{52}{5}
\]
\end{solution}

\examspace[0.3in]

\ppart
Let $H_{NP}$ be the set of all hands that include no pairs;
that is, no two cards in the hand have the same rank.
What is $\card{H_{NP}}$?  \hfill\examrule[1.0in]

\begin{solution}
\[
\card{H_{NP}} = \binom{13}{5} \binom{4}{1}^5
\]
\end{solution}

\examspace[0.2in]

\ppart Let $H_{S}$ be the set of all hands that are straights, that
is, the ranks of the five cards are consecutive.  The order of the
ranks is $(A,2,3,4,5,6,7,8,9,10,J,Q,K,A)$; note that $A$ appears
twice.\\
What is $\card{H_{S}}$?\hfill\examrule[1.0in]

\begin{solution}
\[
\card{H_{S}} = \binom{10}{1} \binom{4}{1}^5
\]
\end{solution}

\examspace[0.4in]

\ppart Let $H_{F}$ be the set of all hands that are flushes, that is,
the suits of the five cards are identical.
What is $\card{H_{F}}$? \hfill\examrule[1.0in]

\begin{solution}
\[
\card{H_{F}} = \binom{4}{1}\binom{13}{5} 
\]
\end{solution}

\examspace[0.4in]

\ppart Let $H_{SF}$ be the set of all straight flush hands, that is,
the hand is both a straight and a flush.
What is $\card{H_{SF}}$? \hfill\examrule[1.0in]

\begin{solution}
\[
\card{H_{SF}} = \binom{10}{1} \binom{4}{1}
\]
\end{solution}

\examspace[0.4in]

\ppart Let $H_{HC}$ be the set of all high-card hands; that is, hands
that do not include pairs, are not straights, and are not flushes.
Write a formula for $\card{H_{HC}}$ in terms of $\card{H_{NP}},
\card{H_{S}}, \card{H_{F}}, \card{H_{SF}}$.

\exambox{4.0in}{0.4in}{0in}

\begin{solution}
\begin{equation}\label{HHCHNP}
\card{H_{HC}}  = \card{H_{NP}} - \card{H_{S}} - \card{H_{F}} + \card{H_{SF}}.
\end{equation}

This follows because
\[
H_\text{NP} = H_S \union H_F \union H_{HC}
\]
by definition of the poker hands.  Also, $H_{HC}$ is disjoint from
$H_S$ and $H_F$.  So by inclusion exclusion,
\[
\card{H_{NP}}  = \card{H_{HC}} + \card{H_S} + \card{H_F} - 0 - 0 - \card{H_{SF}} + 0,
\]
which immediately yields~\eqref{HHCHNP}.

\iffalse
\begin{align*}
\card{H_{HC}} & = \card{H_{NP}} - \card{H_{S}} - \card{H_{F}} + \card{H_{SF}}\\
             & = \binom{13}{5} \binom{4}{1}^5 - \binom{10}{1}\binom{4}{1}^5\\
             & \quad  - \binom{13}{5} \binom{4}{1} + \binom{10}{1}\binom{4}{1}\\
             & = \paren{\binom{4}{1}^5 - \binom{4}{1}} \paren{\binom{13}{5} - \binom{10}{1}}
\end{align*}
\fi

\end{solution}

\eparts
\end{problem}

%%%%%%%%%%%%%%%%%%%%%%%%%%%%%%%%%%%%%%%%%%%%%%%%%%%%%%%%%%%%%%%%%%%%%
% Problem ends here
%%%%%%%%%%%%%%%%%%%%%%%%%%%%%%%%%%%%%%%%%%%%%%%%%%%%%%%%%%%%%%%%%%%%%

\endinput

\documentclass[problem]{mcs}

\begin{pcomments}
  \pcomment{PS_set_identity_counterexample}
  \pcomment{ARM 9/22/17}
\end{pcomments}

\pkeywords{
  logic
  set_theory
  identity
  propositional
  chain_of_iff
}

%%%%%%%%%%%%%%%%%%%%%%%%%%%%%%%%%%%%%%%%%%%%%%%%%%%%%%%%%%%%%%%%%%%%%
% Problem starts here
%%%%%%%%%%%%%%%%%%%%%%%%%%%%%%%%%%%%%%%%%%%%%%%%%%%%%%%%%%%%%%%%%%%%%

\begin{problem}
Looking at set expressions involving just the operations $\union,
\intersect, -$ and complement, Section~\bref{set_equality_sec}
explains how to reduce proving the equality of two such expressions to
verifying validity of a corresponding propositional equivalence.  The
same approach can provide counter-examples to set-theoretic equalities
that are not valid.

For example, a counter-example to the equality
\textcolor{red}{
\[
A-(B\union C) = A-B \union A-C,
\]
}
is $A = \set{1}, B = \set{1}, C = \emptyset$.  Now the left hand
expression evaluates to $\set{1} - (\set{1} \union \emptyset) =
\emptyset$, while the right hand expression evaluates to $\emptyset
\union (\set{1}-\emptyset = \set{1}$.

Explain how, given a set-theoretic equality between two set
expressions that is not valid, to construct a counter-example using
any truth assignment that falsifies the corresponding propositional
equivalence.  Conclude that any set equality that is valid in a domain
of size one will be valid in \emph{all} domains.
\begin{solution}
\TBA{TBA}.
\end{solution}

\end{problem}

%%%%%%%%%%%%%%%%%%%%%%%%%%%%%%%%%%%%%%%%%%%%%%%%%%%%%%%%%%%%%%%%%%%%%
% Problem ends here
%%%%%%%%%%%%%%%%%%%%%%%%%%%%%%%%%%%%%%%%%%%%%%%%%%%%%%%%%%%%%%%%%%%%%

\endinput

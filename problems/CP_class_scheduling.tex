%CP_class_scheduling

\documentclass[problem]{mcs}

\begin{pcomments}
  \pcomment{from: F04.rec9, revised by ARM 10/1/09}
\end{pcomments}

\pkeywords{
  partial_orders
  scheduling
  chain
  antichain
}

%%%%%%%%%%%%%%%%%%%%%%%%%%%%%%%%%%%%%%%%%%%%%%%%%%%%%%%%%%%%%%%%%%%%%
% Problem starts here
%%%%%%%%%%%%%%%%%%%%%%%%%%%%%%%%%%%%%%%%%%%%%%%%%%%%%%%%%%%%%%%%%%%%%

\begin{problem} The table below lists some prerequisite information for some subjects in
  the MIT Computer Science program (in 2006).  This defines an indirect
  prerequisite relation, $\prec$, that is a strict partial order on these subjects.
%
\begin{align*}
18.01 & \to 6.042 & 18.01 & \to 18.02 \\
18.01 & \to 18.03 & 6.046 & \to 6.840 \\
8.01 & \to 8.02 & 6.001 & \to 6.034 \\
6.042 & \to 6.046 & 18.03, 8.02 & \to 6.002 \\
6.001, 6.002 & \to 6.003 & 6.001, 6.002 & \to 6.004 \\
6.004 & \to 6.033 & 6.033 & \to 6.857
\end{align*}
\end{problem}

\bparts
\iffalse

\ppart Draw a Hasse diagram for the corresponding partially-ordered
set.  (A \term{Hasse diagram} is a way of representing a poset $(A,
\prec)$ as a directed acyclic graph.  The vertices are the element
of $A$, and there is generally an edge $u \to v$ if $u \prec v$.
However, self-loops and edges implied by transitivity are omitted.)
You'll need this diagram for all the subsequent problem parts, so be
neat!

\solution[\vspace{3in}]{

\mfigure{!}{2.5in}{rec9-hasse}
}
\fi

\ppart\label{sixterms} Explain why exactly six terms are required to finish all these
subjects, if you can take as many subjects as you want per term.
Using a \emph{greedy} subject selection strategy, you should take as many
subjects as possible each term.  Exhibit your complete class schedule each
term using a greedy strategy.

\begin{solution}
There is a $\prec$-chain of length six:
\[
8.01 \prec 8.02 \prec 6.002 \prec 6.004 \prec 6.033 \prec 6.857
\]
So six terms are necessary, because at most one of these
subjects can be taken each term.

There is no  longer chain, so with the greedy strategy you will take six terms.
Here are the subjects you take in successive terms.
\begin{center}
\begin{tabular}{lccccc}
1: & 6.001 &  8.01 & 18.01 \\
2: & 6.034 & 6.042 & 8.02 & 18.02 &  18.03\\
3: & 6.002 & 6.046 \\
4: & 6.003 & 6.004 & 6.840 \\
5: & 6.033\\
6: & 6.857
\end{tabular}
\end{center}

\end{solution}

\ppart In the second term of the greedy schedule, you took five subjects
including 18.03.  Identify a set of five subjects not including 18.03 that
it would be posssible to take in one term (using some nongreedy schedule).
Can you figure out how many such sets there are?

\begin{solution}
We're looking for an antichain in the $\prec$ relation that does not
include 18.03.  Every such antichain will have to include 18.02 6.003,
6.034.  Then a fourth subject could be any of 6.042, 6.048, and 6.840.
The fifth subject could then be any of 6.004, 6.033, and 6.857.  This
gives a total of nine antichains of five subjects.
\end{solution}


\ppart Exhibit a schedule for taking all the courses but only one per term.

\begin{solution}
We're asking for a topological sort of $\prec$.  There are many.  One is 18.01,
8.01, 6.001, 18.02, 6.042, 18.03, 8.02, 6.034, 6.046, 6.002, 6.840,
6.004, 6.003, 6.033, 6.857.
\end{solution}

\ppart Suppose that you want to take all of the subjects, but can
handle only two per term.  Exactly how many terms are required to
graduate?  Explain why.
\begin{solution}

There are $\ceil{15/2} =8$ subjects are necessary.  The schedule below
shows that 8 terms are sufficient as well:

\begin{center}
\begin{tabular}{rcc}
1: & 18.01 & 8.01 \\
2: & 6.001 & 18.02 \\
3: & 6.042 & 18.03 \\
4: & 8.02 & 6.034 \\
5: & 6.046 & 6.002 \\
6: & 6.840 & 6.004 \\
7: & 6.003 & 6.033 \\
8: & 6.857
\end{tabular}
\end{center}

\end{solution}

\ppart What if you could take three subjects per term?

\begin{solution}
  From part~\eqref{sixterms} we know six terms are required even if there
  is no limit on the number of subjects per term.  Six terms are also
  sufficient, as the following schedule shows:

\begin{center}
\begin{tabular}{rccc}
1: & 18.01 & 8.01 & 6.001 \\
2: & 6.042 & 18.03 & 8.02 \\
3: & 18.02 & 6.046 & 6.002 \\
4: & 6.004 & 6.003 & 6.034 \\
5: & 6.840 & 6.033 \\
6: & 6.857 \\
\end{tabular}
\end{center}

\end{solution}

\eparts

\end{problem}

%%%%%%%%%%%%%%%%%%%%%%%%%%%%%%%%%%%%%%%%%%%%%%%%%%%%%%%%%%%%%%%%%%%%%
% Problem ends here
%%%%%%%%%%%%%%%%%%%%%%%%%%%%%%%%%%%%%%%%%%%%%%%%%%%%%%%%%%%%%%%%%%%%%

\endinput

\documentclass[problem]{mcs}

\begin{pcomments}
  \pcomment{PS_dependent_pairs}
  \pcomment{sequel to PS_equal_birthdays}
  \pcomment{by ARM 4/28/11; corrected with complete soln 8/12/12}
  \pcomment{revised to stand alone, ARM 11/21/13}
  \pcomment{staff comment re intransitive independence, ARM 11/22/17}
\end{pcomments}

\pkeywords{
  random_variable
  independence
  mutual
  pairwise
  uniform
  distribution
}

%%%%%%%%%%%%%%%%%%%%%%%%%%%%%%%%%%%%%%%%%%%%%%%%%%%%%%%%%%%%%%%%%%%%%
% Problem starts here
%%%%%%%%%%%%%%%%%%%%%%%%%%%%%%%%%%%%%%%%%%%%%%%%%%%%%%%%%%%%%%%%%%%%%

\begin{problem}
Let $R$, $S$ and $T$ be mutually independent indicator variables.

In general, the event that $S=T$ is not independent of $R=S$.  We can
explain this intuitively as follows: suppose for simplicity that $S$
is uniform, that is, equally likely to be 0 or 1.  This implies that
$S$ is equally likely as not to equal $R$, that is $\pr{R=S} =1/2$;
likewise, $\pr{S=T} = 1/2$.

Now suppose further that both $R$ and $T$ are more likely to equal 1
than to equal 0.  This implies that $R=S$ makes it more likely than
not that $S=1$, and knowing that $S=1$, makes it more likely than not
that $S=T$.  So knowing that $R=S$ makes it more likely than not that
$S=T$, that is, $\prcond{S=T}{R=S} >1/2$.

Now prove rigorously (without any appeal to intuition)
\begin{lemma}\label{RST}
Events $[R=S]$ and $[S=T]$ are independent iff either $R$ is
uniform\footnote{That is, $\pr{R=1} = 1/2$.}, or $T$ is uniform, or
$S$ is constant\footnote{That is, $\pr{S=1}$ is one or zero.}.
\end{lemma}

\begin{solution}
Let $r,s,t$ be the probabilities that $R=1, S=1, T=1$, respectively.
Then
\begin{align*}
\pr{R=S} & = rs + (1-r)(1-s)\\
\pr{S=T} & = st + (1-s)(1-t)\\
\pr{R=S \QAND S=T} & = rst + (1-r)(1-s)(1-t).
\end{align*}
So $[R=S]$ and $[S=T]$ are independent iff
\begin{equation}\label{rs+1-r.1-s}
[rs + (1-r)(1-s)][st + (1-s)(1-t)] = rst + (1-r)(1-s)(1-t).
\end{equation}
Subtracting the left from the right-hand side of this equation and
factoring, we find that~\eqref{rs+1-r.1-s} holds iff
\[
s(s-1)(2r-1)(2t-1) = 0,
\]
namely, iff $s=0$ or $s=1$ or $r=1/2$ or $t=1/2$.  That is,
independence holds iff $S$ is constant or $R$ or $T$ is uniform.
\end{solution}

\begin{staffnotes}
This a nontrivial example of independence intransitivity: if $\pr{R} = 1/2$ and
$\pr{S}, \pr{T} \notin \set{0, 1/2, 1}$, then by Lemma~\eqref{RST}:
\begin{align*}
R = S\text{ is independent of } S = T,\\
S = T\text{ is independent of } R = T,\text{ but}\\
R = S\text{ is \textbf{not} independent of } R = T.
\end{align*}
\end{staffnotes}

\end{problem}

%%%%%%%%%%%%%%%%%%%%%%%%%%%%%%%%%%%%%%%%%%%%%%%%%%%%%%%%%%%%%%%%%%%%%
% Problem ends here
%%%%%%%%%%%%%%%%%%%%%%%%%%%%%%%%%%%%%%%%%%%%%%%%%%%%%%%%%%%%%%%%%%%%%

\endinput

\documentclass[problem]{mcs}

\begin{pcomments}
  \pcomment{FP_consecutive_coin_flips}
  \pcomment{part of CP_consecutive_coin_flips}
  \pcomment{from: S09.cp14m}
\end{pcomments}

\pkeywords{
  expectation
  total_expectation
}

%%%%%%%%%%%%%%%%%%%%%%%%%%%%%%%%%%%%%%%%%%%%%%%%%%%%%%%%%%%%%%%%%%%%%
% Problem starts here
%%%%%%%%%%%%%%%%%%%%%%%%%%%%%%%%%%%%%%%%%%%%%%%%%%%%%%%%%%%%%%%%%%%%%

\begin{problem}
Suppose we flip a fair coin and let $N_\mathtt{TT}$ be the number of
flips until the first time two consecutive Tails appear.  What is
$\expect{N_\mathtt{TT}}$?  \hfill\examrule[0.7in]

\hint Figure~\ref{tree_D}.

\begin{figure}
\graphic{tree_D}
  \caption{Sample space tree for coin toss until two consecutive tails.}
  \label{tree_D}
\end{figure}

\begin{solution}
\[
\expect{N_\mathtt{TT}}=6.
\]

Let $H$ be the event that a Head appears on the first flip, $TH$ the
event that the first flips are Tail then Head, and likewise $TT$.
From $D$ and the Law of Total Expectation:

\begin{align*}
\expect{N_\mathtt{TT}}
    & = \expcond{N_\mathtt{TT}}{H}    \cdot \pr{H}
       + \expcond{N_\mathtt{TT}}{TH}  \cdot \pr{TH}
       + \expcond{N_\mathtt{TT}}{TT}  \cdot \pr{TT}\\ 
  & = \paren{1 + \expect{N_\mathtt{TT}}}\cdot \frac{1}{2} +
      \paren{2 + \expect{N_\mathtt{TT}}}\cdot \frac{1}{4} +
       2 \cdot \frac{1}{4}\\
  & = \frac{1}{2} + \frac{\expect{N_\mathtt{TT}}}{2}
       + \frac{1}{2} + \frac{\expect{N_\mathtt{TT}}}{4}
       + \frac{1}{2}\\
  & = \frac{3}{2} + \frac{3\expect{N_\mathtt{TT}}}{4}
\end{align*}
So
\[
\expect{N_\mathtt{TT}} = \frac{3}{2} \cdot 4 = 6.
\]
\end{solution}

\end{problem}


%%%%%%%%%%%%%%%%%%%%%%%%%%%%%%%%%%%%%%%%%%%%%%%%%%%%%%%%%%%%%%%%%%%%%
% Problem ends here%%%%%%%%%%%%%%%%%%%%%%%%%%%%%%%%%%%%%%%%%%%%%%%%%%%%%%%%%%%%%%%%%%%%%

\endinput

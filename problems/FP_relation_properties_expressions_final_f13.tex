\documentclass[problem]{mcs}

\begin{pcomments}
  \pcomment{FP_relation_properties_expressions_final_f13}
  \pcomment{scramble of TP_relation_properties_expressions}
  \pcomment{10/12/13 ARM}
\end{pcomments}

\pkeywords{
  irreflexive
  transitive
  reflexive
  antisymmetric
  asymmetric
  identity_relation
}

%%%%%%%%%%%%%%%%%%%%%%%%%%%%%%%%%%%%%%%%%%%%%%%%%%%%%%%%%%%%%%%%%%%%%
% Problem starts here
%%%%%%%%%%%%%%%%%%%%%%%%%%%%%%%%%%%%%%%%%%%%%%%%%%%%%%%%%%%%%%%%%%%%%

\begin{problem}
Let $R$ be a binary relation on a set $D$.  Each of the following
formulas expresses the fact that $R$ has a familiar relational
property such as reflexivity, asymmetry, transitivity.  Predicate
formulas are numbered (1), (2), \dots, and relational formulas
(equalities and containments) are labelled with letters (a),(b),\dots.

Next to each of the relational formulas, write the number of the
predicate formula equivalent to it.  It is not necessary to name the
property expressed, but you can get partial credit if you do.  For
example, part~\eqref{irrpart} gets the label ``(1).''  It expresses
\emph{irreflexivity}.


\begin{align}
%\item\label{reflexform} \forall d.\quad d\mrel{R}d
\label{irreflexform} \forall d. &\quad \QNOT(d\mrel{R}d)\\
\label{symmiffform} \forall c,d. &\quad c\mrel{R}d\ \QIFF\ d\mrel{R}c\\
%\label{symmimpform} \forall c,d. &\quad c\mrel{R}d \QIMPLIES d\mrel{R}c
\label{asymmform} \forall c, d. &\quad c\mrel{R}d\ \QIMPLIES\ \QNOT(d\mrel{R}c)\\
%\label{antisymmform} \forall c \neq d. &\quad c\mrel{R}d \QIMPLIES \QNOT(d\mrel{R}c)
%\label{tournamentform} \forall c \neq d. &\quad c\mrel{R}d \QIFF \QNOT(d\mrel{R}c)
%\label{transform} \forall b,c,d. &\quad (b\mrel{R}c \ \QAND\  c\mrel{R}d)\ \QIMPLIES\ b\mrel{R}d\\
\label{transexform} \forall b,d. &\quad [\exists c.\ (b\mrel{R}c \ \QAND\  c\mrel{R}d)]\ \QIMPLIES\ b\mrel{R}d\\
\label{denseform} \forall b,d. &\quad  b\mrel{R}d\ \QIMPLIES\ [\exists c.\ (b\mrel{R}c \ \QAND\  c\mrel{R}d)]
\end{align}

\bparts

\ppart\label{irrpart} $R \intersect \ident{D} = \emptyset$
\hfill $\underline{\quad \text{(1)}\quad}$

\begin{solution}
\eqref{irreflexform}: \textbf{irreflexive}
\end{solution}

\ppart $R \subseteq \inv{R}$ \hfill \examrule[0.5in]
\begin{solution}
\eqref{symmiffform}: \textbf{symmetric}
\end{solution}


%% \ppart $R = \inv{R}$ \hfill \examrule[0.5in]
%% \begin{solution}
%% \eqref{symmimpform}, \eqref{symmiffform}: \textbf{symmetric}
%% \end{solution}

%% \ppart $\ident{D} \subseteq R$ \hfill \examrule[0.5in]
%% \begin{solution}
%% \eqref{reflexform}: \textbf{reflexive}
%%\end{solution}


\ppart $R \compose R \subseteq R$ \hfill \examrule[0.5in]
\begin{solution}
\eqref{transexform}: \textbf{transitive}  %\eqref{transform}
\end{solution}

\ppart $R \subseteq R \compose R$ \hfill \examrule[0.5in]
\begin{solution}
\eqref{denseform}: \textbf{dense}
\end{solution}

%% \ppart $R \intersect \inv{R} \subseteq \ident{D}$ \hfill \examrule[0.5in]
%% \begin{solution}
%% \eqref{antisymmform}: \textbf{antisymmetric}
%% \end{solution}

%% \ppart $\bar{R} \subseteq \inv{R} $ \hfill \examrule[0.5in]
%% \begin{solution}
%% \eqref{asymmform}: \textbf{asymmetric}
%% \end{solution}

%% \ppart $\bar{R} \intersect \ident{R} = \inv{R} \intersect \ident{R}$ \hfill \examrule[0.5in]
%% \begin{solution}
%% \eqref{tournamentform}: \textbf{tournament graph}
%% \end{solution}

\ppart $R \intersect \inv{R} = \emptyset$ \hfill \examrule[0.5in]
\begin{solution}
\eqref{asymmform}: \textbf{asymmetric}
\end{solution}

\eparts
\end{problem}
%%%%%%%%%%%%%%%%%%%%%%%%%%%%%%%%%%%%%%%%%%%%%%%%%%%%%%%%%%%%%%%%%%%%%
% Problem ends here
%%%%%%%%%%%%%%%%%%%%%%%%%%%%%%%%%%%%%%%%%%%%%%%%%%%%%%%%%%%%%%%%%%%%%

\endinput


\documentclass[handout]{mcs}

\begin{document}

\inclassproblems{14, Wed.}

%%%%%%%%%%%%%%%%%%%%%%%%%%%%%%%%%%%%%%%%%%%%%%%%%%%%%%%%%%%%%%%%%%%%%
% Problems start here
%%%%%%%%%%%%%%%%%%%%%%%%%%%%%%%%%%%%%%%%%%%%%%%%%%%%%%%%%%%%%%%%%%%%%

\pinput{CP_random_walk_stationary_distributions_F13}
\pinput{CP_uniform_stationary}
\pinput{CP_simple_google_graph}

%\pinput{TP_stable_distribution}
%\pinput{FP_uniform_stationary_distribution}
%\pinput{PS_random_walk_strongly_connected}
%\pinput{CP_stationary_distribution}

%%%%%%%%%%%%%%%%%%%%%%%%%%%%%%%%%%%%%%%%%%%%%%%%%%%%%%%%%%%%%%%%%%%%%
% Problems end here
%%%%%%%%%%%%%%%%%%%%%%%%%%%%%%%%%%%%%%%%%%%%%%%%%%%%%%%%%%%%%%%%%%%%%

\iffalse
\inhandout{
\section*{Appendix}
\newcommand{\vout}[1]{\text{out}(#1)}
\newcommand{\vin}[1]{\text{in}(#1)}

A \term{random-walk graph} is a digraph such that each edge,
$\diredge{x}{y}$, is labelled with a number, $p(x,y) > 0$, which will
indicate the probability of following that edge starting at vertex $x$.
Formally, we simply require that the sum of labels leaving each vertex is
1.  That is, if we define for each vertex, $x$,
\[
\vout{x} \eqdef \set{y \suchthat \diredge{x}{y} \text{ is an edge of the
    graph}},
\]
then
\[
\sum_{y \in \vout{x}} p(x,y) = 1.
\]

A \term{distribution}, $d$, is a labelling of each vertex, $x$, with a
number, $d(x) \geq 0$, which will indicate the probability of being at $x$.
Formally, we simply require that the sum of all the vertex labels is 1,
that is,
\[
\sum_{x \in V} d(x) = 1,
\]
where $V$ is the set of vertices.

The distribution, $\widehat{d}$, \term{after a single step} of a random walk from
distribution, $d$, is given by
\[
\widehat{d}(x) \eqdef \sum_{y \in \vin{x}} d(y) \cdot p(y,x),
\]
where
\[
\vin{x} \eqdef \set{y \suchthat \diredge{y}{x} \text{ is an edge of the
    graph}}.
\]

A distribution $d$ is \term{stationary} if $\widehat{d} = d$, where 
$\widehat{d}$ is the distribution after a single step of a random walk
starting from $d$.
In other words, $d$ stationary implies
\[
d(x) \eqdef \sum_{y \in \vin{x}} d(y) \cdot p(y,x).
\]
}\fi


\end{document}

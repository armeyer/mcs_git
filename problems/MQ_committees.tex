\documentclass[problem]{mcs}

\begin{pcomments}
  \pcomment{MQ_committees}
  \pcomment{variation of MQ_one_committee}
\end{pcomments}

\pkeywords{
 counting
 inclusion-exclusion
}

%%%%%%%%%%%%%%%%%%%%%%%%%%%%%%%%%%%%%%%%%%%%%%%%%%%%%%%%%%%%%%%%%%%%%
% Problem starts here
%%%%%%%%%%%%%%%%%%%%%%%%%%%%%%%%%%%%%%%%%%%%%%%%%%%%%%%%%%%%%%%%%%%%%

\begin{problem}
Twenty people work at CantorCorp, a small, unsuccessful start-up.  Two
six-person committees are to be formed.  (One will be charged with the
sole task of working to prove the Continuum Hypothesis; the other will
work, day and night, to disprove it.)
\iffalse Employees appointed to serve on the same committee join as
equals---they do not get assigned distinct roles or ranks within a
committee. 
\fi
Each employee can serve on at most one committee.  An assignment of
employees to the two committees is called an ``arrangement''.

\begin{problemparts}

\problempart
Let $D$ denote the set of all possible arrangements.  Find $\card{D}$.

\examspace[1in]

\begin{solution}
There are $\displaystyle\binom{20}{6}$ ways to choose employees for
the first committee.   For each of these, there are 14 people left over
and so $\displaystyle\binom{14}{6}$ ways to assign employees to the
second committee.  By the Generalized Product Rule (remember, the
committees do different things),
\[
\card{D} = \binom{20}{6}\binom{14}{6}.
\]

For a more rigorous approach, start by assigning a unique identifier
(ID) to each employee at CantorCorp.  Denote the set of all IDs by
$I$.  Then each arrangement in the set $D$, can be represented by a
pair of sets $\paren{S_1,S_2}$ where $S_i$ is the set of ID's of
employees on committee number $i$.  Formally, $\paren{S_1,S_2}$ is any
pair of sets such that
\begin{itemize}
\item $S_1 \subseteq I$.
\item $S_2 \subseteq I$.
\item $\card{S_1}=|S_2|=6$.
\item $S_1 \cap S_2 = \emptyset$.
\end{itemize}

There are then $\displaystyle\binom{20}{6}$ ways to choose $S_1$, and
for each of these, $\displaystyle\binom{14}{6}$ ways to choose $S_2$.
Now by the generalized product rule,
\[
\card{D}=\binom{20}{6}\binom{14}{6}.
\]

\end{solution}

\problempart Two of the workers Aleph and Beth will be unhappy if
they are to serve on the same committee.

Let $P$ denote the set of all possible arrangments in which Aleph and
Beth would serve on the same committee.  Find $\card{P}$.

\examspace[1.5in]

\begin{solution}
Either Aleph and Beth serve on the first committee or they serve on
the second.  If the first, then there are 18 employees left to choose
from to fill the remaining four spots on that committee.  Thus there
are $\displaystyle\binom{18}{4}$ ways to build the first committee (to
choose $S_1$).  For each such selection of the first committee, there
are 14 employees left for the second committee.  Of these, six must be
chosen.  There are $\displaystyle\binom{14}{6}$ ways to do this (to
choose $S_2$, for each choice of $S_1$).  So by the Generalized
Product Rule, there are $\displaystyle\binom{18}{4}\binom{14}{6}$
possible pairs $\paren{S_1,S_2}$ where Aleph and Beth serve together
on the first committee.  There are the same number of possible
arrangements where Aleph and Beth serve together on the second
committee.  These two kinds of arrangements don't overlap, so by the
Sum Rule, the number of possible pairs $\paren{S_1,S_2}$ in which
Aleph and Beth serve on the same committee is
\[
\card{P} = 2\binom{18}{4}\binom{14}{6}.
\]
\end{solution}

\problempart Beth will also be unhappy if she has to serve on the same
committee as \textbf{both} Ferdinand and Georg.

Let $Q$ denote the set of all possible arrangements in which Beth,
Ferdinand, and Georg would all serve on the same committee.  Find
$\card{Q}$.

\examspace[1in]

\begin{solution}
By reasoning similar to that used to find $|P|$, obtain 
\[
|Q|=2\binom{17}{3}\binom{14}{6}.
\]
\end{solution}

\problempart
Find $|P\cap Q|$.
\examspace[1in]

\begin{solution}
$P\cap Q$ is the set of all possible arrangements in which Aleph,
  Beth, Ferdinand and Georg all serve on the same committee.
  Evidently,
\[
|P\cap Q|=2\binom{16}{2}\binom{14}{6}.
\]
\end{solution}

\problempart Let $S$ denote the set of all possible arrangements in
which there is at least one unhappy employee.  Express $S$ in terms of
$P$ and $Q$ \textbf{only}.
\examspace[1in]

\begin{solution}
\[
S = P\cup Q.
\]
\end{solution}

\problempart
Find $|S|$.
\examspace[1.5in]
\begin{solution}
Applying inclusion/exclusion, have
\begin{align*}
|S|       &= |P \cup Q| \\         
          &= |P| + |Q| - |P\cap Q| \\
          &= 2\binom{18}{4}\binom{14}{6} + 2\binom{17}{3}\binom{14}{6} - 2\binom{16}{2}\binom{14}{6} \\
          &= 2\paren{\binom{18}{4} + \binom{17}{3} - \binom{16}{2}}\binom{14}{6}
\end{align*}
\end{solution}

\problempart If we want an arrangement with no unhappy employees, how
many choices do we have to choose from?
 
You may leave your answer in terms of $|D|$, $|P|$, $|Q|$ and $|S|$.
\examspace[1.5in]

\begin{solution}
Let $R$ denote the set of all possible arrangements in which no
employee is unhappy.  Clearly, $D = R\cup S$ and $R\cap S =
\emptyset$.  So apply the Sum Rule: $|D| = |R| + |S|$.  Hence:
\begin{align*}
|R| &= |D| - |S| \\
    &= \binom{20}{6}\binom{14}{6} - 2\paren{\binom{18}{4} + \binom{17}{3} - \binom{16}{2}}\binom{14}{6} \\
    &= \paren{\binom{20}{6} - 2\paren{\binom{18}{4} + \binom{17}{3} - \binom{16}{2}}}\binom{14}{6}
\end{align*}
\end{solution}

\end{problemparts}

\end{problem}

%%%%%%%%%%%%%%%%%%%%%%%%%%%%%%%%%%%%%%%%%%%%%%%%%%%%%%%%%%%%%%%%%%%%%
% Problem ends here
%%%%%%%%%%%%%%%%%%%%%%%%%%%%%%%%%%%%%%%%%%%%%%%%%%%%%%%%%%%%%%%%%%%%%

\endinput

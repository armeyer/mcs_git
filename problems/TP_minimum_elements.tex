\documentclass[problem]{mcs}

\begin{pcomments}
  \pcomment{TP_minimum_elements}
  \pcomment{ARM 9/20/15}
\end{pcomments}

\pkeywords{
  minimum
  real
  rational 
  irrational
}

%%%%%%%%%%%%%%%%%%%%%%%%%%%%%%%%%%%%%%%%%%%%%%%%%%%%%%%%%%%%%%%%%%%%%
% Problem starts here
%%%%%%%%%%%%%%%%%%%%%%%%%%%%%%%%%%%%%%%%%%%%%%%%%%%%%%%%%%%%%%%%%%%%%


\begin{problem}
For each of the following sets, describe a subset with no minimum element:
\bparts
\ppart The integers $\integers$.

\examspace[0.5in]

\begin{solution}
The negative integers. Alternatively, the set $\integers$ itself.
\end{solution}
 
\ppart The nonnegative rational numbers $\rationals^{\geq 0}$.

\examspace[0.5in]

\begin{solution}
The positive rationals $\rationals^{>0}$.
\end{solution}

\ppart The irrational real numbers $\geq \sqrt{3}$.

\examspace[0.5in]

\begin{solution}
The whole set has a minimum element, namely the irrational numbere
$\sqrt{3}$.  But throwing out this minimum element, we get the set
irrational real numbers $> \sqrt{3}$ which has no minimum element.
This follows because if $a$ is an irrational real number $> \sqrt{3}$,
then there is always another irrational number in any positive length
interval strictly between $a$ and $\sqrt{3}$.
\end{solution}

\eparts

\end{problem}
 
\endinput

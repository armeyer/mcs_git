\documentclass[handout]{mcs}

\begin{document}

\inclassproblems{7, Fri.}

%%%%%%%%%%%%%%%%%%%%%%%%%%%%%%%%%%%%%%%%%%%%%%%%%%%%%%%%%%%%%%%%%%%%%
% Problems start here
%%%%%%%%%%%%%%%%%%%%%%%%%%%%%%%%%%%%%%%%%%%%%%%%%%%%%%%%%%%%%%%%%%%%%

%equivalence relations
\begin{staffnotes}

Students took much longer than expected to go through problem 1 on basic relations in the class.

\end{staffnotes}
\pinput{TP_basic_relations}
%\pinput{TP_strictPOs_are_DAGs}
%\pinput{TP_divisibility_partial_order}
\pinput{CP_partially_ordered_by_divisibility}

\pinput{CP_equiv_partition_proof}
\pinput{CP_equivalence_same_property}

\pinput{CP_binary_relations_on_01}

%\pinput{TP_transitive_irreflexive_implies_asymmetric}

%\pinput{CP_bogus_reflexive_proof}

%\pinput{PS_preserve_transitivity}
%\pinput{CP_strict_PO_irreflexive}
%\pinput{CP_inverse_partial_order}
%\pinput{CP_prerequisite_relation}
%\pinput{CP_class_scheduling}


%%%%%%%%%%%%%%%%%%%%%%%%%%%%%%%%%%%%%%%%%%%%%%%%%%%%%%%%%%%%%%%%%%%%%
% Problems end here
%%%%%%%%%%%%%%%%%%%%%%%%%%%%%%%%%%%%%%%%%%%%%%%%%%%%%%%%%%%%%%%%%%%%%

\iffalse
\instatements{\newpage}

\begin{pagesidebar}[to \textheight]
\textboxtitle{Properties of a Relation $R: A \to A$\ / \ Digraph $G$ with $\vertices{G}=A$}

\begin{description}

\item[Reflexivity]

$R$ is \term{reflexive} when
\[
\forall x \in A. \; x \mrel{R} x.
\]

%``Everyone likes themselves.''

Every node in~$G$ has a self-loop.

\item[Irreflexivity]

$R$ is \term{irreflexive} when
\[
\QNOT \exists x \in A. \; x \mrel{R} x.
\]

%``No one likes themselves.''

There are no self-loops in~$G$.

\item[Symmetry]

$R$ is \term{symmetric} when
\[
\forall x, y \in A. \; x \mrel{R} y \QIMP y \mrel{R} x.
\]

%``If $x$ likes~$y$, then $y$ likes~$x$.''

If there is an edge from $x$ to~$y$ in~$G$, then there is an edge back from
$y$ to~$x$ in~$G$ as well.

\item[Asymmetry]
$R$ is \emph{asymmetric} when
\[
\forall x, y \in A. \; x \mrel{R} y \QIMPLIES \QNOT( y \mrel{R} x ).
\]
There is at most one directed edge between any two nodes in $G$; there
are no self-loops.

\item[Antisymmetry]
$R$ is \term{antisymmetric} when
\[
\forall x \neq y \in A. \; x \mrel{R} y \QIMPLIES \QNOT( y \mrel{R} x ).
\]
There is at most one directed edge between any two nodes; there may be self-loops.

\item[Transitivity]
$R$ is \term{transitive} if
\[
 \forall x, y, z \in A. \; (x \mrel{R} y \QAND y \mrel{R} z) \QIMPLIES x \mrel{R} z.
\]
%``If $x$ likes~$y$ and $y$ likes~$z$, then $x$ likes~$z$ too.''

If there is a positive length path from $u$ to $v$, then there is an edge from $u$ to $v$.

\iffalse
For any walk $v_0, v_1, \dots, v_k$ in~$G$ where $k \ge 2$,
$\diredge{v_0}{v_k}$ is in~$G$ (and, hence, $\diredge{v_i}{v_j}$ is
also in~$G$ for all $i < j$.
\fi

\item[Total] $R$ is \term{total} when
\[
 \forall x \neq y \in A. \; (x \mrel{R} y \QOR y \mrel{R} x)
\]
Given any two vertices in $G$, there is an edge in one direction or the
other between them.

\item[Strict Partial Order] $R$ is a \term{strict partial order} iff
  it is transitive and asymmetric iff it is transitive and
  irreflexive.

\item[Weak Partial Order] $R$ is a \term{weak partial order} iff it is
  transitive and anti-symmetric and reflexive.

\end{description}
\end{pagesidebar}
\fi

\end{document}

\documentclass[problem]{mcs}

\begin{pcomments}
  \pcomment{FP_linear_recur_squares}
  \pcomment{ZDz 11/22/15}
\end{pcomments}

\pkeywords{
  recurrence
  linear_recurrence
  generating_function
}

%%%%%%%%%%%%%%%%%%%%%%%%%%%%%%%%%%%%%%%%%%%%%%%%%%%%%%%%%%%%%%%%%%%%%
% Problem starts here
%%%%%%%%%%%%%%%%%%%%%%%%%%%%%%%%%%%%%%%%%%%%%%%%%%%%%%%%%%%%%%%%%%%%%

\begin{problem}
Express as a quotient of polynomials or products of polynomials:

\bparts

\ppart
The generating function $A(x) \eqdef \sum_0^\infty a_n x^n$ of the series
%$1, 3, 5, 7, \ldots$, i..e.,
$a_n = 2n+1, n \geq 0$.

\examspace[1.3in]

\begin{solution}
Using that the generating function for $n$ is $\frac{x}{(1-x)^2}$ and for $1$ is $\frac{1}{1-x}$,
the generating function for $a_n$ is
\[
A(x) = \frac{2x}{(1-x)^2} + \frac{1}{1-x} = \frac{1+x}{(1-x)^2} \, .
\]
\end{solution}

\ppart
The generating function $B(x) \eqdef \sum_0^\infty b_n x^n$ of the series
%$0, 1, 3, 5, \ldots$, i..e.,
\[b_n = \begin{cases}
           2n-1 & \text{for } n\geq 1,\\
           0       &  \text{for } n = 0.
       \end{cases}
\]
You may express your answer in terms of $A(x)$.

\hint{Note that $b_n = a_{n-1}$ for $n \geq 1$.}

\examspace[1in]

\begin{solution}
\[
B(x) = x A(x) = \frac{x(1+x)}{(1-x)^2} \, ,
\]
which follows from the fact that $b_n$ is obtained by shifting $a_n$ to the right by
one position.
\end{solution}

\ppart
The generating function $C(x) \eqdef \sum_0^\infty c_n x^n$ of the series
\[c_n = \begin{cases}
           c_{n-1} + (2n-1) & \text{for } n\geq 1,\\
           0                           &  \text{for } n = 0.
       \end{cases}
\]
You may express your answer in terms of $B(x)$.
You do \emph{not} have to find a closed form for $c_n$.

\examspace[2in]

\begin{solution}
We have:
\[
\begin{array}{rcrcrcrcrcl}
C(x)       & = & c_0 & + & c_1 x & + & c_2 x^2 & + & c_3 x^3  & + & \cdots\\
- xC(x)   & = &        & - & c_0 x & -  & c_1 x^2 & - & c_2 x^3 & - & \cdots\\
- B(x)     & = &        & - &    1 x & -  &    3 x^2 & - &  5 x^3 & - & \cdots\\
\hline
%         & = & r_0 & + & 0 x & + &    0 x^2 & + &    0x^3 & + &  \cdots\\
         & = &  0  & + &  0  x & + &    0 x^2 & + &    0x^3 & + &  \cdots .
\end{array}
\]
Therefore,
\[
C(x)(1-x) - B(x) = 0,
\]
so
\[
C(x) = \frac{B(x)}{1-x} = \frac{x A(x)}{1-x} = \frac{x(1+x)}{(1-x)^3} \ .
\]
\end{solution}

\eparts

\end{problem}


%%%%%%%%%%%%%%%%%%%%%%%%%%%%%%%%%%%%%%%%%%%%%%%%%%%%%%%%%%%%%%%%%%%%%
% Problem ends here
%%%%%%%%%%%%%%%%%%%%%%%%%%%%%%%%%%%%%%%%%%%%%%%%%%%%%%%%%%%%%%%%%%%%%
\endinput

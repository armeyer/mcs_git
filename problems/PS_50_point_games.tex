\documentclass[problem]{mcs}

\begin{pcomments}
  \pcomment{from: S09.ps4}
  \pcomment{based on problem 6 of F07.ps3}
\end{pcomments}

\pkeywords{
  structural_induction
  games
}

%%%%%%%%%%%%%%%%%%%%%%%%%%%%%%%%%%%%%%%%%%%%%%%%%%%%%%%%%%%%%%%%%%%%%
% Problem starts here
%%%%%%%%%%%%%%%%%%%%%%%%%%%%%%%%%%%%%%%%%%%%%%%%%%%%%%%%%%%%%%%%%%%%%

\begin{problem}

\newcommand{\fiftypg}{\text{50-PG}}

  Define \term{2-person 50-point games of perfect information}\footnote{
    Some helpful (we hope) motivation for this definition appears in an
    attached excerpt from Spring '08 class notes which went further into
    this subject.  The discussion of infinite games can be skipped.},
  $\fiftypg$'s, recursively as follows:

\textbf{Base case}: An integer, $k$, is a $\fiftypg$ for $-50\leq k \leq
50$.  This $\fiftypg$ called the \term{terminated game with payoff} $k$.
A \term{play} of this $\fiftypg$ is the length one integer sequence, $k$.

\textbf{Constructor case}: If $G_0,\dots, G_n$ is a finite sequence of
$\fiftypg$'s for some $n \in \naturals$, then the following game, $G$, is a
$\fiftypg$: the possible first moves in $G$ are the choice of an integer $i$
between 0 and $n$, the possible second moves in $G$ are the possible first
moves in $G_i$, and the rest of the game $G$ proceeds as in $G_i$.

A \emph{play} of the $\fiftypg$, $G$, is a sequence of nonnegative integers
starting with a possible move, $i$, of $G$, followed by a play of $G_i$.
If the play ends at the game terminated game, $k$, then $k$ is called the
\term{payoff} of the play.

There are two players in a $\fiftypg$ who make moves alternately.  The
objective of one player (call him the \emph{max}-player) is to have the
play end with as high a payoff as possible, and the other player (called
the \emph{min}-player) aims to have play end with as low a payoff as
possible.

Given which of the players moves first in a game, a strategy for the
max-player is said to \emph{ensure} the payoff, $k$, if play ends with a
payoff of at least $k$, no matter what moves the min-player makes.
Likewise, a strategy for the min-player is said to \emph{hold down} the
payoff to $k$, if play ends with a payoff of at most $k$, no matter what
moves the max-player makes.

A $\fiftypg$ is said to have \emph{max value}, $k$, if the max-player has a
strategy that ensures payoff $k$, and the min-player has a strategy that
holds down the payoff to $k$, when the \emph{max-player moves first}.
Likewise, the $\fiftypg$ has \emph{min value}, $k$, if the max-player has a
strategy that ensures $k$, and the min-player has a strategy that holds
down the payoff to $k$, when the \emph{min-player moves first}.

The \emph{Fundamental Theorem} for 2-person 50-point games of perfect
information is that is that every game has both a max value and a min
value.  (Note: the two values are usually different.)

What this means is that there's no point in playing a game: if the max
player gets the first move, the min-player should just pay the max-player
the max value of the game without bothering to play (a negative payment
means the max-player is paying the min-player).  Likewise, if the
min-player gets the first move, the min-player should just pay the
max-player the min value of the game.

\bparts

\ppart\label{finpg} Prove this Fundamental Theorem for 50-valued
$\fiftypg$'s by structural induction.

\solution{The proof is by structural induction on the definition of a
  $\fiftypg$, $G$.  The induction hypothesis is that there is that
\begin{quote}
  $G$ has a max value and a min value.
\end{quote}

\textbf{Base case}: [$G$ is the terminated game with payoff $k$].  The only
possible play is $k$.  So the max value and the min value are both $k$.

\textbf{Constructor case}: [$G = (G_0,\dots, G_n)$].  By structural
induction we may assume that each of the games $G_i$ have both max values
and min values.

We first show that $G$ has max value, $k$, where $k$ is the largest min
value among the games $G_0,\dots,G_n$.

To prove the max value of $G$ is $k$, we must show how the max-player,
moving first in $G$, can ensure $k$, and how the min-player, moving second
in $G$, can hold down the payoff to $k$.

To ensure $k$, the max-player simply chooses $i$ as her first move where
game $G_i$ has this largest min value, $k$.  The min-player then has the
first move in $G_i$, so by definition of min value, the max-player has a
strategy in $G_i$ that ensures $k$, which she can now follow.  So this
first move, combined with the ensuring strategy in $G_i$, defines a
strategy for the max-player in $G$ that ensures $k$.

Likewise, there is a simple strategy for the min-player, moving second in
$G$, to hold down the payoff to $k$.  Namely, suppose the max-player's
first move is $i$.  Then $G_i$ has a min value of $m \leq k$, since $k$ is
the largest min value.  So by definition of min value, there is a strategy
in $G_i$ for the min-player to hold down the payoff to $m$, which he can
now follow, thereby holding down the payoff of play on $G$ to $m \leq k$.

The existence of these ensuring and holding down strategies for $G$
implies that the max value of $G$ is $k$.

Second, to show that $G$ has a min value, we can repeat the previous
argument with min and max exchanged.

Therefore, by structural induction, we can conclude that all $\fiftypg$'s have
min and max values.}

\ppart

\iffalse A meta-$\fiftypg$ game has as possible first moves the
choice of \emph{any} $\fiftypg$ to play.  Meta-$\fiftypg$ games aren't any
harder to understand than $\fiftypg$'s, but there is one notable
difference, they have an infinite number of possible first moves.  We
could also define meta-meta-$\fiftypg$'s in which the first move was a
choice of any $\fiftypg$ \emph{or} the meta-$\fiftypg$ game to play.  In
meta-meta-$\fiftypg$'s there are an infinite number of possible first
\emph{and} second moves.  \fi

The 2D-origin game in a Week 4 class problem is a game in which there are
an infinite number of possible first moves, an infinite number of possible
second moves, \dots.  \iffalse (with two values, ``win'' or ``lose''
instead of values from -50 to 50)\fi

To model such infinite games, we could have modified the recursive
definition of $\fiftypg$'s to allow first moves that choose any one of an
infinite sequence $G_0,G_1,\dots,G_n,G_{n+1}, \dots$ of $\fiftypg$'s.  Now
a $\fiftypg$ can be a mind-bendingly infinite datum instead of a finite
one.

Do these infinite $\fiftypg$'s still have max and min values?  In
particular, do you think it would be correct to use structural induction
as in part~\eqref{finpg} to prove a Fundamental Theorem for such infinite
$\fiftypg$'s?  Offer an answer to this question, and briefly indicate why
you believe in it.

\solution{It may not be obvious, but structural induction is a perfectly
  sound proof technique even for infinite recursively defined data, and
  the proof of the Fundamental Theorem for $\fiftypg$'s applies without
  change to infinite ones.}

\eparts

\end{problem}

%%%%%%%%%%%%%%%%%%%%%%%%%%%%%%%%%%%%%%%%%%%%%%%%%%%%%%%%%%%%%%%%%%%%%
% Problem ends here
%%%%%%%%%%%%%%%%%%%%%%%%%%%%%%%%%%%%%%%%%%%%%%%%%%%%%%%%%%%%%%%%%%%%%

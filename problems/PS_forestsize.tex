\documentclass[problem]{mcs}

\begin{pcomments}
  \pcomment{PS_forestsize}
  \pcomment{ARM 11/10/17, revised to call for induction 5/11/18}
\end{pcomments}

\pkeywords{
  forest
  tree
  component
  vertices
  edges
}

%%%%%%%%%%%%%%%%%%%%%%%%%%%%%%%%%%%%%%%%%%%%%%%%%%%%%%%%%%%%%%%%%%%%%
% Problem starts here
%%%%%%%%%%%%%%%%%%%%%%%%%%%%%%%%%%%%%%%%%%%%%%%%%%%%%%%%%%%%%%%%%%%%%

\begin{problem}
If a finite simple graph $G$ has exactly $\card{\vertices{G}} -
\card{\edges{G}}$ components, then $G$ is a forest.  Prove this by
induction on the number of vertices.

\begin{solution}
  \TBA{The proof is similar to the proof of
    Theorem~\bref{th:forestsize}.}

There is also a simple proof that uses Theorem~\bref{th:forestsize}:
\begin{proof}
  There is a set $E \subseteq \edges{G}$ of edges on cycles of
  $G$ such that removing these edges leaves an acyclic graph $G-E$
  with the same number of connected components as $G$.  Therefore
\begin{align*}
\lefteqn{\card{\text{components of $G$}}}\\
   & = \card{\text{components of $G-E$}} 
   & = \card{\vertices{G-E}} - \card{\edges{G-E}}
           & \text{(by Theorem~\bref{th:forestsize})}\\
   & = \card{\vertices{G}} - (\card{\edges{G}}- \card{E})\\
   & = (\card{\vertices{G}} - \card{\edges{G}}) +\card{E})\\
   & = \card{\text{components of $G$}} + \card{E} & \text{(by Theorem~\bref{th:forestsize}{.\bref{itm:finconn}})}.
\end{align*}
So $\card{E} = 0$, which means that $G$ must have been acyclic---that
is, a forest---to begin with.
\end{proof}
\end{solution}

%%%%%%%%%%%%%%%%%%%%%%%%%%%%%%%%%%%%%%%%%%%%%%%%%%%%%%%%%%%%%%%%%%%%%
% Problem ends here
%%%%%%%%%%%%%%%%%%%%%%%%%%%%%%%%%%%%%%%%%%%%%%%%%%%%%%%%%%%%%%%%%%%%%

\endinput

\documentclass[problem]{mcs}

\begin{pcomments}
  \pcomment{PS_rewrite_assertions_prime_goldbach_bertrand_fermat}
  \pcomment{from: S03.ps4, f03.ps1, f09.ps2}
  \pcomment{subsumed PS_translate_to_predicate_logic}
\end{pcomments}

\pkeywords{
  predicate
  quantifier
  Goldbach
  Bertrand
  Fermat
}

%%%%%%%%%%%%%%%%%%%%%%%%%%%%%%%%%%%%%%%%%%%%%%%%%%%%%%%%%%%%%%%%%%%%%
% Problem starts here
%%%%%%%%%%%%%%%%%%%%%%%%%%%%%%%%%%%%%%%%%%%%%%%%%%%%%%%%%%%%%%%%%%%%%

\begin{problem}
In this problem we'll examine predicate logic formulas where the domain
of discourse is $\nngint$.  In addition to the logical symbols, the
formulas may contain ternary predicate symbols $A,M$ and $X$, where
\begin{align*}
A(k,m,n) &\ \text{means}\ k = m + n,\\
M(k,m,n) &\ \text{means}\ k = m \cdot n,\\
A(k,m,n) &\ \text{means}\ k = m^n.
\end{align*}
For example, a formula ``$\text{Zero}(n)$'' meaning that $n$ is zero
could be defined as
\[
\text{Zero}(n) \eqdef A(n,n,n).
\]
Having defined ``Zero,'' it is now OK to use it in subsequent
formulas.  So a formula ``$\text{Greater}(m,n)$'' meaning $m > n$
could be defined as
\[
\text{Greater}(m,n) \eqdef \exists k. \QNOT(\text{Zero}(k)) \QAND A(m,n,k).
\]
This makes it OK to use ``Greater'' in subsequent formulas.

Write logic formulas using only the allowed predicates that define the following predicates:

\begin{problemparts}
\ppart\label{defone} $\text{One}(n)$ meaning that $n=1$.

\begin{solution}
\[
\text{One}(n) \eqdef M(n,n,n) \QAND \QNOT(\text{Zero}(n))
\]
clearly works.  So does
\[
\text{One}(n) \eqdef \forall m. M(m,m,n).
\]
Another approach uses Greater:
\[
\text{One}(n) \eqdef \forall m.\,  \text{Greater}(n,m) \QIMP \text{Zero}(m).
\]
\end{solution}

\problempart $\text{Prime}(p)$ meaning $p$ is a prime number.

\begin{solution}
\begin{align*}
\text{Prime}(p) & \eqdef\\
      & \exists w. \text{One}(w) \QAND \text{Greater}(p,w) \QAND\\
      & \quad \QNOT(\exists m.\, \exists n.\, 
          (\text{Greater}(m,w) \QAND \text{Greater}(n,w) \QAND M(p,m,n))
\end{align*}
\end{solution}

\ppart\label{deftwo} $\text{Two}(n)$ meaning that $n=2$.
\begin{solution}
\[
\text{Two}(n) \eqdef \forall m.\, \text{Greater}(n,m) \QIMP (\text{Zero}(m) \QOR \text{One}(m)).
\]
Another approach is
\[
\text{Two}(n) \eqdef \exists w. \text{One}(w) \QAND A(n, w, w).
\]
\end{solution}

\eparts

The results of part~\eqref{deftwo} will entend to formulas
$\text{Three}(n), \text{Four}(n), \text{Five}(n), \dots$ which are
allowed from now on.

\bparts

\iffalse
\problempart There is no largest prime number.

\begin{solution}
\[
\QNOT(\exists p.\, (\text{Prime}(p) \QAND (\forall q.\, (\text{Prime}(q) \QIMPLIES p \geq q))))
\]
\end{solution}
\fi

\ppart $\text{Even}(n)$  meaning $n$ is even.

\begin{solution}
A standard definition would be:
\[
\text{Even}(n)\eqdef \exists t.\, (\text{Two}(t) \QAND  \exists k.\, M(n,t,k))
\]
A simpler alternative is:
\[
\text{Even}(n)\eqdef \exists k. M(n,k,k)]
\]
\end{solution}


\problempart (\idx{Goldbach Conjecture}) Every even
integer $n \geq 4$ can be expressed as the sum of two primes.

\begin{solution}
\begin{align*}
\exists r.\\
  & \text{Three}(r) \QAND\\
  & \forall n. (\text{Greater}(n,r) \QAND \text{Even}(n))\\
  & \qquad \QIMPLIES \exists p \exists q. \text{Prime}(p) \QAND \text{Prime}(q) \QAND A(n, p, q).
\end{align*}
\end{solution}

\iffalse
\problempart (\idx{Bertrand's Postulate}) If $n > 1$, then there is always
at least one prime $p$ such that $n < p < 2n$.

\begin{solution}
\[
\forall n.\,
\( (n > 1) \QIMPLIES (\exists p ( \text{Prime}(p)  \QAND (n < p) \QAND (p < 2n))) 
)
\]
\end{solution}

\problempart (\idx{Fermat's Last Theorem}) There are no solutions to the
equation:

\begin{equation*}
    x^n + y^n = z^n
\end{equation*}

where $n > 2$ and $x$, $y$ and $z$ are positive.

\begin{solution}
\[
\forall x, y, z, n.\,
    (
    (x > 0 \QAND y > 0 \QAND z > 0 \QAND n > 2)
    \QIMPLIES
    \QNOT(x^n + y^n = z^n)
    )
\]
\end{solution}
\fi

\ppart (Twin Prime Conjecture) There are infinitely primes that differ by two.

\begin{solution}
\begin{align*}
\exists t.\, \text{Two}(t)
   & \QAND \forall n \exists p. \exists q\\
   & \text{Greater}(p,n) \QAND A(q,p,t) \QAND \text{Prime}(q) \QAND \text{Prime}(q).
\end{align*}
\end{solution}

\eparts

\end{problem}

%%%%%%%%%%%%%%%%%%%%%%%%%%%%%%%%%%%%%%%%%%%%%%%%%%%%%%%%%%%%%%%%%%%%%
% Problem ends here
%%%%%%%%%%%%%%%%%%%%%%%%%%%%%%%%%%%%%%%%%%%%%%%%%%%%%%%%%%%%%%%%%%%%%

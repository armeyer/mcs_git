\documentclass[problem]{mcs}

\begin{pcomments}
  \pcomment{FP_coloring_complete_triangles}
  \pcomment{variation of FP_coloring_triangles from F03 final}
  \pcomment{from S06.final}
  \pcomment{minor edit ARM 5/20/12}
  \pcomment{soln to last part need rewrite ala Drew Minnear}
\end{pcomments}

\pkeywords{
 coloring
 probability
 variance
 graphs
 expectation
 independence
 Chebyshev
 law_of_large_numbers
 asymptotic_notation
 big_oh 
}

%%%%%%%%%%%%%%%%%%%%%%%%%%%%%%%%%%%%%%%%%%%%%%%%%%%%%%%%%%%%%%%%%%%%%
% Problem starts here
%%%%%%%%%%%%%%%%%%%%%%%%%%%%%%%%%%%%%%%%%%%%%%%%%%%%%%%%%%%%%%%%%%%%%


\begin{problem}

\iffalse
\begin{staffnotes}
(a) 2, (b) 2, (c) 1, (d) 2, (e) 2, (f) 3
\end{staffnotes}
\fi

Let $K_n$ be the complete graph with $n$ vertices.  Each of the edges of
the graph will be randomly assigned one of the colors red, green, or blue.
The assignments of colors to edges are mutually independent, and the
probability of an edge being assigned red is $r$, blue is $b$, and green is
$g$ (so $r+b+g=1$).

A set of three vertices in the graph is called a \emph{triangle}.  A
triangle is \emph{monochromatic} if the three edges connecting the vertices
are all the same color.

\bparts 

\ppart\label{mr3b3g3} Let $m$ be the probability that any given triangle $T$ is
monochromatic.  Write a simple formula for $m$ in terms of $r,b,$ and $g$.

\examspace[0.5in]

\begin{solution}
$m = r^3+b^3 +g^3$
\end{solution}

\ppart Let $I_T$ be the indicator variable for whether $T$ is
monochromatic.  Write simple formulas in terms of $m,r,b$ and $g$ for
$\expect{I_T}$ and $\variance{I_T}$.

\begin{align*}
\expect{I_T}   & = \text{\examboxplain{1.5in}{0.4in}{-0.2in}}\\
\variance{I_T} & = \text{\examboxplain{1.5in}{0.4in}{-0.2in}}
\end{align*}

\examspace[1.0in]

\begin{solution}
\begin{align*}
\expect{I_T} & = m,\\
\variance{I_T} & = m(1-m).
\end{align*}
\end{solution}

\eparts

\bigskip

Let $T$ and $U$ be distinct triangles.

\bparts

\ppart\label{tumono} What is the probability that $T$ and $U$ are both
monochromatic if they do not share an edge?\dots if they do share an
edge?

\exambox{0.5in}{0.4in}{0in}

\begin{solution}
%Both $T$ and $U$ are monochromatic iff $I_T \cdot I_U =1$.

\[
\pr{\text{$T$ and $U$ are monochromatic}} =
    \begin{cases}
      m^2 & \text{if they do not share an edge},\\
      r^5+b^5+g^5 & \text{if they do share an edge}.
    \end{cases}
\]

If $T$ and $U$ do not share an edge, then the three edges of $T$ match
with probability $m$, and independently, the three edges of $U$ match
with probability $m$, so both match with probability $m^2$.  If they
do share an edge, the five edges among them must all match one of
colors $r,g,b$.
\end{solution}
\eparts

\bigskip
\begin{center}
{\large \textbf{Now assume $r=b=g = \dfrac{1}{3}$.}}
\end{center}

\bparts
\ppart Show that $I_T$ and $I_U$ are independent random variables.

\examspace[2.0in]

\begin{solution}
Since $I_T$ and $I_U$ are indicators for events, it suffices to verify
that
\begin{equation}\label{it1iu1}
\prob{I_T = 1}\cdot \prob{I_U = 1} = \prob{I_T\cdot I_U =1}.
\end{equation}
In the case that $T$ and $U$ do not share an edge,~\eqref{it1iu1}
follows immediately from parts~\eqref{mr3b3g3} and~\eqref{tumono}.
If they do share an edge, this follows because
\[
m^2= \paren{3(1/3)^3}^2 = (1/3)^4 = 3(1/3)^5 = r^5+b^5+g^5.
\]
\end{solution}

\ppart\label{part:exMvarM} Let $M$ be the number of monochromatic
triangles.  Write simple formulas in terms of $n$ and $m$ for
$\expect{M}$ and $\variance{M}$.

\begin{align*}
\expect{M}   & = \text{\examboxplain{1.5in}{0.4in}{-0.2in}}\\
\variance{M} & = \text{\examboxplain{1.5in}{0.4in}{-0.2in}}
\end{align*}

\examspace[1.0in]

\begin{solution}
\begin{align}
\expect{M}   & = m \cdot \text{(\# triangles)}\notag\\
             & = m\binom{n}{3},\label{expMmsh}\\
\variance{M} & = \variance{I_T} \cdot \text{(\# triangles)}\notag\\
             & = m(1-m)\binom{n}{3} = (1-m)\expect{M}.\label{varMITsh}
\end{align}
\end{solution}

\ppart Let $\mu \eqdef \expect{M}$.  Use Chebyshev's Bound to prove
that
\[
\Prob{\abs{M - \mu} > \sqrt{\mu \log \mu}} \leq \frac{1}{\log \mu}.
\]

\examspace[2.5in]

\begin{staffnotes}
Have students work out rough numbers for $n=10^3$.  THESE NUMBERS NEED
TO BE CHECKED: $\mu \approx 10^9$, $\log \mu \approx 30$, so
there is only a 3\% chance that $M$ will differ from its expected
value by more than 15\%.
\end{staffnotes}

\begin{solution}
According to Chebyshev's Bound:
\[%\label{CMmucsig}
\Prob{\abs{M - \mu} > c \sigma} \leq \frac{1}{c^2}
\]
So
\begin{align*}
\Prob{\abs{M - \mu} > \sqrt{\mu \log \mu}}
  & = \Prob{\abs{M - \mu} > \sqrt{\log \mu} \sqrt{\mu}}\\
  & \leq \Prob{\abs{M - \mu} > \sqrt{\log \mu} \sqrt{(1-m)\mu}}
       & (0 \leq 1- m \leq 1)\\
  & = \Prob{\abs{M - \mu} > \sqrt{\log \mu}\ \sigma}
       & \text{(by~\eqref{varMITsh})}\\
  & \leq \frac{1}{\log \mu}
       & \text{(by Chebyshev)}\\%~\eqref{CMmucsig}
%  & = O \paren{\frac{1}{\log n}}
%       & \text{(by~\eqref{expMmsh}).}
\end{align*}
\iffalse
The last step follows because according to~\eqref{expMmsh},
\[
\mu = m\binom{n}{3} = \Theta(n^3),
\]
so
\[
\frac{1}{\log \mu} = \Theta\paren{\frac{1}{\log n^3}} = \Theta\paren{\frac{1}{\log n}}.
\]
\fi
\end{solution}


\ppart Conclude that
\[
\lim_{n\to \infty} \Prob{\abs{M - \mu} > \sqrt{\mu \log \mu}} = 0
\]

\begin{solution}
According to~\eqref{expMmsh},
\[
\mu = m\binom{n}{3} = \Theta(n^3),
\]
so
\[
\frac{1}{\log \mu} = \Theta\paren{\frac{1}{\log n^3}} = \Theta\paren{\frac{1}{\log n}}.
\]

Since $1/(\log n) \to 0$ as $n \to \infty$, the $O()$ bound on the
probability goes to zero, which means an upper bound on the limit is
zero.  Since the probability is nonnegative, the limit must be exactly
zero.
\end{solution}

\eparts
\end{problem}

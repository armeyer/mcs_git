\documentclass[problem]{mcs}

\begin{pcomments}
  \pcomment{from: S09.ps3}
\end{pcomments}

\pkeywords{
  relations
  relational_properties
  partial_orders
}

%%%%%%%%%%%%%%%%%%%%%%%%%%%%%%%%%%%%%%%%%%%%%%%%%%%%%%%%%%%%%%%%%%%%%
% Problem starts here
%%%%%%%%%%%%%%%%%%%%%%%%%%%%%%%%%%%%%%%%%%%%%%%%%%%%%%%%%%%%%%%%%%%%%

\begin{problem}
For each of the binary relations below, state whether it is a strict
partial order, a weak partial order, or neither.  If it is not a partial
order, indicate which of the axioms for partial order it violates.
If it is a partial order, state whether or not it is a total order.

\bparts
\ppart The subset relation on the power set $\power{\set{1, 2, 3, 4, 5}}$.

\solution{This is a strict partial order, but not a total one, as, for example, 
the sets of size 3 form an antichain.}

\ppart The relation between any two integers, $a$, $b$ that the remainder
of $a$ divided by 8 equals the remainder of $b$ divided by 8.

\solution{This is not a partial order, as, for example, $8 R 16$, $16 R 8$ but $8$ is not equal to $16$.}

\ppart The relation between propositional formulas, $G$, $H$, that $G
\QIMPLIES H$ is valid.

\solution{This isn't a relation as $G$ can imply $H$ and $H$ imply $G$ without the 
statements actually being the same.  This does define a relation between equivalence classes,
if we consider the set of all statements that are logically equivalent instead of the individual 
statements.}

\ppart The relation 'beats' on Rock, Paper and Scissor (for those who don't
know the game Rock, Paper, Scissors, Rock beats Scissors, Scissors beats
Paper and Paper beats Rock).

\solution{Not a relation, as it isn't transitive.}
\eparts

\end{problem}

%%%%%%%%%%%%%%%%%%%%%%%%%%%%%%%%%%%%%%%%%%%%%%%%%%%%%%%%%%%%%%%%%%%%%
% Problem ends here
%%%%%%%%%%%%%%%%%%%%%%%%%%%%%%%%%%%%%%%%%%%%%%%%%%%%%%%%%%%%%%%%%%%%%

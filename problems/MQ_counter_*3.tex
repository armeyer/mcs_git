\documentclass[problem]{mcs}

\begin{pcomments}
  \pcomment{MQ_counter_*3}
  \pcomment{F16, midterm1}
  \pcomment{ARM 2/24/16}
  \pcomment{was CP_TBA6}
\end{pcomments}

\pkeywords{
  predicate
  logic
  domain
}

\begin{problem}
Write a program for a counter machine with two counters \texttt{R} and
\texttt{S} that adds 3 times the contents of \texttt{R} to the
contents of \texttt{S}, while setting the contents of \texttt{R} to 0.
That is, the machine simulates the assignment statements
\begin{center}
\begin{tabular}{rl}
\texttt{S :=} &\texttt{S + 3*R;}\\
\texttt{R :=} &\texttt{0}
\end{tabular}
\end{center}
Your program should only use the basic Counter Machine instructions
\texttt{T+} (increment counter \texttt{T}), \texttt{T-} (decrement
counter \texttt{T} unless it contains 0), \texttt{[T? $m$, $n$]} (if
\texttt{T} contains 0, goto line number $m$, otherwise goto line
$n$) for any counter \texttt{T}.

\hint There is a six line program that needs no extra counters beyond
$R$ and $S$.  \inhandout{No penalty for longer \emph{correct} programs
  using extra registers.}

\begin{solution}
\texttt{1.[R? halt 2], 2.R-, 3.S+, 4.S+, 5.S+, 6.goto 1}

which is shorthand for:

\texttt{1.[R? 7 2], 2.R-, 3.S+, 4.S+, 5.S+, 6.[S? 1 1]}

Even nicer:
\texttt{1.[R? 7 2], 2.R-, 3.S+, 4.S+, 5.S+, 6.[R? 7 2]}
\end{solution}

\end{problem}

\endinput

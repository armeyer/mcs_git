\documentclass[problem]{mcs}

\begin{pcomments}
  \pcomment{from: S09.cp7r}
\end{pcomments}

\pkeywords{
  digraphs
  cycles
}

%%%%%%%%%%%%%%%%%%%%%%%%%%%%%%%%%%%%%%%%%%%%%%%%%%%%%%%%%%%%%%%%%%%%%
% Problem starts here
%%%%%%%%%%%%%%%%%%%%%%%%%%%%%%%%%%%%%%%%%%%%%%%%%%%%%%%%%%%%%%%%%%%%%

\begin{problem}
\bparts

\ppart Give an example showing that two vertices in a digraph may be on
the same cycle, but \emph{not} necessarily on the same \emph{simple}
cycle.

\solution{Let the vertices be $a,b,c$ and edges be $(a,b), (b,a), (b,c),
(c,b)$.  Now $a$ and $c$ are on the cyle $a,b,c,b,a$, but every cycle from
$a$ to $c$ must go through $b$ at least twice, and so will not be simple.}

\ppart Prove that if two vertices in a digraph are connected, then they are
connected by a simple path.  \hint the shortest path.

\solution{
Consider a shortest path from $a$ to $b \neq a$:
\[
a=a_0,a_1,\dots,a_i,\dots, a_j, \dots ,a_k=b,
\]
and suppose this path is not simple.  That is, suppose $a_i=a_j$ for some
$i,j$.  Then
\[
a=a_0,a_1,\dots,a_i, a_{j+1}, \dots ,a_k=b.
\]
is a shorter path from $a$ to $b$, a contradiction.
}

\eparts


\end{problem}

%%%%%%%%%%%%%%%%%%%%%%%%%%%%%%%%%%%%%%%%%%%%%%%%%%%%%%%%%%%%%%%%%%%%%
% Problem ends here
%%%%%%%%%%%%%%%%%%%%%%%%%%%%%%%%%%%%%%%%%%%%%%%%%%%%%%%%%%%%%%%%%%%%%

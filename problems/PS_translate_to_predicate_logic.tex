\documentclass[problem]{mcs}

\begin{pcomments}
  \pcomment{PS_translate_to_predicate_logic}
  \pcomment{from: F09.ps2, S03.ps4}
  \pcomment{edited ARM 2/13/12}
\end{pcomments}

\pkeywords{
  logic
  predicate_calculus
  domain_of_discourse
  english_translation
  quantifier
}

%%%%%%%%%%%%%%%%%%%%%%%%%%%%%%%%%%%%%%%%%%%%%%%%%%%%%%%%%%%%%%%%%%%%%
% Problem starts here
%%%%%%%%%%%%%%%%%%%%%%%%%%%%%%%%%%%%%%%%%%%%%%%%%%%%%%%%%%%%%%%%%%%%%

\begin{problem}
Translate the following statements into predicate logic.  For each,
specify the domain of discourse.  In addition to logic symbols, you
may build predicates using arithmetic, relational symbols and
constants.  For example, the statement ``$n$ is an odd number'' could
be translated as $\exists m (2m+1 = n)$ where the domain of discourse
is $\integers$, the set of integers.  Another example is
``$p$ is a prime number,'' could be 
\[
(p > 1) \QAND\ \QNOT \paren{\exists m \exists n (m > 1 \QAND\ n > 1 \QAND\ mn = p)}
\]
Let $\text{prime}(p)$ be an abbreviation for this formula which you
may use in further formulas below.

\bparts

\ppart (Lagrange's Four-Square Theorem) Every nonnegative integer is
expressible as the sum of four perfect squares.

\begin{solution}
The domain of discourse is $\nngint$.

\[
\forall n \exists w \exists x \exists y \exists z (n = w^2 + x^2 + y^2 + z^2)
\]

\end{solution}

\ppart (Goldbach Conjecture) Every even integer greater than two is
the sum of two primes.

\begin{solution}
The domain of discourse is $\nngint$.  The statement could be
translated as
\[
\forall n\,
\paren{((n > 2) \QAND\ \exists m (n = 2m)) \QIMP\
      \exists p \exists q (\text{prime}(p) \QAND\ \text{prime}(q) \QAND\ (n = p + q))}
\]
\end{solution}

\ppart The function $f : \reals \mapsto \reals$ is continuous.

\begin{solution}
The domain of discourse is the real numbers $\reals$.
\[
\forall a \forall x \exists b \forall y
\paren{(a > 0 \QAND\ b > 0 \QAND\ \abs{x-y} < b) \QIMP\ \abs{f(x) - f(y)} < a}
\]
\end{solution}

\ppart
(Fermat's Last Theorem) There are no nontrivial solutions
to the equation:
\[
x^n + y^n = z^n
\]
over the nonnegative integers when $n > 2$.

\begin{solution}
The domain of discourse is $\nngint$.
\[
\forall x \forall y \forall z \forall n
\paren{(x > 0 \QAND\ y > 0 \QAND\ z > 0 \QAND\ n > 2)
    \QIMP\ \QNOT (x^n + y^n = z^n)}
\]
\end{solution}

\ppart
There is no largest prime number.

\begin{solution}
The domain of discourse is $\integers$.

\[
\QNOT \paren{\exists p (\text{Prime}(p) \QAND\ (\forall q (\text{Prime}(q) \QIMP\ p \geq q)))}
\]
\end{solution}

\ppart
(Bertrand's Postulate) If $n > 1$, then there is always
at least one prime $p$ such that $n < p < 2n$.

\begin{solution}
The domain of discourse is $\integers$.
\[
\forall n\,
\paren{ (n > 1) \QIMP\ (\exists p ( \text{Prime}(p)  \QAND\ (n < p) \QAND\ (p < 2n))) }
\]
\end{solution}

\eparts
\end{problem}

%%%%%%%%%%%%%%%%%%%%%%%%%%%%%%%%%%%%%%%%%%%%%%%%%%%%%%%%%%%%%%%%%%%%%
% Problem ends here
%%%%%%%%%%%%%%%%%%%%%%%%%%%%%%%%%%%%%%%%%%%%%%%%%%%%%%%%%%%%%%%%%%%%%

\endinput

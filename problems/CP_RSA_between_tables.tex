\documentclass[problem]{mcs}

\begin{pcomments}
  \pcomment{from: S09.cp8r, S06.cp7w}
%  \pcomment{}
%  \pcomment{}
\end{pcomments}

\pkeywords{
  number_theory
  modular_arithmetic
  primes
  RSA
  Pulverizer
}

%%%%%%%%%%%%%%%%%%%%%%%%%%%%%%%%%%%%%%%%%%%%%%%%%%%%%%%%%%%%%%%%%%%%%
% Problem starts here
%%%%%%%%%%%%%%%%%%%%%%%%%%%%%%%%%%%%%%%%%%%%%%%%%%%%%%%%%%%%%%%%%%%%%

\begin{problem}
Let's try out RSA!  There is a complete description of the algorithm at
the bottom of the page.  You'll probably need extra paper.  \textbf{Check
your work carefully!}

\bparts

\ppart As a team, go through the \textbf{beforehand} steps.

\begin{itemize}

\item Choose primes $p$ and $q$ to be relatively small, say in the
range 10-40.  In practice, $p$ and $q$ might contain several hundred
digits, but small numbers are easier to handle with pencil and paper.

\item Try $e = 3, 5, 7, \ldots$ until you find something that works.
Use Euclid's algorithm to compute the gcd.

\item Find $d$ (using the Pulverizer ---see appendix for a reminder on how
the Pulverizer works ---or Euler's Theorem).

\end{itemize}

When you're done, put your public key on the board.  This lets another
team send you a message.

\ppart Now send an encrypted message to another team using their
public key.  Select your message $m$ from the codebook below:

\begin{itemize}

\item 2 = Greetings and salutations!

\item 3 = Yo, wassup?

\item 4 = You guys are slow!

\item 5 = All your base are belong to us.

\item 6 = Someone on {\em our} team thinks someone on {\em your} team
is kinda cute.

\item 7 = You {\em are} the weakest link.  Goodbye.

\end{itemize}

\ppart Decrypt the message sent to you and verify that you received
what the other team sent!

\eparts

\TBA{Should the algorithm appear here or somewhere else?}

\begin{center}
RSA Public Key Encryption
\fbox{
\begin{minipage}[t]{6in}
\vspace{0.1cm}
\begin{description}

\item[Beforehand] The receiver creates a public key and a secret key
as follows.

\begin{enumerate}

\item Generate two distinct primes, $p$ and $q$.

\item Let $n = pq$.

\item Select an integer $e$ such that $\gcd(e, (p-1)(q-1)) = 1$.\\ The
{\em public key} is the pair $(e, n)$.  This should be distributed
widely.

\item Compute $d$ such that $de \equiv 1 \pmod{(p-1)(q-1)}$.\\ The
{\em secret key} is the pair $(d, n)$.  This should be kept hidden!

\end{enumerate}

\item[Encoding] The sender encrypts message $m$, where $0 \leq m < n$, to
  produce $m^\prime$ using the public key:
\[
m' = \rem{m^e}{n}.
\]

\item[Decoding] The receiver decrypts message $m'$ back to message $m$
using the secret key:
\[
m = \rem{(m')^d}{n}.
\]

\end{description}

\vspace{0.1cm}
\end{minipage}
}
\end{center}

\end{problem}

%%%%%%%%%%%%%%%%%%%%%%%%%%%%%%%%%%%%%%%%%%%%%%%%%%%%%%%%%%%%%%%%%%%%%
% Problem ends here
%%%%%%%%%%%%%%%%%%%%%%%%%%%%%%%%%%%%%%%%%%%%%%%%%%%%%%%%%%%%%%%%%%%%%

\endinput

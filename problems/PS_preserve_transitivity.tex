%PS_preserve_transitivity

\documentclass[problem]{mcs}

\begin{pcomments}
  \pcomment{from: F09.ps3 revised by ARM 9/26/09 from symmetry problem S02.ps3}
\end{pcomments}

\pkeywords{
  relations
  relational_properties
  composition
  transitive
}

%%%%%%%%%%%%%%%%%%%%%%%%%%%%%%%%%%%%%%%%%%%%%%%%%%%%%%%%%%%%%%%%%%%%%
% Problem starts here
%%%%%%%%%%%%%%%%%%%%%%%%%%%%%%%%%%%%%%%%%%%%%%%%%%%%%%%%%%%%%%%%%%%%%

\begin{problem}
Let $R$ and $S$ be binary relations on the same set, $A$.

\begin{definition}
  The \term{composition}, $S \compose R$, of $R$ and $S$\footnote{Note
    the reversal in the order of $R$ and $S$.  This is so that relational
    composition generalizes function composition, Composing the functions
    $f$ and $g$ means that $f$ is applied first, and then $g$ is applied
    to the result.  That is, value of the composition of $f$ and $g$
    applied to an argument, $x$, is $g(f(x))$.  To reflect this, the
    notation $g \compose f$ is commonly used for the composition of $f$
    and $g$.  Some texts do define $g \compose f$ the other way around.}
  is defined to be the binary relation on $A$ defined by the rule:
\begin{displaymath}
a \mrel{(S \compose R)} c  \qiff \exists b\, [a\, \mrel{R} b \QAND b \mrel{S} c].
\end{displaymath}
\end{definition}

Suppose both $R$ and $S$ are transitive.  Which of the following new
relations must also be transitive?  For each part, justify your answer
with a brief argument if the new relation is transitive and a
counterexample if it is not.

\bparts
\ppart $R^{-1}$

\begin{solution}
  $R^{-1}$: \textbf{Yes.}  Because $a \mrel{R} b \mrel{R} c$ iff $c
  \mrel{\inv{R}} b \mrel{\inv{R}} a$.
\end{solution}

\ppart $R \intersect S$

\begin{solution}
  \textbf{Yes.} $a \mrel{R\intersect S} b \mrel{R \intersect S} c$ iff [$a
  \mrel{R} b \mrel{R} c$ and $a \mrel{S} b \mrel{S} c]$ implies [$a
  \mrel{R} c$ and $a \mrel{S} c]$ iff $a \mrel{R\intersect S} b$.
\end{solution}


\ppart $R \composition R$

\begin{solution}
\textbf{Yes.}  $a R^2 c$ means $a \mrel{R} b \mrel{R} c$ for some $b$, and
since $R$ is transitive, $a \mrel{R} c$.  But then $a \mrel{R^2} b
\mrel{R^2} c$ implies $a \mrel{R} b \mrel{R} c$, which implies $a \mrel{R^2} c$.
\end{solution}

\ppart $R \composition S$

\begin{solution}
  \textbf{No.}  Suppose $A = \set{1,2,3,4,5}$, $\graph{R} = \set{(1, 2),
    (3,4)}$ , and $\graph{S} = \set{(2, 3), (4, 5)}$ .  Then $\graph{S
    \compose R} = \set{(1,3) (3,5)}$.  Now $R$ and $S$ are vacuously
  transitive, but $S \compose R$ is missing the $(1.5)$ arrow, and so is
  not transitive.
\end{solution}


\eparts

\end{problem}

%%%%%%%%%%%%%%%%%%%%%%%%%%%%%%%%%%%%%%%%%%%%%%%%%%%%%%%%%%%%%%%%%%%%%
% Problem ends here
%%%%%%%%%%%%%%%%%%%%%%%%%%%%%%%%%%%%%%%%%%%%%%%%%%%%%%%%%%%%%%%%%%%%%

\endinput

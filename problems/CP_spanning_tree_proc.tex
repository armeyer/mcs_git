\documentclass[problem]{mcs}

\begin{pcomments}
  \pcomment{from: S09.cp6t}
\end{pcomments}

\pkeywords{
  graphs
  spanning_trees
  state_machines
  increasing_decreasing_variables
  termination
}

%%%%%%%%%%%%%%%%%%%%%%%%%%%%%%%%%%%%%%%%%%%%%%%%%%%%%%%%%%%%%%%%%%%%%
% Problem starts here
%%%%%%%%%%%%%%%%%%%%%%%%%%%%%%%%%%%%%%%%%%%%%%%%%%%%%%%%%%%%%%%%%%%%%

\begin{problem}

Given a simple graph $G$, we apply the following operation to the
graph: pick two vertices $u \neq v$ such that either
\begin{enumerate}
\item there is an edge of $G$ between $u$ and $v$ and there is
also a path from $u$ to $v$ which does \emph{not} include this edge;
in this case, delete the edge $\edge{u}{v}$.

\item or, there is no path from $u$ to $v$; in which case, add the edge
  $\edge{u}{v}$.
\end{enumerate}

We keep repeating these operations until it is no longer possible to
find two vertices $u \neq v$ to which an operation applies.

Assume the vertices of $G$ are the integers $1,2,\dots,n$ for some $n \geq
2$.  This procedure can be modeled as a state machine whose states are
all possible simple graphs with vertices $1,2,\dots,n$.  The start state
is $G$, and the final states are the graphs on which no operation is
possible.

\bparts

\ppart Let $G$ be the graph with vertices $\set{1,2,3,4}$ and edges
\[
\set{\edge{1}{2},\edge{3}{4}}
\]
What are the possible final states reachable from start state $G$?  Draw
them.

\solution{It's not possible to delete any edge.  The procedure can only
add an edge connecting exactly one of vertices 1 or 2 to exactly one of
vertices 3 or 4, and then terminate.  So there are four possible final
states.}

\ppart \label{derived} For any state, $G'$, let $e$ be the number of edges
in $G'$, $c$ be the number of connected components it has, and $s$ be the
number of simple cycles.  For each of the derived variables below, indicate
the \emph{strongest} of the properties that it is guaranteed to satisfy, no
matter what the starting graph $G$ is and be prepared to briefly explain
your answer.

The choices for properties are: \emph{constant}, \emph{strictly
increasing}, \emph{strictly decreasing}, \emph{weakly increasing},
\emph{weakly decreasing}, \emph{none of these}.  The derived variables are
 
\begin{enumerate}

\item[(i)] $e$ \solution{none of these}

\item[(ii)] $c$ \solution{weakly decreasing}

\item[(iii)] $s$ \solution{weakly decreasing}

\item[(iv)] $e-s$  \solution{weakly increasing}

\item[(v)] $c+e$  \solution{weakly decreasing}

\item[(vi)] $3c + 2e$  \solution{strictly decreasing}

\item[(vii)] $c+s$ \solution{strictly decreasing}

\item[(viii)] $(c,e)$, partially ordered coordinatewise (the \emph{product}
partial order). \solution{none of these}

\item[(ix)] $(c,e)$, ordered lexicographically  \solution{strictly decreasing}

\end{enumerate}

\ppart Conclude that the procedure terminates by proving that one of the
derived variables above is strictly decreasing under some well founded
partial order.

\solution{To show that the variable (vi) strictly decreases, note that
  the rule for deleting an edge ensures that the connectedness relation
  does not change, so neither does the number of connected components $c$.
  Meanwhile the number of edges $e$ decreases by one when an edge is
  deleted.  Therefore the variable $3c+2e$ decreases by $2$.  The rule for
  adding an edge ensures that the number of connected components $c$
  decreases by one and the number of edges $e$ increases by one.
  Therefore the variable $3c+2e$ decreases by $1$.

  To show that the variable (vii) strictly decreases, note that the rule
  for deleting an edge ensures that the number of connected components $c$
  does not change and the number of simple cycles $s$ decreases by $n$,
  where $n \geq 1$. Therefore the variable $c+s$ decreases by $n$.  The
  rule for adding an edge ensures that the number of connected components
  $c$ decreases by one and the number of simple cycles $s$ does not
  change.  Therefore the variable $c+s$ decreases by one.

To show that the lexicographically ordered $(c,e)$ strictly decreases,
note that the rule for deleting an edge ensures that the number of
connected components $c$ does not change and the number of edges $e$
decreases by one.  The rule for adding an edge ensures that the number of
connected components $c$ decreases by one.}

\ppart Prove that any final state must be a tree on the vertices.

\solution{We use the characterization of a tree as an acyclic connected
graph.

A final state must be connected, because otherwise there would be two
vertices with no path between them, and then a transition adding the edge
between them would be possible, contradicting finality of the state.

A final state can't have a simple cycle, because deleting any edge on the
cycle would be a possible transition.}

\eparts
\end{problem}

%%%%%%%%%%%%%%%%%%%%%%%%%%%%%%%%%%%%%%%%%%%%%%%%%%%%%%%%%%%%%%%%%%%%%
% Problem ends here
%%%%%%%%%%%%%%%%%%%%%%%%%%%%%%%%%%%%%%%%%%%%%%%%%%%%%%%%%%%%%%%%%%%%%

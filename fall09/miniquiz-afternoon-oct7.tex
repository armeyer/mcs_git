\documentclass[quiz]{mcs}

\begin{document}

\begin{center}
{\Large Afternoon}
\end{center}

\miniquiz{Oct. 7}


%%%%%%%%%%%%%%%%%%%%%%%%%%%%%%%%%%%%%%%%%%%%%%%%%%%%%%%%%%%%%%%%%%%%%
% Problems start here
%%%%%%%%%%%%%%%%%%%%%%%%%%%%%%%%%%%%%%%%%%%%%%%%%%%%%%%%%%%%%%%%%%%%%
\instatements{\newpage}
\pinput[points = 4]{MQ_swapping_quantifiers}

\instatements{\newpage}
\pinput[points = 8]{MQ_partial_order}

\instatements{\newpage}
\pinput[points = 8]{MQ_10_and_15_cent_stamps_by_induction}

%%%%%%%%%%%%%%%%%%%%%%%%%%%%%%%%%%%%%%%%%%%%%%%%%%%%%%%%%%%%%%%%%%%%%
% Problems end here
%%%%%%%%%%%%%%%%%%%%%%%%%%%%%%%%%%%%%%%%%%%%%%%%%%%%%%%%%%%%%%%%%%%%%


%%%%%%%%%%%%%%%%%%%%%%%%%%%%%%%%%%%%%%%%%%%%%%%%%%%%%%%%%%%%%%%%%%%%%
% Appendix start here
%%%%%%%%%%%%%%%%%%%%%%%%%%%%%%%%%%%%%%%%%%%%%%%%%%%%%%%%%%%%%%%%%%%%%

\instatements{\newpage}
\section*{Appendix: Partial Order}

A binary relation, $R$, on a set, $A$, is
\begin{itemize}

\item \term{transitive} iff\quad $[(a\mrel{R}b\ \QAND\ b\mrel{R}c)\ \QIMPLIES\
  a\mrel{R} c]$, for all $a,b,c\in A$,

\item \term{asymmetric} iff\quad $[a\mrel{R}b\ \QIMPLIES \ \QNOT(b\mrel{R}a)]$,
  for all $a,b\in A$,

\item \term{reflexive} iff\quad $a \mrel{R} a$ for every $a \in A$,

\item \term{irreflexive} iff\quad $\QNOT(a \mrel{R} a)$ for all $a \in A$.

\item \term{antisymmetric} iff\quad $[a\mrel{R}b\ \QIMPLIES \QNOT(b\mrel{R}a)]$,
  for all $a \neq b\in A$.

\end{itemize}

A binary relation is a \term{strict partial order} iff it is transitive
and asymmetric.  It is a \term{weak partial order} iff it is transitive,
reflexive, and antisymmetric.

Let $\prec$ be a partial order (strict or weak) on a set, $A$.
\begin{itemize}

\item An element $a \in A$ is \term{minimal} iff there is no \emph{other}
  element in $A$ that is $\prec a$.  Similarly, an element $a \in A$ is
  \term{maximal} iff there is no \emph{other} element in $A$ that $a$ is
  $\prec$ to.

\item An element $a \in A$ is the \term{minimum} element iff $a$ is
  $\prec$ to every \emph{other} element of $A$.  Similarly, an element $a
  \in A$ is the \term{maximum} element iff every \emph{other} element of
  $A$ is $\prec a$.

\item Elements $a,b \in A$ are \term{comparable} iff either $a \prec b$ or
  $b \prec a$.  Two elements are \term{incomparable} iff they are not
  comparable.

\item A subset, $S \subseteq A$ is \term{totally ordered} iff every
two distinct elements in $S$ are comparable.

\item A \term{chain} is a totally ordered subset of $A$.  

\item An \term{antichain} is a subset of $A$, such that no two distinct
  elements in it are comparable.

\iffalse
\item An element $a \in A$ is the \term{greatest lower bound (glb)} of a
subset, $S \subseteq A$ iff $a$ is a lower bound for $S$, and if $b \in A$
is also a lower bound for $S$, then $b \preceq a$.  Similarly for
\term{least upper bound (lub)}.
\fi

\iffalse
\item The partial order, $\prec$, is \term{well-founded} iff every
  nonempty subset of $A$ has at least one minimal element.
\fi

\end{itemize}

\end{document}

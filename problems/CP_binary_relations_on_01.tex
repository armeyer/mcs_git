\documentclass[problem]{mcs}

\begin{pcomments}
  \pcomment{from: S09.cp4m}
  \pcomment{A bit messy - could use some revision to remove the ellipses.}
%  \pcomment{}
\end{pcomments}

\pkeywords{
  binary
  relations
  relational_properties
  partial_orders
}

%%%%%%%%%%%%%%%%%%%%%%%%%%%%%%%%%%%%%%%%%%%%%%%%%%%%%%%%%%%%%%%%%%%%%
% Problem starts here
%%%%%%%%%%%%%%%%%%%%%%%%%%%%%%%%%%%%%%%%%%%%%%%%%%%%%%%%%%%%%%%%%%%%%

\begin{problem}
How many binary relations are there on the set $\set{0,1}$?

How many are there that transitive?, \dots asymmetric?, \dots reflexive?,
\dots irreflexive?, \dots strict partial orders?, \dots weak partial
orders?

\hint There are easier ways to find these numbers than listing all
the relations and checking which properties each one has.

\solution{
There are $2^4 = 16$ such relations, since in any such relation
  there are four possible arrows between $\set{0,1}$ and itself, each of
  which may or may not be there.

There are 3 \textbf{in}transitive transitive relations, because the only
way transitivity can fail in a relation on two elements is when there is
an arrow in both directions between the elements, but one or the other or
both the elements are missing a \term{self-loop}, that is, an arrow that
starts and ends at the element.  So there are $13=16-3$ transitive
relations.

There are 3 asymmetric relations.  Asymmetry implies no self-loops, and at
most one of the two possible arrows between 0 and 1.  So the only
3 possibilities are no arrows, arrow from 0 to 1, arrow from 1 to 0.

There are 4 reflexive relations, because two of the four possible arrows
(the self-loops) must be present, the remaining two arrows can be either
present or not present, which yields $2^2$ relations.  There are 4
irreflexive relations for the same reason.

There are 3 strict partial orders, because the 3 asymmetric relations are
all transitive.

There are 3 weak partial orders, because the 3 strict partial orders
remain distinct after adding self-loops to both elements.
}
\end{problem}

%%%%%%%%%%%%%%%%%%%%%%%%%%%%%%%%%%%%%%%%%%%%%%%%%%%%%%%%%%%%%%%%%%%%%
% Problem ends here
%%%%%%%%%%%%%%%%%%%%%%%%%%%%%%%%%%%%%%%%%%%%%%%%%%%%%%%%%%%%%%%%%%%%%

\endinput

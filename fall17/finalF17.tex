%finalF17

\documentclass[quiz]{mcs}

%\renewcommand{\examspace}{}  %DISABLE EXAM SPACINGG

\renewcommand{\exampreamble}{
  \begin{itemize}

  \item
   This exam is \textbf{closed book} except for two 2-sided cribsheets.
   Total time is 180 minutes.

  \item
   Write your solutions in the space provided.  If you need more
   space, \textbf{write on the back} of the sheet containing the
   problem.

   \item In answering the following questions, you may use without
     proof any of the results from class or text.

  \end{itemize}}

\begin{document}

\final

%%%%%%%%%%%%%%%%%%%%%%%%%%%%%%%%%%%%%%%%%%%%%%%%%%%%%%%%%%%%%%%%%%%%%
% Problems start here
%%%%%%%%%%%%%%%%%%%%%%%%%%%%%%%%%%%%%%%%%%%%%%%%%%%%%%%%%%%%%%%%%%%%%

\begin{center}
{\LARGE \textbf{Short-Answer Questions}}
\end{center}

\emph{\large The following questions are short-answer.  The graders
  will not read explanations, so do not spend time including them.}

\iffalse
\pinput[points = 9, title = \textbf{Logical
    Formulas}]{TP_Propositions_with_Quantifiers}
\fi

\pinput[points = 12, title =
  \textbf{Quantifiers, Law of Large Numbers}]{FP_large_numbers_quantifiers}

\pinput[points = 15, title =
  \textbf{Counting Relations}]{FP_counting_graphs_f13}

\pinput[points = 12, title = \textbf{Remainder
    Arithmetic}]{TP_Fermats_Little_Theorem_F13}

\pinput[points = 12, title= \textbf{Scheduling}]{FP_schedule_B4_F17}

\pinput[points = 10, title = \textbf{Stable
    Marriage}]{TP_Stable_Marriage_Invariants_F17}

\begin{center}
{\LARGE \textbf{Proof and Concept Questions}}
\end{center}

\pinput[points = 12, title = \textbf{GCD}]{TP_lin_neq2}
\examspace

\pinput[points = 20, title = \textbf{Well Ordering
    Principle}]{MQ_prove_by_wop_odds}
\examspace

\pinput[points = 20, title = \textbf{Induction, Simple
    Graphs}]{FP_cycles_components_induction}
\examspace

\pinput[points = 20, title = \textbf{Structural
    Induction}]{FP_binary_tree_induction}
\examspace

\pinput[points = 20, title =
  \textbf{Diagonalization/Jections}]{FP_diagonalization_lonely_subsets}
\examspace

\pinput[points = 12, title = \textbf{Big Oh}]{MQ_asymptotic_incomparable}
\examspace

\pinput[points = 15, title = \textbf{Variance}]
       {FP_variance_dice_sum}
\examspace

\pinput[points = 20, title= \textbf{Chebyshev Bound}]{FP_chebyshev_pq_F17}
\examspace

%%%%%%%%%%%%%%%%%%%%%%%%%%%%%%%%%%%%%%%%%%%%%%%%%%%%%%%%%%%%%%%%%%%%%
% Problems end here
%%%%%%%%%%%%%%%%%%%%%%%%%%%%%%%%%%%%%%%%%%%%%%%%%%%%%%%%%%%%%%%%%%%%%
\end{document}

\iffalse

\pinput[points = 12, title = \textbf{Graph Properties}]{FP_graph_variables_F17}

\pinput[points = 8, title = \textbf{Modular Inverse}]{FP_353_factor}

\pinput[points = 10, title = \textbf{Partial Orders, Multiple choice}]
       {FP_partial_order_short_answer_f17}

\pinput[points = 18, title = \textbf{Simple Graphs \& Trees}]
       {FP_simple_graphs_trees_short_answer_F17}

\pinput[points = 20, title = \textbf{Induction}]{FP_induction_nn1}

\pinput[points = 12, title=
  \textbf{Inclusion-Exclusion}]{FP_permutations_inc_exc}

\pinput[points = 8, title = \textbf{Transitivity,
    Independence}]{FP_independent_intransitive}

\pinput[points = 10, title = \textbf{Simple Graph, State
    Machine}]{MQ_graph_state_machine_f15}

\pinput[points = 10, title = \textbf{Minimum Weight Spanning
    Trees}]{MQ_local_min_gray_edge}

\pinput[points = 15, title = \textbf{Expectation}]
       {MQ_expectHH_TT_S16}

\pinput[points = 10, title= \textbf{Bijections}]{FP_card_bij}

\pinput[points = 10, title=
  \textbf{Probability}]{FP_red_and_blue_goats}

\pinput[points = 10, title=
  \textbf{Probability/Counting}]{FP_string_counting}

\pinput[points = 15, title= \textbf{Probable
    Satisfiability}]{CP_probable_satisfiability_nk}

%cp12m
\pinput[points = 10, title=
  \textbf{Bayes}]{FP_conditional_beaver_fever}

%cp12m
\pinput[points = 10, title=
  \textbf{Probability}]{FP_college_probability}

%mechanical
\pinput[points = 10, title=
  \textbf{Pulverizer}]{FP_pulverizer}

%NEEDS QA:
\pinput[points = 15, title= \textbf{Probable
    Coloring}]{FP_coloring_complete_triangles}

\pinput[points = 15, title= \textbf{Fermat's
    Theorem}]{FP_Fermat_primes}

%too much an IQ test.
\pinput[points = 10, title= \textbf{Bijections}]{FP_sum_of_digits_bijection}

%cumbersome argument
\pinput[points = 10, title=
  \textbf{Cardinalilty}]{FP_infinite_binary_sequences_S14}

%over-simplified version of mid4 problem
\pinput[points = 10, title=
  \textbf{Pigeonholes}]{FP_monochromatic_rectangles}

%mid4 problem
\pinput[points = 10, title=
  \textbf{Counting}]{FP_bijection_counting}

%awkward ellipsis
\pinput[points = 10, title=
  \textbf{Counting}]{FP_hockey_stick_formula}

%doable IQ test since like class prob; not interesting
\pinput[points = 10, title=
  \textbf{Pigeonholes}]{FP_15_pigeons}

\pinput[points = 10, title=
  \textbf{Inclusion-Exclusion}]{FP_poker_inc_exc}

%complicated
\pinput[points = 10, title= \textbf{Structural
    Induction}]{FP_red_black_tree_induction}

%didn't do 'eval' in F17
\begin{staffnotes}
ONLY ONE OF NEXT TWO:
\end{staffnotes}

\pinput[points = 10, title= \textbf{Structural
    Induction}]{FP_structural_ind_polynomials}

\pinput[points = 10, title= \textbf{Structural
    Induction}]{FP_structural_induction_polynomial}

\pinput[points = 8, title = \textbf{Modular
    Inverse}]{FP_check_factor_by_digits_proof_conflict}

%midterm3
\pinput[points = 10, title= \textbf{Vertex
    Degrees}]{FP_degree_sequencesS15}

\pinput[points = 10, title= \textbf{Search Trees}]{PS_search1-n}

%nice problem, not important
\pinput[points = 10, title= \textbf{Independent
    Variables}]{MQ_4_way_dependent_ions}

\pinput[points = 10, title = \textbf{Matching,   
    Counting}]{FP_magic_trick_27_cards}

\pinput[points = 10, title =
  \textbf{Simple graphs: Walks \& Cycles}]{FP_odd_length_walk_simple}

\pinput[points = 10, title = \textbf{Propositional Formulas,
    Structural Induction}]{CP_XOR_AND_recursive}

\pinput[points = 10, title = \textbf{Propositional Formulas,
    Structural Induction}]{FP_OR_AND_recursive_multivar}

\pinput[points = 7, title = \textbf{GCD, True/False}]
       {FP_gcd_TF}

\pinput[points = 5, title = \textbf{Congruence, True/False}]
       {FP_congruence_TF}

\pinput[points = 5, title = \textbf{Bogus
    WOP}]{TP_bogus_well_ordering_fibonacci_proof}

\pinput[points = 18, title= \textbf{Counting}]
{FP_counting_given_answers2}

\pinput[points = 10, title = \textbf{Sampling, Confidence}]{TP_sampling_perturbed}

%F15 final conflict_2
\pinput[points = 16, title = \textbf{Number Theory}]
{FP_numbers_short_answer2_F15}

\pinput[points = 15, title= \textbf{Relations, Probability}]
{FP_probability_relations_short_answer2}

\pinput[points = 4, title = \textbf{Graphs, True/False}]
       {FP_simple_graphs_TF}

\pinput[points = 10, title = \textbf{Relations,
    Predicates}]{FP_relation_properties_expressions_final_f13}

\pinput[points = 10, title =
  \textbf{Quantifiers}]{FP_limit_quantifiers}

\pinput[points = 10, title = \textbf{Logical
    Formulas}]{TP_Quantifiers}

%too hard for final?  hard to grade?:
\pinput[points = 10, title= \textbf{Minimum Spanning Trees}]{CP_maxweight_edge}

%ps3
\pinput{FP_rogue_pair}

%ps4

\pinput[points = 10, title= \textbf{Cardinality}]{TP_cardinality_class}
\pinput[points = 10, title= \textbf{Cardinality}]{MQ_set_cardinality}
\pinput{FP_cardinality}  

\pinput[points = 10, title= \textbf{Cardinality}]{FP_infinite_sequence_injection} 

%needs figures
\pinput[points = 10, title= \textbf{Partial Orders}]{TP_which_are_partial_orders}

\pinput[points = 10, title= \textbf{Partial Orders}]{FP_partial_order_short_answer}

\pinput[points = 10, title= \textbf{Partial Orders}]{TP_equivalence_relations_partial_orders}

\pinput[points = 10, title= \textbf{Matching}]{MQ_degree_constrained}

%OY
\pinput[points = 10, title= \textbf{Asymptotics}]{PS_asymptotics_table}

%repeat? if so, could perturb or skip
\pinput[points = 10, title= \textbf{Pigeonholes}]{PS_monochromatic_rectangle}

\fi

\documentclass[problem]{mcs}

\begin{pcomments}
  \pcomment{FP_polynomial_modk}
  \pcomment{stimulated by FP_sequence_gcd}
  \pcomment{ARM 4/30/18}
\end{pcomments}

\pkeywords{
  polynomial
  gcd
  lcm
  number_theory
  modular
}

%%%%%%%%%%%%%%%%%%%%%%%%%%%%%%%%%%%%%%%%%%%%%%%%%%%%%%%%%%%%%%%%%%%%%
% Problem starts here
%%%%%%%%%%%%%%%%%%%%%%%%%%%%%%%%%%%%%%%%%%%%%%%%%%%%%%%%%%%%%%%%%%%%%

\begin{problem}
Let $p(x)$ be an integer polynomial, that is, $p(x) = \sum_{i=0}^d c_i
x^i$ where the coefficients $c_i$ are integers.

\bparts

\ppart\label{pkprem} Explain why $p(k) \equiv p(\rem{k}{n})
\pmod n$ for all integers $k, n$ where $n>1$.

\examspace[1.5in]
\begin{solution}
By the General Principle of Remainder Arithmetic \inbook{of
  Section~\bref{remainder_arithmetic_sec}}, in any sequence of integer
additions and multiplications, we can replace any integer by its
remainder on division by $n$, and the result will still have the same
remainder on division by $n$.  So replacing $k$ by $\rem{k}{n}$ in any
integer polynomial will leave the same remainder on division by $n$.
That is, $\rem{p(k)}{n} = \rem{p(\rem{k}{n})}{n}$, which means $p(k)
\equiv p(\rem{k}{n}) \pmod n$.
\end{solution}

\eparts


Now let
\[
q(x) \eqdef (x^2-4)(x^2-9),
\]
and let $q(\nngint) \eqdef \set{q(0),q(1),q(2),\dots}$.

\bparts
\ppart\label{3dQ}
Verify that 3 divides every element of $q(\nngint)$.  \inhandout{\hint part~\eqref{pkprem}}

\examspace[1.5in]
\begin{solution}
Every element of $q(\nngint)$ is equivalent modulo three to $p(0) = 36$, $p(1)
= 24$, or $p(2) = 0$ by part~\eqref{pkprem}.  Since 36, 24, and 0 are
all equivalent to zero modulo three, so is every element of $q(\nngint)$.
\end{solution}

\ppart\label{4dQ}
Verify that 4 divides every element of $q(\nngint)$.

\examspace[1.5in]

\begin{solution}
Similarly, every element of $q(\nngint)$ is equivalent modulo four to $p(0) =
36$, $p(1) = 24$, or $p(2) = 0 = p(3)$ by part~\eqref{pkprem}.  Since
36, 24, and 0 are all equivalent to zero modulo four, so is every
element of $q(\nngint)$.
\end{solution}

\ppart
Prove that $\gcd(q(\nngint)) = 12$.



\iffalse 12 is the largest integer that divides every element of
$q(\nngint)$.\fi


\begin{solution}
Any integer that divides $q(0) = 36$ and $q(1)= 24$ must divide
their gcd $12$.  So any integer that divides every element of
$q(\nngint)$ must divide 12.

But by parts~\eqref{3dQ},~\eqref{4dQ}, every element of $q(\nngint)$ is
divisible by three and four, and so is divisible by $\lcm(3,4) = 12$.
So the largest integer that divides every element of $q(\nngint)$ must divide
twelve and be divisible by twelve, and this implies that it equals twelve.
\end{solution}

\eparts

\end{problem}

\endinput

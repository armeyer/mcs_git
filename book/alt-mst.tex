
 \iffalse In the rest of this section, subgraphs and forests will
 always be spanning forests and spanning subgraphs of $G$, that is
 they will have the same vertices as $G$.\fi

If $e$ is an edge of $G$ and $S$ is a spanning subgraph, we'll write
$S+e$ for the spanning subgraph with edges $\edges{S} \union \set{e}$.

\begin{definition}
Let $F$ be a pre-MST, and color the vertices in each connected
component of $F$ either all black or all white.  At least one
component of each color is required.  Call this a \term{solid
  coloring}\index{coloring!solid} of $F$.  A \term{gray edge} of a
solid coloring is an edge of $G$ with different colored endpoints.
\end{definition}

Any path in $G$ from a white vertex to a black vertex obviously must
include a gray edge, so for any solid coloring, there is guaranteed to
be at least one gray edge.  In fact, there will have to be at least as
many gray edges as there are components with the same color.  Here's
the punchline:

\begin{lemma}\label{lem:edgeextends}
An edge extends a pre-MST $F$ if it is a minimum weight gray edge in
some solid coloring of $F$.
\end{lemma}

\begin{editingnotes}
I think the converse can be proved by coloring the ends of an
extending edge different colors, but this needs to be checked.  Could
be a nice pset problem.  Would also imply uniqueness of MST when
weights are different, since there is only one way to grow the tree
using MST1 or MST2.
\end{editingnotes}

So to extend a pre-MST, choose any solid coloring, find the gray
edges, and among them choose one with minimum weight.  Each of these
steps is easy to do, so it is easy to keep extending and arrive at an
MST.  For example, here are three known algorithms that are explained
by Lemma~\ref{lem:edgeextends}:

\begin{algorithm}\label{alg:MST1}[Prim]
  Grow a tree one edge at a time by adding a minimum weight edge
  among the edges that have exactly one endpoint in the tree.
\end{algorithm}

This is the algorithm that comes from coloring the growing tree white
and all the vertices not in the tree black.  Then the gray edges are
the ones with exactly one endpoint in the tree.

\begin{algorithm}\label{alg:MST2}[Kruskal]
  Grow a forest one edge at a time by adding a minimum weight edge
  among the edges with endpoints in different connected components.
\end{algorithm}

An edge does not create a cycle iff it connects different components.
The edge chosen by Kruskal's algorithm will be the minimum weight gray
edge when the components it connects are assigned different colors.

\iffalse  %earlier version
The edges between different components are exactly the edges that are
gray under some solid coloring, namely any coloring where the
components it connects have different colors.
\fi

For example, in the weighted graph we have been considering, we might
run Algorithm~\ref{alg:MST1} as follows.  Start by choosing one of the
weight~1 edges, since this is the smallest weight in the graph.
Suppose we chose the weight~1 edge on the bottom of the triangle of
weight~1 edges in our graph.  This edge is incident to the same vertex
as two weight~1 edges, a weight~4 edge, a weight~7 edge, and a
weight~3 edge.  We would then choose the incident edge of minimum
weight.  In this case, one of the two weight~1 edges.  At this point,
we cannot choose the third weight~1 edge: it won't be gray because its
endpoints are both in the tree, and so are both colored white.  But we
can continue by choosing a weight~2 edge.  We might end up with the
spanning tree shown in Figure~\ref{fig:5KC}, which has weight~17, the
smallest we've seen so far.

\begin{figure}

\graphic{Fig_5KC}

\caption{A spanning tree found by Algorithm~\ref{alg:MST1}.}

\label{fig:5KC}

\end{figure}

Now suppose we instead ran Algorithm~\ref{alg:MST2} on our graph.  We
might again choose the weight~1 edge on the bottom of the triangle of
weight~1 edges in our graph.  Now, instead of choosing one of the
weight~1 edges it touches, we might choose the weight~1 edge on the
top of the graph.  This edge still has minimum weight, and will be
gray if we simply color its endpoints differently, so
Algorithm~\ref{alg:MST2} can choose it.  We would then choose one of
the remaining weight~1 edges.  Note that neither causes us to form a
cycle.  Continuing the algorithm, we could end up with the same
spanning tree in Figure~\ref{fig:5KC}, though this will depend on
the tie breaking rules used to choose among gray edges with the same
minimum weight.  For example, if the weight of every edge in $G$ is
one, then all spanning trees are MST's with weight
$\card{\vertices{G}}-1$, and both of these algorithms can arrive at
each of these spanning trees by suitable tie-breaking.

\begin{editingnotes}
Verify that Prim \& Kruskal can create every MST.
\end{editingnotes}

The coloring that explains Algorithm~\ref{alg:MST1} also justifies a more
flexible algorithm which has Algorithm~\ref{alg:MST1} as a special case:
\begin{algorithm}\label{alg:MST3}
  Grow a forest one edge at a time by picking any component and adding a
  minimum weight edge among the edges leaving that component.
\end{algorithm}
This algorithm allows components that are not too close to grow in
parallel and independently, which is great for ``distributed''
computation where separate processors share the work with limited
communication between processors.\footnote{The idea of growing trees
  seems first to have been developed in by Bor\.{u}vka (1926), ref
  TBA.  Efficient MST algorithms running in parallel time $O(\log
  \card{V})$ are described in Karger, Klein, and Tarjan (1995), ref
  TBA.}

These are examples of greedy approaches to optimization.  Sometimes
greediness works and sometimes it doesn't.  The good news is that it
does work to find the MST.  Therefore, we can be sure that the MST for
our example graph has weight~17, since it was produced by
Algorithm~\ref{alg:MST2}.  Furthermore we have a fast algorithm for
finding a minimum weight spanning tree for any graph.

Ok, to wrap up this story, all that's left is the proof that minimal
gray edges are extending edges.  This might sound like a chore, but it
just uses the same reasoning we used to be sure there would be a gray
edge when you need it.

\begin{proof} (of Lemma~\ref{lem:edgeextends})

  Let $F$ be a pre-MST that is a subgraph of some MST $M$ of $G$, and
  suppose $e$ is a minimum weight gray edge under some solid coloring of
  $F$.  We want to show that $F+e$ is also a pre-MST.

  If $e$ happens to be an edge of $M$, then $F+e$ remains a subgraph of
  $M$, and so is a pre-MST.

  The other case is when $e$ is not an edge of $M$.  In that case,
  $M+e$ will be a connected, spanning subgraph.  Also $M$ has a path
  $\walkv{p}$ between the different colored endpoints of $e$, so $M+e$
  has a cycle consisting of $e$ together with $\walkv{p}$.  Now
  $\walkv{p}$ has both a black endpoint and a white one, so it must
  contain some gray edge $g \neq e$.  The trick is to remove $g$ from
  $M+e$ to obtain a subgraph $M+e-g$.  Since gray edges by definition
  are not edges of $F$, the graph $M+e-g$ contains $F+e$.  We claim
  that $M+e-g$ is an MST, which proves the claim that $e$ extends $F$.

\begin{editingnotes}
  CLR illustrate $M+e-g$ in a figure, but I don't think the figure helped
  --ARM.
\end{editingnotes}

   To prove this claim, note that $M+e$ is a connected, spanning
   subgraph, and $g$ is on a cycle of $M+e$, so by
   Lemma~\ref{lem:cutiffcycle}, removing $g$ won't disconnect
   anything.  Therefore, $M+e-g$ is still a connected, spanning
   subgraph.  Moreover, $M+e-g$ has the same number of edges as $M$,
   so Lemma~\ref{lem:iffe=v-1} implies that it must be a spanning
   tree.  Finally, since $e$ is minimum weight among gray
   edges,\iffalse $w(e) \leq w(g)$, and therefore\fi
   \[
   w(M+e-g) = w(M) + w(e) - w(g) \leq w(M).
   \]
   This means that $M+e-g$ is a spanning tree whose weight is at most
   that of an MST, which implies that $M+e-g$ is also an MST.
\end{proof}

\begin{editingnotes}
  Verify that the converse of Lemma~\ref{lem:edgeextends} is true.
  That is, every extending edge is a minimal gray edge for some
  coloring.
\end{editingnotes}

Another interesting fact falls out of the proof of Lemma~\ref{lem:edgeextends}:
\begin{corollary}\label{cor:uniqMST}
If all edges in a weighted graph have distinct weights, then the graph has
a \emph{unique} MST.
\end{corollary}

The proof of Corollary~\ref{cor:uniqMST} is left to Problem~\ref{PS_unique_MST}.

\begin{editingnotes}
  Another approach to proving uniqueness: show that any MST can be the
  result of Algorithm~\ref{alg:MST1} (also true for
  Algorithm~\ref{alg:MST2}).  But Algorithm~\ref{alg:MST1} is
  deterministic when there are no ties, so the unique tree its
  produces must be all the trees there are.
\end{editingnotes}

\begin{editingnotes}
  CLR have lots of great exercises about all this.
\end{editingnotes}

\iffalse  FTL's earlier draft

It's a little easier
to prove it for Algorithm~\ref{alg:MST2}, so we'll do that one here.

\begin{theorem}\label{thm:MST2}
For any connected, weighted graph~$G$, Algorithm~\ref{alg:MST2}
produces an MST for~$G$.
\end{theorem}

\begin{proof}
The proof is a bit tricky.  We need to show the algorithm terminates,
namely, if we have selected fewer than $n - 1$ edges, then we can
always find an edge to add that does not create a cycle.  We also need
to show the algorithm creates a tree of minimum weight.

The key to doing all of this is to show that the algorithm never gets
stuck or goes in a bad direction by adding an edge that will keep us
from ultimately producing an MST\@.  The natural way to prove this is
to show that the set of edges selected at any point is contained in
some MST, in other words, we can always get to where we need to be.
We'll state this as a lemma.

\begin{lemma}\label{lemma:MST2}
  For any $m \ge 0$, let $S$ consist of the first $m$ edges selected by
  Algorithm~\ref{alg:MST2}.  Then there exists some MST~$T$ for~$G$ such
  that $S \subseteq \edges{T}$, that is, the set of edges that we are
  growing is always contained in some MST\@.
\end{lemma}

We'll prove this momentarily, but first let's see why it helps prove
the theorem.  Assume the lemma is true.  Then how do we know
Algorithm~\ref{alg:MST2} can always find an edge to add without
creating a cycle?  Well, as long as there are fewer than $n - 1$ edges
picked, there exists some edge in $\edges{T} - S$ and so there is an
edge that we can add to~$S$ without forming a cycle.  Next, how do we
know that we get an MST at the end?  Well, once $m = n - 1$, we know
that $S$ will be the edges of an MST.

Ok, so the theorem is an easy corollary of the lemma.  To prove the
lemma, we'll use induction on the number of edges chosen by the
algorithm so far.  This is very typical in proving that an algorithm
preserves some kind of invariant condition---induct on the number of
steps taken, namely, the number of edges added.

Our inductive hypothesis $P(m)$ is the following: for any $G$ and any
set~$S$ of $m$ edges initially selected by Algorithm~\ref{alg:MST2},
there exists an MST $T$ of~$G$ such that $S \subseteq \edges{T}$.

For the base case, we need to show $P(0)$.  In this case, $S =
\emptyset$, so $S \subseteq \edges{T}$ holds trivially for any MST
$T$.

For the inductive step, we assume $P(m)$ holds and show that it
implies $P(m + 1)$.  Let $e$ denote the $(m+1)$st edge selected by
Algorithm~\ref{alg:MST2}, and let $S$ denote the first $m$ edges
selected by Algorithm~\ref{alg:MST2}.  Let $T^*$ be the
MST such that $S \subseteq \edges{T^*}$, which exists by the inductive
hypothesis.  There are now two cases:
\begin{description}

\item[Case 1:] $e \in \edges{T^*}$, in which case $S \union \{e\}
  \subseteq \edges{T^*}$, and thus $P(m+1)$ holds.

\item[Case 2:]
$e \notin \edges{T^*}$, as illustrated in Figure~\ref{fig:5KD}.  Now we need
  to find a different MST that contains $S$ and~$e$.

\begin{figure}

\graphic{Fig_5KD}

\caption{The graph formed by adding $e$ to $\edges{T^*}$.  Edges of~$S$ are
  denoted with solid lines and edges of $\edges{T^*} - S$ are denoted with
  dashed lines.}

\label{fig:5KD}
\end{figure}

\end{description}

What happens when we add $e$ to~$\edges{T^*}$?  Since $T^*$ is a tree,
we get a cycle.  (Here we used part~3 of Theorem~\ref{th:treeprops}.)
Moreover, the cycle cannot only contains edges in~$S$, since $e$ was
chosen so that together with the edges in~$S$, it does not form a
cycle.  This implies that $\set{e} \union \edges{T^*}$ contains a
cycle that contains an edge $e'$ of $\edges{T^*} - S$.  For example,
such an $e'$ is shown in Figure~\ref{fig:5KD}.

Note that the weight of~$e$ is at most that of~$e'$.  This is because
Algorithm~\ref{alg:MST2} picks the minimum weight edge that does not
make a cycle with~$S$.  Since $e' \in \edges{T^*}$, edge $e'$ cannot
make a cycle with~$S$, and if the weight of~$e$ were greater than the
weight of~$e'$, Algorithm!\ref{alg:MST2} would no have selected~$e$
ahead of~$e'$.

Okay, we're almost done.  Now we'll make an MST that contains $S
\union \set{e}$.  Let $T^{**}$ be the graph with
\begin{align*}
\vertices{T^{**}} & \eqdef \vertices{T},\\
\edges{T^{**}} & \eqdef (\edges{T^*} - \set{e'}) \union \set{e}.
\end{align*}
That is, we swap $e$ and~$e'$ in~$\edges{T^*}$.

\begin{claim}\label{claim:MST2}
$T^{**}$ is an MST.
\end{claim}

\begin{proof}[Proof of claim]
We first show that $T^{**}$ is a spanning tree.  $T^{**}$ is acyclic
because it was produced by removing an edge from the only cycle in
$T^{*} \union \set{e}$.  $T^{**}$ is connected since the edge we
deleted from $\edges{T^*} \union \set{e}$ was on a cycle.  Since
$T^{**}$ contains all the nodes of~$G$, it must be a spanning tree
for~$G$.

Now let's look at the weight of~$T^{**}$.  Well, since the weight
of~$e$ was at most that of~$e'$, the weight of~$T^{**}$ is at most
that of~$T^*$, and thus $T^{**}$ is an MST for~$G$.
\end{proof}

Since $S \union \set{e} \subseteq \edges{T^{**}}$, we have shown that
  $P(m + 1)$ holds.  Thus, Algorithm~\ref{alg:MST2} must eventually
  produce an MST\@.  This will happens when it adds $n - 1$ edges to
  the subgraph it builds.
\end{proof}
\fi

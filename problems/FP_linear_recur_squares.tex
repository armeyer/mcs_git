\documentclass[problem]{mcs}

\begin{pcomments}
  \pcomment{FP_linear_recur_squares}
  \pcomment{ZDz 11/22/15}
\end{pcomments}

\pkeywords{
  recurrence
  linear_recurrence
  generating_function
}

%%%%%%%%%%%%%%%%%%%%%%%%%%%%%%%%%%%%%%%%%%%%%%%%%%%%%%%%%%%%%%%%%%%%%
% Problem starts here
%%%%%%%%%%%%%%%%%%%%%%%%%%%%%%%%%%%%%%%%%%%%%%%%%%%%%%%%%%%%%%%%%%%%%

\begin{problem}
Express the generating function $R(x) \eqdef \sum_0^\infty
r_nx^n$ as a quotient of polynomials or products of polynomials, where $r_n$ is defined as
\[r_n = \begin{cases}
           r_{n-1} + (2n-1) & \text{for } n\geq 1,\\
           0                           &  \text{for } n = 0.
       \end{cases}
\]

You do \emph{not} have to find a closed form for $r_n$.

\begin{solution}
Let $c_n = 2n+1$ for $n \geq 0$. Using that the generating function for $n$ is
$\frac{x}{(1-x)^2}$ and for $1$ is $\frac{1}{1-x}$, the generating function for $c_n$ is
\[
C(x) = \frac{2x}{(1-x)^2} + \frac{1}{1-x} = \frac{1+x}{(1-x)^2} \, .
\]
Then, the generating function for
\[a_n = \begin{cases}
           2n-1 & \text{for } n\geq 1,\\
           0       &  \text{for } n = 0.
       \end{cases}
\]
is
\[
A(x) = xC(x) = \frac{x(1+x)}{(1-x)^2} \, ,
\]
which follows from the fact that $a_n$ is obtained by shifting $c_n$ to the right by
one position.

We have:
\[
\begin{array}{rcrcrcrcrcl}
R(x)                                & = & r_0 & + & r_1 x & + & r_2 x^2 & + & r_3 x^3  & + & \cdots\\
- xR(x)                            & = &        & - & r_0 x & -  & r_1 x^2 & - & r_2 x^3 & - & \cdots\\
- \frac{x(1+x)}{(1-x)^2} & = &        & - &    1 x & -  &    3 x^2 & - &  5 x^3 & - & \cdots\\
\hline
%         & = & r_0 & + & 0 x & + &    0 x^2 & + &    0x^3 & + &  \cdots\\
         & = &  0  & + &  0  x & + &    0 x^2 & + &    0x^3 & + &  \cdots .
\end{array}
\]
Therefore,
\[
R(x)(1-x) - \frac{x(1+x)}{(1-x)^2} = 0,
\]
\iffalse
Therefore,
\begin{align*}
R(x)(1-7x-4x^2)
 & = \frac{1}{(1-x)^2} - (1 + 2x)\\
 & = \frac{1- (1+2x)(1-x)^2}{(1-x)^2}\\
 & = \frac{x^2(3+2x)}{(1-x)^2},
\end{align*}
\fi
so
\[
R(x) = \frac{x(1+x)}{(1-x)^3} \ .
\]

\end{solution}

\end{problem}


%%%%%%%%%%%%%%%%%%%%%%%%%%%%%%%%%%%%%%%%%%%%%%%%%%%%%%%%%%%%%%%%%%%%%
% Problem ends here
%%%%%%%%%%%%%%%%%%%%%%%%%%%%%%%%%%%%%%%%%%%%%%%%%%%%%%%%%%%%%%%%%%%%%
\endinput

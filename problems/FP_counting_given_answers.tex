\documentclass[problem]{mcs}

\begin{pcomments}
  \pcomment{FP_counting_given_answers}
%  \pcomment{edited excerpt of FP_counting_given_answers}
  \pcomment{longer version of CP_counting_given_answers}
  \pcomment{from: F15.final, S08.final,prob7; F03.final,prob5; F01.final}
  \pcomment{ARM 12/15/15}
\end{pcomments}

\pkeywords{
  counting
  binomial_coefficient
  bijection
  distinguishable
}

%%%%%%%%%%%%%%%%%%%%%%%%%%%%%%%%%%%%%%%%%%%%%%%%%%%%%%%%%%%%%%%%%%%%%
% Problem starts here
%%%%%%%%%%%%%%%%%%%%%%%%%%%%%%%%%%%%%%%%%%%%%%%%%%%%%%%%%%%%%%%%%%%%%

\begin{problem}

\begin{staffnotes}
\textbf{S17 final 8pts: part(a) 2 pts, others 1 pt each}
\end{staffnotes}


Formulas~(1)--(8) below are the answers to
questions~\eqref{x1x2xnm}--\eqref{mindnu} below, in no particular
order.  Enter the number of the correct solution in the box below each
question.

\[\begin{array}{rcrcrcrc}
(1).& \dfrac{n!}{(n-m)!} & (2). & \dbinom{n+m}{m} & (3). & (n-m)! & (4). & m^n\\
\\
(5). & \dbinom{n-1+m}{m} & (6). & \dbinom{n-1+m}{n} & (7). & 2^{mn} & (8). & n^m
\end{array}\]

\begin{staffnotes}
In class, ask that a brief explanation be written for each answer.
\end{staffnotes}

\bparts

\ppart\label{x1x2xnm} How many solutions over the nonnegative integers
are there to the inequality
\[
x_1 + x_2 + \cdots + x_n \leq m\ ?
\]

\exambox{0.6in}{0.5in}{-0.1in}

\begin{solution}
\[
\binom{n+m}{m}
\]
This is the same as the number of solutions to the equation the
equality $x_1 + x_2 + \cdots + x_n + y = m$, and which has a bijection
to bit-sequences with $m$ \STR{0}'s and $n$ \STR{1}'s.
\end{solution}

\iffalse
\ppart{How many $\sqrt{n} \times \sqrt{n}$ matrices are there with
entries drawn from $\set{1, 2, \dots, m}$?} 

\exambox{0.6in}{0.5in}{-0.1in}

\begin{solution}
$m^n$ This follows from the Product Rule, since there are $n^2$ matrix
  entries and $m$ choices for each.
\end{solution}
\fi

\inbook{
\ppart How many different subsets of the set $A \times B$ are there,
if $\card{A} = m$ and $\card{B} = n$? 

\exambox{0.6in}{0.5in}{-0.1in}

\begin{solution}
$2^{mn}$, because $\card{A \times B}= mn$.
\end{solution}

\ppart How many total functions are there from a set of size $n$ to a set of
  size $m$? 

\exambox{0.6in}{0.5in}{-0.1in}

\begin{solution}
$m^n$
\end{solution}
}

\ppart How many length $m$ words can be formed from an $n$-letter
alphabet, if no letter is used more than once?

\exambox{0.6in}{0.5in}{-0.1in}

\begin{solution}
\[
\frac{n!}{(n-m)!}
\]
There are $n$ choices for the first letter, $n-1$ choices for the
second letter, \dots $n - m +1$ choices for the $m$th letter, so by
the Generalized Product rule, the number of words is
\[
n \cdot (n-1) \cdots (n-m +1).
\]

%that is, $P(n,m)$.
\end{solution}

\ppart How many length $m$ words can be formed from an $n$-letter
  alphabet, if letters can be reused?

\exambox{0.6in}{0.5in}{-0.1in}

\begin{solution}
$n^m$ by the Product Rule.
\end{solution}

\ppart How many binary relations are there from set of size $m$ to set
  of size $n$?

\exambox{0.6in}{0.5in}{-0.1in}

\iffalse
\ppart How many different subsets of the set $A \times B$ are there
  when $\card{A} = m$ and $\card{B} = n$?

\exambox{0.6in}{0.5in}{-0.1in}
\fi

\begin{solution}
\[
2^{mn}
\]
The graph of a binary relations from a set $A$ to a set $B$ is a
subset of $A \times B$, so there are $2^{\card{A \times B}}$ such
relations.  If $\card{A} = m$ and $\card{B}=n$, then $\card{A \times B}= mn$.
\end{solution}

\inbook{
\ppart How many total injective functions are there from set of size
  $m$ to a set of size $n$ where $n \geq m$?

\exambox{0.6in}{0.5in}{-0.1in}

\begin{solution}
\[
\frac{n!}{(n-m)!}
\]
There is a bijection between the injections and the length-$m$
sequences of distinct elements of the size-$n$ set.  By the
Generalized Product rule, the number of such sequences is
\[
n \cdot (n-1) \cdots (n-m +1).
\]

%that is, $P(n,m)$.
\end{solution}
}

\ppart How many ways are there to place a total of $m$ distinguishable
  balls into $n$ distinguishable urns, with some urns possibly empty or
  with several balls?

\exambox{0.6in}{0.5in}{-0.1in}

\begin{solution}
\[
n^m
\]

There is a bijection between a placement of the balls and a length-$m$
sequence whose $i$th element is the urn where the $i$th ball is
placed.  So the number of placements is the same as the number of
length $m$ sequences of elements from a size-$n$ set.
\end{solution}

\ppart\label{mbdnd} How many ways are there to put a total of $m$ distinguishable
  balls into $n$ distinguishable urns with at most one ball in each
  urn?

\exambox{0.6in}{0.5in}{-0.1in}

\begin{solution}
\[
\frac{n!}{(n-m)!}
\]
There is a bijection between a placement of balls and a length-$m$
sequence whose $i$th element is the urn containing the $i$th ball.  So
the number of ball placements is the same as number of length $m$
sequences of distinct elements from a set of $n$ elements.
%that is, $P(n,m)$.
\end{solution}


\ppart\label{mindnu} How many ways are there to place a total of $m$
\textbf{in}distinguishable balls into $n$ distinguishable urns, with
some urns possibly empty or with several balls?

\exambox{0.6in}{0.5in}{-0.1in}

\begin{solution}
\[
\binom{n - 1 + m}{m}
\]
There is a bijection between placements and bit-sequences with $m$
\STR{0}'s and $n - 1$ \STR{1}'s, the length of the $i$th block of
\STR{0}'s being the number of balls in the $i$th urn.
\end{solution}


\eparts
\end{problem}

%%%%%%%%%%%%%%%%%%%%%%%%%%%%%%%%%%%%%%%%%%%%%%%%%%%%%%%%%%%%%%%%%%%%%
% Problem ends here
%%%%%%%%%%%%%%%%%%%%%%%%%%%%%%%%%%%%%%%%%%%%%%%%%%%%%%%%%%%%%%%%%%%%%

\endinput

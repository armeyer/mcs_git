\documentclass[handout]{mcs}

\begin{document}
\paper{}{Oct. 7, 2009}{Appendix for Oct. 7 Miniquiz}

\iffalse

\section*{Appendix: Mapping Rule}

Let $R: A \to B$ be a binary relation, where $A$ and $B$ are finite sets.

\begin{itemize}

\item\label{mapping-sur} If $R$ is a surjective function, then $\card{A} \geq \card{B}$.

\item\label{mapping-inj} If $R$ is total and injective, then $\card{A} \leq \card{B}$.

\item\label{mapping-bij} If $R$ is a bijection, then $\card{A} = \card{B}$.

\end{itemize}

\section*{Appendix: Russell's Paradox}

\textbox{
\begin{quote}
Let $S$ be a variable ranging over all sets, and define
\[W \eqdef \set{S \suchthat S \not\in S}.\]
So by definition,
\[S \in W  \mbox{  iff  } S \not\in S,\]
for every set $S$.  In particular, we can let $S$ be $W$, and obtain
the contradictory result that
\[W \in W  \mbox{  iff  } W \not\in W.\]
\end{quote}}

\section*{Appendix: Size of Power Sets}

\begin{quote}
  Define a surjection relation $A \surj B$ iff there is a surjective 
  function from $A$ to $B$. Then
  \[ NOT (A \surj \power(A)). \]
\end{quote}

\fi

\section*{Appendix: Partial Order}

A binary relation, $R$, on a set, $A$, is
\begin{itemize}

\item \term{transitive} iff\quad $[(a\mrel{R}b\ \QAND\ b\mrel{R}c)\ \QIMPLIES\
  a\mrel{R} c]$, for all $a,b,c\in A$,

\item \term{asymmetric} iff\quad $[a\mrel{R}b\ \QIMPLIES \ \QNOT(b\mrel{R}a)]$,
  for all $a,b\in A$,

\item \term{reflexive} iff\quad $a \mrel{R} a$ for every $a \in A$,

\item \term{irreflexive} iff\quad $\QNOT(a \mrel{R} a)$ for all $a \in A$.

\item \term{antisymmetric} iff\quad $[a\mrel{R}b\ \QIMPLIES \QNOT(b\mrel{R}a)]$,
  for all $a \neq b\in A$.

\end{itemize}

A binary relation is a \term{strict partial order} iff it is transitive
and asymmetric.  It is a \term{weak partial order} iff it is transitive,
reflexive, and antisymmetric.

Let $\prec$ be a partial order (strict or weak) on a set, $A$.
\begin{itemize}

\item An element $a \in A$ is \term{minimal} iff there is no \emph{other}
  element in $A$ that is $\prec a$.  Similarly, an element $a \in A$ is
  \term{maximal} iff there is no \emph{other} element in $A$ that $a$ is
  $\prec$ to.

\item An element $a \in A$ is the \term{minimum} element iff $a$ is
  $\prec$ to every \emph{other} element of $A$.  Similarly, an element $a
  \in A$ is the \term{maximum} element iff every \emph{other} element of
  $A$ is $\prec a$.

\item Elements $a,b \in A$ are \term{comparable} iff either $a \prec b$ or
  $b \prec a$.  Two elements are \term{incomparable} iff they are not
  comparable.

\item A subset, $S \subseteq A$ is \term{totally ordered} iff every
two distinct elements in $S$ are comparable.

\item A \term{chain} is a totally ordered subset of $A$.  

\item An \term{antichain} is a subset of $A$, such that no two distinct
  elements in it are comparable.

\iffalse
\item An element $a \in A$ is the \term{greatest lower bound (glb)} of a
subset, $S \subseteq A$ iff $a$ is a lower bound for $S$, and if $b \in A$
is also a lower bound for $S$, then $b \preceq a$.  Similarly for
\term{least upper bound (lub)}.
\fi

\iffalse
\item The partial order, $\prec$, is \term{well-founded} iff every
  nonempty subset of $A$ has at least one minimal element.
\fi

\end{itemize}

\iffalse

\section*{Appendix: Dilworth's Lemma}

\hyperdef{rule}{Dilworth}{For} all $t>0$, every partially ordered set with
$n$ elements must have either a chain of size greater than $t$ or an
antichain of size at least $n / t$.

\fi

\end{document}

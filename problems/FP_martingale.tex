\documentclass[problem]{mcs}

\begin{pcomments}
  \pcomment{FP_martingale}
  \pcomment{close to CP_fair_martingale}
  \pcomment{F07.rec14h}
\end{pcomments}

\pkeywords{
  paradox
  roulette
  mean_time_to_failure
  martingale
}

%%%%%%%%%%%%%%%%%%%%%%%%%%%%%%%%%%%%%%%%%%%%%%%%%%%%%%%%%%%%%%%%%%%%%
% Problem starts here
%%%%%%%%%%%%%%%%%%%%%%%%%%%%%%%%%%%%%%%%%%%%%%%%%%%%%%%%%%%%%%%%%%%%%

\begin{problem}
A gambler bets \$10 on ``red'' at a roulette table (the odds of red are
18/38, slightly less than even) to win \$10.  If he wins, he gets
back twice the amount of his bet, and he quits.  Otherwise, he doubles his
previous bet and continues.

For example, if he loses his first two bets but wins his third bet,
the total spent on his three bets is $10+20+40$ dollars, but he gets
back $2 \cdot 40$ dollars after his win on the third bet, for a net
profit of \$10.

\bparts

\ppart What is the expected number of bets the gambler makes before he
wins?

\begin{solution}
This is mean time to failure, with failure being a red number coming up.  So
the expected time (number of bets) is
\[
\frac{1}{18/38} = 2\ \frac{1}{9}.
\]
\end{solution}

\ppart What is his probability of winning?

\begin{solution}
He is certain to win, since $\pr{> k \text{ bets}} = (20/38)^{k}$ which
goes to zero as $k$ goes to infinity.  More fully,
\[
\pr{\text{win}} \geq \pr{\text{win in $\leq k$ bets}}
= 1 - \pr{> k \text{ bets}}
\]
and this last expression goes to 1 as $k$ goes to infinity.
\end{solution}

\ppart What is his expected final profit (amount won minus amount lost)?

\begin{solution}
His final profit is always \$10 whenever he finally wins, and he is
certain to win, so \$10 is also his expected final profit.
\end{solution}

\ppart You can beat a biased game by bet doubling, but bet doubling is
not feasible because it requires an infinite bankroll.  Verify this by
proving that the expected size of the gambler's last bet is infinite.

\begin{solution}
Let $B$ be the size of his last bet in dollars.  Now if he wins his
$\$10$ final profit on the $k$th bet, then $B=10\cdot 2^{k-1}$, so
\[
\pr{B=10\cdot 2^{k-1}} = (20/38)^{k-1}(18/38)
\]
Therefore
\begin{align*}
\expect{B} & = \sum_{k \in \nngint^{+}} 10\cdot
      2^{k-1}\paren{\frac{20}{38}}^{k-1}(18/38)\\
  & = 10(18/38) \sum_{k \in \nngint} 2^k\paren{(20/38)^k}\\
  & > \sum_{k \in \nngint} \paren{1+ \frac{1}{19}}^k= \infty.
\end{align*}

\end{solution}

\begin{staffnotes}
If used as class problem, try adding:

\ppart Prove that if the gambler only has enough money to make $n$
bets, then his loss grows exponentially with $n$.

\begin{solution}
Let $p \eqdef 18/38$ be the probability of red on one spin, and let $q
\eqdef 1-p$.  The probability of making a $k$th bet is the
probability of $k-1$ consecutive not-red spins, namely, $q^{k-1}$.
The amount bet on the $k$th spin is $10\cdot 2^{k-1}$.  So the
expected win on the $k$th bet is
\[
10\cdot 2^{k-1}(p-q)q^{k-1} = 10(2p-1)(2q)^{k-1}.
\]
So the expected win with $n$ bets is
\[
\sum_{k=1}^n 10(2p-1)(2q)^{k-1} = 10(2p-1)\frac{(2q)^n -1}{2q-1} = -10((2q)^n -1).
\]
Since $q > 1/2$, this loss grows exponentially in $n$.
\end{solution}
\end{staffnotes}

\eparts

\end{problem}


%%%%%%%%%%%%%%%%%%%%%%%%%%%%%%%%%%%%%%%%%%%%%%%%%%%%%%%%%%%%%%%%%%%%%
% Problem ends here
%%%%%%%%%%%%%%%%%%%%%%%%%%%%%%%%%%%%%%%%%%%%%%%%%%%%%%%%%%%%%%%%%%%%%

\endinput

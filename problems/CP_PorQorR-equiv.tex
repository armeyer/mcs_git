\documentclass[problem]{mcs}

\begin{pcomments}
  \pcomment{from: S09}
  \pcomment{ARM renamed something else PorQorR by accident, 9/11/09 5:25PM}
\end{pcomments}

\pkeywords{
Proposition
equivalence
}

%%%%%%%%%%%%%%%%%%%%%%%%%%%%%%%%%%%%%%%%%%%%%%%%%%%%%%%%%%%%%%%%%%%%%
% Problems start here
%%%%%%%%%%%%%%%%%%%%%%%%%%%%%%%%%%%%%%%%%%%%%%%%%%%%%%%%%%%%%%%%%%%%%


\begin{problem}
Prove that the propositional formulas
\[
P \QOR Q \QOR R
\]
and
\[
(P \QAND \QNOT Q) \QOR (Q \QAND \QNOT R) \QOR (R \QAND \QNOT P) \QOR (P \QAND Q \QAND R).
\]
are equivalent.

\begin{solution}
We compare $P  \vee  Q  \vee  R$ and $K = (P  \wedge  \neg Q)  \vee  (Q  \wedge  \neg R)  \vee  (R  \wedge  \neg P)  \vee  (P  \wedge  Q  \wedge  R)$ using a truth table:

\begin{center}
\begin{tabular}{ c c c | c | c c c c | c }
  $P$ & $Q$ & $R$ & $P  \vee  Q  \vee  R$ & $P  \wedge  \neg Q & Q  \wedge  \neg R &  R  \wedge  \neg P & P  \wedge  Q  \wedge  R & K$\\
  \hline
  T & T & T & \textbf{T} & F & F & F & T & \textbf{T} \\
  T & T & F & \textbf{T} & F & T & F & F & \textbf{T} \\
  T & F & T & \textbf{T} & T & F & F & F & \textbf{T} \\
  T & F & F & \textbf{T} & T & F & F & F & \textbf{T} \\
  F & T & T & \textbf{T} & F & F & T & F & \textbf{T} \\
  F & T & F & \textbf{T} & F & T & F & F & \textbf{T} \\
  F & F & T & \textbf{T} & F & F & T & F & \textbf{T} \\
  F & F & F & \textbf{F} & F & F & F & F & \textbf{F} \\
\end{tabular}
\end{center}

Both $P  \vee  Q  \vee  R$ and K have identical truth tables, thus the two statements are equivalent.

\end{solution}
\end{problem}


%%%%%%%%%%%%%%%%%%%%%%%%%%%%%%%%%%%%%%%%%%%%%%%%%%%%%%%%%%%%%%%%%%%%%
% Problems end here
%%%%%%%%%%%%%%%%%%%%%%%%%%%%%%%%%%%%%%%%%%%%%%%%%%%%%%%%%%%%%%%%%%%%%

\endinput

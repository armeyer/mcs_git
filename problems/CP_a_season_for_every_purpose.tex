\documentclass[problem]{mcs}

\begin{pcomments}
  \pcomment{from: S09.cp2t}
%  \pcomment{}
\end{pcomments}

\pkeywords{
  quantifiers
  predicate_calculus
  translating_english_statements
}

%%%%%%%%%%%%%%%%%%%%%%%%%%%%%%%%%%%%%%%%%%%%%%%%%%%%%%%%%%%%%%%%%%%%%
% Problem starts here
%%%%%%%%%%%%%%%%%%%%%%%%%%%%%%%%%%%%%%%%%%%%%%%%%%%%%%%%%%%%%%%%%%%%%

\begin{problem} 
When the Poet says ``There is a season for every purpose under heaven.''
Which of the following does he mean:
\begin{equation}\label{EA}
\exists s \in {\text{Season}}\,\forall p \in \text{Purpose}.\ s \text{ is
for }p
\end{equation}
or
\begin{equation}\label{AE}
\forall p \in \text{Purpose}\,\exists s \in \text{Season}.\
s \text{ is for }p
\end{equation}
Briefly explain.

\solution{ This poetic statement is meant to offer solace: this may be a
  bad season for you now, but be hopeful, a season that suits your purpose
  will come.  So the appropriate translation would be formula~\eqref{AE},
  namely that given your Purpose, you can find a season that's good for
  it.  For example, if your purpose is planting, take heart: even though
  it's Winter now, Spring is coming.

Formula~\eqref{EA} says you can find a single season, say Spring, that's
good for every possible Purpose like skiiing, leaf watching, \dots.  This
is false, so it's clearly not what the Poet meant.  But even though he
really meant~\eqref{AE}, he used his poetic license to express~\eqref{AE}
in a way that mechanically would translate into~\eqref{EA}.

Note that a similar statement, ``There is a man for all seasons,'' is
famously used to describe one extraordinarily versatile man, Sir Thomas
More.  So this statement would actually best be translated as
\[
\exists x \in\, \text{men}\; \forall s \in\, \text{seasons}.\
x\ \text{is (good) for}\ s
\]
}
\end{problem}

%%%%%%%%%%%%%%%%%%%%%%%%%%%%%%%%%%%%%%%%%%%%%%%%%%%%%%%%%%%%%%%%%%%%%
% Problem ends here
%%%%%%%%%%%%%%%%%%%%%%%%%%%%%%%%%%%%%%%%%%%%%%%%%%%%%%%%%%%%%%%%%%%%%

\endinput
